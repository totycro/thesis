\documentclass[,%fontsize=11pt,%
	%landscape,
	%DIV8, % mehr text pro seite als defaultyyp
	%DIV10,
	%DIV=calc,%
	draft=false,% final|draft % draft ist platzsparender (kein code, bilder..)
	%titlepage,
	numbers=noendperiod
	11pt,
	a4paper,
	oneside,% apparently, this should stay below some other parameter to have an effect
	openany,
	%]{scrartcl}
]{memoir}



\usepackage[utf8]{inputenc}
\usepackage[T1]{fontenc}
\usepackage[english]{babel}


\newcommand{\changefont}[3]{
\fontfamily{#1}\fontseries{#2}\fontshape{#3}\selectfont}

%\renewcommand{\familydefault}{ \sfdefault }
%\renewcommand{\rmdefault}{ppl}


%\usepackage[urw-garamond]{mathdesign}

\usepackage{lscape}
\usepackage{stackengine}
\usepackage{enumerate}
\usepackage{paralist}
\usepackage{tikz}
\usetikzlibrary{shapes,arrows,backgrounds,graphs,%
	matrix,patterns,arrows,decorations.pathmorphing,decorations.pathreplacing,%
	positioning,fit,calc,decorations.text,shadows%
}


\usepackage{comment} 

\usepackage{etoolbox} % fixes fatal error caused by combining bm, stackengine, hyperref (seriously?)
% http://tex.stackexchange.com/questions/22995/package-incompatibilites-etoolbox-hyperref-and-bm-standalone

\usepackage{etex} % else error on too many packages

% includes
\usepackage{algorithm}
%\usepackage{algorithmic} % conflicts with algpseudocode
\usepackage{algpseudocode}
%\newcommand*\Let[2]{\State #1 $\gets$ #2}
\algrenewcommand\alglinenumber[1]{
{\scriptsize #1}}
\algrenewcommand{\algorithmicrequire}{\textbf{Input:}}
\algrenewcommand{\algorithmicensure}{\textbf{Output:}}


%\usepackage[multiple]{footmisc} % footnotes at the same character separated by ','

\usepackage{multicol}

\usepackage{afterpage}

\usepackage{changepage} % for adjustwidth
\usepackage{caption} % for \ContinuedFloat

\usepackage{tikz}
\usetikzlibrary{shapes,arrows,backgrounds,graphs,%
matrix,patterns,arrows,decorations.pathmorphing,decorations.pathreplacing,%
positioning,fit,calc,decorations.text,shadows%
}

\usepackage{bussproofs}
\EnableBpAbbreviations


\usepackage{amsmath}
\usepackage{amsthm}
\usepackage{amssymb} % the reals
\usepackage{mathtools} % smashoperator

\usepackage{bm} % bm, bold math symbols

\usepackage{thm-restate} % restatable env

% needs extra work and fails on some label here
%\usepackage{cleveref} % cref, apparently better than autoref of hyperref 

\usepackage{nicefrac} % nicefrac

\usepackage{mathrsfs} % mathscr

\usepackage{pst-node} % http://tex.stackexchange.com/questions/35717/how-to-draw-arrows-between-parts-of-an-equation-to-show-the-math-distributive-pr

\usepackage{stackengine}

\usepackage{thmtools} % advanced thm commands (declaretheorem)


\usepackage{nameref} % reference name of thm instead of counter

\usepackage{todonotes}

% conflict with beamer
%\usepackage{paralist} % compactenum

\usepackage{hyperref}
%\hypersetup{hidelinks}  % don't give options to usepackage, it doesn't work with beamer
%\hypersetup{colorlinks=false}  % don't give options to usepackage, it doesn't work with beamer


% \usepackage{enumitem} % labels for enumerate % breaks beamer and memoir itemize


\usepackage{url} 


\usepackage[format=hang,justification=raggedright]{caption}% or e.g. [format=hang]

\usepackage{cancel} % \cancel

\usepackage{lineno}


% commands

% logic etcs
%\newcommand{\ex}[2]{\bigskip\section*{Exercise #1: \begin{minipage}[t]{.80\linewidth} \small \textnormal{\it #2} \end{minipage} } }

\newcommand{\ex}[2]{\bigskip \noindent\textbf{Exercise #1.} \textit{#2} \smallskip}

\newcommand{\comm}[1]{{\color{gray} // #1 }}


\newcommand{\true}[0]{\textbf{1}}
\newcommand{\false}[0]{\textbf{0}}
\newcommand{\tr}{\true}
\newcommand{\fa}{\false}

\newcommand{\ra}{\rightarrow}
\newcommand{\Ra}{\Rightarrow}
\newcommand{\la}{\leftarrow}
\newcommand{\La}{\Leftarrow}

\newcommand{\lra}{\leftrightarrow}
\newcommand{\Lra}{\Leftrightarrow}

\newcommand{\NKZ}{\textbf{NK2}}

%\DeclareMathOperator{\syneq}{\equiv} %spacing seems wrong, therefore defined as newcommand below
\DeclareMathOperator{\limpl}{\supset}
\DeclareMathOperator{\liff}{\lra}
\DeclareMathOperator{\semiff}{\Lra}
\newcommand{\syneq}{\equiv}
\newcommand{\union}{\cup}
\newcommand{\bigunion}{\bigcup}
\newcommand{\intersection}{\cap}
\newcommand{\bigintersection}{\bigcap}
\newcommand{\intersect}{\intersection}
\newcommand{\bigintersect}{\bigintersection}

\newcommand{\powerset}{\mathcal{P}}

\newcommand{\entails}{\vDash}
\newcommand{\notentails}{\nvDash}
\newcommand{\proves}{\vdash}

\newcommand{\vm}{\ensuremath{\vv_\mathcal{M}}}
\newcommand{\Dia}{\ensuremath{\lozenge}}

\newcommand{\spaced}[1]{\ \ #1 \ \ }
\newcommand{\spa}[1]{\spaced{#1}}
\newcommand{\spas}[1]{\;{#1}\;}
\newcommand{\spam}[1]{\;\,{#1}\;\,}

% functions
\DeclareMathOperator{\sk}{sk}
\DeclareMathOperator{\mgu}{mgu}
\DeclareMathOperator{\dom}{dom}
\DeclareMathOperator{\ran}{ran}

\DeclareMathOperator{\id}{id}
\DeclareMathOperator{\Fun}{FS}
\DeclareMathOperator{\Pred}{PS}
\DeclareMathOperator{\Lang}{L}
\DeclareMathOperator{\ar}{ar}
\DeclareMathOperator{\PI}{PI}
\DeclareMathOperator{\LI}{LI}
\DeclareMathOperator{\Congr}{Congr}
\DeclareMathOperator{\Refl}{Refl}
\DeclareMathOperator{\aiu}{au}
\DeclareMathOperator{\expa}{unfold-lift}

\newcommand{\PIinc}{\LI}
\newcommand{\PIincde}{\LIde}

\newcommand{\LIde}{\ensuremath{\LI^\Delta}}

\newcommand{\LIcl}{\ensuremath{\LI_{\operatorname{cl}}}}
\newcommand{\LIclde}{\ensuremath{\LI_{\operatorname{cl}}^\Delta}}

\newcommand{\cll}{\ensuremath{_{\operatorname{LIcl}}}}
\newcommand{\cllde}{\ensuremath{_{\operatorname{LIcl}^\Delta}}}

%\newcommand{\lifi}{\mathop{\ell\text{}i}}
\newcommand{\lifiboth}[1]{\ensuremath{\LIcl(#1)}}
\newcommand{\lifidelta}[1]{\ensuremath{\LIclde(#1)}}


%\DeclareMathOperator{\abstraction}{abstraction}

%\newcommand{\sk}{\ensuremath{\mathrm{sk}}}
%\newcommand{\mgu}{\ensuremath{\mathrm{mgu}}}
%\newcommand{\Fun}{\ensuremath{\mathrm{FS}}}
%\newcommand{\Pred}{\ensuremath{\mathrm{PS}}}
%\newcommand{\PI}{\ensuremath{\mathrm{PI}}}
%\newcommand{\Lang}{\ensuremath{\mathrm{L}}}
%\newcommand{\ar}{\ensuremath{\mathrm{ar}}}

\DeclareMathOperator{\AI}{AI}
\newcommand{\AIde}{\ensuremath{\AI^\Delta}}
\newcommand{\AImatrix}{\ensuremath{\AI_\mathrm{mat}}}
\newcommand{\AImatrixde}{\ensuremath{\AI_\mathrm{mat}^\Delta}}
\newcommand{\AImat}{\AImatrix}
\newcommand{\AImatde}{\AImatrixde}
\newcommand{\AIclause}{\ensuremath{\AI_\mathrm{cl}}}
\newcommand{\AIcl}{\AIclause}
\newcommand{\AIclde}{\AIclausede}
\newcommand{\AIclausede}{\ensuremath{\AIclause^\Delta}}
\newcommand{\fromclause}{\ensuremath{_{\operatorname{AIcl}}}}
\newcommand{\fromclausede}{\ensuremath{_{\operatorname{AIcl}^\Delta}}}
\newcommand{\cl}{\fromclause}
\newcommand{\clde}{\fromclausede}

\newcommand{\Q}{\ensuremath{Q}}

\newcommand{\AIcol}{\ensuremath{\AI_\mathrm{col}}}
\newcommand{\AIcolde}{\AIcol^\Delta}

\newcommand{\AIany}{\ensuremath{\AI_\mathrm{*}}}
\newcommand{\AIanyde}{\AIany^\Delta}

\newcommand{\AIclpre}{\AIclause^\bullet}
\newcommand{\AImatpre}{\AImatrix^\bullet}

\newcommand{\PS}{\Pred}
\newcommand{\FS}{\Fun}

\DeclareMathOperator{\LangSym}{\mathcal{L}}

%\newcommand{\mguarr}{\sim_\ra}
\newcommand{\mguarr}{\mapsto_{\mgu}}


%\newcommand{\Trans}{\ensuremath{\mathrm{T}}}
%\newcommand{\Trans}{\ensuremath{\mathrm{T}}}
\DeclareMathOperator{\Trans}{T}
\DeclareMathOperator{\TransInv}{T^{-1}}

\DeclareMathOperator{\FAX}{F_{Ax}}
\DeclareMathOperator{\EAX}{E_{Ax}}
%\newcommand{\FAX}{\ensuremath{\mathrm{F_{Ax}}}}
%\newcommand{\EAX}{\ensuremath{\mathrm{E_{Ax}}}}

%\newcommand{\TransAll}{\ensuremath{\Trans_{\mathrm{Ax}}}}
\DeclareMathOperator{\TransAll}{\Trans_{Ax}}
%\newcommand{\FAX}{\ensuremath{\mathrm{F_{Ax}}}}

\DeclareMathOperator{\defeq}{\stackrel{\mathrm{def}}{=}}

\newcommand{\subst}[1]{[#1]}
\newcommand{\abstractionOp}[1]{\{#1\}}

\newcommand{\subformdefinitional}[1]{\ensuremath{D_{\Sigma(#1)}}}


%\newcommand{\lift}[3]{\operatorname{Lift}_{#1}(#2; #3)}
%\newcommand{\lift}[3]{\operatorname{Lift}_{#1,#3}(#2)}
%\newcommand{\lift}[3]{\operatorname{Lift}_{#1,#3}[#2]}
%\newcommand{\lift}[3]{\overline{#2}_{#1,#3}}
\newcommand{\lifsym}{\ell}
%\newcommand{\lift}[3]{\lifsym_{#1,#3}[#2]}
\newcommand{\lift}[3]{\lifsym_{#1}^{#3}[#2]}
\newcommand{\liftnovar}[2]{\lifsym_{#1}[#2]}

%\newcommand{\lft}[3]{\lifsym_{#1,#2}[#3]}
\newcommand{\lft}[3]{\lift{#1}{#3}{#2}}
\newcommand{\lifboth}[1]{\lifsym[#1]}

%\newcommand{\lifi}{\mathop{\ell\text{}i}}
%\newcommand{\lifiboth}[1]{\lifi[#1]}
%\newcommand{\lifidelta}[1]{\lifi_\Delta^x[#1]}
%\newcommand{\lifideltanovar}[1]{\lifi_\Delta[#1]}

\newcommand{\lifdelta}[1]{\lift{\Delta}{#1}{x}}
\newcommand{\lifdeltanovar}[1]{\liftnovar{\Delta}{#1}}
\newcommand{\lifgamma}[1]{\lift{\Gamma}{#1}{y}}
\newcommand{\lifgammanovar}[1]{\liftnovar{\Gamma}{#1}}
\newcommand{\lifphinovar}[1]{\liftnovar{\Phi}{#1}}
\newcommand{\lifphi}[1]{\lift{\Phi}{#1}{z}}

\DeclareMathOperator{\arr}{\mathcal{A}}
%\DeclareMathOperator{\arrFinal}{{\mathcal{A}^{\bm*}}}
\DeclareMathOperator{\arrFinal}{{\mathcal{\bm{\hat}A}}}
\DeclareMathOperator{\warr}{\marr}
\DeclareMathOperator{\marr}{\mathcal{M}}

\DeclareMathOperator{\apath}{\leadsto}
\DeclareMathOperator{\mpath}{\leadsto_=}
\DeclareMathOperator{\notapath}{\not\leadsto}
\DeclareMathOperator{\notmpath}{\not\leadsto_=}

\newcommand{\ltArrC}{<_{\arrFinal(C)}}
\newcommand{\ltAC}{<_{\arr(C)}}
\newcommand{\ltArrCOne}{<_{\arrFinal(C_1)}}
\newcommand{\ltArrCTwo}{<_{\arrFinal(C_2)}}
%\newcommand{\ltArrC}{<_{\scalebox{0.6}{$\arrFinal(C)$}}}
\newcommand{\ltArr}{<_{\scalebox{0.6}{$\arrFinal$}}}

\newcommand{\bhat}{\bm\hat}
\newcommand{\bbar}{\bm\bar}
\newcommand{\bdot}{\bm\dot}

%\usepackage{yfonts}
\usepackage{upgreek}
\DeclareMathAlphabet{\mathpzc}{OT1}{pzc}{m}{it}
%\DeclareMathOperator{\pos}{\mathscr{P}}
%\DeclareMathOperator{\pos}{\mathpzc{p}}
%\DeclareMathOperator{\pos}{{\rho}}
\DeclareMathOperator{\pos}{{\operatorname P}}
%\DeclareMathOperator{\pos}{P}
\DeclareMathOperator{\poslit}{\pos_\mathrm{lit}}
\DeclareMathOperator{\posterm}{\pos_\mathrm{term}}
%\newcommand{\poslit}[1]{\ensuremath{p_\text{lit}(#1)}}
%\newcommand{\posterm}[1]{\ensuremath{p_\text{term}(#1)}}
\newcommand{\at}[1]{|_{#1}}

\newcommand{\UICm}[1]{\UnaryInfCm{#1}}
\newcommand{\UnaryInfCm}[1]{\UnaryInfC{$#1$}}
\newcommand{\BICm}[1]{\BinaryInfCm{#1}}
\newcommand{\BinaryInfCm}[1]{\BinaryInfC{$#1$}}
\newcommand{\RightLabelm}[1]{\RightLabel{$#1$}}
\newcommand{\LeftLabelm}[1]{\LeftLabel{$#1$}}
\newcommand{\AXCm}[1]{\AxiomCm{#1}}
\newcommand{\AxiomCm}[1]{\AxiomC{$#1$}}
\newcommand{\mt}[1]{\textnormal{#1}}

\newcommand{\UnaryInfm}[1]{\UnaryInf$#1$}
\newcommand{\BinaryInfm}[1]{\BinaryInf$#1$}
\newcommand{\Axiomm}[1]{\Axiom$#1$}



% math
\newcommand{\calI}{\ensuremath{\mathcal{I}}}

\newcommand{\tupleShort}[2]{\ensuremath{(#1_1,\dotsc,#1_{#2})}}
\newcommand{\tuple}[2]{\ensuremath{(#1_1,\:#1_2\:,\dotsc,\:#1_{#2})}}
\newcommand{\setelements}[2]{\ensuremath{\{#1_1,\:#1_2\:,\dotsc,\:#1_{#2}\}}}
\newcommand{\pathelements}[2]{\ensuremath{ (#1_1,\:#1_2\:,\dotsc,\:#1_{#2}) }}

\newcommand{\elems}[1]{\ensuremath{#1_1,\dotsc, #1_{n}) }}

\newcommand{\defiemph}[1]{\emph{#1}}

\newcommand{\setofbases}{\ensuremath{\mathcal{B}}}
\newcommand{\setofcircuits}{\ensuremath{\mathcal{C}}}

\newcommand{\reals}{\ensuremath{\mathbb{R}}}
\newcommand{\integers}{\ensuremath{\mathbb{Z}}} 
\newcommand{\naturalnumbers}{\ensuremath{\mathbb{N}}}

% general
\newcommand{\zit}[3]{#1\ \cite{#2}, #3}
\newcommand{\zitx}[2]{#1\ \cite{#2}}
\newcommand{\footzit}[3]{\footnote{\zit{#1}{#2}{#3}}}
\newcommand{\footzitx}[2]{\footnote{\zitx{#1}{#2}}}

\newcommand{\ite}{\begin{itemize}}
\newcommand{\ete}{\end{itemize}}

\newcommand{\bfr}{\begin{frame}}
\newcommand{\efr}{\end{frame}}

\newcommand{\ilc}[1]{\texttt{#1}}


% misc

% multiframe
\usepackage{xifthen}% provides \isempty test
% new counter to now which frame it is within the sequence
\newcounter{multiframecounter}
% initialize buffer for previously used frame title
\gdef\lastframetitle{\textit{undefined}}
% new environment for a multi-frame
\newenvironment{multiframe}[1][]{%
\ifthenelse{\isempty{#1}}{%
% if no frame title was set via optional parameter,
% only increase sequence counter by 1
\addtocounter{multiframecounter}{1}%
}{%
% new frame title has been provided, thus
% reset sequence counter to 1 and buffer frame title for later use
\setcounter{multiframecounter}{1}%
\gdef\lastframetitle{#1}%
}%
% start conventional frame environment and
% automatically set frame title followed by sequence counter
\begin{frame}%
\frametitle{\lastframetitle~{\normalfont \Roman{multiframecounter}}}%
}{%
\end{frame}%
}




% http://texfragen.de/hurenkinder_und_schusterjungen
\usepackage[all]{nowidow}



% force no overlong lines:
%\tolerance=1 % tolerance for how badly spaced lines are allowed, less means "better" lines
\tolerance=500 %  need more tolerance for equations
%\emergencystretch=\maxdimen
%\emergencystretch=200pt
%\setlength{\emergencystretch}{3em}
%\hyphenpenalty=10000 % forces no hyphenation
%\hbadness=10000


% http://tex.stackexchange.com/questions/35717/how-to-draw-arrows-between-parts-of-an-equation-to-show-the-math-distributive-pr
\tikzset{square arrow/.style={to path={ -- ++(.0,-.15)  -| (\tikztotarget)}}}
\tikzset{square arrow2/.style={to path={ -- ++(.0,-.25)  -| (\tikztotarget)}}}
%\tikzset{square arrow/.style={to path={ -- ++(00,-.01) -- ++(0.5,-0.1) -- ++(0.5,-0.1) -| (\tikztotarget)},color=red}}


% have arrows from a to b and from c to d here
% just use: texttext\arrowA texttest \arrowB texttext
\newcommand{\arrowA}{\tikz[overlay,remember picture] \node (a) {};}
\newcommand{\arrowB}{\tikz[overlay,remember picture] \node (b) {};}
\newcommand{\drawAB}{
	\tikz[overlay,remember picture]
	{\draw[->,bend left=5,color=red] (a.south) to (b.south);}
	%{\draw[->,square arrow,color=red] (a.south) to (b.south);}
}
\newcommand{\arrowAP}{\tikz[overlay,remember picture] \node (ap) {};}
\newcommand{\arrowBP}{\tikz[overlay,remember picture] \node (bp) {};}
\newcommand{\drawABP}{
	\tikz[overlay,remember picture]
	{\draw[->,bend right=5,color=red] (ap.south) to (bp.south);}
	%{\draw[->,square arrow,color=red] (a.south) to (b.south);}
}

\newcommand{\arrowAB}{\tikz[overlay,remember picture] \node (ab) {};}
\newcommand{\arrowBA}{\tikz[overlay,remember picture] \node (ba) {};}
\newcommand{\drawAABB}{
	\tikz[overlay,remember picture]
	%{\draw[->,bend left=80] (a.north) to (b.north);}
	{\draw[->,square arrow,color=brown] (ab.south) to (ba.south);
	\draw[->,square arrow,color=brown] (ba.south) to (ab.south);}
}


\newcommand{\arrowCD}{\tikz[overlay,remember picture] \node (cd) {};}
\newcommand{\arrowDC}{\tikz[overlay,remember picture] \node (dc) {};}
\newcommand{\drawCCDD}{
	\tikz[overlay,remember picture]
	%{\draw[->,bend left=80] (a.north) to (b.north);}
	{\draw[<->,dashed,square arrow,color=green] (cd.south) to (dc.south); }
}



\newcommand{\arrowC}{\tikz[overlay,remember picture] \node (c) {};}
\newcommand{\arrowD}{\tikz[overlay,remember picture] \node (d) {};}
\newcommand{\drawCD}{
	\tikz[overlay,remember picture]
	{\draw[->,square arrow,color=blue] (c.south) to (d.south);}
}

\newcommand{\arrowE}{\tikz[overlay,remember picture] \node (e) {};}
\newcommand{\arrowF}{\tikz[overlay,remember picture] \node (f) {};}
\newcommand{\drawEF}{
	\tikz[overlay,remember picture]
	{\draw[->,square arrow2,color=orange] (e.south) to (f.south);}
}


\newcommand{\arrAP}{\arrowAP}
\newcommand{\arrBP}{\arrowBP}
\newcommand{\arrA}{\arrowA}
\newcommand{\arrB}{\arrowB}
\newcommand{\arrC}{\arrowC}
\newcommand{\arrD}{\arrowD}
\newcommand{\arrE}{\arrowE}
\newcommand{\arrF}{\arrowF}


\DeclareMathOperator{\simgeq}{\scalebox{0.92}{$\gtrsim$}}

\newcommand{\refsub}[2]{\hyperref[#2]{\ref*{#1}.\ref*{#2}}}

%\newcommand{\sigmarange}[2]{\sigma_{#1}^{#2} }
\newcommand{\sigmarange}[2]{\sigma_{(#1,#2)} }
\newcommand{\sigmaz}[1]{\sigmarange{0}{#1} }
\newcommand{\sigmazi}[0]{\sigmaz{i} }

\DeclareMathOperator{\lit}{lit}

%\def\fCenter{\ \proves\ }
\def\fCenter{\proves}

\newcommand{\prflbl}[2]{\RightLabel{\footnotesize $#1, #2$} }
%\newcommand{\prflblid}[1]{\RightLabel{$#1, \id$} }
\newcommand{\prflblid}[1]{\RightLabel{\footnotesize $#1$} }

\DeclareMathOperator{\resruleres}{res}
\DeclareMathOperator{\resrulefac}{fac}
\DeclareMathOperator{\resrulepar}{par}
\newcommand{\lkrule}[2]{\ensuremath{\operatorname{#1}:#2}} % operatorname fixes spacing issues for =

\newcommand{\parti}[4]{\ensuremath{ \langle (#1; #2), (#3; #4)\rangle  }}

\newcommand{\partisym}{\ensuremath{\chi}}

\newcommand{\occur}[1]{\ensuremath{[#1]}}
\newcommand{\occ}[1]{\occur{#1}}

\newcommand{\occurat}[2]{\ensuremath{{\occur{#1}_{#2}}}}
\newcommand{\occat}[2]{\occurat{#1}{#2}}
\newcommand{\occatp}[1]{\occurat{#1}{p}}
\newcommand{\occatq}[1]{\occurat{#1}{q}}

\newcommand{\colterm}[1]{\zeta_{#1}}



% fix restateable spacing 
%http://tex.stackexchange.com/questions/111639/extra-spacing-around-restatable-theorems

\makeatletter

\def\thmt@rst@storecounters#1{%
%THIS IS THE LINE I ADDED:
\vspace{-1.9ex}%
  \bgroup
        % ugly hack: save chapter,..subsection numbers
        % for equation numbers.
  %\refstepcounter{thmt@dummyctr}% why is this here?
  %% temporarily disabled, broke autorefname.
  \def\@currentlabel{}%
  \@for\thmt@ctr:=\thmt@innercounters\do{%
    \thmt@sanitizethe{\thmt@ctr}%
    \protected@edef\@currentlabel{%
      \@currentlabel
      \protect\def\@xa\protect\csname the\thmt@ctr\endcsname{%
        \csname the\thmt@ctr\endcsname}%
      \ifcsname theH\thmt@ctr\endcsname
        \protect\def\@xa\protect\csname theH\thmt@ctr\endcsname{%
          (restate \protect\theHthmt@dummyctr)\csname theH\thmt@ctr\endcsname}%
      \fi
      \protect\setcounter{\thmt@ctr}{\number\csname c@\thmt@ctr\endcsname}%
    }%
  }%
  \label{thmt@@#1@data}%
  \egroup
}%

\makeatother




\newcommand{\mymark}[1]{\ensuremath{(#1)}}
\newcommand{\markA}{\mymark \circ}
\newcommand{\markB}{\mymark *}
\newcommand{\markC}{\mymark \divideontimes}

\newcommand{\wrong}[1]{{\color{red}WRONG: #1}}
\newcommand{\NB}[1]{{\color{blue}NB: #1}}
\newcommand{\hl}[1]{{\color{orange} #1}}
\newcommand{\mytodo}[1]{{\color{red}TODO: #1}}
\newcommand{\largered}[1]{{

	  \LARGE\bfseries\color{red}
		#1

}}
\newcommand{\largeblue}[1]{{

	  \large\bfseries\color{blue}
		#1

}}




\usepackage{ulem} %  \dotuline{dotty} \dashuline{dashing} \sout{strikethrough}
\normalem

\usepackage{tabu} % tabular also in math mode (and much more)

\usepackage[color]{changebar} %  \cbstart, \cbend
\cbcolor{red}



% http://tex.stackexchange.com/questions/7032/good-way-to-make-textcircled-numbers
\newcommand*\circled[1]{\tikz[baseline=(char.base)]{
\node[shape=circle,draw,inner sep=2pt] (char) {#1};}}



% http://tex.stackexchange.com/questions/43346/how-do-i-get-sub-numbering-for-theorems-theorem-1-a-theorem-1-b-theorem-2

\makeatletter
\newenvironment{subtheorem}[1]{%
  \def\subtheoremcounter{#1}%
  \refstepcounter{#1}%
  \protected@edef\theparentnumber{\csname the#1\endcsname}%
  \setcounter{parentnumber}{\value{#1}}%
  \setcounter{#1}{0}%
  \expandafter\def\csname the#1\endcsname{\theparentnumber.\Alph{#1}}%
  \ignorespaces
}{%
  \setcounter{\subtheoremcounter}{\value{parentnumber}}%
  \ignorespacesafterend
}
\makeatother
\newcounter{parentnumber}


\usepackage{tabularx}% http://ctan.org/pkg/tabularx
\newcolumntype{Y}{>{\centering\arraybackslash}X}

\newcommand{\mycols}[2][3]{
	\noindent\begin{tabularx}{\textwidth}{*{#1}{Y}}
		#2
	\end{tabularx}%
}


\newcommand{\definethms}{

	%\declaretheorem[title=Theorem,qed=$\triangle$,parent=chapter]{thm}
	\newcommand{\thmqed}{$\square$} % for thms without proof
	\newcommand{\propqed}{$\square$} % for props without proof
	\declaretheorem[title=Theorem]{thm}
	\declaretheorem[title=Proposition,sibling=thm]{prop}
	\declaretheorem[title=Conjectured Proposition,sibling=thm]{cprop}

	%\declaretheorem[title=Lemma,parent=chapter]{lemma}
	\declaretheorem[sibling=thm]{lemma}
	\declaretheorem[sibling=thm,title=Conjectured Lemma]{clemma}
	\declaretheorem[title=Corollary,sibling=thm]{corr}
	\declaretheorem[sibling=thm,title=Definition,style=definition,qed=$\triangle$]{defi}
	%\declaretheorem[title=Definition,qed=$\triangle$,parent=chapter]{defi}
	\declaretheorem[title=Example,style=definition,qed=$\triangle$,sibling=thm]{exa}

	\declaretheorem[sibling=thm,title=Conjecture]{conj}

	\declaretheorem[title=Remark,style=remark,numbered=no,qed=$\triangle$]{remark}


}

\usepackage[matha]{mathabx} % the locial operators here have more space around them and [ and ] are thicker, also langle and rangle are a bit nicer; subseteq looks a bit weird

%\usepackage{MnSymbol} % again other symbols


\newcommand{\inference}{\ensuremath{\iota}}

\usepackage{cases} % numcases



% subsections also in toc
\setcounter{tocdepth}{2}
\setsecnumdepth{subsection}


\definethms

\def\proofSkipAmount{ \vskip -0.1em }


%\usepackage{bussproof}

%\usepackage{vaucanson-g}
%\usepackage{amssymb}
\usepackage{latexsym}

% for color-highlighted code
%\usepackage{color} % for grey comments
%\usepackage{alltt}

%\usepackage[doublespacing]{setspace}
%\usepackage[onehalfspacing]{setspace}
%\usepackage[singlespacing]{setspace}


\usepackage{amsthm}


\chapterstyle{madsen}

% define page numbering styles
\makepagestyle{numberCorner}
\makeevenfoot{numberCorner}{\thepage}{}{}
\makeoddfoot{numberCorner}{}{}{\thepage}

\makepagestyle{numberCenter}
%\makeevenfoot{numberCenter}{}{\thepage}{}
%\makeoddfoot{numberCenter}{}{\thepage}{}
%
%\makeevenhead{numberCenter}{\thechapter}{}{\thesection}
%\makeoddhead{numberCenter}{\thesection }{}{\thechapter}
\makeheadrule{numberCenter}{\textwidth}{1pt}

\makeevenhead{numberCenter}{\thepage}{}{\leftmark}
\makeoddhead{numberCenter}{\rightmark}{}{\thepage}


\makeatletter
\makepsmarks{numberCenter}{
	\def\chaptermark##1{\markboth{%
			\ifnum \value{secnumdepth} > -1
			\if@mainmatter
			\chaptername\ \thechapter\ --- %
			\fi
			\fi
	##1}{}}
	\def\sectionmark##1{\markright{%
			\ifnum \value{secnumdepth} > 0
			\thesection. \ %
			\fi
	##1}}
}
\makeatother
\newcommand{\mysetpagestyle}{
	%\pagestyle{numberCorner}
	\pagestyle{numberCenter}
}
\mysetpagestyle





\usepackage{refcheck}

%\settypeblocksize{0.64\stockheight}{0.64\stockwidth}{*}
%%\settypeblocksize{0.63\stockheight}{0.63\stockwidth}{*}
%\setlrmargins{*}{*}{1.0}
%\setulmargins{*}{*}{1.4}
%\checkandfixthelayout[nearest]

  

\begin{document}



\tableofcontents

\section{Lemmas from other pdf}
\begin{lemma}
	\label{lemma:lifting_order_not_relevant}
	$\lifgammanovar{\lifdeltanovar{\varphi}} = \lifdeltanovar{\lifgammanovar{\varphi}}$.
\end{lemma}

\clearpage

\section{Interpolant extraction from resolution proofs in one phase \hl{lifting terms whose quantifier position can be determined -- nested}}

\section{Incremental lifting and substitutions of lifting variables}

\begin{defi}[Substitution $\tau(\inference)$]
	For an inference $\inference$ with $\sigma = \mgu(\inference)$, we define the infinite substitution\todo{define infinite substitutions properly and apply definition here}{} $\tau(\inference)$ with $\dom(\tau(\inference)) = \dom(\sigma) \cup \{z_s \mid s\sigma \neq s\}$ as follows for a variable $x$:

	\[
		x\tau(\inference) =
		\begin{cases}
			x\sigma & \text{$x$ is a non-lifting variable} \\
			z_{t\sigma} & \text{$x$ is a lifting variable $z_t$}
		\end{cases} 
	\]
		%\qedhere

	If the inference $\inference$ is clear from the context, we abbreviate $\tau(\inference)$ by $\tau$. 
\end{defi}


\begin{lemma}
	\label{lemma:lifting_tau_commute}
	For a formula or term $\varphi$ and an inference $\inference$ such that $\tau = \tau(\inference)$,
	$\lifboth{\lifboth{ \varphi} \tau} =\nolinebreak \lifboth{ \varphi \tau } $.
\end{lemma}
\begin{proof}
	We proceed by induction.

	\begin{itemize}
		\item Suppose that $t$ is a grey constant or function symbol of the form $f(t_1, \dots, t_n)$.
			Then we can derive the following, where (IH) signifies a deduction by virtue of the induction hypothesis. 
			\begin{align*}
				\lifboth{\lifboth{t}\tau} &= \lifboth{\lifboth{ f(t_1, \dots, t_n)}\tau} \\
																  &= \lifboth{ f(\lifboth{t_1}\tau, \dots, \lifboth{t_n}\tau) } \\
																	&= f(\lifboth{\lifboth{t_1}\tau}, \dots, \lifboth{\lifboth{t_n}\tau}) \\
																 	&\stackrel{\mathclap{\text{(IH)}}}= f(\lifboth{t_1\tau}, \dots, \lifboth{t_n\tau}) \\
																	&= \lifboth{f(t_1, \dots, t_n)\tau} \\
																	&= \lifboth{t\tau}
			\end{align*}
		\item Suppose that $t$ is a colored constant or function symbol. Then:
			\[
				\lifboth{\lifboth{t}\tau}= \lifboth{z_t\tau} 
				= \lifboth{z_{t\sigma}} 
				= z_{t\sigma} 
				= \lifboth{t\sigma} 
				= \lifboth{t\tau}
			\]
		\item Suppose that $t$ is a variable $x$. Then:
			\[
				\lifboth{\lifboth{t}\tau} = \lifboth{\lifboth{ x}\tau} = \lifboth{x\tau} = \lifboth{t\tau}
			\]
		\item Suppose that $t$ is a lifting variable $z_t$. Then:
			\[
				\lifboth{\lifboth{z_t}\tau} = \lifboth{z_t\tau} 
				\qedhere
			\]
	\end{itemize}

\end{proof}



\begin{defi}[Incrementally lifted interpolant]
	Let $\pi$ be a resolution refutation of $\Gamma \cup \Delta$.
	We define $\LI(\pi)$ and $\LIcl(\pi)$ to be $\LI(\square)$ and $\LIcl(\square)$ respectively, where $\square$ is the empty clause derived in $\pi$.

	Let $C$ be a clause in $\pi$. 
	%For a literal $\lambda$ in $C$, we denote the corresponding literal in $\LIcl(C)$ by $\lambda\cll$, whose existence is ensured Lemma~\ref{lemma:li_vs_clause_plus_literals_equal}.

	We define $\LIcl(C) \defeq C$. \mytodo{if this version is final, drop $\LIcl(C)$ everywhere}

	We define the preliminary formula $\LI^\bullet(C)$ as follows:

	\begin{itemize}
		\item[Base case.]
			If $C \in \Gamma$, $\LI(C) \defeq \bot$.
			If otherwise $C \in \Delta$, $\LI(C) \defeq \top$.
		\item[Resolution.]

			If the clause $C$ is the result of a resolution step $\inference$ of $C_1: D \lor l$ and $C_2: E \lor \lnot l'$ using a unifier $\sigma$ such that $l\sigma =  l'\sigma$, then define $\LI(C)$ as follows:

			\begin{enumerate}

				\item If $l$ is $\Gamma$-colored:
					$\LIpre(C) \defeq \LI(C_1)\tau\spas\lor \LI(C_2)\tau $

				\item If $l$ is $\Delta$-colored:
					$\LIpre(C) \defeq \LI(C_1)\tau\spas\land \LI(C_2)\tau$

				\item If $l$ is grey:
					$\LIpre(C) \defeq
					(l\cll\tau \land \LI(C_2)\tau) \spas\lor
					(\lnot l'\cll\tau\land \LI(C_1)\tau)
					$

			\end{enumerate}

		\item[Factorisation.]
			If the clause $C$ is the result of a factorisation step $\inference$ of $C_1: l \lor l' \lor D$ using a unifier $\sigma$ such that $l\sigma = l'\sigma$, then $\LIpre(C) \defeq \lifboth{\LI(C_1)\tau}$.

		\item[Paramodulation.]
			If the clause $C$ is the result of a paramodulation step $\inference$ of $C_1 : s = t \lor D$ and $C_2: E\occ{r}$ with $\sigma = \mgu(\inference)$.
			Let $h\occ{r}$ be the maximal colored term in which $r$ occurs in $E\occ{r}$.
			Then define $\LI(C)$ as follows:

			\begin{enumerate}

				\item If $h\occur{r}$ is $\Delta$-colored and $h\occur{r}$ occurs more than once in $E\occur{r} \lor \LI(E\occur{r})$:
					%\label{def:PI_paramod_1}
					\newline
					$\LIpre(C) \defeq  ( s=t \land \LI(C_2) )\tau \lor (s\neq t \land \LI(C_1))\tau \lor (s=t \land h\occur{s} \neq      h\occur{t})\tau$

				\item If $h\occur{r}$ is $\Gamma$-colored and $h\occur{r}$ occurs more than once in $E\occur{r} \lor \LI(E\occur{r})$:
					%\label{def:PI_paramod_2}
					\newline
					$\LIpre(C)\defeq{} [ ( s=t \land \LI(C_2) )\tau \lor (s\neq t \land \LI(C_1))\tau ] \land\allowbreak (s\neq t \lor h\occur{s} = h\occur{t})\tau$

				\item If $r$ does not occur in a colored term in $E\occur{r}$ which occurs more than once in $E\occur{r} \lor \LI(E\occur{r})$:
					%\label{def:PI_paramod_3}
					\newline
					$\LIpre(C) \defeq{} ( s=t \land \LI(C_2) )\tau \lor (s\neq t \land \LI(C_1))\tau $ 

			\end{enumerate}

	\end{itemize}

	$\LI(C)$ is built from $\LIpre(C)$ as follows:

	\begin{enumerate}
		\item Lift all maximal colored terms in $\LIpre(C)$ which contains some variable which does not occur in $C$.
		\item Let $X$ ($Y$) be the $\Delta$-($\Gamma$-)lifting variables created in the previous step.
		\item Prefix the resulting formula with an arrangement of $\{\forall x_t \mid x_t \in X\}\cup\allowbreak\{\exists y_t \mid y_t \in Y\}$ such that if $s$ and $r$ are terms such that $s$ is a subterm of $r$, then $z_s$ precedes $z_r$.
			\qedhere
	\end{enumerate}
\end{defi}

\section{Properties of $\LI$ and $\LIcl$}


{correct but useless here:

	\tiny 

\begin{lemma}
	\label{lemma:li_vs_clause_plus_literals_equal}
	Let $C$ be a clause in a resolution refutation of $\Gamma\cup\Delta$.

	Then for every literal $\lambda$ in $C$, there exists a literal $\lambda\cll$ in\nolinebreak{} $\LIcl(C)$ such that $\lambda\cll = \lifboth{\lambda}$ and for resolved or factorised literals $l$ and $l'$ of a resolution or factorisation inference $\inference$, we have that $\lifboth{l\cll\tau} = \lifboth{l'\cll\tau}$.
\end{lemma}
\begin{proof}
	We proceed by induction.
	\begin{description}
		\item{} Base case.
			For $C\in\Gamma\cup\Delta$, $\LIcl(C)$ is defined to be $\lifboth{C}$.

		\item{} Induction step.
			Suppose the clause $C$ is the result of a resolution, factorisation or paramodulation inference \inference{} of the clauses $\bbar C$ with $\sigma = \mgu(\inference)$.

			%Suppose the clause $C$ is the result of a resolution step \inference{} of $C_1: D \lor l$ and $C_2: E \lor \lnot l'$ with $\sigma = \mgu(\inference)$.

			Every literal in $C$ is of the form $\lambda\sigma$ for a literal $\lambda$ in $C_i\in\bbar C$.
			%For every literal in $C$, there exists a predecessor $\lambda$ in a clause $C_i$ in $\bbar C$.
			%Let $\lambda$ be a literal in $C_i\in\bbar C$, such that $\lambda$ is not the predecessor of the literal being resolved or factorised in $\inference$.
			%Then $\lambda\sigma$ is occurs in $C$.

			By the induction hypothesis, $\lifboth{\lambda}$ occurs in $\LIcl(C_i)$.
			By the construction of $\LIcl(C)$ and as $\lambda$ is not a resolved or factorised literal, $\LIcl(C)$ contains a literal of the form $\lifboth{\lifboth{\lambda}\tau}$.
			But by Lemma~\ref{lemma:lifting_tau_commute}, this is nothing else than $\lifboth{\lambda\tau}$.
			As $\lambda$ occurs in the resolution derivation, it does not contain lifting variables.
			Hence we get by the definition of $\tau$ that $\lifboth{\lambda\tau} = \lifboth{\lambda\sigma}$.

			Let $l$ and $l'$ be the resolved or factorised literals of \inference.
			In order to show that $\lifboth{l\cll\tau} = \lifboth{l'\cll\tau}$,
			consider that by the induction hypothesis, this is nothing else than
			$\lifboth{\lifboth{l}\tau} = \lifboth{\lifboth{l'}\tau}$.
			But by applying a similar argument as above, this equation is equivalent to
			$\lifboth{l\sigma} = \lifboth{l'\sigma}$, which is implied by $l\sigma =\nolinebreak l'\sigma$.
			\qedhere

			\begin{comment}
				Let $l$ and $l'$ be the resolved or factorised literals.
				By the induction hypothesis,
				$l\cll = \lifboth{l}$
				and
				$l'\cll = \lifboth{l'}$.

				By a similar reasoning as above, we get that $\lifboth{\lifboth{\lambda}\tau} = \lifboth{\lambda\sigma}$ for any literal $\lambda$ in $\bbar C$.

				$\lifboth{\lambda\cll\tau} = \lifboth{\lifboth{\lambda}\tau}$
				But as $l\sigma = l'\sigma$ and

				$\lifboth{\lifboth{l}\tau } =
				\lifboth{\lifboth{l'}\tau }
				$

				As no lifting variables occur in $l$ or $l'$, we get that $l\tau = l'\tau$, which we can lift to $\lifboth{l\tau} = \lifboth{l'\tau}$.

				Note that $l\sigma = l'\sigma$.

				$l\cll = \lifboth{l}$

				$\lifboth{ \lifboth{l} \tau} = \lifboth{l\tau}$

			\end{comment}
	\end{description}
\end{proof}


\begin{lemma}
	\label{lemma:no_colored_terms}
	Let $C$ be a clause of a resolution refutation of $\Gamma\cup\Delta$.
	$\LI(C)$ and $\LIcl(C)$ do not contain colored symbols.
\end{lemma}
\begin{proof}
	For $\LI(C)$ and $\LIcl(C)$, consider the following:
	In the base case of the inductive definitions of $\LI(C)$ and $\LIcl(C)$, no colored symbols occur.
	In the inductive steps, any colored symbol which is added by $\tau$ to intermediary formulas is lifted.
\end{proof}

}

\begin{lemma}
	\label{lemma:substitute_and_lift}
	Let $\sigma$ be a substitution and $F$ a formula without $\Phi$-colored terms such that for a set of formulas $\Psi$ which does not contain $\Phi$-lifting variables, $\Psi \entails F$.
	Then $\Psi \entails \lifphinovar{F\sigma}$.
\end{lemma}
\begin{proof}
	$\lifphinovar{F\sigma}$ is an instance of $F$:
	$\sigma$ substitutes variables either for terms which do not contain $\Phi$-colored symbols or by terms containing $\Phi$-colored symbols.
	For the first kind, the lifting has no effect.
	For the latter, the lifting only replaces subterms of the terms introduced by the substitution by a lifting variable such that the original structure of $F$ remains invariant as it by assumption does not contain colored terms.
\end{proof}

\begin{clemma}
	Let $C$ be a clause in a resolution refutation of $\Gamma \cup \Delta$.
	Then
	$\Gamma \entails \lifdeltanovar{ \LI(C) } \lor \lifdeltanovar{C} $
\end{clemma}
\begin{proof}
	We proceed by induction on the strengthening
	$\Gamma \entails \lifdeltanovar{ \LI(C_\Gamma) } \lor \lifdeltanovar{C_\Gamma} $\footnote{Recall that $D_\Phi$ denotes the clause created from the clause $D$ by removing all literals which are not contained $\Lang(\Phi)$.}

	\begin{description}
		\item{} Base case.
			If $C\in \Gamma$, then $\lifdeltanovar{C} = C$ and $\Gamma \entails C$.
			If otherwise $C \in \Delta$, then $\LI(C) = \top$.

		\item{} Resolution.
			Suppose the clause $C$ is the result of a resolution step \inference{} of $C_1: D \lor l$ and $C_2: E \lor \lnot l'$ with $\sigma = \mgu(\inference)$.

			By the induction hypothesis we obtain the following:

			$\Gamma \entails \lifdeltanovar{\LI(C_1)} \lor \lifdeltanovar{D_\Gamma} \lor \lifdeltanovar{l_\Gamma}$

			$\Gamma \entails \lifdeltanovar{\LI(C_2)} \lor \lifdeltanovar{E_\Gamma} \lor \lnot \lifdeltanovar{l'_\Gamma}$

			Hence by Lemma~\ref{lemma:substitute_and_lift} and Lemma~\ref{lemma:lifting_tau_commute}, we get:

			$\Gamma \stackrel\markA\entails \lifdeltanovar{\LI(C_1)\tau} \lor \lifdeltanovar{D_\Gamma\tau} \lor \lifdeltanovar{l_\Gamma\tau}$

			$\Gamma \stackrel\markB\entails \lifdeltanovar{\LI(C_2)\tau} \lor \lifdeltanovar{E_\Gamma\tau} \lor \lnot \lifdeltanovar{l'_\Gamma\tau}$

			As $l_\Gamma\sigma = l'_\Gamma\sigma$ and both $l_\Gamma$ and $l'_\Gamma$ are devoid of lifting variables, it holds that $l_\Gamma\tau = l'_\Gamma\tau$.

			We proceed by a case distinction on the color of the resolved literal to show that in each case
			$\Gamma \entails \lifdeltanovar{\LIpre(C)} \lor \lifdeltanovar{C_\Gamma}$:
			\begin{itemize}
				\item Suppose that $l$ is $\Gamma$-colored.
					Then $l_\Gamma = l$ and $l'_\Gamma = l$, and we can perform a resolution step on \markA{} and \markB{} to obtain that
					$\Gamma \entails
					\lifdeltanovar{\LI(C_1)\tau} \lor
					\lifdeltanovar{\LI(C_2)\tau} \lor 
					\lifdeltanovar{D_\Gamma\tau}  \lor
					\lifdeltanovar{E_\Gamma\tau}$.
					This however is nothing else than $\Gamma \entails \lifdeltanovar{\LIpre(C)} \lor \lifdeltanovar{C_\Gamma}$.

				\item Suppose that $l$ is $\Delta$-colored.
					Then \markA{} and \markB{} reduce to 
					$\Gamma \entails \lifdeltanovar{\LI(C_1)\tau} \lor \lifdeltanovar{D_\Gamma\tau}$
					as well as
					$\Gamma \entails \lifdeltanovar{\LI(C_2)\tau} \lor \lifdeltanovar{E_\Gamma\tau}$, 
					which clearly implies that 
					$\Gamma \entails \lifdeltanovar{\LI(C_1)\tau} \lor \lifdeltanovar{\LI(C_2)\tau} \lor (\lifdeltanovar{D_\Gamma\tau} \land \lifdeltanovar{E_\Gamma\tau})$.
					This is turn is however the same as
					$\Gamma \entails \lifdeltanovar{\LIpre(C)} \lor \lifdeltanovar{C_\Gamma}$.

				\item Suppose that $l$ is grey.

					Then \markA{} and \markB{} imply that
					$\Gamma \entails
					\lifdeltanovar{\LI(C_1)\tau} \lor
					\lifdeltanovar{\LI(C_2)\tau} \spas\lor\allowbreak
					(\lifdeltanovar{l_\Gamma\tau} \land \lifdeltanovar{E_\Gamma\tau}) \spas\lor\allowbreak
					(\lnot\lifdeltanovar{l'_\Gamma\tau} \land \lifdeltanovar{D_\Gamma\tau})$.
					This however is equivalent to
					$\Gamma \entails \lifdeltanovar{\LIpre(C)} \lor \lifdeltanovar{C_\Gamma}$.

			\end{itemize}


			We now conclude by showing that 
			$\Gamma \entails \lifdeltanovar{\LIpre(C)} \lor \lifdeltanovar{C}$
			implies that 
			$\Gamma \entails \lifdeltanovar{\LI(C)} \lor \lifdeltanovar{C}$.

			The difference between $\lifdeltanovar{\LIpre(C)}$ and $\lifdeltanovar{\LI(C)}$ lies only in maximal colored terms which are lifted in $\lifdeltanovar{\LI(C)}$, hence it suffices to consider these.
			Let $t$ be a term in $\LIpre(C)$ at position $p$ such that $\LI(C)\atp = \lifboth{t}$.
			Then $t$ is a maximal colored term and contains a variable which does not occur in $C$.

			If $t$ is $\Delta$-colored, then $\lifdeltanovar{\LIpre(C)}\atp = \LI(C)\atp = x_t$.
			Note that as $t$ occurs at $p$ in $\LIpre(C)$, $x_t$ is not bound at $\lifdeltanovar{\LIpre(C)}\atp$.
			Hence it is implicitly universally quantified and therefore implies that an explicit universal quantification in $\LI(C)$ is valid with an arbitrarily placed quantifier.  

			If otherwise $t$ is a $\Gamma$-term, then $\lifdeltanovar{\LIpre(C)}\atp = \lifdeltanovar{t}$.
			Hence $\lifdeltanovar{t}$ is a witness term for the lifting variable $y_t$ at $\LI(C)\atp$.
			In general, $\lifdeltanovar{t}$ however contains $\Delta$-lifting variables, which require being lifted in the scope of the existential quantifier for $y_t$. 
			Let $x_s$ be a $\Delta$-lifting variable which occurs in $\lifdeltanovar{t}$. 

			It is essential to see that neither $s$ nor a predecessor of $s$ is lifted in a previous step of the interpolant extraction.
			Suppose to the contrary that is in the inference creating the clause $C'$.
			Let $s'$ and $t'$ be the respective predecessors of $s$ and $t$ in $C'$.
			Then one of the following two contradictions eventuate:
			\begin{itemize}
				\item Suppose that $s'$ is a subterm of the corresponding predecessor $t'$.
					Then $s'$ contains a variable which does not occur in $C'$. But as $t'$ contains $s'$, $t'$ contains this variable as well and would be lifted at this stage already.
				\item Otherwise $t'$ does not contain $s'$.
					Clearly $s'$ contains a variable which does not occur in $C'$.
					As all clauses are variable-disjoint, no other clause contains this variable.
					But then it does not occur in any subsequent unifier, and in particular, it never enters $t'$ by means of substitution, which implies that $s'$ due to containing this variable does not become a subterm of a successor of $t'$.

			\end{itemize}

			Hence there are three possibilities for quantification of $x_s$:
			\begin{enumerate}
			\item $s$ nor a successor of $s$ in the derivation does not occur at a grey position. Then $x_s$ is never explicitly quantified.
			\item A variable which does not occur in $C$ entered $s$ by means of the current subtitution $\sigma$ or a variable is contained in $s$ such that the only occurrences of it in $C_1$ and $C_2$ are in $l$ and $l'$.
				Then $x_s$ is lifted in the current step and as $s$ is a subterm of $t$, $y_t$ is quantified in the scope of $x_s$.
			\item $x_s$ or a respective successor is quantified at a later stage in the derivation.
				Then as the quantifier for $y_t$ is contained in $\LI(C)$ and for any successor $C'$ of $C$, $\LI(C')$ contains a successor $\LI(C)$, $y_t$ is quantified in the scope of the quantifier for $x_s$.
			\end{enumerate}


			%			Suppose that $s$ contains a variable which does not occur in $C$. Then $x_s$ is quantified in $\LI(C)$ as well.
			%			As however the quantifier block of $\LI(C)$ is ordered with respect to the subterm relation of the index of the lifting variables, $x_s$ is quantified prior to $y_t$.

	\end{description}


\end{proof}

\largered{TODO: make sure this proof is valid; then define what happens at the end: lift all remaining terms (as in huang?). also check if symmetry works out. compare with other proofs.}




\part*{old stuff}

\section{Lifting the $\Delta$-terms}

\begin{defi}
	$\LIde(C)$ and $\lifidelta{C}$ for a clause $C$ are defined as $\LI(C)$ and $\lifiboth{C}$ respectively with the difference that in its inductive definition, every lifting $\lifboth{\varphi}$ for a formula or term $\varphi$ is replaced by a lifting of only the $\Delta$-terms $\lifdeltanovar{\varphi}$.
\end{defi}


\tiny
\begin{remark}
	Many results involving $\LI(C)$ or $\LIcl(C)$ are valid for $\LIde(C)$ or $\LIclde(C)$
	in a formulation which is adapted accordingly.
	This can easily be seen by the following proof idea:

	Let $f_1, \dots, f_n$ be all $\Gamma$-colored function or constant symbols occurring in $C$,
	$c$ a fresh constant symbol and $g$ a fresh $n$-ary function symbol.
	Construct a formula $\varphi:  g(t_1, \dots, t_n) = g(t_1, \dots, t_n)$,
	such that $t_i = f_i(c_1, \dots, c_m)$ for $1\varleq i \varleq n$ where $m$ is the arity of $f_i$ and $c_j = c$ for $1\varleq j \varleq m$. Let $\Delta' = \Delta \cup \{\varphi\}$ and apply the desired result to the initial clause sets $\Gamma$ and $\Delta'$.

	Under this construction, every originally $\Gamma$-colored symbol is now grey, which implies that $\LI(C) = \LIde(C)$ as well as $\LIcl(C) = \LIclde(C)$.
	But $\Delta\entails \psi \semiff \Delta'\entails \psi$ for any formula $\psi$.
\end{remark}




\begin{lemma}
	\label{lemma:gamma_entails_lide}
	Let $C$ be a clause in a resolution refutation of $\Gamma\cup\Delta$. Then
	$\Gamma\entails \LIde(C) \lor \lifidelta{C}$.
\end{lemma}
\begin{proof}
	We proceed by induction of the strengthening $\Gamma\entails \LIde(C) \lor \lifidelta{C_\Gamma}$.

	\begin{description}
		\item{} Base case.
			For $C\in\Gamma$, $\LIclde(C_\Gamma) = \lifdeltanovar{C} = C$. Hence $\Gamma \entails \LIclde(C_\Gamma)$.

			For $C\in\Delta$, $\LIde(C) = \top$, so $\Gamma \entails \LIde(C)$.

		\item{} Resolution.
			Suppose the clause $C$ is the result of a resolution step \inference{} of $C_1: D \lor l$ and $C_2: E \lor \lnot l'$ with $\sigma = \mgu(\inference)$.

			\newcommand{\clauseOnePrime}{\LIclde( (C_1)_\Gamma )^*}
			\newcommand{\clauseTwoPrime}{\LIclde( (C_2)_\Gamma )^*}

			We define the following abbreviations:

			$\clauseOnePrime = \LIclde( (C_1)_\Gamma \setminus \{l\cllde\} )$

			$\clauseTwoPrime = \LIclde( (C_2)_\Gamma \setminus \{\lnot l'\cllde\} )$

			Hence the induction hypothesis can be stated as follows:

			$\Gamma \entails \LIde(C_1) \lor \clauseOnePrime \lor (l\cllde)_\Gamma$

			$\Gamma \entails \LIde(C_2) \lor \clauseTwoPrime \lor \lnot (l'\cllde)_\Gamma$

			By Lemma~\ref{lemma:no_colored_terms}, $\LIde(C_i)$ and $\LIclde(C_i)$ for $i\in\{1,2\}$ do not contain $\Delta$-colored terms. 
			Hence we are able to apply Lemma~\ref{lemma:substitute_and_lift} in order to obtain

			$\Gamma \stackrel\markA\entails \lifdeltanovar{\LIde(C_1)\tau} \lor \lifdeltanovar{\clauseOnePrime\tau} \lor \lifdeltanovar{(l\cllde)_\Gamma\tau}$

			$\Gamma \stackrel\markB\entails \lifdeltanovar{\LIde(C_2)\tau} \lor \lifdeltanovar{\clauseTwoPrime\tau} \lor \lnot \lifdeltanovar{(l'\cllde)_\Gamma\tau}$

			By Lemma~\ref{lemma:li_vs_clause_plus_literals_equal}, we obtain that
			$\lifdeltanovar{l\cllde\tau} = 
			\lifdeltanovar{l'\cllde\tau}$.

			Now we distinguish cases based on the color of the resolved literal:

			\begin{itemize}
				\item Suppose that $l$ is $\Gamma$-colored.
					Then as
					$\lifdeltanovar{l\cllde\tau} = 
					\lifdeltanovar{l'\cllde\tau}$, 
					we can perform a resolution step on \markA{} and \markB{}, which gives that
					$\Gamma \entails
					\lifdeltanovar{\LIde(C_1)\tau} \spas\lor \lifdeltanovar{\clauseOnePrime\tau} \spam\lor 
					\lifdeltanovar{\LIde(C_2)\tau} \spas\lor \lifdeltanovar{\clauseTwoPrime\tau}$.
					This however is nothing else than $\Gamma\entails \LIde(C) \lor \LIclde(C)$.

				\item Suppose that $l$ is $\Delta$-colored. Then \markA{} and \markB{} simply to the following:

					$\Gamma \entails \lifdeltanovar{\LIde(C_1)\tau} \lor \lifdeltanovar{\clauseOnePrime\tau}$

					$\Gamma \entails \lifdeltanovar{\LIde(C_2)\tau} \lor \lifdeltanovar{\clauseTwoPrime\tau}$

					These however imply that 
					$\Gamma \entails 
					\clauseOnePrime \spas\lor \clauseTwoPrime \spas\lor\allowbreak
					(\lifdeltanovar{\LIde(C_1)\tau} \land
					\lifdeltanovar{\LIde(C_2)\tau} )$, which is nothing else than
					$\Gamma \entails \LIde(C) \lor \LIclde(C)$.

				\item Suppose that $l$ is grey.
					Suppose that $M$ is a model of $\Gamma$ such that
					%$M\notentails \lifdeltanovar{\LIclde(C_1)\tau} \lor \lifdeltanovar{\LIclde(C_2)\tau}$.
					$M\notentails \LIclde(C)$, i.e.\ 
					$M\notentails \lifdeltanovar{\clauseOnePrime\tau} \lor \lifdeltanovar{\clauseTwoPrime\tau}$.
					Then $M \entails \lifdeltanovar{\LIde(C_1)\tau} \lor \lifdeltanovar{l\cllde\tau}$
					as well as 
					$M \entails \lifdeltanovar{\LIde(C_2)\tau} \lor \lnot \lifdeltanovar{l'\cllde\tau}$.
					Due to $\lifdeltanovar{l\cllde\tau} = 
					\lifdeltanovar{l'\cllde\tau}$,
					we obtain that
					$M\entails (\lifdeltanovar{l\cllde\tau} \land \lifdeltanovar{\LIde(C_2)\tau}) \spam\lor
					(\lnot \lifdeltanovar{l'\cllde\tau} \land \lifdeltanovar{\LIde(C_1)\tau})$,
					which is nothing else than $M\entails \LIde(C)$.

			\end{itemize}


		\item{} Factorisation. 
			Suppose the clause $C$ is the result of a factorisation inference $\inference$ of $C_1: l \lor l' \lor D$ with $\sigma=\mgu(\inference)$.

			We introduce the abbreviation
			$\LIclde( (C_1)_\Gamma)^* = \LIclde( (C_1)_\Gamma \setminus \{l\cllde, \lnot l'\cllde\} )$
			and express the induction hypothesis as follows:

			$\Gamma \entails \LIde(C_1) \lor \LIclde( (C_1)_\Gamma)^* \lor (l\cllde)_\Gamma \lor \lnot (l'\cllde)_\Gamma$

			By Lemma~\ref{lemma:no_colored_terms}, $\LIde(C_i)$ and $\LIclde(C_i)$ for $i\in\{1,2\}$ do not contain $\Delta$-colored terms. 
			Hence we are able to apply Lemma~\ref{lemma:substitute_and_lift} in order to obtain

			$\Gamma \stackrel\markC\entails \lifdeltanovar{\LIde(C_1)\tau} \lor \lifdeltanovar{\LIclde( (C_1)_\Gamma)^*\tau} \lor \lifdeltanovar{(l\cllde)_\Gamma\tau} \lor \lnot \lifdeltanovar{(l'\cllde)_\Gamma\tau}$

			As by Lemma~\ref{lemma:li_vs_clause_plus_literals_equal} we get that 
			$\lifdeltanovar{l\cllde\tau} = 
			\lifdeltanovar{l'\cllde\tau}$,
			we can perform a factorisation step on \markC{} to obtain that 
			$\Gamma \entails \lifdeltanovar{\LIde(C_1)\tau} \lor \lifdeltanovar{\LIclde( (C_1)_\Gamma)^*\tau} \lor \lifdeltanovar{(l\cllde)_\Gamma\tau}$.
			But this is nothing else than $\Gamma \entails \LIde(C) \lor \LIclde(C_\Gamma)$.
			\qedhere


		\item{} Paramodulation.
			Suppose the clause $C$ is the result of a paramodulation inference $\inference$ of $C_1: s=t \lor D$ and $C_2: E\occatp{r}$ with $\sigma=\mgu(\inference)$.

			We introduce the abbreviation $\LIclde((C_1)_\Gamma)^* = \LIclde( (C_1)_\Gamma \setminus \{ (s=t)\cll\})$ and express the induction hypothesis as follows:

			$\Gamma \stackrel\markA\entails \LIde(C_1) \lor \LIclde(C_1)^* \lor (s=t)\cll $

			$\Gamma \stackrel\markB\entails \LIde(C_2) \lor \LIclde(C_2)$

			%By Lemma~\ref{lemma:li_vs_clause_plus_literals_equal},

			%$\Gamma \entails \LIde(C_1) \lor \LIclde(C_1)^* \lor (s=t)\cll $

			%$\Gamma \entails \LIde(C_2) \lor \lifdeltanovar{ E\occatp{r} }$

			%By Lemma~\ref{lemma:no_colored_terms} and Lemma~\ref{lemma:substitute_and_lift}, we obtain that 

			%$\Gamma \stackrel\markA\entails \lifdeltanovar{\LIde(C_1)\tau} \lor \lifdeltanovar{\LIcl(C_1)^*\tau} \lor \lifdeltanovar{(s=t)\cll\tau}$

			%$\Gamma \stackrel\markB\entails \lifdeltanovar{\LIde(C_2)\tau} \lor \lifdeltanovar{\LIcl(C_2)\tau}$

			Suppose now that for a model $M$ of $\Gamma$ that $M \entails \lifdeltanovar{s} \neq \lifdeltanovar{t}$.
			Then by Lemma~\ref{lemma:li_vs_clause_plus_literals_equal}, 
			$M \entails s\cll \neq t\cll$.
			Hence we get by \markA{} that 
			$\Gamma \entails \LIde(C_1) \lor \LIcl(C_1)^*$
			and consequently 
			by Lemma~\ref{lemma:no_colored_terms} and Lemma~\ref{lemma:substitute_and_lift} that 
			$\Gamma \entails \lifdeltanovar{\LIde(C_1)\tau} \lor \lifdeltanovar{\LIcl(C_1)^*\tau}$.
			But this however implies that $\Gamma \entails \LIde(C) \lor \LIcl(C)$.

			Now suppose to the contrary that for a model $M$ of $\Gamma$ that $M \entails \lifdeltanovar{s} =\nolinebreak \lifdeltanovar{t}$.
			Note that by Lemma~\ref{lemma:li_vs_clause_plus_literals_equal}, $\lifdeltanovar{ (E\occatp{r})_\Gamma} = \LIclde(C_2)$.
			Hence \markB{} is nothing else than
			$\Gamma \entails \LIde(C_2) \lor \lifdeltanovar{ (E\occatp{r})_\Gamma }$.
			From this, it also follows by Lemma~\ref{lemma:no_colored_terms} and Lemma~\ref{lemma:substitute_and_lift} that
			$\Gamma \entails \lifdeltanovar{\LIde(C_2)\tau} \lor \lifdeltanovar{\lifdeltanovar{ (E\occatp{r})_\Gamma }\tau}$, 
			which by Lemma~\ref{lemma:lifting_tau_commute} simplifies to
			$\Gamma \entails \lifdeltanovar{\LIde(C_2)\tau} \lor \lifdeltanovar{(E\occatp{r})_\Gamma \tau}$, 

			%We show that
			%$
			%\lifdeltanovar{ (E\occatp{r})_\Gamma\sigma }
			%\semiff
			%\lifdeltanovar{ (E\occatp{t})_\Gamma\sigma }
			%$.

			%then, $\Gamma\entails\LIde(C_2) \lor \lifdeltanovar{ (E\occatp{t})_\Gamma\sigma }$

			Due to $\sigma = \mgu(r, s)$, $r\tau\syneq s\tau$ and consequently $\lifdeltanovar{r\tau} = \lifdeltanovar{s\tau}$.
			Hence 
			%$\lifdeltanovar{(E\occatp{r})_\Gamma\sigma} \equiv \lifdeltanovar{(E\occatp{s})_\Gamma\sigma}$.
			$\Gamma \entails \lifdeltanovar{\LIde(C_2)\tau} \lor \lifdeltanovar{(E\occatp{s})_\Gamma \tau}$, 


			We proceed by a case distinction:
			\begin{itemize}
				\item
					Suppose that $p$ in $E\occatp{s}$ is not contained in a $\Delta$-term.
					Then $\lifdeltanovar{ (E\occatp{s})_\Gamma \tau }$ and $\lifdeltanovar{ (E\occatp{t})_\Gamma \tau}$ only differ at position $p$.
					But as $M\entails \lifdeltanovar{s} = \lifdeltanovar{t}$, 
					by Lemma~\ref{lemma:substitute_and_lift} and
					Lemma~\ref{lemma:lifting_tau_commute} we derive that
					$M\entails \lifdeltanovar{s\tau} = \lifdeltanovar{t\tau}$. 
					Then however 
					$M\entails \lifdeltanovar{ (E\occatp{s})_\Gamma \tau } \semiff \lifdeltanovar{ (E\occatp{t})_\Gamma \tau}$
					and thus
					$M \entails \lifdeltanovar{\LIde(C_2)\tau} \lor \lifdeltanovar{ (E\occatp{t})_\Gamma \tau }$,
					which is sufficient for
					$M \entails \LIde(C) \lor \LIclde(C)$.

					%From this we can however conclude that $M\entails \lifdeltanovar{(E\occatp{r})_\Gamma\sigma} \semiff \lifdeltanovar{(E\occatp{t})_\Gamma\sigma}$.

				\item 
					Suppose that $p$ in $E\occatp{r}$ is contained in a maximal $\Delta$-term $h\occ{r}$, which occurs more than once in $E\occatp{r} \lor \LIde(E\occatp{r})$. 
					Suppose furthermore that $M\entails \lifdeltanovar{h\occ{s}} = \lifdeltanovar{h\occ{t}}$ as otherwise $M \entails s=\nolinebreak t \land \lifdeltanovar{h\occ{s}} = \lifdeltanovar{h\occ{t}}$, which implies that $M \entails \LIde(C)$.

					But then we again obtain that 
					$\lifdeltanovar{ (E\occatp{s})_\Gamma \tau }$ and $\lifdeltanovar{ (E\occatp{t})_\Gamma \tau}$ only differ at position $p$ and
					by a similar line of argument as in the former case, we can deduce that 
					$M \entails \LIde(C) \lor \LIclde(C)$.

				\item 
					Suppose that $p$ in $E\occatp{s}$ is contained in a maximal $\Delta$-term $h\occ{s}$, which occurs exactly once in $E\occatp{s} \lor \LIde(E\occatp{r})$. 
					Then the lifting variable $z_{h\occ{s}}$ occurs exactly once in \markB, where it is implicitly universally quantified. 
					Therefore we can instantiate this variable by any term, in particular by $z_{h\occ{t}}$, 
					so we obtain that 
					$\Gamma \entails \lifdeltanovar{\LIde(C_2)\tau} \lor \lifdeltanovar{(E\occatp{t})_\Gamma \tau}$.
					which again is sufficient for
					$M \entails \LIde(C) \lor \LIclde(C)$.
					%But this is nothing else than stating that \dots formula from above adapted appropriately \dots
					\qedhere
			\end{itemize}

	\end{description}
\end{proof}

\begin{lemma}
	\label{lemma:gamma_lifted_lide}
	For a clause $C$ of a resolution refutation of $\Gamma\cup\Delta$, 
	$\lifgammanovar{\LIde(C)} = \LI(C)$ and $\lifgammanovar{\LIclde(C)} = \LIcl(C)$.
\end{lemma}
\begin{proof}
	We proceed by induction.
	\begin{description}
		\item{} Base case.
			For $C \in \Gamma\cup\Delta$, $\LIclde(C) = \lifdeltanovar{C}$.
			By Lemma~\ref{lemma:lifting_order_not_relevant}, $\lifgammanovar{\lifdeltanovar{C}} = \lifboth{C}$,
			so $\lifgammanovar{\LIclde{C}} = \lifboth{C} = \LIclde(C)$.

			$\LIde(C)$ does not contain colored symbols.

		\item{} Inductions step.
			Suppose the clause $C$ is the result of a resolution, factorisation or paramodulation inference \inference{} of the clauses $\bbar C$.

			%$\LIde(C)$ and $\LI(C)$ as well as $\LIclde(C)$ and $\LIcl(C)$ differ only in the liftings, or more precisely:

			As liftings do not affect the predicate, we do not consider them further.
			Note that every term in $\LI(C)$ or $\LIcl(C)$ is of the form $\lifboth{t\tau}$ for some term $t$ in $\LI(C_i)$ or $\LIcl(C_i)$ for some $C_i \in \bbar C$.
			Furthermore,
			every term in $\LIde(C)$ or $\LIclde(C)$ is of the form $\lifdeltanovar{t\tau}$ for some term $t$ in $\LIde(C_i)$ or $\LIclde(C_i)$ for some $C_i \in \bbar C$.

			%Every literal in $\LI(C)$ ($\LIde(C)$) is of the form $\lifboth{\lambda\tau}$ ($\lifdelta{\lambda\tau}$) for $\lambda$ in $\LI(C_i)$ ($\LIde(C_i)$) or $\LIcl(C_i)$ ($\LIclde(C_i)$), $i\in\{1,2\}$.

			%Every literal in $\LIcl(C)$ ($\LIclde(C)$) is of the form $\lifboth{\lambda\tau}$ ($\lifdelta{\lambda\tau}$) for $\lambda$ in $\LIcl(C_i)$ ($\LIclde(C_i)$), $i\in\{1,2\}$.

			Hence it suffices to show that for a term $t$ in $\LIde(C_i)$ or $\LIclde(C_i)$ and its corresponding term $\lifgammanovar{t}$ in $\LI(C_i)$ or $\LIcl(C_i)$ for some $C_i \in \bbar C$
			that $\lifgammanovar{ \lifdeltanovar{t\tau} } = \lifboth{\lifgammanovar{t}\tau}$.
			%As $\lambda$ corresponds to a literal in $\LIde$ or $\LIclde$, we may by Lemma~\ref{lemma:no_colored_terms} additionally assume that no $\Delta$-colored term occurs in $\lambda$.

			By Lemma~\ref{lemma:no_colored_terms}, no $\Delta$-terms occur in $t$.
			Hence 
			$\lifboth{t} = \lifgammanovar{t}$ and consequently
			$\lifboth{\lifboth{t}\tau} = \lifboth{\lifgammanovar{t}\tau}$.
			By Lemma~\ref{lemma:lifting_tau_commute}, $\lifboth{\lifboth{t}\tau} = \lifboth{t\tau}$ and by Lemma~\ref{lemma:lifting_order_not_relevant}, $\lifboth{t\tau} = \lifgammanovar{\lifdeltanovar{t\tau}}$.
			Hence $\lifgammanovar{\lifdeltanovar{t\tau}} = \lifboth{\lifgammanovar{t}\tau}$.
			%By Lemma~\ref{lemma:lifting_tau_commute},
			%$\lifboth{\lambda\tau} = \lifboth{\lifgammanovar{\lambda}\tau}$, 
			%which by Lemma~\ref{lemma:lifting_order_not_relevant} is nothing else than
			%$\lifgammanovar{\lifdeltanovar{\lambda\tau}} = \lifboth{\lifgammanovar{\lambda}\tau}$.
			\begin{comment}
				Hence it suffices to show that for a literal $\lambda$ in $\LIde(C_i)$ or $\LIclde(C_i)$ and its corresponding literal $\kappa$ in $\LI(C_i)$ or $\LIcl(C_i)$ for some $C_i \in \bbar C$
				that $\lifgammanovar{ \lifdeltanovar{\lambda\tau} } = \lifboth{\kappa\tau}$.
				%As $\lambda$ corresponds to a literal in $\LIde$ or $\LIclde$, we may by Lemma~\ref{lemma:no_colored_terms} additionally assume that no $\Delta$-colored term occurs in $\lambda$.

				By the induction hypothesis, $\lifgammanovar{\lambda} = \kappa$.
				By Lemma~\ref{lemma:no_colored_terms}, no $\Delta$-terms occur in $\lambda$.
				Hence 
				$\lifboth{\lambda} = \kappa$ and also
				$\lifboth{\lifboth{\lambda}\tau} = \lifboth{\kappa\tau}$.
				By Lemma~\ref{lemma:lifting_tau_commute},
				$\lifboth{\lambda\tau} = \lifboth{\kappa\tau}$, 
				which by Lemma~\ref{lemma:lifting_order_not_relevant} is nothing else than
				$\lifgammanovar{\lifdeltanovar{\lambda\tau}} = \lifboth{\kappa\tau}$.
			\end{comment}
			\begin{comment}

				For the induction step, it suffices to show that for a literal $\lambda$ with $\lifgammanovar{\lambda} = \kappa$
				that $\lifgammanovar{\lifdeltanovar{\lambda} \tau} = \lifboth{\kappa\tau}$.

				We abbreviate\
				$C_1\setminus \{l\}$ by $C_1^*$ and
				$C_2\setminus \{\lnot l'\}$ by $C_2^*$.

				Note that each literal in $\LIclde(C)$ is of the form $\lifdeltanovar{\lambda\tau}$ for some literal $\lambda$ in $\LIclde(C_1)$ or $\LIclde(C_2)$.
				Hence by the induction hypothesis, $\lifgammanovar{\lambda}$ is contained in $\LIcl(C_1)$ or $\LIcl(C_1)$.
				Note that every literal in $\LIcl(C)$ is of this form $\lifboth{\lifgammanovar{\lambda}\tau}$ for some $\lambda$ in $\LIclde(C_1)$ or $\LIclde(C_2)$.
				We show that $\lifboth{\lifgammanovar{\lambda}\tau} = \lifgammanovar{\lifdeltanovar{\lambda\tau}}$.

				As $\lambda$ is contained in $\LIclde(C_1)$ or $\LIclde(C_2)$, by Lemma~\ref{lemma:no_colored_terms}, $\lambda$ does not contain $\Delta$-colored terms.
				Hence $\lifboth{\lifgammanovar{\lambda}\tau} =
				\lifboth{\lifboth{\lambda}\tau}$.
				But by Lemma~\ref{lemma:lifting_tau_commute}, 
				$\lifboth{\lifboth{\lambda}\tau} = \lifboth{\lambda\tau}$, which by Lemma~\ref{lemma:lifting_order_not_relevant} is nothing else than $\lifgammanovar{\lifdeltanovar{\lambda\tau}}$.


				We now distinguish cases based on the color of the resolved literal:
				\begin{itemize}
					\item Suppose $l$ is $\Gamma$-colored.
				\end{itemize}

				~

				$\LIclde$:

				$\lifgammanovar{\LIclde(C_1)} = \LIcl(C_1)$

				$\lifdeltanovar{\LIclde(C_1)\tau} \subseteq \LIclde(C)$ 

				%$C_1\sigma \subseteq C$

				$\lifboth{\LIcl(C_1)\tau} \subseteq \LIcl(C)$

				to show: $\lifgammanovar{ \LIclde( C)} = \LIcl(C)$

				$ \lifboth{ \lifgammanovar{\LIclde(C_1)} \tau} = \lifboth{ \LIcl(C_1)\tau}$ $\quad$ IH + same op on both sides

				new lemma above

				$ \lifboth{ \lifgammanovar{\LIclde(C_1)} \tau} = 
				\lifboth{ \LIclde(C_1) \tau} $
				ub

				$\LIde$:

				\begin{itemize}
					\item Supp $\Gamma$:

						IH: $\lifgammanovar{ \LIde(C_1) } = \LI(C_1)$

						hence also: $\lifboth{ \LIde(C_1) } = \LI(C_1)$ (by lemma: no $\Delta$-terms in \dots)

						+ $\tau$:
						$\lifboth{ \LIde(C_1) }\tau = \LI(C_1)\tau$ 

						+ $\ell$:
						$\lifboth{\lifboth{ \LIde(C_1) }\tau} = \lifboth{\LI(C_1)\tau}$ 

						by new lemma
						$\lifboth{ \LIde(C_1) \tau} = \lifboth{\LI(C_1)\tau}$ 

						hence by Lemma~\ref{lemma:lifting_order_not_relevant}, $\lifgammanovar{\lifdeltanovar{\LIde(C_1)\tau}} \subseteq \LIde(C)$

						hence $\lifgammanovar{\LIde(C)} \subseteq \LIde(C)$



				\end{itemize}

			\end{comment}
			\qedhere
	\end{description}

\end{proof}

\section{One-sided interpolants}

As we have just seen, the formula $\LIde(C) \lor \LIclde(C)$ now satisfies one condition of interpolants.
Using this, we are able to formulate a result on one-sided interpolants, which are defined as follows:

\begin{defi}
	Let $\Gamma$ and $\Delta$ be sets of first-order formulas.
	A \defiemph{one-sided interpolant} of $\Gamma$ and $\Delta$ is a first-order formula $I$ such that
	\begin{enumerate}
		\item $\Gamma \entails I$
		\item $\Lang(I) \subseteq \Lang(\Gamma) \cap \Lang(\Delta)$
			\qedhere
	\end{enumerate}
\end{defi}

\begin{prop}
	Let $\Gamma$ and $\Delta$ be sets of first-order formulas such that $\Gamma\cup\Delta$ is unsatisfiable.
	Then there is a one-sided interpolant of $\Gamma$ and $\Delta$ which is a $\Pi_1$ formula.
\end{prop}
\begin{proof}
	Let $\pi$ be a resolution refutation of $\Gamma\cup\Delta$.
	By Lemma~\ref{lemma:gamma_entails_lide}, $\Gamma \entails \LIde(\pi) \lor \LIclde(\pi)$,
	or in other words
	$\Gamma \entails \forall x_1 \dots \forall x_n  \LIde(\pi) \lor \LIclde(\pi)$, where $x_1, \dots, x_n$ are the $\Delta$-lifting variables occurring in $\LIde(\pi) \lor \LIclde(\pi)$.
	%Let $x_1, \dots, x_n$ be the $\Delta$-lifting variables in $I$.
	By Lemma~\ref{lemma:no_colored_terms}, the formula $\LIde(\pi) \lor \LIclde(\pi)$ does not contain $\Delta$-colored symbols.

	Let $y_1, \dots y_m$ be the $\Gamma$-lifting variables of $\lifgamma{\LIde(\pi) \lor \LIclde(\pi)}$
	and
	\[I = \forall x_1 \dots \forall x_n \exists y_1 \dots \exists y_m \lifgamma{\LIde(\pi) \lor \LIclde(\pi)}.\]
	Note that $I$ does not contain any $\Gamma$-terms.
	As $\LIde(\pi) \lor \LIclde(\pi)$ contains witness terms for every existential quantifier in $I$ with respect to $\Gamma$, $\Gamma \entails I$.
	Hence $I$ is a $\Pi_1$ formula which is a one-sided interpolant for $\Gamma \cup \Delta$.
\end{proof}





\section{Quantifying over the lifting variables}

\begin{defi}[Quantifier block]
	\label{def:arrow_quantifier_block}
	Let $C$ be a clause in a resolution refutation $\pi$ of $\Gamma\cup\Delta$
	and $\bar x$ the $\Delta$-lifting variables and $\bar y$ the $\Gamma$-lifting variables occurring in $\LI(C)$ and $\lifiboth{C}$.
	$\Q(C)$ denotes an arrangement of the elements of  $\{ \forall x_t \mid\nolinebreak x_t \in \bar x\} \cup \{ \exists y_t \mid y_t \in \bar y\}$ such that for two lifting variable $z_s$ and $z_r$, if $s$ is a subterm of $r$, then $z_s$ precedes $z_r$.
	We denote $\Q(\square)$ by $\Q(\pi)$.
\end{defi}

Note that at a certain stage of the interpolant extraction, the quantifier block possesses a certain partial ordering based on the subterm relation of the indices of the lifting variables.
This implies that the ordering is monotonous in the sense that in the subsequent course of the extraction, this ordering is only extended but existing order-relations are not modified, even though the indices of the lifting variables are altered by means of substitution.


\begin{lemma}
	\label{lemma:gamma_entails_quantified_lide}
	Let $C$ be a clause of a resolution refutation of $\Gamma\cup\Delta$. Then
	$\Gamma\entails \Q(C)(\LI(C) \lor \LIcl(C))$.
\end{lemma}
\begin{proof}
	By Lemma~\ref{lemma:gamma_entails_lide},
	$\Gamma\entails \LIde(C) \lor \LIclde(C)$ and 
	by Lemma~\ref{lemma:gamma_lifted_lide}
	$\lifgammanovar{ \LIde(C) \lor \LIclde(C)} = \LI(C) \lor \LIcl(C)$.
	Hence the terms in $\LIde(C) \lor \LIclde(C)$ provide witness terms for the $\Gamma$-lifting variables in $\LI(C) \lor \LIcl(C)$, which are existentially quantified in $\Q(C) (\LI(C) \lor \LIcl(C))$.

	Furthermore, the ordering imposed on the quantifiers in $\Q(C)$ implies that if a $\Delta$-lifting variable $x_s$ occurs in a witness term for a $\Gamma$-lifting variable $y_r$, $y_r$ is quantified in the scope of the quantifier of $x_s$ as $s$ is a subterm of $r$.
	This however ensures that the witness terms are valid.
\end{proof}

\begin{lemma}
	\label{lemma:li_symmetry}
	Let $\pi$ be a refutation of $\Gamma\cup\Delta$ and $\bhat \pi$ be $\pi$ with $\bhat \Gamma = \Delta$ and $\bhat \Delta = \Gamma$.
	Then for a clause $C$ in $\pi$ and its corresponding clause $\bhat C$ in $\bhat \pi$, $\Q(C)(\LI(C)) \spas\semiff \lnot \Q(\bhat C)(\LI(\bhat C))$.
\end{lemma}
\begin{proof}
	%Note that $\LIcl$ is defined irrespective of the coloring, so $\LIcl(C) = \LIcl(\bhat C)$.

	Consider furthermore that liftings variables of $C$ and $\bhat C$ only differ in the variable symbol, but not in the index, and that the quantifier type of any given lifting variable in $C$ is dual to the corresponding one in $\bhat C$.
	Hence for any formula $\phi$, $\Q(C) \lnot \phi \spas\semiff \lnot \Q(\bhat C) \phi$.

	It remains to show that $\LI(C) \semiff \lnot \LI(\bhat C)$, which we establish by induction:

	\begin{itemize}
		\item[Base case.]
			If $C \in \Gamma$, then $\LI(C) = \bot \semiff \lnot \top \semiff \lnot \LI(\bhat C)$ as $\bhat C\in\Delta$. 
			The case for $C\in\Delta$ can be argued analogously.

		\item[Resolution.]
			Suppose the clause $C$ is the result of a resolution step \inference{} of $C_1: D \lor l$ and $C_2: E \lor \lnot l'$ with $\sigma = \mgu(\inference)$.

			As $\tau$ depends only on $\sigma$,
			$\tau$ is the same for both $\pi$ and $\bhat \pi$.

			We now distinguish the following cases:

			\begin{enumerate}

				\item $l$ is $\Gamma$-colored:
					\begin{align*}
						\hspace*{\dimexpr-\leftmargini-\leftmarginii}
						\LI(C)	&= \lifboth{\LI(C_1)\tau}\spas\lor \lifboth{\LI(C_2)\tau} \\
													&\semiff \lnot (\lnot\lifboth{\LI(C_1)\tau}\spas\land \lnot \lifboth{\LI(C_2)\tau}) \\
						%&\semiff \lnot (\lifboth{(\lnot \LI(C_1))\tau}\spas\land \lifboth{(\lnot \LI(C_2))\tau}) \\
													&\semiff \lnot (\lifboth{\LI(\bhat C_1)\tau}\spas\land \lifboth{\LI(\bhat C_2)\tau}) \\
													&= \lnot \LI(\bhat C)
					\end{align*}

				\item $l$ is $\Delta$-colored:
					This case can be argued analogously.

				\item $l$ is grey:
					Note that by Lemma~\ref{lemma:li_vs_clause_plus_literals_equal}, $\lifboth{l\cll\tau} = \lifboth{l'\cll\tau}$ \markB.
					\begin{align*}
						\hspace*{\dimexpr-\leftmargini-\leftmarginii}
						\LI(C) &\stackrel{{\phantom{\markB}}}=
						(\lnot {\lifboth{l'\cll\tau}} \land \lifboth{\LI(C_1)\tau}) \spam\lor 
						(\lifboth{l\cll\tau} \land \lifboth{\LI(C_2)\tau})\\
						&\stackrel{{\markB}}\semiff\,
						({\lifboth{l'\cll\tau}} \lor \lifboth{\LI(C_1)\tau}) \spam\land 
						(\lnot \lifboth{l\cll\tau} \lor \lifboth{\LI(C_2)\tau})\\
						&\stackrel{{\phantom{\markB}}}\semiff \lnot \Big( (\lnot {\lifboth{l'\cll\tau}} \land \lnot \lifboth{\LI(C_1)\tau}) \spam\lor 
						(\lifboth{l\cll\tau} \land \lnot\lifboth{\LI(C_2)\tau}) \Big) \\
						&\stackrel{{\phantom{\markB}}}=\lnot \Big( (\lnot {\lifboth{\bhat l'\cll\tau}} \land \lifboth{\LI(\bhat C_1)\tau}) \spam\lor 
						(\lifboth{\bhat l\cll\tau} \land \lifboth{\LI(\bhat C_2)\tau}) \Big)\\
						& \stackrel{{\phantom{\markB}}}= \lnot \LI(\bhat C) 
					\end{align*}


			\end{enumerate}



		\item[Factorisation.]
			Suppose the clause $C$ is the result of a factorisation $\inference$ of $C_1: l \lor l' \lor D$ 
			with $\sigma = \mgu(\inference)$.

			Then $\LI(C) = \lifboth{\LI(C_1)\tau}$, so the construction is not influenced by the coloring and the induction hypothesis gives the result.

		\item[Paramodulation.]
			Suppose the clause $C$ is the result of a paramodulation inference $\inference$ of $C_1: s=t \lor D$ and $C_2: E\occatp{r}$ with $\sigma=\mgu(\inference)$.


			We proceed by a case distinction:
			\begin{itemize}
				\item Suppose that $p$ in $E\occatp{r}$ is contained in a maximal $\Delta$-term $h\occ{r}$, which occurs more than once in $E\occatp{r} \lor \LI(E\occatp{r})$. 
					Then $p$ in $\bhat E\occatp{r}$ is contained in a maximal $\Gamma$-term $h\occ{r}$, which occurs more than once in $\bhat E\occatp{r} \lor \LI(\bhat E\occatp{r})$. 

					\hspace*{\dimexpr-\leftmargini-\leftmarginii-\leftmarginiii}\parbox{\linewidth}{%
						%\hspace*{\dimexpr-\leftmargini-\leftmarginii}\parbox{\linewidth}{%
						\begin{align*}
							%\hspace*{\dimexpr-\leftmargini-\leftmarginii}
							&\LI(C) \\
						 &= (\lifboth{s\tau=t\tau} \land \lifboth{\LI(C_2)\tau} ) \lor (\lifboth{s\tau\neq t\tau} \land \lifboth{\LI(C_1)\tau}) \lor (\lifboth{s\tau=t\tau} \land \lifboth{h\occur{s}\tau \neq h\occur{t}\tau}) \\
						 &\semiff \lnot [ (\lifboth{s\tau\neq t\tau} \lor \lnot \lifboth{\LI(C_2)\tau} ) \land (\lifboth{s\tau= t\tau} \lor \lnot\lifboth{\LI(C_1)\tau}) \land (\lifboth{s\tau\neq t\tau} \lor \lifboth{h\occur{s}\tau = h\occur{t}\tau}) ] \\
						 &= \lnot [ (\lifboth{s\tau\neq t\tau} \lor \lifboth{\LI(\bhat C_2)\tau} ) \land (\lifboth{s\tau= t\tau} \lor \lifboth{\LI(\bhat C_1)\tau}) \land (\lifboth{s\tau\neq t\tau} \lor \lifboth{h\occur{s}\tau = h\occur{t}\tau}) ] \\
						 &\semiff \lnot [ (\lifboth{s\tau= t\tau} \land \lifboth{\LI(\bhat C_2)\tau} ) \lor (\lifboth{s\tau\neq t\tau} \land \lifboth{\LI(\bhat C_1)\tau}) \land (\lifboth{s\tau\neq t\tau} \lor \lifboth{h\occur{s}\tau = h\occur{t}\tau}) ]\\
						 &= \lnot \LI(\bhat C)
						\end{align*}
					}\par

				\item Suppose that $p$ in $E\occatp{r}$ is contained in a maximal $\Gamma$-term $h\occ{r}$, which occurs more than once in $E\occatp{r} \lor \LI(E\occatp{r})$. 
					This case can be argued analogously.

				\item Otherwise:
					%\hspace*{\dimexpr-\leftmargini-\leftmarginii-\leftmarginiii}\parbox{\linewidth}{%
					%\hspace*{\dimexpr-\leftmargini-\leftmarginii}\parbox{\linewidth}{%
					\begin{align*}
						%\hspace*{\dimexpr-\leftmargini-\leftmarginii}
						&\LI(C) \\
						&= (\lifboth{s\tau=t\tau} \land \lifboth{\LI(C_2)\tau} ) \spam\lor (\lifboth{s\tau\neq t\tau} \land \lifboth{\LI(C_1)\tau}) \\
						&\semiff \lnot [ (\lifboth{s\tau\neq t\tau} \lor \lnot \lifboth{\LI(C_2)\tau} ) \spam\land (\lifboth{s\tau= t\tau} \lor \lnot \lifboth{\LI(C_1)\tau}) ] \\
						&= \lnot [ (\lifboth{s\tau\neq t\tau} \lor \lifboth{\LI(\bhat C_2)\tau} ) \spam\land (\lifboth{s\tau= t\tau} \lor \lifboth{\LI(\bhat C_1)\tau}) ] \\
						&\semiff \lnot [ (\lifboth{s\tau= t\tau} \land \lifboth{\LI(\bhat C_2)\tau} ) \spam\lor (\lifboth{s\tau\neq t\tau} \land \lifboth{\LI(\bhat C_1)\tau}) ] \\
						&= \lnot \LI(\bhat C)
						\qedhere
					\end{align*}
					%}\par



			\end{itemize}


			\qedhere

	\end{itemize}
\end{proof}

\begin{thm}
	Let $\pi$ be a resolution refutation of $\Gamma\cup\Delta$.
	Then $\LI( \pi )$ is an interpolant.
\end{thm}
\begin{proof}
	By Lemma~\ref{lemma:gamma_entails_quantified_lide}
	$\Gamma \entails \Q(\pi)(\LI(\pi) \lor \LIcl(\pi))$.
	But as $\LIcl(\pi) = \square$, this simplifies to
	$\Gamma \entails \Q(\pi)\LI(\pi)$.

	By constructing a proof $\bhat \pi$ from $\pi$ with $\bhat \Gamma = \Delta$ and $\bhat \Delta = \Gamma$, we obtain by Lemma~\ref{lemma:gamma_entails_quantified_lide} that $\bhat\Gamma \entails \Q(\bhat \pi)\LI(\bhat \pi)$.
	By Lemma \ref{lemma:li_symmetry}, this however is nothing else than
	$\Delta \entails \lnot \Q(\pi) \LI(\pi)$. 

	As furthermore by construction no colored symbols occur in $\Q(\pi) \LI(\pi)$, this formula is an interpolant for $\Gamma\cup\Delta$.
\end{proof}




\end{document}
