\documentclass[,%fontsize=11pt,%
	%landscape,
	%DIV8, % mehr text pro seite als defaultyyp
	%DIV10,
	%DIV=calc,%
	draft=false,% final|draft % draft ist platzsparender (kein code, bilder..)
	%titlepage,
	numbers=noendperiod,
	11pt,
	a4paper,
	oneside,% apparently, this should stay below some other parameter to have an effect
	openany,
	%]{scrartcl}
]{memoir}



\usepackage[utf8]{inputenc}
\usepackage[T1]{fontenc}
\usepackage[english]{babel}

\usepackage{microtype} % better kerning and general typesetting

% for screen display:
\usepackage[bitstream-charter]{mathdesign}
%\usepackage{amssymb} % conflicts with bitstream-charter
%\usepackage{newcent}
%\usepackage{lmodern}



\newcommand{\changefont}[3]{
\fontfamily{#1}\fontseries{#2}\fontshape{#3}\selectfont}

%\renewcommand{\familydefault}{ \sfdefault }
%\renewcommand{\rmdefault}{ppl}


%\usepackage[urw-garamond]{mathdesign}

\usepackage{lscape}
\usepackage{stackengine}
\usepackage{enumerate}
\usepackage{paralist}
\usepackage{tikz}
\usetikzlibrary{shapes,arrows,backgrounds,graphs,%
	matrix,patterns,arrows,decorations.pathmorphing,decorations.pathreplacing,%
	positioning,fit,calc,decorations.text,shadows%
}


\usepackage{comment} 

\usepackage{etoolbox} % fixes fatal error caused by combining bm, stackengine, hyperref (seriously?)
% http://tex.stackexchange.com/questions/22995/package-incompatibilites-etoolbox-hyperref-and-bm-standalone

\usepackage{etex} % else error on too many packages

% includes
\usepackage{algorithm}
%\usepackage{algorithmic} % conflicts with algpseudocode
\usepackage{algpseudocode}
%\newcommand*\Let[2]{\State #1 $\gets$ #2}
\algrenewcommand\alglinenumber[1]{
{\scriptsize #1}}
\algrenewcommand{\algorithmicrequire}{\textbf{Input:}}
\algrenewcommand{\algorithmicensure}{\textbf{Output:}}


%\usepackage[multiple]{footmisc} % footnotes at the same character separated by ','

\usepackage{multicol}

\usepackage{afterpage}

\usepackage{changepage} % for adjustwidth
\usepackage{caption} % for \ContinuedFloat

\usepackage{tikz}
\usetikzlibrary{shapes,arrows,backgrounds,graphs,%
matrix,patterns,arrows,decorations.pathmorphing,decorations.pathreplacing,%
positioning,fit,calc,decorations.text,shadows%
}

\usepackage{bussproofs}
\EnableBpAbbreviations


\usepackage{amsmath}
\usepackage{amsthm}
\usepackage{amssymb} % the reals
\usepackage{mathtools} % smashoperator

\usepackage{bm} % bm, bold math symbols

\usepackage{thm-restate} % restatable env

% needs extra work and fails on some label here
%\usepackage{cleveref} % cref, apparently better than autoref of hyperref 

\usepackage{nicefrac} % nicefrac

\usepackage{mathrsfs} % mathscr

\usepackage{pst-node} % http://tex.stackexchange.com/questions/35717/how-to-draw-arrows-between-parts-of-an-equation-to-show-the-math-distributive-pr

\usepackage{stackengine}

\usepackage{thmtools} % advanced thm commands (declaretheorem)


\usepackage{nameref} % reference name of thm instead of counter

\usepackage{todonotes}

% conflict with beamer
%\usepackage{paralist} % compactenum

\usepackage{hyperref}
%\hypersetup{hidelinks}  % don't give options to usepackage, it doesn't work with beamer
%\hypersetup{colorlinks=false}  % don't give options to usepackage, it doesn't work with beamer


% \usepackage{enumitem} % labels for enumerate % breaks beamer and memoir itemize


\usepackage{url} 


\usepackage[format=hang,justification=raggedright]{caption}% or e.g. [format=hang]

\usepackage{cancel} % \cancel

\usepackage{lineno}


% commands

% logic etcs
%\newcommand{\ex}[2]{\bigskip\section*{Exercise #1: \begin{minipage}[t]{.80\linewidth} \small \textnormal{\it #2} \end{minipage} } }

\newcommand{\ex}[2]{\bigskip \noindent\textbf{Exercise #1.} \textit{#2} \smallskip}

\newcommand{\comm}[1]{{\color{gray} // #1 }}


\newcommand{\true}[0]{\textbf{1}}
\newcommand{\false}[0]{\textbf{0}}
\newcommand{\tr}{\true}
\newcommand{\fa}{\false}

\newcommand{\ra}{\rightarrow}
\newcommand{\Ra}{\Rightarrow}
\newcommand{\la}{\leftarrow}
\newcommand{\La}{\Leftarrow}

\newcommand{\lra}{\leftrightarrow}
\newcommand{\Lra}{\Leftrightarrow}

\newcommand{\NKZ}{\textbf{NK2}}

%\DeclareMathOperator{\syneq}{\equiv} %spacing seems wrong, therefore defined as newcommand below
\DeclareMathOperator{\limpl}{\supset}
\DeclareMathOperator{\liff}{\lra}
\DeclareMathOperator{\semiff}{\Lra}
\newcommand{\syneq}{\equiv}
\newcommand{\union}{\cup}
\newcommand{\bigunion}{\bigcup}
\newcommand{\intersection}{\cap}
\newcommand{\bigintersection}{\bigcap}
\newcommand{\intersect}{\intersection}
\newcommand{\bigintersect}{\bigintersection}

\newcommand{\powerset}{\mathcal{P}}

\newcommand{\entails}{\vDash}
\newcommand{\notentails}{\nvDash}
\newcommand{\proves}{\vdash}

\newcommand{\vm}{\ensuremath{\vv_\mathcal{M}}}
\newcommand{\Dia}{\ensuremath{\lozenge}}

\newcommand{\spaced}[1]{\ \ #1 \ \ }
\newcommand{\spa}[1]{\spaced{#1}}
\newcommand{\spas}[1]{\;{#1}\;}
\newcommand{\spam}[1]{\;\,{#1}\;\,}

% functions
\DeclareMathOperator{\sk}{sk}
\DeclareMathOperator{\mgu}{mgu}
\DeclareMathOperator{\dom}{dom}
\DeclareMathOperator{\ran}{ran}

\DeclareMathOperator{\id}{id}
\DeclareMathOperator{\Fun}{FS}
\DeclareMathOperator{\Pred}{PS}
\DeclareMathOperator{\Lang}{L}
\DeclareMathOperator{\ar}{ar}
\DeclareMathOperator{\PI}{PI}
\DeclareMathOperator{\LI}{LI}
\DeclareMathOperator{\Congr}{Congr}
\DeclareMathOperator{\Refl}{Refl}
\DeclareMathOperator{\aiu}{au}
\DeclareMathOperator{\expa}{unfold-lift}

\newcommand{\PIinc}{\LI}
\newcommand{\PIincde}{\LIde}

\newcommand{\LIde}{\ensuremath{\LI^\Delta}}

\newcommand{\LIcl}{\ensuremath{\LI_{\operatorname{cl}}}}
\newcommand{\LIclde}{\ensuremath{\LI_{\operatorname{cl}}^\Delta}}

\newcommand{\cll}{\ensuremath{_{\operatorname{LIcl}}}}
\newcommand{\cllde}{\ensuremath{_{\operatorname{LIcl}^\Delta}}}

%\newcommand{\lifi}{\mathop{\ell\text{}i}}
\newcommand{\lifiboth}[1]{\ensuremath{\LIcl(#1)}}
\newcommand{\lifidelta}[1]{\ensuremath{\LIclde(#1)}}


%\DeclareMathOperator{\abstraction}{abstraction}

%\newcommand{\sk}{\ensuremath{\mathrm{sk}}}
%\newcommand{\mgu}{\ensuremath{\mathrm{mgu}}}
%\newcommand{\Fun}{\ensuremath{\mathrm{FS}}}
%\newcommand{\Pred}{\ensuremath{\mathrm{PS}}}
%\newcommand{\PI}{\ensuremath{\mathrm{PI}}}
%\newcommand{\Lang}{\ensuremath{\mathrm{L}}}
%\newcommand{\ar}{\ensuremath{\mathrm{ar}}}

\DeclareMathOperator{\AI}{AI}
\newcommand{\AIde}{\ensuremath{\AI^\Delta}}
\newcommand{\AImatrix}{\ensuremath{\AI_\mathrm{mat}}}
\newcommand{\AImatrixde}{\ensuremath{\AI_\mathrm{mat}^\Delta}}
\newcommand{\AImat}{\AImatrix}
\newcommand{\AImatde}{\AImatrixde}
\newcommand{\AIclause}{\ensuremath{\AI_\mathrm{cl}}}
\newcommand{\AIcl}{\AIclause}
\newcommand{\AIclde}{\AIclausede}
\newcommand{\AIclausede}{\ensuremath{\AIclause^\Delta}}
\newcommand{\fromclause}{\ensuremath{_{\operatorname{AIcl}}}}
\newcommand{\fromclausede}{\ensuremath{_{\operatorname{AIcl}^\Delta}}}
\newcommand{\cl}{\fromclause}
\newcommand{\clde}{\fromclausede}

\newcommand{\Q}{\ensuremath{Q}}

\newcommand{\AIcol}{\ensuremath{\AI_\mathrm{col}}}
\newcommand{\AIcolde}{\AIcol^\Delta}

\newcommand{\AIany}{\ensuremath{\AI_\mathrm{*}}}
\newcommand{\AIanyde}{\AIany^\Delta}

\newcommand{\AIclpre}{\AIclause^\bullet}
\newcommand{\AImatpre}{\AImatrix^\bullet}

\newcommand{\PS}{\Pred}
\newcommand{\FS}{\Fun}

\DeclareMathOperator{\LangSym}{\mathcal{L}}

%\newcommand{\mguarr}{\sim_\ra}
\newcommand{\mguarr}{\mapsto_{\mgu}}


%\newcommand{\Trans}{\ensuremath{\mathrm{T}}}
%\newcommand{\Trans}{\ensuremath{\mathrm{T}}}
\DeclareMathOperator{\Trans}{T}
\DeclareMathOperator{\TransInv}{T^{-1}}

\DeclareMathOperator{\FAX}{F_{Ax}}
\DeclareMathOperator{\EAX}{E_{Ax}}
%\newcommand{\FAX}{\ensuremath{\mathrm{F_{Ax}}}}
%\newcommand{\EAX}{\ensuremath{\mathrm{E_{Ax}}}}

%\newcommand{\TransAll}{\ensuremath{\Trans_{\mathrm{Ax}}}}
\DeclareMathOperator{\TransAll}{\Trans_{Ax}}
%\newcommand{\FAX}{\ensuremath{\mathrm{F_{Ax}}}}

\DeclareMathOperator{\defeq}{\stackrel{\mathrm{def}}{=}}

\newcommand{\subst}[1]{[#1]}
\newcommand{\abstractionOp}[1]{\{#1\}}

\newcommand{\subformdefinitional}[1]{\ensuremath{D_{\Sigma(#1)}}}


%\newcommand{\lift}[3]{\operatorname{Lift}_{#1}(#2; #3)}
%\newcommand{\lift}[3]{\operatorname{Lift}_{#1,#3}(#2)}
%\newcommand{\lift}[3]{\operatorname{Lift}_{#1,#3}[#2]}
%\newcommand{\lift}[3]{\overline{#2}_{#1,#3}}
\newcommand{\lifsym}{\ell}
%\newcommand{\lift}[3]{\lifsym_{#1,#3}[#2]}
\newcommand{\lift}[3]{\lifsym_{#1}^{#3}[#2]}
\newcommand{\liftnovar}[2]{\lifsym_{#1}[#2]}

%\newcommand{\lft}[3]{\lifsym_{#1,#2}[#3]}
\newcommand{\lft}[3]{\lift{#1}{#3}{#2}}
\newcommand{\lifboth}[1]{\lifsym[#1]}

%\newcommand{\lifi}{\mathop{\ell\text{}i}}
%\newcommand{\lifiboth}[1]{\lifi[#1]}
%\newcommand{\lifidelta}[1]{\lifi_\Delta^x[#1]}
%\newcommand{\lifideltanovar}[1]{\lifi_\Delta[#1]}

\newcommand{\lifdelta}[1]{\lift{\Delta}{#1}{x}}
\newcommand{\lifdeltanovar}[1]{\liftnovar{\Delta}{#1}}
\newcommand{\lifgamma}[1]{\lift{\Gamma}{#1}{y}}
\newcommand{\lifgammanovar}[1]{\liftnovar{\Gamma}{#1}}
\newcommand{\lifphinovar}[1]{\liftnovar{\Phi}{#1}}
\newcommand{\lifphi}[1]{\lift{\Phi}{#1}{z}}

\DeclareMathOperator{\arr}{\mathcal{A}}
%\DeclareMathOperator{\arrFinal}{{\mathcal{A}^{\bm*}}}
\DeclareMathOperator{\arrFinal}{{\mathcal{\bm{\hat}A}}}
\DeclareMathOperator{\warr}{\marr}
\DeclareMathOperator{\marr}{\mathcal{M}}

\DeclareMathOperator{\apath}{\leadsto}
\DeclareMathOperator{\mpath}{\leadsto_=}
\DeclareMathOperator{\notapath}{\not\leadsto}
\DeclareMathOperator{\notmpath}{\not\leadsto_=}

\newcommand{\ltArrC}{<_{\arrFinal(C)}}
\newcommand{\ltAC}{<_{\arr(C)}}
\newcommand{\ltArrCOne}{<_{\arrFinal(C_1)}}
\newcommand{\ltArrCTwo}{<_{\arrFinal(C_2)}}
%\newcommand{\ltArrC}{<_{\scalebox{0.6}{$\arrFinal(C)$}}}
\newcommand{\ltArr}{<_{\scalebox{0.6}{$\arrFinal$}}}

\newcommand{\bhat}{\bm\hat}
\newcommand{\bbar}{\bm\bar}
\newcommand{\bdot}{\bm\dot}

%\usepackage{yfonts}
\usepackage{upgreek}
\DeclareMathAlphabet{\mathpzc}{OT1}{pzc}{m}{it}
%\DeclareMathOperator{\pos}{\mathscr{P}}
%\DeclareMathOperator{\pos}{\mathpzc{p}}
%\DeclareMathOperator{\pos}{{\rho}}
\DeclareMathOperator{\pos}{{\operatorname P}}
%\DeclareMathOperator{\pos}{P}
\DeclareMathOperator{\poslit}{\pos_\mathrm{lit}}
\DeclareMathOperator{\posterm}{\pos_\mathrm{term}}
%\newcommand{\poslit}[1]{\ensuremath{p_\text{lit}(#1)}}
%\newcommand{\posterm}[1]{\ensuremath{p_\text{term}(#1)}}
\newcommand{\at}[1]{|_{#1}}

\newcommand{\UICm}[1]{\UnaryInfCm{#1}}
\newcommand{\UnaryInfCm}[1]{\UnaryInfC{$#1$}}
\newcommand{\BICm}[1]{\BinaryInfCm{#1}}
\newcommand{\BinaryInfCm}[1]{\BinaryInfC{$#1$}}
\newcommand{\RightLabelm}[1]{\RightLabel{$#1$}}
\newcommand{\LeftLabelm}[1]{\LeftLabel{$#1$}}
\newcommand{\AXCm}[1]{\AxiomCm{#1}}
\newcommand{\AxiomCm}[1]{\AxiomC{$#1$}}
\newcommand{\mt}[1]{\textnormal{#1}}

\newcommand{\UnaryInfm}[1]{\UnaryInf$#1$}
\newcommand{\BinaryInfm}[1]{\BinaryInf$#1$}
\newcommand{\Axiomm}[1]{\Axiom$#1$}



% math
\newcommand{\calI}{\ensuremath{\mathcal{I}}}

\newcommand{\tupleShort}[2]{\ensuremath{(#1_1,\dotsc,#1_{#2})}}
\newcommand{\tuple}[2]{\ensuremath{(#1_1,\:#1_2\:,\dotsc,\:#1_{#2})}}
\newcommand{\setelements}[2]{\ensuremath{\{#1_1,\:#1_2\:,\dotsc,\:#1_{#2}\}}}
\newcommand{\pathelements}[2]{\ensuremath{ (#1_1,\:#1_2\:,\dotsc,\:#1_{#2}) }}

\newcommand{\elems}[1]{\ensuremath{#1_1,\dotsc, #1_{n}) }}

\newcommand{\defiemph}[1]{\emph{#1}}

\newcommand{\setofbases}{\ensuremath{\mathcal{B}}}
\newcommand{\setofcircuits}{\ensuremath{\mathcal{C}}}

\newcommand{\reals}{\ensuremath{\mathbb{R}}}
\newcommand{\integers}{\ensuremath{\mathbb{Z}}} 
\newcommand{\naturalnumbers}{\ensuremath{\mathbb{N}}}

% general
\newcommand{\zit}[3]{#1\ \cite{#2}, #3}
\newcommand{\zitx}[2]{#1\ \cite{#2}}
\newcommand{\footzit}[3]{\footnote{\zit{#1}{#2}{#3}}}
\newcommand{\footzitx}[2]{\footnote{\zitx{#1}{#2}}}

\newcommand{\ite}{\begin{itemize}}
\newcommand{\ete}{\end{itemize}}

\newcommand{\bfr}{\begin{frame}}
\newcommand{\efr}{\end{frame}}

\newcommand{\ilc}[1]{\texttt{#1}}


% misc

% multiframe
\usepackage{xifthen}% provides \isempty test
% new counter to now which frame it is within the sequence
\newcounter{multiframecounter}
% initialize buffer for previously used frame title
\gdef\lastframetitle{\textit{undefined}}
% new environment for a multi-frame
\newenvironment{multiframe}[1][]{%
\ifthenelse{\isempty{#1}}{%
% if no frame title was set via optional parameter,
% only increase sequence counter by 1
\addtocounter{multiframecounter}{1}%
}{%
% new frame title has been provided, thus
% reset sequence counter to 1 and buffer frame title for later use
\setcounter{multiframecounter}{1}%
\gdef\lastframetitle{#1}%
}%
% start conventional frame environment and
% automatically set frame title followed by sequence counter
\begin{frame}%
\frametitle{\lastframetitle~{\normalfont \Roman{multiframecounter}}}%
}{%
\end{frame}%
}




% http://texfragen.de/hurenkinder_und_schusterjungen
\usepackage[all]{nowidow}



% force no overlong lines:
%\tolerance=1 % tolerance for how badly spaced lines are allowed, less means "better" lines
\tolerance=500 %  need more tolerance for equations
%\emergencystretch=\maxdimen
%\emergencystretch=200pt
%\setlength{\emergencystretch}{3em}
%\hyphenpenalty=10000 % forces no hyphenation
%\hbadness=10000


% http://tex.stackexchange.com/questions/35717/how-to-draw-arrows-between-parts-of-an-equation-to-show-the-math-distributive-pr
\tikzset{square arrow/.style={to path={ -- ++(.0,-.15)  -| (\tikztotarget)}}}
\tikzset{square arrow2/.style={to path={ -- ++(.0,-.25)  -| (\tikztotarget)}}}
%\tikzset{square arrow/.style={to path={ -- ++(00,-.01) -- ++(0.5,-0.1) -- ++(0.5,-0.1) -| (\tikztotarget)},color=red}}


% have arrows from a to b and from c to d here
% just use: texttext\arrowA texttest \arrowB texttext
\newcommand{\arrowA}{\tikz[overlay,remember picture] \node (a) {};}
\newcommand{\arrowB}{\tikz[overlay,remember picture] \node (b) {};}
\newcommand{\drawAB}{
	\tikz[overlay,remember picture]
	{\draw[->,bend left=5,color=red] (a.south) to (b.south);}
	%{\draw[->,square arrow,color=red] (a.south) to (b.south);}
}
\newcommand{\arrowAP}{\tikz[overlay,remember picture] \node (ap) {};}
\newcommand{\arrowBP}{\tikz[overlay,remember picture] \node (bp) {};}
\newcommand{\drawABP}{
	\tikz[overlay,remember picture]
	{\draw[->,bend right=5,color=red] (ap.south) to (bp.south);}
	%{\draw[->,square arrow,color=red] (a.south) to (b.south);}
}

\newcommand{\arrowAB}{\tikz[overlay,remember picture] \node (ab) {};}
\newcommand{\arrowBA}{\tikz[overlay,remember picture] \node (ba) {};}
\newcommand{\drawAABB}{
	\tikz[overlay,remember picture]
	%{\draw[->,bend left=80] (a.north) to (b.north);}
	{\draw[->,square arrow,color=brown] (ab.south) to (ba.south);
	\draw[->,square arrow,color=brown] (ba.south) to (ab.south);}
}


\newcommand{\arrowCD}{\tikz[overlay,remember picture] \node (cd) {};}
\newcommand{\arrowDC}{\tikz[overlay,remember picture] \node (dc) {};}
\newcommand{\drawCCDD}{
	\tikz[overlay,remember picture]
	%{\draw[->,bend left=80] (a.north) to (b.north);}
	{\draw[<->,dashed,square arrow,color=green] (cd.south) to (dc.south); }
}



\newcommand{\arrowC}{\tikz[overlay,remember picture] \node (c) {};}
\newcommand{\arrowD}{\tikz[overlay,remember picture] \node (d) {};}
\newcommand{\drawCD}{
	\tikz[overlay,remember picture]
	{\draw[->,square arrow,color=blue] (c.south) to (d.south);}
}

\newcommand{\arrowE}{\tikz[overlay,remember picture] \node (e) {};}
\newcommand{\arrowF}{\tikz[overlay,remember picture] \node (f) {};}
\newcommand{\drawEF}{
	\tikz[overlay,remember picture]
	{\draw[->,square arrow2,color=orange] (e.south) to (f.south);}
}


\newcommand{\arrAP}{\arrowAP}
\newcommand{\arrBP}{\arrowBP}
\newcommand{\arrA}{\arrowA}
\newcommand{\arrB}{\arrowB}
\newcommand{\arrC}{\arrowC}
\newcommand{\arrD}{\arrowD}
\newcommand{\arrE}{\arrowE}
\newcommand{\arrF}{\arrowF}


\DeclareMathOperator{\simgeq}{\scalebox{0.92}{$\gtrsim$}}

\newcommand{\refsub}[2]{\hyperref[#2]{\ref*{#1}.\ref*{#2}}}

%\newcommand{\sigmarange}[2]{\sigma_{#1}^{#2} }
\newcommand{\sigmarange}[2]{\sigma_{(#1,#2)} }
\newcommand{\sigmaz}[1]{\sigmarange{0}{#1} }
\newcommand{\sigmazi}[0]{\sigmaz{i} }

\DeclareMathOperator{\lit}{lit}

%\def\fCenter{\ \proves\ }
\def\fCenter{\proves}

\newcommand{\prflbl}[2]{\RightLabel{\footnotesize $#1, #2$} }
%\newcommand{\prflblid}[1]{\RightLabel{$#1, \id$} }
\newcommand{\prflblid}[1]{\RightLabel{\footnotesize $#1$} }

\DeclareMathOperator{\resruleres}{res}
\DeclareMathOperator{\resrulefac}{fac}
\DeclareMathOperator{\resrulepar}{par}
\newcommand{\lkrule}[2]{\ensuremath{\operatorname{#1}:#2}} % operatorname fixes spacing issues for =

\newcommand{\parti}[4]{\ensuremath{ \langle (#1; #2), (#3; #4)\rangle  }}

\newcommand{\partisym}{\ensuremath{\chi}}

\newcommand{\occur}[1]{\ensuremath{[#1]}}
\newcommand{\occ}[1]{\occur{#1}}

\newcommand{\occurat}[2]{\ensuremath{{\occur{#1}_{#2}}}}
\newcommand{\occat}[2]{\occurat{#1}{#2}}
\newcommand{\occatp}[1]{\occurat{#1}{p}}
\newcommand{\occatq}[1]{\occurat{#1}{q}}

\newcommand{\colterm}[1]{\zeta_{#1}}



% fix restateable spacing 
%http://tex.stackexchange.com/questions/111639/extra-spacing-around-restatable-theorems

\makeatletter

\def\thmt@rst@storecounters#1{%
%THIS IS THE LINE I ADDED:
\vspace{-1.9ex}%
  \bgroup
        % ugly hack: save chapter,..subsection numbers
        % for equation numbers.
  %\refstepcounter{thmt@dummyctr}% why is this here?
  %% temporarily disabled, broke autorefname.
  \def\@currentlabel{}%
  \@for\thmt@ctr:=\thmt@innercounters\do{%
    \thmt@sanitizethe{\thmt@ctr}%
    \protected@edef\@currentlabel{%
      \@currentlabel
      \protect\def\@xa\protect\csname the\thmt@ctr\endcsname{%
        \csname the\thmt@ctr\endcsname}%
      \ifcsname theH\thmt@ctr\endcsname
        \protect\def\@xa\protect\csname theH\thmt@ctr\endcsname{%
          (restate \protect\theHthmt@dummyctr)\csname theH\thmt@ctr\endcsname}%
      \fi
      \protect\setcounter{\thmt@ctr}{\number\csname c@\thmt@ctr\endcsname}%
    }%
  }%
  \label{thmt@@#1@data}%
  \egroup
}%

\makeatother




\newcommand{\mymark}[1]{\ensuremath{(#1)}}
\newcommand{\markA}{\mymark \circ}
\newcommand{\markB}{\mymark *}
\newcommand{\markC}{\mymark \divideontimes}

\newcommand{\wrong}[1]{{\color{red}WRONG: #1}}
\newcommand{\NB}[1]{{\color{blue}NB: #1}}
\newcommand{\hl}[1]{{\color{orange} #1}}
\newcommand{\mytodo}[1]{{\color{red}TODO: #1}}
\newcommand{\largered}[1]{{

	  \LARGE\bfseries\color{red}
		#1

}}
\newcommand{\largeblue}[1]{{

	  \large\bfseries\color{blue}
		#1

}}




\usepackage{ulem} %  \dotuline{dotty} \dashuline{dashing} \sout{strikethrough}
\normalem

\usepackage{tabu} % tabular also in math mode (and much more)

\usepackage[color]{changebar} %  \cbstart, \cbend
\cbcolor{red}



% http://tex.stackexchange.com/questions/7032/good-way-to-make-textcircled-numbers
\newcommand*\circled[1]{\tikz[baseline=(char.base)]{
\node[shape=circle,draw,inner sep=2pt] (char) {#1};}}



% http://tex.stackexchange.com/questions/43346/how-do-i-get-sub-numbering-for-theorems-theorem-1-a-theorem-1-b-theorem-2

\makeatletter
\newenvironment{subtheorem}[1]{%
  \def\subtheoremcounter{#1}%
  \refstepcounter{#1}%
  \protected@edef\theparentnumber{\csname the#1\endcsname}%
  \setcounter{parentnumber}{\value{#1}}%
  \setcounter{#1}{0}%
  \expandafter\def\csname the#1\endcsname{\theparentnumber.\Alph{#1}}%
  \ignorespaces
}{%
  \setcounter{\subtheoremcounter}{\value{parentnumber}}%
  \ignorespacesafterend
}
\makeatother
\newcounter{parentnumber}


\usepackage{tabularx}% http://ctan.org/pkg/tabularx
\newcolumntype{Y}{>{\centering\arraybackslash}X}

\newcommand{\mycols}[2][3]{
	\noindent\begin{tabularx}{\textwidth}{*{#1}{Y}}
		#2
	\end{tabularx}%
}


\newcommand{\definethms}{

	%\declaretheorem[title=Theorem,qed=$\triangle$,parent=chapter]{thm}
	\newcommand{\thmqed}{$\square$} % for thms without proof
	\newcommand{\propqed}{$\square$} % for props without proof
	\declaretheorem[title=Theorem]{thm}
	\declaretheorem[title=Proposition,sibling=thm]{prop}
	\declaretheorem[title=Conjectured Proposition,sibling=thm]{cprop}

	%\declaretheorem[title=Lemma,parent=chapter]{lemma}
	\declaretheorem[sibling=thm]{lemma}
	\declaretheorem[sibling=thm,title=Conjectured Lemma]{clemma}
	\declaretheorem[title=Corollary,sibling=thm]{corr}
	\declaretheorem[sibling=thm,title=Definition,style=definition,qed=$\triangle$]{defi}
	%\declaretheorem[title=Definition,qed=$\triangle$,parent=chapter]{defi}
	\declaretheorem[title=Example,style=definition,qed=$\triangle$,sibling=thm]{exa}

	\declaretheorem[sibling=thm,title=Conjecture]{conj}

	\declaretheorem[title=Remark,style=remark,numbered=no,qed=$\triangle$]{remark}


}

\usepackage[matha]{mathabx} % the locial operators here have more space around them and [ and ] are thicker, also langle and rangle are a bit nicer; subseteq looks a bit weird

%\usepackage{MnSymbol} % again other symbols


\newcommand{\inference}{\ensuremath{\iota}}

\usepackage{cases} % numcases



% subsections also in toc
\setcounter{tocdepth}{2}
\setsecnumdepth{subsection}


\definethms

\def\proofSkipAmount{ \vskip -0.1em }


%\usepackage{bussproof}

%\usepackage{vaucanson-g}
%\usepackage{amssymb}
\usepackage{latexsym}

% for color-highlighted code
%\usepackage{color} % for grey comments
%\usepackage{alltt}

%\usepackage[doublespacing]{setspace}
%\usepackage[onehalfspacing]{setspace}
%\usepackage[singlespacing]{setspace}


\usepackage{amsthm}


\chapterstyle{madsen}
%\chapterstyle{bianchi}


% define page numbering styles
\makepagestyle{numberCorner}
\makeevenfoot{numberCorner}{\thepage}{}{}
\makeoddfoot{numberCorner}{}{}{\thepage}

\makepagestyle{numberCenter}
%\makeevenfoot{numberCenter}{}{\thepage}{}
%\makeoddfoot{numberCenter}{}{\thepage}{}
%
%\makeevenhead{numberCenter}{\thechapter}{}{\thesection}
%\makeoddhead{numberCenter}{\thesection }{}{\thechapter}
\makeheadrule{numberCenter}{\textwidth}{1pt}

\makeevenhead{numberCenter}{\thepage}{}{\leftmark}
\makeoddhead{numberCenter}{\rightmark}{}{\thepage}


\makeatletter
\makepsmarks{numberCenter}{
	\def\chaptermark##1{\markboth{%
			\ifnum \value{secnumdepth} > -1
			\if@mainmatter
			\chaptername\ \thechapter\ --- %
			\fi
			\fi
	##1}{}}
	\def\sectionmark##1{\markright{%
			\ifnum \value{secnumdepth} > 0
			\thesection. \ %
			\fi
	##1}}
}
\makeatother
\newcommand{\mysetpagestyle}{
	%\pagestyle{numberCorner}
	\pagestyle{numberCenter}
}
\mysetpagestyle





\usepackage{refcheck}

\settypeblocksize{0.64\stockheight}{0.64\stockwidth}{*}
%\settypeblocksize{0.63\stockheight}{0.63\stockwidth}{*}
\setlrmargins{*}{*}{1.0}
\setulmargins{*}{*}{1.4}
\checkandfixthelayout[nearest]

  
\binoppenalty=9999
\relpenalty=9999

\begin{document}



\tableofcontents

%\section{Lemmas from other pdf}
\begin{lemma}
	\label{lemma:lifting_order_not_relevant}
	$\lifgammanovar{\lifdeltanovar{\varphi}} = \lifdeltanovar{\lifgammanovar{\varphi}}$.
\end{lemma}

\clearpage

\chapter{Interpolant extraction from resolution proofs in one phase \hl{lifting terms whose quantifier position can be determined -- nested}}

%\section{Incremental lifting and substitutions of lifting variables}

\begin{defi}[Substitution $\tau(\inference)$]
	For an inference $\inference$ with $\sigma = \mgu(\inference)$, we define the infinite substitution\todo{define infinite substitutions properly and apply definition here}{} $\tau(\inference)$ with $\dom(\tau(\inference)) = \dom(\sigma) \cup \{z_s \mid s\sigma \neq s\}$ as follows for a variable $x$:

	\[
		x\tau(\inference) =
		\begin{cases}
			x\sigma & \text{$x$ is a non-lifting variable} \\
			z_{t\sigma} & \text{$x$ is a lifting variable $z_t$}
		\end{cases} 
	\]
		%\qedhere

	If the inference $\inference$ is clear from the context, we abbreviate $\tau(\inference)$ by $\tau$. 
\end{defi}


\begin{lemma}
	\label{lemma:lifting_tau_commute}
	For a formula or term $\varphi$ and an inference $\inference$ such that $\tau = \tau(\inference)$,
	$\lifboth{\lifboth{ \varphi} \tau} =\nolinebreak \lifboth{ \varphi \tau } $.
\end{lemma}
\begin{proof}
	We proceed by induction.

	\begin{itemize}
		\item Suppose that $t$ is a grey constant or function symbol of the form $f(t_1, \dots, t_n)$.
			Then we can derive the following, where (IH) signifies a deduction by virtue of the induction hypothesis. 
			\begin{align*}
				\lifboth{\lifboth{t}\tau} &= \lifboth{\lifboth{ f(t_1, \dots, t_n)}\tau} \\
																  &= \lifboth{ f(\lifboth{t_1}\tau, \dots, \lifboth{t_n}\tau) } \\
																	&= f(\lifboth{\lifboth{t_1}\tau}, \dots, \lifboth{\lifboth{t_n}\tau}) \\
																 	&\stackrel{\mathclap{\text{(IH)}}}= f(\lifboth{t_1\tau}, \dots, \lifboth{t_n\tau}) \\
																	&= \lifboth{f(t_1, \dots, t_n)\tau} \\
																	&= \lifboth{t\tau}
			\end{align*}
		\item Suppose that $t$ is a colored constant or function symbol. Then:
			\[
				\lifboth{\lifboth{t}\tau}= \lifboth{z_t\tau} 
				= \lifboth{z_{t\sigma}} 
				= z_{t\sigma} 
				= \lifboth{t\sigma} 
				= \lifboth{t\tau}
			\]
		\item Suppose that $t$ is a variable $x$. Then:
			\[
				\lifboth{\lifboth{t}\tau} = \lifboth{\lifboth{ x}\tau} = \lifboth{x\tau} = \lifboth{t\tau}
			\]
		\item Suppose that $t$ is a lifting variable $z_t$. Then:
			\[
				\lifboth{\lifboth{z_t}\tau} = \lifboth{z_t\tau} 
				\qedhere
			\]
	\end{itemize}

\end{proof}



\begin{defi}[Incrementally lifted interpolant]
	Let $\pi$ be a resolution refutation of $\Gamma \cup \Delta$.
	We define $\LI(\pi)$ to be $\LI(\square)$, where $\square$ is the empty clause derived in $\pi$.

	Let $C$ be a clause in $\pi$. 
	%For a literal $\lambda$ in $C$, we denote the corresponding literal in $\LIcl(C)$ by $\lambda\cll$, whose existence is ensured Lemma~\ref{lemma:li_vs_clause_plus_literals_equal}.

	%We define $\LIcl(C) \defeq C$. \mytodo{if this version is final, drop $\LIcl(C)$ everywhere}

	We define the preliminary formula $\LI^\bullet(C)$ as follows:
	\begin{description}
	\item{} Base case.
			If $C \in \Gamma$, $\LI(C) \defeq \bot$.
			If otherwise $C \in \Delta$, $\LI(C) \defeq \top$.

		\item{} Resolution.

			If the clause $C$ is the result of a resolution step $\inference$ of $C_1: D \lor l$ and $C_2: E \lor \lnot l'$ using a unifier $\sigma$ such that $l\sigma =  l'\sigma$, then define $\LI(C)$ as follows:

			\begin{enumerate}

				\item If $l$ is $\Gamma$-colored:
					$\LIpre(C) \defeq \LI(C_1)\tau\spas\lor \LI(C_2)\tau $

				\item If $l$ is $\Delta$-colored:
					$\LIpre(C) \defeq \LI(C_1)\tau\spas\land \LI(C_2)\tau$

				\item If $l$ is grey:
					$\LIpre(C) \defeq
					(l\tau \land \LI(C_2)\tau) \spas\lor
					(\lnot l'\tau\land \LI(C_1)\tau)
					$

			\end{enumerate}

	\item{} Factorisation.
			If the clause $C$ is the result of a factorisation step $\inference$ of $C_1: l \lor l' \lor D$ using a unifier $\sigma$ such that $l\sigma = l'\sigma$, then $\LIpre(C) \defeq \lifboth{\LI(C_1)\tau}$.

		\item{} Paramodulation.
			If the clause $C$ is the result of a paramodulation step $\inference$ of $C_1 : s = t \lor D$ and $C_2: E\occ{r}$ with $\sigma = \mgu(\inference)$.
			Let $h\occ{r}$ be the maximal colored term in which $r$ occurs in $E\occ{r}$
			and define $\LI(C)$ as follows:

			\begin{enumerate}

				\item If $h\occur{r}$ is $\Delta$-colored and $h\occur{r}$ occurs more than once in $\LI(C_2)\tau \lor E\occ{r}\tau$:
					%\label{def:PI_paramod_1}
					\newline
					$\LIpre(C) \defeq  ( s=t \land \LI(C_2) )\tau \lor (s\neq t \land \LI(C_1))\tau \lor (s=t \land h\occur{s} \neq      h\occur{t})\tau$

				\item If $h\occur{r}$ is $\Gamma$-colored and $h\occur{r}$ occurs more than once in $\LI(C_2)\tau \lor E\occ{r}\tau$:
					%\label{def:PI_paramod_2}
					\newline
					$\LIpre(C)\defeq{} [ ( s=t \land \LI(C_2) )\tau \lor (s\neq t \land \LI(C_1))\tau ] \land\allowbreak (s\neq t \lor h\occur{s} = h\occur{t})\tau$

				\item If $r$ does not occur in a colored term in $E\occur{r}$ which occurs more than once in $\LI(C_2)\tau \lor E\occ{r}\tau$:
					%\label{def:PI_paramod_3}
					\newline
					$\LIpre(C) \defeq{} ( s=t \land \LI(C_2) )\tau \lor (s\neq t \land \LI(C_1))\tau $ 

			\end{enumerate}
	\end{description}

	\noindent
	$\LI(C)$ is built from $\LIpre(C)$ according to the following steps:

	\begin{enumerate}
		\item Lift all maximal colored terms $t$ in $\LIpre(C)$
			for at least which one of the following conditions applies:

			\begin{itemize} 
				\item The term $t$ contains some variable $x$ such that $x$ does not occur in $C$.
				\item The term $t$ is ground and $C$ does not contain $t$.
			\end{itemize} 
		\item Let $X$ ($Y$) be the $\Delta$-($\Gamma$-)lifting variables created in the previous step.
		\item Prefix the resulting formula with an arrangement $\Q(C)$ of the elements of $\{\forall x_t \mid x_t \in X\}\cup\allowbreak\{\exists y_t \mid y_t \in Y\}$ such that if $s$ and $r$ are terms such that $s$ is a subterm of $r$, then $z_s$ precedes $z_r$.
			\qedhere
	\end{enumerate}
\end{defi}

%\section{Properties of $\LI$ and $\LIcl$}

\begin{lemma}
	\label{lemma:substitute_and_lift}
	Let $\sigma$ be a substitution and $F$ a formula without $\Phi$-colored terms such that for a set of formulas $\Psi$ which does not contain $\Phi$-lifting variables, $\Psi \entails F$.
	Then $\Psi \entails \lifphinovar{F\sigma}$.
\end{lemma}
\begin{proof}
	$\lifphinovar{F\sigma}$ is an instance of $F$:
	$\sigma$ substitutes variables either for terms which do not contain $\Phi$-colored symbols or by terms containing $\Phi$-colored symbols.
	For the first kind, the lifting has no effect.
	For the latter, the lifting only replaces subterms of the terms introduced by the substitution by a lifting variable such that the original structure of $F$ remains invariant as it by assumption does not contain colored terms.
\end{proof}


\begin{lemma}
	If a variable $x$ occurs in $\LI(C)$ but not in $C$, then $x$ does not occur in the subsequent resolution derivation.
\end{lemma}


\begin{lemma}
	If a ground term $t$ occurs in $\LI(C)$ but not in $C$, and a successor of $C$ contains $t$, then ``the second $t$ has been derived from first principles in a different way''.
\end{lemma}

\begin{lemma}
	\label{lemma:gamma_entails_delta_lifted_invariant}
	Let $C$ be a clause in a resolution refutation of $\Gamma \cup \Delta$.
	Then
	$\Gamma \entails \lifdeltanovar{ \LI(C) } \lor \lifdeltanovar{C} $
\end{lemma}
\begin{proof}
	We show the strengthening
	$\Gamma \entails \lifdeltanovar{ \LI(C) } \lor \lifdeltanovar{C_\Gamma}$\footnote{Recall that $D_\Phi$ denotes the clause created from the clause $D$ by removing all literals which are not contained $\Lang(\Phi)$.}.

	As a first step, 
	we prove by induction that
	$\Gamma \entails \lifdeltanovar{ \LIpre(C) } \lor \lifdeltanovar{C_\Gamma}$.
	%First, we proceed by induction to prove that $\Gamma \entails \lifdeltanovar{ \LIpre(C) } \lor \lifdeltanovar{C_\Gamma} $.


	\begin{description}
		\item{} Base case.
			If $C\in \Gamma$, then $\lifdeltanovar{C} = C$ and clearly $\Gamma \entails C$.
			If otherwise $C \in \Delta$, then $\LI(C) = \top$.

		\item{} Resolution.
			Suppose the clause $C$ is the result of a resolution step \inference{} of $C_1: D \lor l$ and $C_2: E \lor \lnot l'$ with $\sigma = \mgu(\inference)$.

			By the induction hypothesis we obtain that
			$\Gamma \entails \lifdeltanovar{\LI(C_1)} \lor \lifdeltanovar{D_\Gamma} \lor \lifdeltanovar{l_\Gamma}$
			as well as 
			$\Gamma \entails \lifdeltanovar{\LI(C_2)} \lor \lifdeltanovar{E_\Gamma} \lor \lnot \lifdeltanovar{l'_\Gamma}$.
			Hence by Lemma~\ref{lemma:substitute_and_lift} and Lemma~\ref{lemma:lifting_tau_commute}, we get:

			$\Gamma \stackrel\markA\entails \lifdeltanovar{\LI(C_1)\tau} \lor \lifdeltanovar{D_\Gamma\tau} \lor \lifdeltanovar{l_\Gamma\tau}$

			$\Gamma \stackrel\markB\entails \lifdeltanovar{\LI(C_2)\tau} \lor \lifdeltanovar{E_\Gamma\tau} \lor \lnot \lifdeltanovar{l'_\Gamma\tau}$

			As $l_\Gamma\sigma = l'_\Gamma\sigma$,
			it holds that $l_\Gamma\tau = l'_\Gamma\tau$ and consequently 
			$\lifdeltanovar{l_\Gamma\tau} = \lifdeltanovar{l'_\Gamma\tau}$.
			We proceed by a case distinction on the color of the resolved literal to show that in each case, we have that
			$\Gamma \entails \lifdeltanovar{\LIpre(C)} \lor \lifdeltanovar{C_\Gamma}$:
			\begin{itemize}
				\item Suppose that $l$ is $\Gamma$-colored.
					Then $l_\Gamma = l$ and $l'_\Gamma = l$, and we can perform a resolution step on \markA{} and \markB{} to obtain that
					$\Gamma \entails
					\lifdeltanovar{\LI(C_1)\tau} \lor
					\lifdeltanovar{\LI(C_2)\tau} \lor 
					\lifdeltanovar{D_\Gamma\tau}  \lor
					\lifdeltanovar{E_\Gamma\tau}$.
					This however is nothing else than $\Gamma \entails \lifdeltanovar{\LIpre(C)} \lor \lifdeltanovar{C_\Gamma}$.

				\item Suppose that $l$ is $\Delta$-colored.
					Then \markA{} and \markB{} reduce to 
					$\Gamma \entails \lifdeltanovar{\LI(C_1)\tau} \lor \lifdeltanovar{D_\Gamma\tau}$
					and
					$\Gamma \entails \lifdeltanovar{\LI(C_2)\tau} \lor \lifdeltanovar{E_\Gamma\tau}$
					respectively,
					which clearly implies that 
					$\Gamma \entails \lifdeltanovar{\LI(C_1)\tau} \lor \lifdeltanovar{\LI(C_2)\tau} \lor (\lifdeltanovar{D_\Gamma\tau} \land \lifdeltanovar{E_\Gamma\tau})$.
					This is turn is however just the unfolding of
					$\Gamma \entails \lifdeltanovar{\LIpre(C)} \lor \lifdeltanovar{C_\Gamma}$.

				\item Suppose that $l$ is grey.
					Then $l_\Gamma = l$ and $l'_\Gamma = l$, and 
					\markA{} and \markB{} imply that
					$\Gamma \entails
					\lifdeltanovar{\LI(C_1)\tau} \spam\lor
					\lifdeltanovar{\LI(C_2)\tau} \spam\lor\allowbreak
					(\lifdeltanovar{l_\Gamma\tau} \land \lifdeltanovar{E_\Gamma\tau}) \spam\lor\allowbreak
					(\lnot\lifdeltanovar{l'_\Gamma\tau} \land \lifdeltanovar{D_\Gamma\tau})$.
					This however is equivalent to
					$\Gamma \entails \lifdeltanovar{\LIpre(C)} \lor \lifdeltanovar{C_\Gamma}$.

			\end{itemize}



			%			Suppose that $s$ contains a variable which does not occur in $C$. Then $x_s$ is quantified in $\LI(C)$ as well.
			%			As however the quantifier block of $\LI(C)$ is ordered with respect to the subterm relation of the index of the lifting variables, $x_s$ is quantified prior to $y_t$.

		\item{} Factorisation. 
			Suppose the clause $C$ is the result of a factorisation inference $\inference$ of $C_1: l \lor l' \lor D$ with $\sigma=\mgu(\inference)$.
			
			The induction hypothesis gives $\Gamma \entails \lifdeltanovar{ \LI(C_1) } \lor \lifdeltanovar{ l_\Gamma \lor l'_\Gamma \lor D_\Gamma }$.
			By Lemma~\ref{lemma:substitute_and_lift}, we obtain
			$\Gamma \entails \lifdeltanovar{ \LI(C_1)\tau } \lor \lifdeltanovar{ l_\Gamma\tau \lor l'_\Gamma\tau \lor D_\Gamma\tau }$.
			As however $l\sigma \equiv l'\sigma$, 
			also $l\tau \equiv l'\tau$, so we can apply a factorisation step and obtain that
			$\Gamma \entails \lifdeltanovar{ \LI(C_1)\tau } \lor \lifdeltanovar{ l_\Gamma\tau \lor D_\Gamma\tau }$,
			which is nothing else than $\Gamma \entails \LIpre(C) \lor \lifdeltanovar{C_\Gamma}$.

		\item{} Paramodulation.
			Suppose the clause $C$ is the result of a paramodulation inference\nolinebreak{} $\inference$ of $C_1: s=t \lor D$ and $C_2: E\occatp{r}$ with $\sigma=\mgu(\inference)$.

			By the induction hypothesis, we obtain the following: 

			$\Gamma \stackrel\markA\entails \lifdeltanovar{\LI(C_1)} \lor \lifdeltanovar{D_\Gamma} \lor \lifdeltanovar{s}=\lifdeltanovar{t}$

			$\Gamma \stackrel\markB\entails \lifdeltanovar{\LI(C_2)} \lor \lifdeltanovar{(E\occatp{r})_\Gamma}$

			Suppose now that for a model $M$ of $\Gamma$ and an assignment $\alpha$ of the free variables of $\lifdeltanovar{s}$ and $\lifdeltanovar{t}$ that $M_\alpha \entails \lifdeltanovar{s}\neq \lifdeltanovar{t}$.
			Then we get by \markA{} that $M_\alpha \entails \lifdeltanovar{\LI(C_1)} \lor \lifdeltanovar{D_\Gamma}$, which by Lemma~\ref{lemma:substitute_and_lift} and Lemma~\ref{lemma:lifting_tau_commute} gives $M_\alpha \entails \lifdeltanovar{\LI(C_1)\tau} \lor \lifdeltanovar{D_\Gamma\tau}$.
			Note that $M_\alpha \entails \lifdeltanovar{s\tau}\neq \lifdeltanovar{t\tau} \land \lifdeltanovar{\LI(C_1)\tau}$ suffices for $M_\alpha \entails \lifdeltanovar{\LIpre(C)}$ and $M_\alpha \entails \lifdeltanovar{D_\Gamma\tau}$ implies that $M_\alpha \entails \lifdeltanovar{C_\Gamma}$.
			Therefore we obtain that 
			$M_\alpha \entails \lifdeltanovar{\LIpre(C)} \lor \lifdeltanovar{C_\Gamma}$.

			Now suppose to the contrary that for a model $M$ of $\Gamma$ that for any assignment of free variables $M \entails \lifdeltanovar{s}\nolinebreak=\lifdeltanovar{t}$.

			By Lemma~\ref{lemma:substitute_and_lift} and  Lemma~\ref{lemma:lifting_tau_commute} we obtain from \markB{} that
			$\Gamma \entails \lifdeltanovar{\LI(C_2)\tau} \lor \lifdeltanovar{(E\occatp{r})_\Gamma\tau}$.
			As however $\sigma = \mgu(r, s)$, $r\tau \equiv s\tau$
			and hence $\lifdeltanovar{r\tau} \equiv \lifdeltanovar{s\tau}$.
			Therefore we also have that 
			$\Gamma \entails \lifdeltanovar{\LI(C_2)\tau} \lor \lifdeltanovar{(E\occatp{s})_\Gamma\tau}$.

			We proceed by a case distinction:
			\begin{itemize}
				\item Suppose that the position $p$ in $E\occatp{s}$ is not contained in a $\Delta$-term.
					Then
					$\lifdeltanovar{(E\occatp{s})_\Gamma\tau}$
					and
					$\lifdeltanovar{(E\occatp{t})_\Gamma\tau}$
					only differ at at position $p$.
					As $M \entails \lifdeltanovar{s} = \lifdeltanovar{t}$, we can apply Lemma~\ref{lemma:substitute_and_lift} and Lemma~\ref{lemma:lifting_tau_commute} to obtain that 
					$M \entails \lifdeltanovar{s\tau} = \lifdeltanovar{t\tau}$.
					Thus
					$M\entails \lifdeltanovar{(E\occatp{s})_\Gamma\tau} \semiff 
					\lifdeltanovar{(E\occatp{t})_\Gamma\tau}$
					and consequently
					$M \entails \lifdeltanovar{\LI(C_2)\tau} \lor \lifdeltanovar{(E\occatp{t})_\Gamma\tau}$.
					As furthermore $\lifdeltanovar{\tau s} =\lifdeltanovar{t} \land \lifdeltanovar{\LI(C_2)\tau}$ entails $\lifdeltanovar{\LIpre(C)}$
					and $\lifdeltanovar{(E\occatp{t})_\Gamma\tau}$ is sufficient for $\lifdeltanovar{C_\Gamma}$,
					we have that 
					$M\entails \lifdeltanovar{\LIpre(C)} \lor \lifdeltanovar{C_\Gamma}$.

				\item
					Suppose that the position $p$ in $E\occatp{s}$ is contained in a maximal $\Delta$-term $h\occ{s}$.
					We distinguish further:

					\begin{itemize}
						\item Suppose $h\occ{s}$ occurs more than once in $\LI(C_2)\tau \lor E\occatp{s}\tau$ and let $\alpha$ be an arbitrary assignment to the variables $\lifdeltanovar{h\occ{s}} = x_{h\occ{s}}$ and $\lifdeltanovar{h\occ{t}} = x_{h\occ{t}}$.

							If $M_\alpha \entails \lifdeltanovar{h\occ{s}} \neq \lifdeltanovar{h\occ{t}}$, then we have that $M_\alpha \entails \lifdeltanovar{s}\nolinebreak=\lifdeltanovar{t} \land\allowbreak \lifdeltanovar{h\occ{s}} \neq \lifdeltanovar{h\occ{t}}$, which implies that $M_\alpha \entails \lifdeltanovar{\LIpre(C)}$.

							%Assume that $M \entails \lifdeltanovar{h\occ{s}} = \lifdeltanovar{h\occ{t}}$ as otherwise $M \entails \lifdeltanovar{s}\nolinebreak=\lifdeltanovar{t} \land\allowbreak \lifdeltanovar{h\occ{s}} \neq \lifdeltanovar{h\occ{t}}$, which implies that $M \entails \lifdeltanovar{\LIpre(C)}$.
							Otherwise it holds that $M_\alpha \entails \lifdeltanovar{h\occ{s}} = \lifdeltanovar{h\occ{t}}$.
							But then 
							$\lifdeltanovar{(E\occatp{s})_\Gamma\tau}$
							and
							$\lifdeltanovar{(E\occatp{t})_\Gamma\tau}$
							differ in subterms which are equal in the given model and with the given interpretation of the free variables,
							%at position of $h\occ{s}$ and $h\occ{t}$ respectively and
							so by a similar line of argument as in the preceding case, we can deduce that $M \entails \lifdeltanovar{\LIpre(C)} \lor \lifdeltanovar{C}$.

						\item Suppose $h\occ{s}$ occurs exactly once in $\LI(C_2)\tau \lor E\occatp{s}\tau$.
							Then the lifting variable $x_{h\occ{s}}$ 
							occurs exactly once in $\lifdeltanovar{\LI(C_2)\tau} \lor \lifdeltanovar{E\occatp{s}\tau}$.

							Note that from \markB{} by applying Lemma~\ref{lemma:substitute_and_lift} as well as Lemma~\ref{lemma:lifting_tau_commute}, we obtain that $M\entails \lifdeltanovar{\LI(C_2)\tau} \lor \lifdeltanovar{(E\occatp{s})_\Gamma\tau}$.
							As $x_{h\occ{s}}$ occurs only once and free in this formula, it is implicitly universally quantified and we can instantiate it arbitrarily, in particular by $z_{h\occ{t}}$.
							But thereby we get that 
							$M\entails \lifdeltanovar{\LI(C_2)\tau} \lor \lifdeltanovar{(E\occatp{t})_\Gamma\tau}$, 
							which implies that
							$\Gamma\entails \lifdeltanovar{\LIpre(C)} \lor \lifdeltanovar{C_\Gamma}$.
					\end{itemize}
			\end{itemize}
	\end{description}


	%
	%\begin{lemma}
	%	\label{lemma:gamma_entails_delta_lifted_invariant}
	%	Let $C$ be a clause in a resolution refutation of $\Gamma \cup \Delta$.
	%	Then
	%	$\Gamma \entails \lifdeltanovar{ \LI(C) } \lor \lifdeltanovar{C} $
	%\end{lemma}
	%\begin{proof}
	%	By Lemma~\ref{lemma:gamma_entails_delta_lifted_preinvariant}, we have that 
	%	$\Gamma \entails \lifdeltanovar{\LIpre(C)} \lor \lifdeltanovar{C_\Gamma}$.

	As we have now established that
	$\Gamma \entails \lifdeltanovar{ \LIpre(C) } \lor \lifdeltanovar{C_\Gamma}$,
	we show that also
	$\Gamma \entails \lifdeltanovar{ \LI(C) } \lor \lifdeltanovar{C_\Gamma}$ holds.


	The difference between $\lifdeltanovar{\LIpre(C)}$ and $\lifdeltanovar{\LI(C)}$ lies only in certain maximal colored terms which are lifted in $\lifdeltanovar{\LI(C)}$, hence it suffices to consider these.
	Let $t$ be a colored term in $\LIpre(C)$ at position $p$ such that $\LI(C)\atp = \lifboth{t}$.
	Then $t$ is a maximal colored term. % and contains a variable which does not occur in\nolinebreak{} $C$.

	If $t$ is $\Delta$-colored, then $\lifdeltanovar{\LIpre(C)}\atp = \LI(C)\atp = x_t$.
	Note that as $t$ occurs at $p$ in $\LIpre(C)$, $x_t$ occurs free at $\lifdeltanovar{\LIpre(C)}\atp$.
	Hence it is implicitly universally quantified and therefore entails that an explicit universal quantification in $\LI(C)$ is valid with an arbitrarily placed universal quantifier.  

	If otherwise $t$ is a $\Gamma$-term, then $\lifdeltanovar{\LIpre(C)}\atp = \lifdeltanovar{t}$.
	Hence $\lifdeltanovar{t}$ represents a witness term for the existentially quantified lifting variable $y_t$ at $\LI(C)\atp$.
	In general, $\lifdeltanovar{t}$ however contains $\Delta$-lifting variables, which require being lifted in the scope of the existential quantifier of $y_t$. 

	Let $x_s$ be a $\Delta$-lifting variable which occurs in $\lifdeltanovar{t}$. 
	It is essential to see that neither $s$ nor a predecessor of $s$ in the resolution derivation is lifted in a previous step of the interpolant extraction.
	Suppose to the contrary that this is the case in the inference creating the clause $C'$.
	Let $s'$ and $t'$ be the respective predecessors of $s$ and $t$ in $C'$.

	\begin{itemize}
		\item Suppose that $t$ is lifted due to containing a variable which does not occur in\nolinebreak{} $C$.
			Then as $s'$ is a subterm of $t'$, $t'$ contains this variable as well and would therefore be lifted in $\LI(C')$.

		\item Suppose that $t$ is lifted due to being a ground term which does not occur in\nolinebreak{} $C$.
			Then $s'$ does not occur in $C'$. But as $t'$ contains $s'$, $t'$ also does not occur in $C'$ and would therefore be lifted in $\LI(C')$.
	\end{itemize}

	\largered{check other paramod proofs}

	Note the the lifting conditions ensure that only terms are lifted which do not influence the subsequent resolution derivation: 
	If a variable $x$ occurs in $\LI(C)$ but not in $C$, then as clauses are variable-disjoint, the variable $x$ does not occur in any other clause.
	For ground terms $r$ however which occur in $\LI(C)$ but not in $C$,
	it is possible for them to cooccur in a subsequent clause. Let $p$ be the occurrence of $r$ in $\LI(C)$ and $q$ the occurrence of $r$ in a successive clause of $C$.
	Then due to the fact that $p$ is not used in any unification, 
	$q$ must have been created or originated from other occurrences of the same function and/or constant symbols.
	Note that the lifting conditions ensure that for these, the order of the quantifiers of their respective lifting variables is established in a fashion appropriate to ensuring the logical validity of the interpolant. 



	\noindent
	\rule{\textwidth}{0.5px}



	~

	~


	{

		\begin{itemize}
			\item Suppose that $y_t$ is lifted due to containing a variable which does not occur in\nolinebreak{} $C$.

				Then one of the following two contradictions eventuate:
				\begin{itemize}
					\item Suppose that $s'$ is a subterm of $t'$.
						Then due to the fact that $s'$ is lifted, $s'$ must contain a variable which does not occur in $C'$. But as $t'$ contains $s'$, $t'$ contains this variable as well and would be lifted at this stage already.

					\item Otherwise $t'$ does not contain $s'$.
						We have already established that $s'$ contains a variable which does not occur in $C'$.
						As all clauses are variable-disjoint, no other clause contains this variable.
						But then it does not occur in any subsequent unifier, and in particular, it never enters $t'$ by means of substitution, which implies that $s'$ due to containing this variable does not become a subterm of a successor of $t'$.
				\end{itemize}

			\item Suppose that $y_t$ is lifted due to being a ground term which does not occur in\nolinebreak{} $C$.

				\begin{itemize}
					\item Suppose that $s'$ is a subterm of $t'$.
						Then as $s'$ occurs in $t'$, $s'$ is not lifted.

					\item Otherwise $t'$ does not contain $s'$.
						Then if $s'$ is lifted, $s'$ is not contained in any term in $C$ or $\LI(C)$.
						Hence $s'$ is not a predecessor of 



				\end{itemize}

				\largered{what is lifted? all occs of some term? all maximal $\Phi$-colored occs?}
				\largered{candidate for answer: effectively, we lift \textbf{maximal colored terms}, conceptually, every occurrence of these terms are lifted. with this version, we argue for correctness, and we should argue, that that's what we get}

				Supp $s'$ subterm of $t'$:
				then $s'$ is not lifted as it occurs in $t'$ which occurs in $\LI(C')$ or in $C$ (well, exactly where $s'$ occurs too)

				Supp $s'$ not subterm of $t'$.
				IDEA: then $s'$ can be lifted, but if some $t'$ contains $s'$, then this is a different occurrence of $s'$ with a different ``history''. 

		\end{itemize}

	}

	Hence there are three possibilities for quantification of $x_s$. We show that in each of them, $y_t$ is quantified in the scope of the quantifier of $x_s$. 
	\begin{enumerate}
		\item Neither $s$ nor a successor of $s$ in the derivation occurs at a grey position. Then $x_s$ is not quantified in the course of the interpolant extraction.
		\item A variable which does not occur in $C$ enters $s$ by means of the current substitution $\sigma$ or a variable is contained in $s$ such that the only occurrences of it in $C_1$ and $C_2$ are in $l$ and $l'$.
			Then $x_s$ is lifted in the current step and as $s$ is a subterm of $t$, $x_s$ is quantified in $\Q(C)$ prior to $y_t$.
		\item The lifting variable $x_s$ or a respective successor is quantified at a later stage in the derivation.
			Then as the quantifier for $y_t$ is contained in $\LI(C)$ and for any successor $C'$ of $C$, $\LI(C')$ contains a successor $\LI(C)$, $y_t$ is quantified in the scope of the quantifier for $x_s$.
			\qedhere
	\end{enumerate}
\end{proof}




\begin{lemma}
	\label{lemma:li_symmetry}
	Let $\pi$ be a refutation of $\Gamma\cup\Delta$ and $\bhat \pi$ be $\pi$ with $\bhat \Gamma = \Delta$ and $\bhat \Delta = \Gamma$.
	Then for a clause $C$ in $\pi$ and its corresponding clause $\bhat C$ in $\bhat \pi$, $\LI(C) \spas\semiff \lnot \LI(\bhat C)$.
\end{lemma}
\begin{proof}
	We proceed by induction to show that $\LIpre(C) \semiff \lnot \LIpre(\bhat C)$:
	\begin{itemize}
		\item[Base case.]
			If $C \in \Gamma$, then $\LI(C) = \bot \semiff \lnot \top \semiff \lnot \LI(\bhat C)$ as $\bhat C\in\Delta$. 
			The case for $C\in\Delta$ can be argued analogously.

		\item[Resolution.]
			Suppose the clause $C$ is the result of a resolution step \inference{} of $C_1: D \lor l$ and $C_2: E \lor \lnot l'$ with $\sigma = \mgu(\inference)$.

			As $\tau$ depends only on $\sigma$,
			$\tau$ is the same for both $\pi$ and $\bhat \pi$.

			We now distinguish the following cases:

			\begin{enumerate}

				\item $l$ is $\Gamma$-colored:
					\begin{align*}
						\hspace*{\dimexpr-\leftmargini-\leftmarginii}
						\LIpre(C)	&= \LI(C_1)\tau\spas\lor {\LI(C_2)\tau} \\
															 &\semiff \lnot (\lnot{\LI(C_1)\tau}\spas\land \lnot {\LI(C_2)\tau}) \\
						%&\semiff \lnot (\lifboth{(\lnot \LI(C_1))\tau}\spas\land \lifboth{(\lnot \LI(C_2))\tau}) \\
															 &\semiff \lnot ({\LI(\bhat C_1)\tau}\spas\land {\LI(\bhat C_2)\tau}) \\
															 &= \lnot \LIpre(\bhat C)
					\end{align*}

				\item $l$ is $\Delta$-colored:
					This case can be argued analogously.

				\item $l$ is grey:
					Note ${l\tau} \stackrel{{\markB}}= {l'\tau}$.
					\begin{align*}
						\hspace*{\dimexpr-\leftmargini-\leftmarginii}
						\LIpre(C) &\stackrel{{\phantom{\markB}}}=
						(\lnot {{l'\tau}} \land {\LI(C_1)\tau}) \spam\lor 
						({l\tau} \land {\LI(C_2)\tau})\\
						&\stackrel{{\markB}}\semiff\,
						({{l'\tau}} \lor {\LI(C_1)\tau}) \spam\land 
						(\lnot {l\tau} \lor {\LI(C_2)\tau})\\
						&\stackrel{{\phantom{\markB}}}\semiff \lnot [ (\lnot {{l'\tau}} \land \lnot {\LI(C_1)\tau}) \spam\lor 
					({l\tau} \land \lnot{\LI(C_2)\tau}) ] \\
					&\stackrel{{\phantom{\markB}}}=\lnot [ (\lnot {{\bhat l'\tau}} \land {\LI(\bhat C_1)\tau}) \spam\lor 
				({\bhat l\tau} \land {\LI(\bhat C_2)\tau}) ]\\
				& \stackrel{{\phantom{\markB}}}= \lnot \LIpre(\bhat C) 
			\end{align*}
	\end{enumerate}


\item[Factorisation.]
	Suppose the clause $C$ is the result of a factorisation $\inference$ of $C_1: l \lor l' \lor D$ 
	with $\sigma = \mgu(\inference)$.

	As the construction is not influenced by the coloring, the induction hypothesis $\LIpre(C) = \LI(C_1)\tau$ suffices.

	Then $\LIpre(C) = \lifboth{\LI(C_1)\tau}$, so the construction is not influenced by the coloring and by the induction hypothesis, $\LIpre(C) \semiff \lnot \LIpre(\bhat C)$.

\item[Paramodulation.]
	Suppose the clause $C$ is the result of a paramodulation inference $\inference$ of $C_1: s=t \lor D$ and $C_2: E\occatp{r}$ with $\sigma=\mgu(\inference)$.


	We proceed by a case distinction:
	\begin{itemize}
		\item Suppose that $p$ in $E\occatp{r}$ is contained in a maximal $\Delta$-term $h\occ{r}$, which occurs more than once in $E\occatp{r} \lor \LI(E\occatp{r})$. 
			Then $p$ in $\bhat E\occatp{r}$ is contained in a maximal $\Gamma$-term $h\occ{r}$, which occurs more than once in $\bhat E\occatp{r} \lor \LI(\bhat E\occatp{r})$. 

			\hspace*{\dimexpr-\leftmargini-\leftmarginii}\parbox{\linewidth}{%
				%\hspace*{\dimexpr-\leftmargini-\leftmarginii}\parbox{\linewidth}{%
				\begin{align*}
					%\hspace*{\dimexpr-\leftmargini-\leftmarginii}
					\LIpre(C) 
					&= ({s\tau=t\tau} \land {\LI(C_2)\tau} ) \lor ({s\tau\neq t\tau} \land {\LI(C_1)\tau}) \lor ({s\tau=t\tau} \land {h\occur{s}\tau \neq h\occur{t}\tau}) \\
					&\semiff \lnot [ ({s\tau\neq t\tau} \lor \lnot {\LI(C_2)\tau} ) \land ({s\tau= t\tau} \lor \lnot{\LI(C_1)\tau}) \land ({s\tau\neq t\tau} \lor {h\occur{s}\tau = h\occur{t}\tau}) ] \\
					&= \lnot [ ({s\tau\neq t\tau} \lor {\LI(\bhat C_2)\tau} ) \land ({s\tau= t\tau} \lor {\LI(\bhat C_1)\tau}) \land ({s\tau\neq t\tau} \lor {h\occur{s}\tau = h\occur{t}\tau}) ] \\
					&\semiff \lnot [ ({s\tau= t\tau} \land {\LI(\bhat C_2)\tau} ) \lor ({s\tau\neq t\tau} \land {\LI(\bhat C_1)\tau}) \land ({s\tau\neq t\tau} \lor {h\occur{s}\tau = h\occur{t}\tau}) ]\\
					&= \lnot \LIpre(\bhat C)
				\end{align*}
			}\par

		\item Suppose that $p$ in $E\occatp{r}$ is contained in a maximal $\Gamma$-term $h\occ{r}$, which occurs more than once in $E\occatp{r} \lor \LI(E\occatp{r})$. 
			This case can be argued analogously.

		\item Otherwise:
			%\hspace*{\dimexpr-\leftmargini-\leftmarginii-\leftmarginiii}\parbox{\linewidth}{%
			%\hspace*{\dimexpr-\leftmargini-\leftmarginii}\parbox{\linewidth}{%
			\begin{align*}
				%\hspace*{\dimexpr-\leftmargini-\leftmarginii}
				\LIpre(C) 
				&= ({s\tau=t\tau} \land {\LI(C_2)\tau} ) \spam\lor ({s\tau\neq t\tau} \land {\LI(C_1)\tau}) \\
				&\semiff \lnot [ ({s\tau\neq t\tau} \lor \lnot {\LI(C_2)\tau} ) \spam\land ({s\tau= t\tau} \lor \lnot {\LI(C_1)\tau}) ] \\
				&= \lnot [ ({s\tau\neq t\tau} \lor {\LI(\bhat C_2)\tau} ) \spam\land ({s\tau= t\tau} \lor {\LI(\bhat C_1)\tau}) ] \\
				&\semiff \lnot [ ({s\tau= t\tau} \land {\LI(\bhat C_2)\tau} ) \spam\lor ({s\tau\neq t\tau} \land {\LI(\bhat C_1)\tau}) ] \\
				&= \lnot \LIpre(\bhat C)
			\end{align*}
			%}\par

	\end{itemize}

	\end{itemize}


	We conclude by showing that 
	$\LIpre(C) \semiff \lnot \LIpre(\bhat C)$ 
	entails that 
	$\LI(C) \semiff \lnot \LI(\bhat C)$:
	Clearly the terms to be lifted in $\LIpre(C)$ and $\LIpre(\bhat C)$ are the same and differ only in their color.
	Even though this results in different lifting variables, that is of no relevance as all lifted variables are instantly bound.
	Additionally, the quantifier type of any given lifting variable in $\Q(C)$ is dual to the respective one in $\Q(\bhat C)$.
	Furthermore note that the subterm-relation is not affected by the coloring, so the ordering of the quantifiers in $\Q(C)$ and $\Q(\bhat C)$ is identical.
	Hence 
	$\LI(C) \semiff \lnot \LI(\bhat C)$.
\end{proof}


\begin{lemma}
	\label{lemma:delta_entails_li}
	Let $C$ be a clause in a resolution refutation of $\Gamma \cup \Delta$.
	Then
	$\Delta \entails \lnot\lifgammanovar{\LI(C)} \lor \lifgammanovar{C}$.
\end{lemma}
\begin{proof}
	Construct $\bhat \pi$ with $\bhat \Gamma = \Delta$ and $\bhat \Delta = \Gamma$. 
	Then by Lemma~\ref{lemma:gamma_entails_delta_lifted_invariant}, $\bhat \Gamma \entails \liftnovar{\bhat \Delta}{\LI(\bhat C)} \lor \liftnovar{\bhat \Delta}{\bhat C}$, 
	which by Lemma~\ref{lemma:li_symmetry} is nothing else than
	$\Delta \entails \lnot \liftnovar{\Gamma}{\LI(C)} \lor \liftnovar{\Gamma}{C}$.
\end{proof}

\begin{thm}
	Let $\pi$ be a resolution refutation of $\Gamma \cup \Delta$.
	Then $\LI(\pi)$ is an interpolant of $\Gamma$ and $\Delta$.
\end{thm}
\begin{proof}
	We obtain by Lemma~\ref{lemma:gamma_entails_delta_lifted_invariant} that  $\Gamma \entails \liftnovar{\Delta}{\LI(\pi)}$ and
	by Lemma~\ref{lemma:delta_entails_li} that
	$\Delta \entails \lnot\lifgammanovar{\LI(\pi)}$.
	As the empty clause derived in $\pi$ trivially contains neither variables nor ground terms, 
	all colored terms are lifted in $\LI(\pi)$.
	Therefore $\lifdeltanovar{\LI(\pi)} = \LI(\pi)$ and $\lifgammanovar{\LI(\pi)} = \LI(\pi)$.
\end{proof}


\end{document}
