\documentclass[%fontsize=11pt,%
  %landscape,
  %DIV8, % mehr text pro seite als defaultyyp
  %DIV10,
  %DIV=calc,%
  %article,
  %draft=false,% final|draft % draft ist platzsparender (kein code, bilder..)
	%final,%
  %titlepage,
  a4paper,%
  numbers=noendperiod,%
  11pt,%
	twoside,%
  %oneside,% apparently, this should stay below some other parameter to have an effect
	%one,%
	%openright,%
  %openany,%
  %]{scrartcl}
]{memoir}

% default for memoir is:
%\documentclass[letterpaper,10pt,twoside,onecolumn,openright,final]{memoir}



%\documentclass[%
%	11pt,%
%	a4paper,%
%	oneside,%
%	openany%
%	%twoside, openany % no blank page after chapter such that chapters start on odd pages
%]{memoir}

%\settrimmedsize{0.9\stockheight}{0.9\stockwidth}{*}


% ftp://ftp.tex.ac.uk/tex-archive/documentation/MemoirChapStyles/MemoirChapStyles.pdf
%\chapterstyle{veelo}

% VZ221 chapter style
%\usepackage{calc}% used to import fourier
%\makeatletter
%\setlength\midchapskip{7pt}
%\makechapterstyle{VZ21}{
%\renewcommand\chapnamefont{\Large\scshape}
%\renewcommand\chapnumfont{\Large\scshape\centering}
%\renewcommand\chaptitlefont{\huge\bfseries\centering}
%\renewcommand\printchaptertitle[1]{%
%\setlength\tabcolsep{7pt}% used as indentation on both sides
%\settowidth\@tempdimc{\chaptitlefont ##1}%
%\setlength\@tempdimc{\textwidth-\@tempdimc-2\tabcolsep}%
%\chaptitlefont
%\ifdim\@tempdimc > 0pt\relax% one line
%\begin{tabular}{c}
%\toprule ##1\\ \bottomrule
%\end{tabular}
%\else% two+ lines
%\begin{tabular}{%
%>{\chaptitlefont\arraybackslash}p{\textwidth-2\tabcolsep}}
%\toprule ##1\\ \bottomrule
%\end{tabular}
%\fi
%}
%}
%\makeatother
%\chapterstyle{VZ21}


%\chapterstyle{ger}
\chapterstyle{madsen}
%\chapterstyle{bianchi}


\usepackage[utf8]{inputenc}
\usepackage[T1]{fontenc}
\usepackage[english,ngerman]{babel}


\usepackage{comment} 

\usepackage{etoolbox} % fixes fatal error caused by combining bm, stackengine, hyperref (seriously?)
% http://tex.stackexchange.com/questions/22995/package-incompatibilites-etoolbox-hyperref-and-bm-standalone

\usepackage{etex} % else error on too many packages

% includes
\usepackage{algorithm}
%\usepackage{algorithmic} % conflicts with algpseudocode
\usepackage{algpseudocode}
%\newcommand*\Let[2]{\State #1 $\gets$ #2}
\algrenewcommand\alglinenumber[1]{
{\scriptsize #1}}
\algrenewcommand{\algorithmicrequire}{\textbf{Input:}}
\algrenewcommand{\algorithmicensure}{\textbf{Output:}}


%\usepackage[multiple]{footmisc} % footnotes at the same character separated by ','

\usepackage{multicol}

\usepackage{afterpage}

\usepackage{changepage} % for adjustwidth
\usepackage{caption} % for \ContinuedFloat

\usepackage{tikz}
\usetikzlibrary{shapes,arrows,backgrounds,graphs,%
matrix,patterns,arrows,decorations.pathmorphing,decorations.pathreplacing,%
positioning,fit,calc,decorations.text,shadows%
}

\usepackage{bussproofs}
\EnableBpAbbreviations


\usepackage{amsmath}
\usepackage{amsthm}
\usepackage{amssymb} % the reals
\usepackage{mathtools} % smashoperator

\usepackage{bm} % bm, bold math symbols

\usepackage{thm-restate} % restatable env

% needs extra work and fails on some label here
%\usepackage{cleveref} % cref, apparently better than autoref of hyperref 

\usepackage{nicefrac} % nicefrac

\usepackage{mathrsfs} % mathscr

\usepackage{pst-node} % http://tex.stackexchange.com/questions/35717/how-to-draw-arrows-between-parts-of-an-equation-to-show-the-math-distributive-pr

\usepackage{stackengine}

\usepackage{thmtools} % advanced thm commands (declaretheorem)


\usepackage{nameref} % reference name of thm instead of counter

\usepackage{todonotes}

% conflict with beamer
%\usepackage{paralist} % compactenum

\usepackage{hyperref}
%\hypersetup{hidelinks}  % don't give options to usepackage, it doesn't work with beamer
%\hypersetup{colorlinks=false}  % don't give options to usepackage, it doesn't work with beamer


% \usepackage{enumitem} % labels for enumerate % breaks beamer and memoir itemize


\usepackage{url} 


\usepackage[format=hang,justification=raggedright]{caption}% or e.g. [format=hang]

\usepackage{cancel} % \cancel

\usepackage{lineno}


% commands

% logic etcs
%\newcommand{\ex}[2]{\bigskip\section*{Exercise #1: \begin{minipage}[t]{.80\linewidth} \small \textnormal{\it #2} \end{minipage} } }

\newcommand{\ex}[2]{\bigskip \noindent\textbf{Exercise #1.} \textit{#2} \smallskip}

\newcommand{\comm}[1]{{\color{gray} // #1 }}


\newcommand{\true}[0]{\textbf{1}}
\newcommand{\false}[0]{\textbf{0}}
\newcommand{\tr}{\true}
\newcommand{\fa}{\false}

\newcommand{\ra}{\rightarrow}
\newcommand{\Ra}{\Rightarrow}
\newcommand{\la}{\leftarrow}
\newcommand{\La}{\Leftarrow}

\newcommand{\lra}{\leftrightarrow}
\newcommand{\Lra}{\Leftrightarrow}

\newcommand{\NKZ}{\textbf{NK2}}

%\DeclareMathOperator{\syneq}{\equiv} %spacing seems wrong, therefore defined as newcommand below
\DeclareMathOperator{\limpl}{\supset}
\DeclareMathOperator{\liff}{\lra}
\DeclareMathOperator{\semiff}{\Lra}
\newcommand{\syneq}{\equiv}
\newcommand{\union}{\cup}
\newcommand{\bigunion}{\bigcup}
\newcommand{\intersection}{\cap}
\newcommand{\bigintersection}{\bigcap}
\newcommand{\intersect}{\intersection}
\newcommand{\bigintersect}{\bigintersection}

\newcommand{\powerset}{\mathcal{P}}

\newcommand{\entails}{\vDash}
\newcommand{\notentails}{\nvDash}
\newcommand{\proves}{\vdash}

\newcommand{\vm}{\ensuremath{\vv_\mathcal{M}}}
\newcommand{\Dia}{\ensuremath{\lozenge}}

\newcommand{\spaced}[1]{\ \ #1 \ \ }
\newcommand{\spa}[1]{\spaced{#1}}
\newcommand{\spas}[1]{\;{#1}\;}
\newcommand{\spam}[1]{\;\,{#1}\;\,}

% functions
\DeclareMathOperator{\sk}{sk}
\DeclareMathOperator{\mgu}{mgu}
\DeclareMathOperator{\dom}{dom}
\DeclareMathOperator{\ran}{ran}

\DeclareMathOperator{\id}{id}
\DeclareMathOperator{\Fun}{FS}
\DeclareMathOperator{\Pred}{PS}
\DeclareMathOperator{\Lang}{L}
\DeclareMathOperator{\ar}{ar}
\DeclareMathOperator{\PI}{PI}
\DeclareMathOperator{\LI}{LI}
\DeclareMathOperator{\Congr}{Congr}
\DeclareMathOperator{\Refl}{Refl}
\DeclareMathOperator{\aiu}{au}
\DeclareMathOperator{\expa}{unfold-lift}

\newcommand{\PIinc}{\LI}
\newcommand{\PIincde}{\LIde}

\newcommand{\LIde}{\ensuremath{\LI^\Delta}}

\newcommand{\LIcl}{\ensuremath{\LI_{\operatorname{cl}}}}
\newcommand{\LIclde}{\ensuremath{\LI_{\operatorname{cl}}^\Delta}}

\newcommand{\cll}{\ensuremath{_{\operatorname{LIcl}}}}
\newcommand{\cllde}{\ensuremath{_{\operatorname{LIcl}^\Delta}}}

%\newcommand{\lifi}{\mathop{\ell\text{}i}}
\newcommand{\lifiboth}[1]{\ensuremath{\LIcl(#1)}}
\newcommand{\lifidelta}[1]{\ensuremath{\LIclde(#1)}}


%\DeclareMathOperator{\abstraction}{abstraction}

%\newcommand{\sk}{\ensuremath{\mathrm{sk}}}
%\newcommand{\mgu}{\ensuremath{\mathrm{mgu}}}
%\newcommand{\Fun}{\ensuremath{\mathrm{FS}}}
%\newcommand{\Pred}{\ensuremath{\mathrm{PS}}}
%\newcommand{\PI}{\ensuremath{\mathrm{PI}}}
%\newcommand{\Lang}{\ensuremath{\mathrm{L}}}
%\newcommand{\ar}{\ensuremath{\mathrm{ar}}}

\DeclareMathOperator{\AI}{AI}
\newcommand{\AIde}{\ensuremath{\AI^\Delta}}
\newcommand{\AImatrix}{\ensuremath{\AI_\mathrm{mat}}}
\newcommand{\AImatrixde}{\ensuremath{\AI_\mathrm{mat}^\Delta}}
\newcommand{\AImat}{\AImatrix}
\newcommand{\AImatde}{\AImatrixde}
\newcommand{\AIclause}{\ensuremath{\AI_\mathrm{cl}}}
\newcommand{\AIcl}{\AIclause}
\newcommand{\AIclde}{\AIclausede}
\newcommand{\AIclausede}{\ensuremath{\AIclause^\Delta}}
\newcommand{\fromclause}{\ensuremath{_{\operatorname{AIcl}}}}
\newcommand{\fromclausede}{\ensuremath{_{\operatorname{AIcl}^\Delta}}}
\newcommand{\cl}{\fromclause}
\newcommand{\clde}{\fromclausede}

\newcommand{\Q}{\ensuremath{Q}}

\newcommand{\AIcol}{\ensuremath{\AI_\mathrm{col}}}
\newcommand{\AIcolde}{\AIcol^\Delta}

\newcommand{\AIany}{\ensuremath{\AI_\mathrm{*}}}
\newcommand{\AIanyde}{\AIany^\Delta}

\newcommand{\AIclpre}{\AIclause^\bullet}
\newcommand{\AImatpre}{\AImatrix^\bullet}

\newcommand{\PS}{\Pred}
\newcommand{\FS}{\Fun}

\DeclareMathOperator{\LangSym}{\mathcal{L}}

%\newcommand{\mguarr}{\sim_\ra}
\newcommand{\mguarr}{\mapsto_{\mgu}}


%\newcommand{\Trans}{\ensuremath{\mathrm{T}}}
%\newcommand{\Trans}{\ensuremath{\mathrm{T}}}
\DeclareMathOperator{\Trans}{T}
\DeclareMathOperator{\TransInv}{T^{-1}}

\DeclareMathOperator{\FAX}{F_{Ax}}
\DeclareMathOperator{\EAX}{E_{Ax}}
%\newcommand{\FAX}{\ensuremath{\mathrm{F_{Ax}}}}
%\newcommand{\EAX}{\ensuremath{\mathrm{E_{Ax}}}}

%\newcommand{\TransAll}{\ensuremath{\Trans_{\mathrm{Ax}}}}
\DeclareMathOperator{\TransAll}{\Trans_{Ax}}
%\newcommand{\FAX}{\ensuremath{\mathrm{F_{Ax}}}}

\DeclareMathOperator{\defeq}{\stackrel{\mathrm{def}}{=}}

\newcommand{\subst}[1]{[#1]}
\newcommand{\abstractionOp}[1]{\{#1\}}

\newcommand{\subformdefinitional}[1]{\ensuremath{D_{\Sigma(#1)}}}


%\newcommand{\lift}[3]{\operatorname{Lift}_{#1}(#2; #3)}
%\newcommand{\lift}[3]{\operatorname{Lift}_{#1,#3}(#2)}
%\newcommand{\lift}[3]{\operatorname{Lift}_{#1,#3}[#2]}
%\newcommand{\lift}[3]{\overline{#2}_{#1,#3}}
\newcommand{\lifsym}{\ell}
%\newcommand{\lift}[3]{\lifsym_{#1,#3}[#2]}
\newcommand{\lift}[3]{\lifsym_{#1}^{#3}[#2]}
\newcommand{\liftnovar}[2]{\lifsym_{#1}[#2]}

%\newcommand{\lft}[3]{\lifsym_{#1,#2}[#3]}
\newcommand{\lft}[3]{\lift{#1}{#3}{#2}}
\newcommand{\lifboth}[1]{\lifsym[#1]}

%\newcommand{\lifi}{\mathop{\ell\text{}i}}
%\newcommand{\lifiboth}[1]{\lifi[#1]}
%\newcommand{\lifidelta}[1]{\lifi_\Delta^x[#1]}
%\newcommand{\lifideltanovar}[1]{\lifi_\Delta[#1]}

\newcommand{\lifdelta}[1]{\lift{\Delta}{#1}{x}}
\newcommand{\lifdeltanovar}[1]{\liftnovar{\Delta}{#1}}
\newcommand{\lifgamma}[1]{\lift{\Gamma}{#1}{y}}
\newcommand{\lifgammanovar}[1]{\liftnovar{\Gamma}{#1}}
\newcommand{\lifphinovar}[1]{\liftnovar{\Phi}{#1}}
\newcommand{\lifphi}[1]{\lift{\Phi}{#1}{z}}

\DeclareMathOperator{\arr}{\mathcal{A}}
%\DeclareMathOperator{\arrFinal}{{\mathcal{A}^{\bm*}}}
\DeclareMathOperator{\arrFinal}{{\mathcal{\bm{\hat}A}}}
\DeclareMathOperator{\warr}{\marr}
\DeclareMathOperator{\marr}{\mathcal{M}}

\DeclareMathOperator{\apath}{\leadsto}
\DeclareMathOperator{\mpath}{\leadsto_=}
\DeclareMathOperator{\notapath}{\not\leadsto}
\DeclareMathOperator{\notmpath}{\not\leadsto_=}

\newcommand{\ltArrC}{<_{\arrFinal(C)}}
\newcommand{\ltAC}{<_{\arr(C)}}
\newcommand{\ltArrCOne}{<_{\arrFinal(C_1)}}
\newcommand{\ltArrCTwo}{<_{\arrFinal(C_2)}}
%\newcommand{\ltArrC}{<_{\scalebox{0.6}{$\arrFinal(C)$}}}
\newcommand{\ltArr}{<_{\scalebox{0.6}{$\arrFinal$}}}

\newcommand{\bhat}{\bm\hat}
\newcommand{\bbar}{\bm\bar}
\newcommand{\bdot}{\bm\dot}

%\usepackage{yfonts}
\usepackage{upgreek}
\DeclareMathAlphabet{\mathpzc}{OT1}{pzc}{m}{it}
%\DeclareMathOperator{\pos}{\mathscr{P}}
%\DeclareMathOperator{\pos}{\mathpzc{p}}
%\DeclareMathOperator{\pos}{{\rho}}
\DeclareMathOperator{\pos}{{\operatorname P}}
%\DeclareMathOperator{\pos}{P}
\DeclareMathOperator{\poslit}{\pos_\mathrm{lit}}
\DeclareMathOperator{\posterm}{\pos_\mathrm{term}}
%\newcommand{\poslit}[1]{\ensuremath{p_\text{lit}(#1)}}
%\newcommand{\posterm}[1]{\ensuremath{p_\text{term}(#1)}}
\newcommand{\at}[1]{|_{#1}}

\newcommand{\UICm}[1]{\UnaryInfCm{#1}}
\newcommand{\UnaryInfCm}[1]{\UnaryInfC{$#1$}}
\newcommand{\BICm}[1]{\BinaryInfCm{#1}}
\newcommand{\BinaryInfCm}[1]{\BinaryInfC{$#1$}}
\newcommand{\RightLabelm}[1]{\RightLabel{$#1$}}
\newcommand{\LeftLabelm}[1]{\LeftLabel{$#1$}}
\newcommand{\AXCm}[1]{\AxiomCm{#1}}
\newcommand{\AxiomCm}[1]{\AxiomC{$#1$}}
\newcommand{\mt}[1]{\textnormal{#1}}

\newcommand{\UnaryInfm}[1]{\UnaryInf$#1$}
\newcommand{\BinaryInfm}[1]{\BinaryInf$#1$}
\newcommand{\Axiomm}[1]{\Axiom$#1$}



% math
\newcommand{\calI}{\ensuremath{\mathcal{I}}}

\newcommand{\tupleShort}[2]{\ensuremath{(#1_1,\dotsc,#1_{#2})}}
\newcommand{\tuple}[2]{\ensuremath{(#1_1,\:#1_2\:,\dotsc,\:#1_{#2})}}
\newcommand{\setelements}[2]{\ensuremath{\{#1_1,\:#1_2\:,\dotsc,\:#1_{#2}\}}}
\newcommand{\pathelements}[2]{\ensuremath{ (#1_1,\:#1_2\:,\dotsc,\:#1_{#2}) }}

\newcommand{\elems}[1]{\ensuremath{#1_1,\dotsc, #1_{n}) }}

\newcommand{\defiemph}[1]{\emph{#1}}

\newcommand{\setofbases}{\ensuremath{\mathcal{B}}}
\newcommand{\setofcircuits}{\ensuremath{\mathcal{C}}}

\newcommand{\reals}{\ensuremath{\mathbb{R}}}
\newcommand{\integers}{\ensuremath{\mathbb{Z}}} 
\newcommand{\naturalnumbers}{\ensuremath{\mathbb{N}}}

% general
\newcommand{\zit}[3]{#1\ \cite{#2}, #3}
\newcommand{\zitx}[2]{#1\ \cite{#2}}
\newcommand{\footzit}[3]{\footnote{\zit{#1}{#2}{#3}}}
\newcommand{\footzitx}[2]{\footnote{\zitx{#1}{#2}}}

\newcommand{\ite}{\begin{itemize}}
\newcommand{\ete}{\end{itemize}}

\newcommand{\bfr}{\begin{frame}}
\newcommand{\efr}{\end{frame}}

\newcommand{\ilc}[1]{\texttt{#1}}


% misc

% multiframe
\usepackage{xifthen}% provides \isempty test
% new counter to now which frame it is within the sequence
\newcounter{multiframecounter}
% initialize buffer for previously used frame title
\gdef\lastframetitle{\textit{undefined}}
% new environment for a multi-frame
\newenvironment{multiframe}[1][]{%
\ifthenelse{\isempty{#1}}{%
% if no frame title was set via optional parameter,
% only increase sequence counter by 1
\addtocounter{multiframecounter}{1}%
}{%
% new frame title has been provided, thus
% reset sequence counter to 1 and buffer frame title for later use
\setcounter{multiframecounter}{1}%
\gdef\lastframetitle{#1}%
}%
% start conventional frame environment and
% automatically set frame title followed by sequence counter
\begin{frame}%
\frametitle{\lastframetitle~{\normalfont \Roman{multiframecounter}}}%
}{%
\end{frame}%
}




% http://texfragen.de/hurenkinder_und_schusterjungen
\usepackage[all]{nowidow}



% force no overlong lines:
%\tolerance=1 % tolerance for how badly spaced lines are allowed, less means "better" lines
\tolerance=500 %  need more tolerance for equations
%\emergencystretch=\maxdimen
%\emergencystretch=200pt
%\setlength{\emergencystretch}{3em}
%\hyphenpenalty=10000 % forces no hyphenation
%\hbadness=10000


% http://tex.stackexchange.com/questions/35717/how-to-draw-arrows-between-parts-of-an-equation-to-show-the-math-distributive-pr
\tikzset{square arrow/.style={to path={ -- ++(.0,-.15)  -| (\tikztotarget)}}}
\tikzset{square arrow2/.style={to path={ -- ++(.0,-.25)  -| (\tikztotarget)}}}
%\tikzset{square arrow/.style={to path={ -- ++(00,-.01) -- ++(0.5,-0.1) -- ++(0.5,-0.1) -| (\tikztotarget)},color=red}}


% have arrows from a to b and from c to d here
% just use: texttext\arrowA texttest \arrowB texttext
\newcommand{\arrowA}{\tikz[overlay,remember picture] \node (a) {};}
\newcommand{\arrowB}{\tikz[overlay,remember picture] \node (b) {};}
\newcommand{\drawAB}{
	\tikz[overlay,remember picture]
	{\draw[->,bend left=5,color=red] (a.south) to (b.south);}
	%{\draw[->,square arrow,color=red] (a.south) to (b.south);}
}
\newcommand{\arrowAP}{\tikz[overlay,remember picture] \node (ap) {};}
\newcommand{\arrowBP}{\tikz[overlay,remember picture] \node (bp) {};}
\newcommand{\drawABP}{
	\tikz[overlay,remember picture]
	{\draw[->,bend right=5,color=red] (ap.south) to (bp.south);}
	%{\draw[->,square arrow,color=red] (a.south) to (b.south);}
}

\newcommand{\arrowAB}{\tikz[overlay,remember picture] \node (ab) {};}
\newcommand{\arrowBA}{\tikz[overlay,remember picture] \node (ba) {};}
\newcommand{\drawAABB}{
	\tikz[overlay,remember picture]
	%{\draw[->,bend left=80] (a.north) to (b.north);}
	{\draw[->,square arrow,color=brown] (ab.south) to (ba.south);
	\draw[->,square arrow,color=brown] (ba.south) to (ab.south);}
}


\newcommand{\arrowCD}{\tikz[overlay,remember picture] \node (cd) {};}
\newcommand{\arrowDC}{\tikz[overlay,remember picture] \node (dc) {};}
\newcommand{\drawCCDD}{
	\tikz[overlay,remember picture]
	%{\draw[->,bend left=80] (a.north) to (b.north);}
	{\draw[<->,dashed,square arrow,color=green] (cd.south) to (dc.south); }
}



\newcommand{\arrowC}{\tikz[overlay,remember picture] \node (c) {};}
\newcommand{\arrowD}{\tikz[overlay,remember picture] \node (d) {};}
\newcommand{\drawCD}{
	\tikz[overlay,remember picture]
	{\draw[->,square arrow,color=blue] (c.south) to (d.south);}
}

\newcommand{\arrowE}{\tikz[overlay,remember picture] \node (e) {};}
\newcommand{\arrowF}{\tikz[overlay,remember picture] \node (f) {};}
\newcommand{\drawEF}{
	\tikz[overlay,remember picture]
	{\draw[->,square arrow2,color=orange] (e.south) to (f.south);}
}


\newcommand{\arrAP}{\arrowAP}
\newcommand{\arrBP}{\arrowBP}
\newcommand{\arrA}{\arrowA}
\newcommand{\arrB}{\arrowB}
\newcommand{\arrC}{\arrowC}
\newcommand{\arrD}{\arrowD}
\newcommand{\arrE}{\arrowE}
\newcommand{\arrF}{\arrowF}


\DeclareMathOperator{\simgeq}{\scalebox{0.92}{$\gtrsim$}}

\newcommand{\refsub}[2]{\hyperref[#2]{\ref*{#1}.\ref*{#2}}}

%\newcommand{\sigmarange}[2]{\sigma_{#1}^{#2} }
\newcommand{\sigmarange}[2]{\sigma_{(#1,#2)} }
\newcommand{\sigmaz}[1]{\sigmarange{0}{#1} }
\newcommand{\sigmazi}[0]{\sigmaz{i} }

\DeclareMathOperator{\lit}{lit}

%\def\fCenter{\ \proves\ }
\def\fCenter{\proves}

\newcommand{\prflbl}[2]{\RightLabel{\footnotesize $#1, #2$} }
%\newcommand{\prflblid}[1]{\RightLabel{$#1, \id$} }
\newcommand{\prflblid}[1]{\RightLabel{\footnotesize $#1$} }

\DeclareMathOperator{\resruleres}{res}
\DeclareMathOperator{\resrulefac}{fac}
\DeclareMathOperator{\resrulepar}{par}
\newcommand{\lkrule}[2]{\ensuremath{\operatorname{#1}:#2}} % operatorname fixes spacing issues for =

\newcommand{\parti}[4]{\ensuremath{ \langle (#1; #2), (#3; #4)\rangle  }}

\newcommand{\partisym}{\ensuremath{\chi}}

\newcommand{\occur}[1]{\ensuremath{[#1]}}
\newcommand{\occ}[1]{\occur{#1}}

\newcommand{\occurat}[2]{\ensuremath{{\occur{#1}_{#2}}}}
\newcommand{\occat}[2]{\occurat{#1}{#2}}
\newcommand{\occatp}[1]{\occurat{#1}{p}}
\newcommand{\occatq}[1]{\occurat{#1}{q}}

\newcommand{\colterm}[1]{\zeta_{#1}}



% fix restateable spacing 
%http://tex.stackexchange.com/questions/111639/extra-spacing-around-restatable-theorems

\makeatletter

\def\thmt@rst@storecounters#1{%
%THIS IS THE LINE I ADDED:
\vspace{-1.9ex}%
  \bgroup
        % ugly hack: save chapter,..subsection numbers
        % for equation numbers.
  %\refstepcounter{thmt@dummyctr}% why is this here?
  %% temporarily disabled, broke autorefname.
  \def\@currentlabel{}%
  \@for\thmt@ctr:=\thmt@innercounters\do{%
    \thmt@sanitizethe{\thmt@ctr}%
    \protected@edef\@currentlabel{%
      \@currentlabel
      \protect\def\@xa\protect\csname the\thmt@ctr\endcsname{%
        \csname the\thmt@ctr\endcsname}%
      \ifcsname theH\thmt@ctr\endcsname
        \protect\def\@xa\protect\csname theH\thmt@ctr\endcsname{%
          (restate \protect\theHthmt@dummyctr)\csname theH\thmt@ctr\endcsname}%
      \fi
      \protect\setcounter{\thmt@ctr}{\number\csname c@\thmt@ctr\endcsname}%
    }%
  }%
  \label{thmt@@#1@data}%
  \egroup
}%

\makeatother




\newcommand{\mymark}[1]{\ensuremath{(#1)}}
\newcommand{\markA}{\mymark \circ}
\newcommand{\markB}{\mymark *}
\newcommand{\markC}{\mymark \divideontimes}

\newcommand{\wrong}[1]{{\color{red}WRONG: #1}}
\newcommand{\NB}[1]{{\color{blue}NB: #1}}
\newcommand{\hl}[1]{{\color{orange} #1}}
\newcommand{\mytodo}[1]{{\color{red}TODO: #1}}
\newcommand{\largered}[1]{{

	  \LARGE\bfseries\color{red}
		#1

}}
\newcommand{\largeblue}[1]{{

	  \large\bfseries\color{blue}
		#1

}}




\usepackage{ulem} %  \dotuline{dotty} \dashuline{dashing} \sout{strikethrough}
\normalem

\usepackage{tabu} % tabular also in math mode (and much more)

\usepackage[color]{changebar} %  \cbstart, \cbend
\cbcolor{red}



% http://tex.stackexchange.com/questions/7032/good-way-to-make-textcircled-numbers
\newcommand*\circled[1]{\tikz[baseline=(char.base)]{
\node[shape=circle,draw,inner sep=2pt] (char) {#1};}}



% http://tex.stackexchange.com/questions/43346/how-do-i-get-sub-numbering-for-theorems-theorem-1-a-theorem-1-b-theorem-2

\makeatletter
\newenvironment{subtheorem}[1]{%
  \def\subtheoremcounter{#1}%
  \refstepcounter{#1}%
  \protected@edef\theparentnumber{\csname the#1\endcsname}%
  \setcounter{parentnumber}{\value{#1}}%
  \setcounter{#1}{0}%
  \expandafter\def\csname the#1\endcsname{\theparentnumber.\Alph{#1}}%
  \ignorespaces
}{%
  \setcounter{\subtheoremcounter}{\value{parentnumber}}%
  \ignorespacesafterend
}
\makeatother
\newcounter{parentnumber}


\usepackage{tabularx}% http://ctan.org/pkg/tabularx
\newcolumntype{Y}{>{\centering\arraybackslash}X}

\newcommand{\mycols}[2][3]{
	\noindent\begin{tabularx}{\textwidth}{*{#1}{Y}}
		#2
	\end{tabularx}%
}


\newcommand{\definethms}{

	%\declaretheorem[title=Theorem,qed=$\triangle$,parent=chapter]{thm}
	\newcommand{\thmqed}{$\square$} % for thms without proof
	\newcommand{\propqed}{$\square$} % for props without proof
	\declaretheorem[title=Theorem]{thm}
	\declaretheorem[title=Proposition,sibling=thm]{prop}
	\declaretheorem[title=Conjectured Proposition,sibling=thm]{cprop}

	%\declaretheorem[title=Lemma,parent=chapter]{lemma}
	\declaretheorem[sibling=thm]{lemma}
	\declaretheorem[sibling=thm,title=Conjectured Lemma]{clemma}
	\declaretheorem[title=Corollary,sibling=thm]{corr}
	\declaretheorem[sibling=thm,title=Definition,style=definition,qed=$\triangle$]{defi}
	%\declaretheorem[title=Definition,qed=$\triangle$,parent=chapter]{defi}
	\declaretheorem[title=Example,style=definition,qed=$\triangle$,sibling=thm]{exa}

	\declaretheorem[sibling=thm,title=Conjecture]{conj}

	\declaretheorem[title=Remark,style=remark,numbered=no,qed=$\triangle$]{remark}


}

\usepackage[matha]{mathabx} % the locial operators here have more space around them and [ and ] are thicker, also langle and rangle are a bit nicer; subseteq looks a bit weird

%\usepackage{MnSymbol} % again other symbols


\newcommand{\inference}{\ensuremath{\iota}}

\usepackage{cases} % numcases


\usepackage{amssymb}

% in conflict with beamer:
\usepackage{paralist} % compactenum
%\usepackage{enumitem} % labels for enumerate % CAN CHANGE ITEMIZE POSITION 

\usepackage[authoryear]{natbib} % \cite ; square|round etc.
%\usepackage[numbers,square]{natbib}
%\usepackage[square, authoryear]{natbib}
%\usepackage[language=english]{biblatex}


%\bibliographystyle{plain}
\bibliographystyle{alpha}
%\bibliographystyle{alphadin}
%\bibliographystyle{dinat}
%\bibliographystyle{chicago}
%\bibliographystyle{plainnat}

\bibdata{bib.bib}

% smaller url style
\makeatletter
\def\url@leostyle{%
\@ifundefined{selectfont}{\def\UrlFont{\sf}}{\def\UrlFont{\small\ttfamily}}}
\makeatother
\urlstyle{leo}

%\definethms
% thesis has other numbering:

%\declaretheorem[title=Theorem,qed=$\triangle$,parent=chapter]{thm}
\newcommand{\thmqed}{$\square$} % for thms without proof
\newcommand{\propqed}{$\square$}
\newcommand{\lemmaqed}{$\square$}
\declaretheorem[title=Theorem,parent=chapter]{thm}
\declaretheorem[title=Proposition,sibling=thm]{prop}
%\declaretheorem[title=Lemma,parent=chapter]{lemma}
\declaretheorem[sibling=thm]{lemma}
\declaretheorem[title=Corollary,sibling=thm]{corr}
\declaretheorem[sibling=thm,title=Definition,style=definition,qed=$\triangle$]{defi}
%\declaretheorem[title=Definition,qed=$\triangle$,parent=chapter]{defi}
\declaretheorem[title=Example,style=definition,qed=$\triangle$,sibling=thm]{exa}
\declaretheorem[sibling=thm,title=Conjecture]{conj}

%\theoremstyle{remark}
%\newtheorem*{remark}{Remark}
\declaretheorem[title=Remark,style=remark,numbered=no,qed=$\triangle$]{remark}

% section numbers for subsections:
%\setsecnumdepth{subsection}
\setsecnumdepth{subsubsection}

% subsections also in toc
%\setcounter{tocdepth}{2}
\setcounter{tocdepth}{3}

%\def\proofSkipAmount{ \vskip -0.5em}
%\def\proofSkipAmount{ \vskip -0.3em}
\def\proofSkipAmount{ \vskip -0.0em } % NOTE: proofSkipAmount is by default something positive, i.e. having a skip of 0 here makes it less already


\usepackage{TUINFDA}
\thesistitle{Interpolation in First-Order Logic with Equality}
%\thesissubtitle{Optional Subtitle} % optional
\thesisdate{TT.MM.JJJJ}

% all titles and designations have to be gender-related!
\thesisdegree{Diplom-Ingenieur}{Diplom-Ingenieur}
\thesiscurriculum{Computational Intelligence}{Computational Intelligence} % your  study
\thesisverfassung{Verfasser} % Verfasser
\thesisauthor{Bernhard Mallinger} % your name
\thesisauthoraddress{Gassergasse 25/17-18, 1050 Wien} % your address
\thesismatrikelno{0707663} % your registration number

\thesisbetreins{Ass.Prof.~Stefan Hetzl}
%\thesisbetrzwei{Dr. Vorname Familienname}
%\thesisbetrdrei{Dr. Vorname Familienname} % optional



\usepackage{url}
\usepackage{hyperref}					% links in pdf
\usepackage{graphicx}            			% Figures
\usepackage{verbatim}            			% Code-Environment
%\usepackage[lined,linesnumbered,algochapter]{algorithm2e} % Algorithm-Environment

\usepackage{pgf}					
\usepackage{tikz}					% tikz graphics
\usetikzlibrary{arrows,automata}

%\usepackage{bibgerm,cite}       % Deutsche Bezeichnungen, Automatisches Zusammenfassen von Literaturstellen

%\usepackage[ngerman]{varioref}  % Querverweise
% to use the german charset include cp850 for MS-DOS, ansinew for Windows and latin1 for Linux.
% \usepackage[latin1]{inputenc}



% define page numbering styles
\makepagestyle{numberCorner}
\makeevenfoot{numberCorner}{\thepage}{}{}
\makeoddfoot{numberCorner}{}{}{\thepage}

\makepagestyle{somePageStyle}
%\makeevenfoot{somePageStyle}{}{}{}
%\makeoddfoot{somePageStyle}{}{}{}
%
%\makeevenhead{somePageStyle}{\thechapter}{}{\thesection}
%\makeoddhead{somePageStyle}{\thesection }{}{\thechapter}
\makeheadrule{somePageStyle}{\textwidth}{1pt}

\makeevenhead{somePageStyle}{\thepage}{}{\leftmark}
\makeoddhead{somePageStyle}{\rightmark}{}{\thepage}


\makepagestyle{emptyPageStyle}
\aliaspagestyle{chapter}{emptyPageStyle}


\makeatletter
\makepsmarks{somePageStyle}{
  \def\chaptermark##1{\markboth{%
        \ifnum \value{secnumdepth} > -1
          \if@mainmatter
            \chaptername\ \thechapter\ --- %
          \fi
        \fi
        ##1}{}}
  \def\sectionmark##1{\markright{%
        \ifnum \value{secnumdepth} > 0
          \thesection. \ %
        \fi
        ##1}}
}
\makeatother

\newcommand{\mysetpagestyle}{ 
	%\pagestyle{numberCorner}
	\pagestyle{somePageStyle} 
}
\mysetpagestyle


\newcommand{\proofcontent}{

	\documentclass[,%fontsize=11pt,%
	paper=a4,% 
	%landscape,
	%DIV12, % mehr text pro seite als defaultyyp
	DIV14, 
	%DIV=calc,%
	%twoside=false,%
	liststotoc,
	bibtotoc,
	draft=false,% final|draft % draft ist platzsparender (kein code, bilder..)
	%titlepage,
	numbers=noendperiod
]{scrartcl}

\usepackage{lscape}
\usepackage{stackengine}


\usepackage[utf8]{inputenc}
\usepackage[T1]{fontenc}
\usepackage[english]{babel}

\usepackage{enumerate}
\usepackage{paralist}
\usepackage{tikz}
\usetikzlibrary{shapes,arrows,backgrounds,graphs,%
	matrix,patterns,arrows,decorations.pathmorphing,decorations.pathreplacing,%
	positioning,fit,calc,decorations.text,shadows%
}


\usepackage{comment} 

\usepackage{etoolbox} % fixes fatal error caused by combining bm, stackengine, hyperref (seriously?)
% http://tex.stackexchange.com/questions/22995/package-incompatibilites-etoolbox-hyperref-and-bm-standalone

\usepackage{etex} % else error on too many packages

% includes
\usepackage{algorithm}
%\usepackage{algorithmic} % conflicts with algpseudocode
\usepackage{algpseudocode}
%\newcommand*\Let[2]{\State #1 $\gets$ #2}
\algrenewcommand\alglinenumber[1]{
{\scriptsize #1}}
\algrenewcommand{\algorithmicrequire}{\textbf{Input:}}
\algrenewcommand{\algorithmicensure}{\textbf{Output:}}


%\usepackage[multiple]{footmisc} % footnotes at the same character separated by ','

\usepackage{multicol}

\usepackage{afterpage}

\usepackage{changepage} % for adjustwidth
\usepackage{caption} % for \ContinuedFloat

\usepackage{tikz}
\usetikzlibrary{shapes,arrows,backgrounds,graphs,%
matrix,patterns,arrows,decorations.pathmorphing,decorations.pathreplacing,%
positioning,fit,calc,decorations.text,shadows%
}

\usepackage{bussproofs}
\EnableBpAbbreviations


\usepackage{amsmath}
\usepackage{amsthm}
\usepackage{amssymb} % the reals
\usepackage{mathtools} % smashoperator

\usepackage{bm} % bm, bold math symbols

\usepackage{thm-restate} % restatable env

% needs extra work and fails on some label here
%\usepackage{cleveref} % cref, apparently better than autoref of hyperref 

\usepackage{nicefrac} % nicefrac

\usepackage{mathrsfs} % mathscr

\usepackage{pst-node} % http://tex.stackexchange.com/questions/35717/how-to-draw-arrows-between-parts-of-an-equation-to-show-the-math-distributive-pr

\usepackage{stackengine}

\usepackage{thmtools} % advanced thm commands (declaretheorem)


\usepackage{nameref} % reference name of thm instead of counter

\usepackage{todonotes}

% conflict with beamer
%\usepackage{paralist} % compactenum

\usepackage{hyperref}
%\hypersetup{hidelinks}  % don't give options to usepackage, it doesn't work with beamer
%\hypersetup{colorlinks=false}  % don't give options to usepackage, it doesn't work with beamer


% \usepackage{enumitem} % labels for enumerate % breaks beamer and memoir itemize


\usepackage{url} 


\usepackage[format=hang,justification=raggedright]{caption}% or e.g. [format=hang]

\usepackage{cancel} % \cancel

\usepackage{lineno}


% commands

% logic etcs
%\newcommand{\ex}[2]{\bigskip\section*{Exercise #1: \begin{minipage}[t]{.80\linewidth} \small \textnormal{\it #2} \end{minipage} } }

\newcommand{\ex}[2]{\bigskip \noindent\textbf{Exercise #1.} \textit{#2} \smallskip}

\newcommand{\comm}[1]{{\color{gray} // #1 }}


\newcommand{\true}[0]{\textbf{1}}
\newcommand{\false}[0]{\textbf{0}}
\newcommand{\tr}{\true}
\newcommand{\fa}{\false}

\newcommand{\ra}{\rightarrow}
\newcommand{\Ra}{\Rightarrow}
\newcommand{\la}{\leftarrow}
\newcommand{\La}{\Leftarrow}

\newcommand{\lra}{\leftrightarrow}
\newcommand{\Lra}{\Leftrightarrow}

\newcommand{\NKZ}{\textbf{NK2}}

%\DeclareMathOperator{\syneq}{\equiv} %spacing seems wrong, therefore defined as newcommand below
\DeclareMathOperator{\limpl}{\supset}
\DeclareMathOperator{\liff}{\lra}
\DeclareMathOperator{\semiff}{\Lra}
\newcommand{\syneq}{\equiv}
\newcommand{\union}{\cup}
\newcommand{\bigunion}{\bigcup}
\newcommand{\intersection}{\cap}
\newcommand{\bigintersection}{\bigcap}
\newcommand{\intersect}{\intersection}
\newcommand{\bigintersect}{\bigintersection}

\newcommand{\powerset}{\mathcal{P}}

\newcommand{\entails}{\vDash}
\newcommand{\notentails}{\nvDash}
\newcommand{\proves}{\vdash}

\newcommand{\vm}{\ensuremath{\vv_\mathcal{M}}}
\newcommand{\Dia}{\ensuremath{\lozenge}}

\newcommand{\spaced}[1]{\ \ #1 \ \ }
\newcommand{\spa}[1]{\spaced{#1}}
\newcommand{\spas}[1]{\;{#1}\;}
\newcommand{\spam}[1]{\;\,{#1}\;\,}

% functions
\DeclareMathOperator{\sk}{sk}
\DeclareMathOperator{\mgu}{mgu}
\DeclareMathOperator{\dom}{dom}
\DeclareMathOperator{\ran}{ran}

\DeclareMathOperator{\id}{id}
\DeclareMathOperator{\Fun}{FS}
\DeclareMathOperator{\Pred}{PS}
\DeclareMathOperator{\Lang}{L}
\DeclareMathOperator{\ar}{ar}
\DeclareMathOperator{\PI}{PI}
\DeclareMathOperator{\LI}{LI}
\DeclareMathOperator{\Congr}{Congr}
\DeclareMathOperator{\Refl}{Refl}
\DeclareMathOperator{\aiu}{au}
\DeclareMathOperator{\expa}{unfold-lift}

\newcommand{\PIinc}{\LI}
\newcommand{\PIincde}{\LIde}

\newcommand{\LIde}{\ensuremath{\LI^\Delta}}

\newcommand{\LIcl}{\ensuremath{\LI_{\operatorname{cl}}}}
\newcommand{\LIclde}{\ensuremath{\LI_{\operatorname{cl}}^\Delta}}

\newcommand{\cll}{\ensuremath{_{\operatorname{LIcl}}}}
\newcommand{\cllde}{\ensuremath{_{\operatorname{LIcl}^\Delta}}}

%\newcommand{\lifi}{\mathop{\ell\text{}i}}
\newcommand{\lifiboth}[1]{\ensuremath{\LIcl(#1)}}
\newcommand{\lifidelta}[1]{\ensuremath{\LIclde(#1)}}


%\DeclareMathOperator{\abstraction}{abstraction}

%\newcommand{\sk}{\ensuremath{\mathrm{sk}}}
%\newcommand{\mgu}{\ensuremath{\mathrm{mgu}}}
%\newcommand{\Fun}{\ensuremath{\mathrm{FS}}}
%\newcommand{\Pred}{\ensuremath{\mathrm{PS}}}
%\newcommand{\PI}{\ensuremath{\mathrm{PI}}}
%\newcommand{\Lang}{\ensuremath{\mathrm{L}}}
%\newcommand{\ar}{\ensuremath{\mathrm{ar}}}

\DeclareMathOperator{\AI}{AI}
\newcommand{\AIde}{\ensuremath{\AI^\Delta}}
\newcommand{\AImatrix}{\ensuremath{\AI_\mathrm{mat}}}
\newcommand{\AImatrixde}{\ensuremath{\AI_\mathrm{mat}^\Delta}}
\newcommand{\AImat}{\AImatrix}
\newcommand{\AImatde}{\AImatrixde}
\newcommand{\AIclause}{\ensuremath{\AI_\mathrm{cl}}}
\newcommand{\AIcl}{\AIclause}
\newcommand{\AIclde}{\AIclausede}
\newcommand{\AIclausede}{\ensuremath{\AIclause^\Delta}}
\newcommand{\fromclause}{\ensuremath{_{\operatorname{AIcl}}}}
\newcommand{\fromclausede}{\ensuremath{_{\operatorname{AIcl}^\Delta}}}
\newcommand{\cl}{\fromclause}
\newcommand{\clde}{\fromclausede}

\newcommand{\Q}{\ensuremath{Q}}

\newcommand{\AIcol}{\ensuremath{\AI_\mathrm{col}}}
\newcommand{\AIcolde}{\AIcol^\Delta}

\newcommand{\AIany}{\ensuremath{\AI_\mathrm{*}}}
\newcommand{\AIanyde}{\AIany^\Delta}

\newcommand{\AIclpre}{\AIclause^\bullet}
\newcommand{\AImatpre}{\AImatrix^\bullet}

\newcommand{\PS}{\Pred}
\newcommand{\FS}{\Fun}

\DeclareMathOperator{\LangSym}{\mathcal{L}}

%\newcommand{\mguarr}{\sim_\ra}
\newcommand{\mguarr}{\mapsto_{\mgu}}


%\newcommand{\Trans}{\ensuremath{\mathrm{T}}}
%\newcommand{\Trans}{\ensuremath{\mathrm{T}}}
\DeclareMathOperator{\Trans}{T}
\DeclareMathOperator{\TransInv}{T^{-1}}

\DeclareMathOperator{\FAX}{F_{Ax}}
\DeclareMathOperator{\EAX}{E_{Ax}}
%\newcommand{\FAX}{\ensuremath{\mathrm{F_{Ax}}}}
%\newcommand{\EAX}{\ensuremath{\mathrm{E_{Ax}}}}

%\newcommand{\TransAll}{\ensuremath{\Trans_{\mathrm{Ax}}}}
\DeclareMathOperator{\TransAll}{\Trans_{Ax}}
%\newcommand{\FAX}{\ensuremath{\mathrm{F_{Ax}}}}

\DeclareMathOperator{\defeq}{\stackrel{\mathrm{def}}{=}}

\newcommand{\subst}[1]{[#1]}
\newcommand{\abstractionOp}[1]{\{#1\}}

\newcommand{\subformdefinitional}[1]{\ensuremath{D_{\Sigma(#1)}}}


%\newcommand{\lift}[3]{\operatorname{Lift}_{#1}(#2; #3)}
%\newcommand{\lift}[3]{\operatorname{Lift}_{#1,#3}(#2)}
%\newcommand{\lift}[3]{\operatorname{Lift}_{#1,#3}[#2]}
%\newcommand{\lift}[3]{\overline{#2}_{#1,#3}}
\newcommand{\lifsym}{\ell}
%\newcommand{\lift}[3]{\lifsym_{#1,#3}[#2]}
\newcommand{\lift}[3]{\lifsym_{#1}^{#3}[#2]}
\newcommand{\liftnovar}[2]{\lifsym_{#1}[#2]}

%\newcommand{\lft}[3]{\lifsym_{#1,#2}[#3]}
\newcommand{\lft}[3]{\lift{#1}{#3}{#2}}
\newcommand{\lifboth}[1]{\lifsym[#1]}

%\newcommand{\lifi}{\mathop{\ell\text{}i}}
%\newcommand{\lifiboth}[1]{\lifi[#1]}
%\newcommand{\lifidelta}[1]{\lifi_\Delta^x[#1]}
%\newcommand{\lifideltanovar}[1]{\lifi_\Delta[#1]}

\newcommand{\lifdelta}[1]{\lift{\Delta}{#1}{x}}
\newcommand{\lifdeltanovar}[1]{\liftnovar{\Delta}{#1}}
\newcommand{\lifgamma}[1]{\lift{\Gamma}{#1}{y}}
\newcommand{\lifgammanovar}[1]{\liftnovar{\Gamma}{#1}}
\newcommand{\lifphinovar}[1]{\liftnovar{\Phi}{#1}}
\newcommand{\lifphi}[1]{\lift{\Phi}{#1}{z}}

\DeclareMathOperator{\arr}{\mathcal{A}}
%\DeclareMathOperator{\arrFinal}{{\mathcal{A}^{\bm*}}}
\DeclareMathOperator{\arrFinal}{{\mathcal{\bm{\hat}A}}}
\DeclareMathOperator{\warr}{\marr}
\DeclareMathOperator{\marr}{\mathcal{M}}

\DeclareMathOperator{\apath}{\leadsto}
\DeclareMathOperator{\mpath}{\leadsto_=}
\DeclareMathOperator{\notapath}{\not\leadsto}
\DeclareMathOperator{\notmpath}{\not\leadsto_=}

\newcommand{\ltArrC}{<_{\arrFinal(C)}}
\newcommand{\ltAC}{<_{\arr(C)}}
\newcommand{\ltArrCOne}{<_{\arrFinal(C_1)}}
\newcommand{\ltArrCTwo}{<_{\arrFinal(C_2)}}
%\newcommand{\ltArrC}{<_{\scalebox{0.6}{$\arrFinal(C)$}}}
\newcommand{\ltArr}{<_{\scalebox{0.6}{$\arrFinal$}}}

\newcommand{\bhat}{\bm\hat}
\newcommand{\bbar}{\bm\bar}
\newcommand{\bdot}{\bm\dot}

%\usepackage{yfonts}
\usepackage{upgreek}
\DeclareMathAlphabet{\mathpzc}{OT1}{pzc}{m}{it}
%\DeclareMathOperator{\pos}{\mathscr{P}}
%\DeclareMathOperator{\pos}{\mathpzc{p}}
%\DeclareMathOperator{\pos}{{\rho}}
\DeclareMathOperator{\pos}{{\operatorname P}}
%\DeclareMathOperator{\pos}{P}
\DeclareMathOperator{\poslit}{\pos_\mathrm{lit}}
\DeclareMathOperator{\posterm}{\pos_\mathrm{term}}
%\newcommand{\poslit}[1]{\ensuremath{p_\text{lit}(#1)}}
%\newcommand{\posterm}[1]{\ensuremath{p_\text{term}(#1)}}
\newcommand{\at}[1]{|_{#1}}

\newcommand{\UICm}[1]{\UnaryInfCm{#1}}
\newcommand{\UnaryInfCm}[1]{\UnaryInfC{$#1$}}
\newcommand{\BICm}[1]{\BinaryInfCm{#1}}
\newcommand{\BinaryInfCm}[1]{\BinaryInfC{$#1$}}
\newcommand{\RightLabelm}[1]{\RightLabel{$#1$}}
\newcommand{\LeftLabelm}[1]{\LeftLabel{$#1$}}
\newcommand{\AXCm}[1]{\AxiomCm{#1}}
\newcommand{\AxiomCm}[1]{\AxiomC{$#1$}}
\newcommand{\mt}[1]{\textnormal{#1}}

\newcommand{\UnaryInfm}[1]{\UnaryInf$#1$}
\newcommand{\BinaryInfm}[1]{\BinaryInf$#1$}
\newcommand{\Axiomm}[1]{\Axiom$#1$}



% math
\newcommand{\calI}{\ensuremath{\mathcal{I}}}

\newcommand{\tupleShort}[2]{\ensuremath{(#1_1,\dotsc,#1_{#2})}}
\newcommand{\tuple}[2]{\ensuremath{(#1_1,\:#1_2\:,\dotsc,\:#1_{#2})}}
\newcommand{\setelements}[2]{\ensuremath{\{#1_1,\:#1_2\:,\dotsc,\:#1_{#2}\}}}
\newcommand{\pathelements}[2]{\ensuremath{ (#1_1,\:#1_2\:,\dotsc,\:#1_{#2}) }}

\newcommand{\elems}[1]{\ensuremath{#1_1,\dotsc, #1_{n}) }}

\newcommand{\defiemph}[1]{\emph{#1}}

\newcommand{\setofbases}{\ensuremath{\mathcal{B}}}
\newcommand{\setofcircuits}{\ensuremath{\mathcal{C}}}

\newcommand{\reals}{\ensuremath{\mathbb{R}}}
\newcommand{\integers}{\ensuremath{\mathbb{Z}}} 
\newcommand{\naturalnumbers}{\ensuremath{\mathbb{N}}}

% general
\newcommand{\zit}[3]{#1\ \cite{#2}, #3}
\newcommand{\zitx}[2]{#1\ \cite{#2}}
\newcommand{\footzit}[3]{\footnote{\zit{#1}{#2}{#3}}}
\newcommand{\footzitx}[2]{\footnote{\zitx{#1}{#2}}}

\newcommand{\ite}{\begin{itemize}}
\newcommand{\ete}{\end{itemize}}

\newcommand{\bfr}{\begin{frame}}
\newcommand{\efr}{\end{frame}}

\newcommand{\ilc}[1]{\texttt{#1}}


% misc

% multiframe
\usepackage{xifthen}% provides \isempty test
% new counter to now which frame it is within the sequence
\newcounter{multiframecounter}
% initialize buffer for previously used frame title
\gdef\lastframetitle{\textit{undefined}}
% new environment for a multi-frame
\newenvironment{multiframe}[1][]{%
\ifthenelse{\isempty{#1}}{%
% if no frame title was set via optional parameter,
% only increase sequence counter by 1
\addtocounter{multiframecounter}{1}%
}{%
% new frame title has been provided, thus
% reset sequence counter to 1 and buffer frame title for later use
\setcounter{multiframecounter}{1}%
\gdef\lastframetitle{#1}%
}%
% start conventional frame environment and
% automatically set frame title followed by sequence counter
\begin{frame}%
\frametitle{\lastframetitle~{\normalfont \Roman{multiframecounter}}}%
}{%
\end{frame}%
}




% http://texfragen.de/hurenkinder_und_schusterjungen
\usepackage[all]{nowidow}



% force no overlong lines:
%\tolerance=1 % tolerance for how badly spaced lines are allowed, less means "better" lines
\tolerance=500 %  need more tolerance for equations
%\emergencystretch=\maxdimen
%\emergencystretch=200pt
%\setlength{\emergencystretch}{3em}
%\hyphenpenalty=10000 % forces no hyphenation
%\hbadness=10000


% http://tex.stackexchange.com/questions/35717/how-to-draw-arrows-between-parts-of-an-equation-to-show-the-math-distributive-pr
\tikzset{square arrow/.style={to path={ -- ++(.0,-.15)  -| (\tikztotarget)}}}
\tikzset{square arrow2/.style={to path={ -- ++(.0,-.25)  -| (\tikztotarget)}}}
%\tikzset{square arrow/.style={to path={ -- ++(00,-.01) -- ++(0.5,-0.1) -- ++(0.5,-0.1) -| (\tikztotarget)},color=red}}


% have arrows from a to b and from c to d here
% just use: texttext\arrowA texttest \arrowB texttext
\newcommand{\arrowA}{\tikz[overlay,remember picture] \node (a) {};}
\newcommand{\arrowB}{\tikz[overlay,remember picture] \node (b) {};}
\newcommand{\drawAB}{
	\tikz[overlay,remember picture]
	{\draw[->,bend left=5,color=red] (a.south) to (b.south);}
	%{\draw[->,square arrow,color=red] (a.south) to (b.south);}
}
\newcommand{\arrowAP}{\tikz[overlay,remember picture] \node (ap) {};}
\newcommand{\arrowBP}{\tikz[overlay,remember picture] \node (bp) {};}
\newcommand{\drawABP}{
	\tikz[overlay,remember picture]
	{\draw[->,bend right=5,color=red] (ap.south) to (bp.south);}
	%{\draw[->,square arrow,color=red] (a.south) to (b.south);}
}

\newcommand{\arrowAB}{\tikz[overlay,remember picture] \node (ab) {};}
\newcommand{\arrowBA}{\tikz[overlay,remember picture] \node (ba) {};}
\newcommand{\drawAABB}{
	\tikz[overlay,remember picture]
	%{\draw[->,bend left=80] (a.north) to (b.north);}
	{\draw[->,square arrow,color=brown] (ab.south) to (ba.south);
	\draw[->,square arrow,color=brown] (ba.south) to (ab.south);}
}


\newcommand{\arrowCD}{\tikz[overlay,remember picture] \node (cd) {};}
\newcommand{\arrowDC}{\tikz[overlay,remember picture] \node (dc) {};}
\newcommand{\drawCCDD}{
	\tikz[overlay,remember picture]
	%{\draw[->,bend left=80] (a.north) to (b.north);}
	{\draw[<->,dashed,square arrow,color=green] (cd.south) to (dc.south); }
}



\newcommand{\arrowC}{\tikz[overlay,remember picture] \node (c) {};}
\newcommand{\arrowD}{\tikz[overlay,remember picture] \node (d) {};}
\newcommand{\drawCD}{
	\tikz[overlay,remember picture]
	{\draw[->,square arrow,color=blue] (c.south) to (d.south);}
}

\newcommand{\arrowE}{\tikz[overlay,remember picture] \node (e) {};}
\newcommand{\arrowF}{\tikz[overlay,remember picture] \node (f) {};}
\newcommand{\drawEF}{
	\tikz[overlay,remember picture]
	{\draw[->,square arrow2,color=orange] (e.south) to (f.south);}
}


\newcommand{\arrAP}{\arrowAP}
\newcommand{\arrBP}{\arrowBP}
\newcommand{\arrA}{\arrowA}
\newcommand{\arrB}{\arrowB}
\newcommand{\arrC}{\arrowC}
\newcommand{\arrD}{\arrowD}
\newcommand{\arrE}{\arrowE}
\newcommand{\arrF}{\arrowF}


\DeclareMathOperator{\simgeq}{\scalebox{0.92}{$\gtrsim$}}

\newcommand{\refsub}[2]{\hyperref[#2]{\ref*{#1}.\ref*{#2}}}

%\newcommand{\sigmarange}[2]{\sigma_{#1}^{#2} }
\newcommand{\sigmarange}[2]{\sigma_{(#1,#2)} }
\newcommand{\sigmaz}[1]{\sigmarange{0}{#1} }
\newcommand{\sigmazi}[0]{\sigmaz{i} }

\DeclareMathOperator{\lit}{lit}

%\def\fCenter{\ \proves\ }
\def\fCenter{\proves}

\newcommand{\prflbl}[2]{\RightLabel{\footnotesize $#1, #2$} }
%\newcommand{\prflblid}[1]{\RightLabel{$#1, \id$} }
\newcommand{\prflblid}[1]{\RightLabel{\footnotesize $#1$} }

\DeclareMathOperator{\resruleres}{res}
\DeclareMathOperator{\resrulefac}{fac}
\DeclareMathOperator{\resrulepar}{par}
\newcommand{\lkrule}[2]{\ensuremath{\operatorname{#1}:#2}} % operatorname fixes spacing issues for =

\newcommand{\parti}[4]{\ensuremath{ \langle (#1; #2), (#3; #4)\rangle  }}

\newcommand{\partisym}{\ensuremath{\chi}}

\newcommand{\occur}[1]{\ensuremath{[#1]}}
\newcommand{\occ}[1]{\occur{#1}}

\newcommand{\occurat}[2]{\ensuremath{{\occur{#1}_{#2}}}}
\newcommand{\occat}[2]{\occurat{#1}{#2}}
\newcommand{\occatp}[1]{\occurat{#1}{p}}
\newcommand{\occatq}[1]{\occurat{#1}{q}}

\newcommand{\colterm}[1]{\zeta_{#1}}



% fix restateable spacing 
%http://tex.stackexchange.com/questions/111639/extra-spacing-around-restatable-theorems

\makeatletter

\def\thmt@rst@storecounters#1{%
%THIS IS THE LINE I ADDED:
\vspace{-1.9ex}%
  \bgroup
        % ugly hack: save chapter,..subsection numbers
        % for equation numbers.
  %\refstepcounter{thmt@dummyctr}% why is this here?
  %% temporarily disabled, broke autorefname.
  \def\@currentlabel{}%
  \@for\thmt@ctr:=\thmt@innercounters\do{%
    \thmt@sanitizethe{\thmt@ctr}%
    \protected@edef\@currentlabel{%
      \@currentlabel
      \protect\def\@xa\protect\csname the\thmt@ctr\endcsname{%
        \csname the\thmt@ctr\endcsname}%
      \ifcsname theH\thmt@ctr\endcsname
        \protect\def\@xa\protect\csname theH\thmt@ctr\endcsname{%
          (restate \protect\theHthmt@dummyctr)\csname theH\thmt@ctr\endcsname}%
      \fi
      \protect\setcounter{\thmt@ctr}{\number\csname c@\thmt@ctr\endcsname}%
    }%
  }%
  \label{thmt@@#1@data}%
  \egroup
}%

\makeatother




\newcommand{\mymark}[1]{\ensuremath{(#1)}}
\newcommand{\markA}{\mymark \circ}
\newcommand{\markB}{\mymark *}
\newcommand{\markC}{\mymark \divideontimes}

\newcommand{\wrong}[1]{{\color{red}WRONG: #1}}
\newcommand{\NB}[1]{{\color{blue}NB: #1}}
\newcommand{\hl}[1]{{\color{orange} #1}}
\newcommand{\mytodo}[1]{{\color{red}TODO: #1}}
\newcommand{\largered}[1]{{

	  \LARGE\bfseries\color{red}
		#1

}}
\newcommand{\largeblue}[1]{{

	  \large\bfseries\color{blue}
		#1

}}




\usepackage{ulem} %  \dotuline{dotty} \dashuline{dashing} \sout{strikethrough}
\normalem

\usepackage{tabu} % tabular also in math mode (and much more)

\usepackage[color]{changebar} %  \cbstart, \cbend
\cbcolor{red}



% http://tex.stackexchange.com/questions/7032/good-way-to-make-textcircled-numbers
\newcommand*\circled[1]{\tikz[baseline=(char.base)]{
\node[shape=circle,draw,inner sep=2pt] (char) {#1};}}



% http://tex.stackexchange.com/questions/43346/how-do-i-get-sub-numbering-for-theorems-theorem-1-a-theorem-1-b-theorem-2

\makeatletter
\newenvironment{subtheorem}[1]{%
  \def\subtheoremcounter{#1}%
  \refstepcounter{#1}%
  \protected@edef\theparentnumber{\csname the#1\endcsname}%
  \setcounter{parentnumber}{\value{#1}}%
  \setcounter{#1}{0}%
  \expandafter\def\csname the#1\endcsname{\theparentnumber.\Alph{#1}}%
  \ignorespaces
}{%
  \setcounter{\subtheoremcounter}{\value{parentnumber}}%
  \ignorespacesafterend
}
\makeatother
\newcounter{parentnumber}


\usepackage{tabularx}% http://ctan.org/pkg/tabularx
\newcolumntype{Y}{>{\centering\arraybackslash}X}

\newcommand{\mycols}[2][3]{
	\noindent\begin{tabularx}{\textwidth}{*{#1}{Y}}
		#2
	\end{tabularx}%
}


\newcommand{\definethms}{

	%\declaretheorem[title=Theorem,qed=$\triangle$,parent=chapter]{thm}
	\newcommand{\thmqed}{$\square$} % for thms without proof
	\newcommand{\propqed}{$\square$} % for props without proof
	\declaretheorem[title=Theorem]{thm}
	\declaretheorem[title=Proposition,sibling=thm]{prop}
	\declaretheorem[title=Conjectured Proposition,sibling=thm]{cprop}

	%\declaretheorem[title=Lemma,parent=chapter]{lemma}
	\declaretheorem[sibling=thm]{lemma}
	\declaretheorem[sibling=thm,title=Conjectured Lemma]{clemma}
	\declaretheorem[title=Corollary,sibling=thm]{corr}
	\declaretheorem[sibling=thm,title=Definition,style=definition,qed=$\triangle$]{defi}
	%\declaretheorem[title=Definition,qed=$\triangle$,parent=chapter]{defi}
	\declaretheorem[title=Example,style=definition,qed=$\triangle$,sibling=thm]{exa}

	\declaretheorem[sibling=thm,title=Conjecture]{conj}

	\declaretheorem[title=Remark,style=remark,numbered=no,qed=$\triangle$]{remark}


}

\usepackage[matha]{mathabx} % the locial operators here have more space around them and [ and ] are thicker, also langle and rangle are a bit nicer; subseteq looks a bit weird

%\usepackage{MnSymbol} % again other symbols


\newcommand{\inference}{\ensuremath{\iota}}

\usepackage{cases} % numcases


% subsections also in toc
\setcounter{tocdepth}{2}

%\declaretheorem[title=Theorem,qed=$\triangle$,parent=chapter]{thm}
\newcommand{\thmqed}{$\square$} % for thms without proof
\newcommand{\propqed}{$\square$} % for props without proof
\declaretheorem[title=Theorem]{thm}
\declaretheorem[title=Proposition,sibling=thm]{prop}
%\declaretheorem[title=Lemma,parent=chapter]{lemma}
\declaretheorem[sibling=thm]{lemma}
\declaretheorem[title=Corollary,sibling=thm]{corr}
\declaretheorem[sibling=thm,title=Definition,style=definition,qed=$\triangle$]{defi}
%\declaretheorem[title=Definition,qed=$\triangle$,parent=chapter]{defi}
\declaretheorem[title=Example,style=definition,qed=$\triangle$,sibling=thm]{exa}

\declaretheorem[sibling=thm,title=Conjecture]{conj}

%\def\proofSkipAmount{ \vskip -0.5em}



%\usepackage{bussproof}

%\usepackage{vaucanson-g}
\usepackage{amssymb}
\usepackage{latexsym}

% for color-highlighted code
%\usepackage{color} % for grey comments
%\usepackage{alltt}

%\usepackage[doublespacing]{setspace}
\usepackage[onehalfspacing]{setspace}
%\usepackage[singlespacing]{setspace}
\usepackage{tabularx}
\usepackage{hyperref}
\usepackage{comment}
\usepackage{color}
\usepackage[final]{listings} % sourcecode in document
\usepackage{url}      % for urls
\usepackage{multicol}
\usepackage{float}
\usepackage{caption}
\usepackage{subfigure}
\usepackage{amsmath}
\usepackage{amssymb}

\usepackage{graphicx}

\usepackage[authoryear]{natbib} % \cite ; square|round etc.
%\usepackage[numbers,square]{natbib}
%\usepackage[square, authoryear]{natbib}
%\usepackage[language=english]{biblatex}

%\bibliographystyle{plain}
\bibliographystyle{alpha}
%\bibliographystyle{alphadin}
%\bibliographystyle{dinat}
%\bibliographystyle{chicago}
%\bibliographystyle{plainnat}

\bibdata{bib.bib}

\renewcommand*{\partformat}{\partname\ \thepart\ -}
\let\partheadmidvskip\

\newcommand{\comp}{\ensuremath{\text{comp}}}
% smaller url style
\makeatletter
\def\url@leostyle{%
\@ifundefined{selectfont}{\def\UrlFont{\sf}}{\def\UrlFont{\small\ttfamily}}}
\makeatother
\urlstyle{leo}

\newcommand{\myfig}[5] {
	\begin{figure}[tbph]
		\centering
		\includegraphics[#3]{#1}
		\caption[#4]{#5}
		\label{fig:#2}
	\end{figure}
}

\setlength{\parindent}{0em}
%\usepackage{thmtools} % actually already in latex_header.tex ...

\usepackage{amsthm}


\usepackage{tikz-qtree}

%\newcommand{\sig}[1]{{#1}_\Sigma}
%\newcommand{\p}[1]{{#1}_\Pi}
\newcommand{\sig}[1]{\stackrel{\Sigma}{#1}}
\newcommand{\p}[1]{\stackrel{\Pi}{#1}}

\newcommand{\e}[1]{\vskip .7em   \subsection*{#1}}


\def\proofSkipAmount{ \vskip -0.3em}

\newcommand{\lif}[1]{\lift{\Delta}{#1}{x}}
\newcommand{\lifboth}[1]{\lft{\Gamma\cup\Delta}{z}{#1}}

\begin{document}


\section{Proof of the correctness of Huang's algorithm without propositional refutations}


Intuition of $\sigma'$:

If we pull a substitution out of a lifting which replaces $\Delta$-terms, we also have to replace the $\Delta$-terms 
in the ``codomain'' of the substitution. This is the second case in the definition of $\sigma'$ below.

There is just a problem in the following case: $\lif{ f(x)\sigma }$, where $x\sigma = a$ and $f$ is a $\Delta$-symbol.
Then $\lif{ f(x)\sigma } = \lif{ f(a) } = x_i$, but $\lif{f(x)}\sigma = x_j$ with $i\neq j$.
The first case of the definition of $x_j$ then fixes this by replacing $x_j$ with $x_i$. 



\begin{lemma}
	\label{lemma:lif}

	Let $C$ be a clause and $\sigma$ a substitution.
	Let $t_1,\ldots,t_n$ be all maximal $\Delta$-terms in this context, i.e.\ those that occur in $C$ or $C\sigma$,  and 
	$x_1, \ldots, x_n$ the corresponding fresh variables to replace the $t_i$.
	Define $\sigma'$ such that for a variable $z$, 
	\[
		z \sigma' =
		\begin{cases} 
			x_l & \text{ if } z = x_k \text{ and } t_k\sigma = t_l  \\
			\lif{z\sigma} & \text{ otherwise}
		\end{cases} 
	\]

	Then
	$\lif{C\sigma} =
	\lif{C}\sigma'$.
\end{lemma}
Note that the definition of $\sigma'$ only depends on the $x_i$ and $t_i$.
\begin{proof}
	We prove this for an atom $P(s_1, \ldots, s_m)$ in $C$, which works since lifting and substitution commute over binary connectives and into an atom.

	We show that 
	$\lif{s_j \sigma} = \lif{s_j}\sigma'$ for $1 \leq j \leq m$.

	%Let $\phi \sigma^{t}$ denote $\phi[t/y]\sigma[t/x]$ for a fresh variable $y$.

	Note that anything in the term structure above a maximal $\Delta$-term is unaffected by both substitution and the lifting.

	Let $t_i$ be a maximal $\Delta$-term in $s_i \sigma$.

	We show that $ \lif{ t_i \sigma } = \lif{ t_i } \sigma'$, which proves the lemma.

	Let $t_i\sigma = t_j$. Then $\lif{ t_i\sigma} = \lif{t_j} = x_j$.

	We show that $x_j = \lif{t_i}\sigma'$.

	Suppose that $t_i = t_j$, i.e.\ $\sigma$ is trivial on $t_i$.
	Then $i=j$ as the $\Delta$-terms have a unique number.
	Hence $\lif{t_i}\sigma' = x_i \sigma' = x_i = x_j$.


	Otherwise $t_i \neq t_j$. Then $i\neq j$ and  $x_j \neq x_i$.\newline
	$\lif{t_i}\sigma' = x_i \sigma'$.
	By the definition of $\sigma'$, as $t_i\sigma = t_j$, $x_i\sigma' = x_j$.
	%
	%
	\begin{comment}
		Suppose no $\Delta$-colored symbol occurs in $s_j$ or $s_j\sigma$. Then $\lif{s_j\sigma} = s_j\sigma $. this equals $  s_j\sigma'$ as there, only the second case applies, where the lifting doesn't affect the term.

		Suppose a maximal $\Delta$-colored term $t_i$ occurs in $s_j\sigma$ but not in $s_j$ and suppose it's the only one in $s_j\sigma$.
		Then $\lif{ s_j\sigma } = \lif{ s_j\sigma^{t_i} }= s_j\sigma^{t_i} \abstraction{t_i/x_i} = \lif{s_j} \sigma^{t_i} \abstraction{t_i/x_i}$
	\end{comment}
	%Note that if a $\Delta$-term $t_i$ occurs in $s_j$, a $\Delta$-term with the same outermost symbol occurs in $s_j$ at the same position.
\end{proof}


\begin{comment}
	\begin{lemma}
	\label{lemma:lif_literal}
	If $l\sigma$ = $l'\sigma$, then $\lif{l}\sigma' = \lif{l'}\sigma'$ for $\sigma'$ defined as in lemma \ref{lemma:lif}
\end{lemma}
\begin{proof}
	$l\sigma$ = $l'\sigma$

	$\ra \lif{l\sigma} = \lif{l'\sigma}$

	by lemma \ref{lemma:lif}, 
	$\lif{l}\sigma' = \lif{l'}\sigma'$

\end{proof}



\begin{lemma} // currently unused

	$(\lif{C}(x_1, \ldots, x_n))\sigma =
	(\lif{C\sigma'}(x_1, \ldots, x_n))$ if $\sigma$ does not change any of $x_1, \ldots, x_n$ or any of $t_1, \ldots, t_n$.\qedhere

	\todo[inline]{it would work to fix substitutions of $x_i$ by substituting $t_i$ for that instead, as long as the result isn't another $t_i$, but this isn't actually relevant here.}

\end{lemma}

\begin{prop}
	$\Gamma = \lif{\Gamma}$.
\end{prop}
\begin{proof}
	By definition, $\Delta$-terms only appear in $\Delta$ and not in $\Gamma$. 
\end{proof}

\end{comment}

 
\begin{lemma}[corresponds to Lemma 4.8 in thesis and Lemma 11 in Huang]
  \label{lemma:lift_commute}
  Let $A$ and $B$ be first-order formulas and $s$ and $t$ be terms. Then it holds that:
  \begin{enumerate}
    \item $\lift{\Phi}{\lnot A}{x} \semiff{} \lnot \lift{\Phi}{A}{x}$
    \item $\lift{\Phi}{A \circ B}{x} \semiff{} ( \lift{\Phi}{A}{x} \circ \lift{\Phi}{B}{x} )$ for  $\circ \in \{\land, \lor\}$
    \item $\lift{\Phi}{s = t}{x} \semiff{} ( \lift{\Phi}{s}{x} = \lift{\Phi}{t}{x} )$
  \end{enumerate}
\end{lemma}

\begin{lemma}
	Let $s$ and $t$ be terms such that no $x_i$ occurs in them, $\Phi$ a set of formulas and $M$ a model.
	Then $M\entails \lft{\Phi}{x}{s} = \lft{\Phi}{x}{t}$ implies that $M\entails s=t$.
	\label{lemma:lift_equality}
\end{lemma}
\begin{proof}
	Suppose no $\Delta$-term occurs in $s$ or $t$. Then $\lft{\Phi}{x}{s} = s$ 
	and $\lft{\Phi}{x}{t} = t$.

	Otherwise let $t_i$ be a maximal $\Delta$-term in $s$. Suppose it occurs at position $p$. In $\lft{\Phi}{x}{s}$, it is replaced by $x_i$.
	But as $M \entails \lft{\Phi}{x}{s} = \lft{\Phi}{x}{t}$, two situations can arise:
	\begin{compactenum}
	\item $x_i$ occurs at $p$ in $\lft{\Phi}{x}{t}$.
		As $x_i$ does not occur in $t$, it is placed there by the lifting.
		But $x_i$ is only employed in order to replace $t_i$, so at position $p$ in $t$, we have $t_i$.
	\item A term $r$ occurs at $p$ in $\lft{\Phi}{x}{t}$ which does not influence the evaluation of $\lft{\Phi}{x}{t}$ in $M$. This can be the case if $r$ is contained in a subterm of $u$ and in $M$, the function symbol of $u$ is interpreted such that it does not depend on the argument that contains $r$.
		
		But as the maximal $\Delta$-term $t_i$ occurs in $s$ at $p$ and $M \entails \lft{\Phi}{x}{s} = \lft{\Phi}{x}{t}$, there is a function symbol $u'$ in $\lft{\Phi}{x}{s}$ corresponding to $u$ which also does not depend on this argument.

		Hence even though $s$ and $t$ are not syntactically equal, $M\entails s=t$ in this case. \qedhere
	\end{compactenum}

\end{proof}

			\begin{lemma}
					\label{aga5tg5ba}
				Let $E$ be a formula and $s$ and $t$ terms.
					Let $h\occur{t}$ be a maximal $\Delta$-colored term containing $t$ at $p$ in $E\occurat{t}{p}$, if such a term exists.
					Then
					$M \entails (\lif{s}) = (\lif{t})$ and $M\notentails \lif{E\occurat{t}{p}}$ imply that $M\notentails \lif{E\occurat{s}{p}}$
					or $M\entails s=t \;\land\;(\lif{h\occur{s}}) \neq (\lif{h\occur{t}})$.
				\end{lemma}
				\begin{proof} 
					Suppose that $t$ at $p$ in $E\occurat{t}{p}$ is not contained in a $\Delta$-colored term.
					Then $\lif{E\occurat{t}{p}}$ and $\lif{E\occurat{s}{p}}$ only differ at position $p$, where at the first, there is $\lif{t}$, and at the latter, there is $\lif{s}$. But in $M$, they are interpreted the same way, hence $M\entails \lif{E\occurat{s}{p}} \semiff \lif{E\occurat{t}{p}}$, which implies the result.

					Otherwise $t$ at $p$ in $E\occurat{t}{p}$ is contained in the maximal $\Delta$-colored term $h\occur{t}$.
					%Suppose that $s$ and $t$ are syntactically equal. Then $h\occur{t}$ and $h\occur{s}$ are lifted by the same variable and $\lif{E\occurat{t}{p}} \semiff{} \lif{E\occurat{s}{p}}$.
					%Otherwise $s$ and $t$ are distinct terms.
					%Then they are replaced by distinct variables by the lifting.
					If $s$ and $t$ are syntactically equal, then $(\lif{h\occur{s}}) = (\lif{h\occur{t}})$ and $\lif{E\occurat{s}{p}} \semiff{} \lif{E\occurat{t}{p}}$.
					Otherwise, as by Lemma \ref{lemma:lift_equality} it holds that $M\entails s=t$,
					we get that $M \entails\nolinebreak s=\nolinebreak t \;\land\;(\lif{h\occur{s}})  \neq (\lif{h\occur{t}})$.
				\end{proof} 




We use basically the same definition of $\PI$ as Huang with minor adaptions for paramodulation (deviations are marked):
\begin{defi}[Propositional interpolant extraction.]
  Let $\pi$ be a resolution refutation of $\Gamma \cup \Delta$.
  \defiemph{${\PI(\pi)}$} is defined to be $\PI(\square)$, where $\square$ is the empty clause derived in $\pi$.

  For a clause $C$ in $\pi$, \defiemph{$\PI(C)$} is defined as follows:
  \label{def:PI}
  \begin{itemize}
    \item[Base case.]
      If $C \in \Gamma$, $\PI(C) = \bot$.
      If otherwise $C \in \Delta$, $\PI(C) = \top$.
    \item[Resolution.]
      \label{def:PI_resolution}
      %Suppose the clause $C$ is the result of a resolution step. Then it has the following form: 

      % \begin{prooftree}
      %   \AxiomCm{C_1: D \lor l}
      %   \AxiomCm{C_2: E \lor \lnot l'}
      %   \RightLabelm{\quad l\sigma = l'\sigma}
      %   \BinaryInfCm{C: (D\lor E)\sigma}
      % \end{prooftree}
      %\todo{write as prooftree? (not necessary, but nicer)}
      If the clause $C$ is the result of a resolution step of $C_1: D \lor l$ and $C_2: E \lor \lnot l'$ using a unifier $\sigma$ such that $l\sigma = l'\sigma$, then $\PI(C)$ is defined as follows:
      %$\PI(C)$ is defined according to this case distinction:
      \begin{enumerate}
        \item If $l$ is $\Gamma$-colored: $\PI(C) = [\PI(C_1) \lor \PI(C_2)]\sigma$
        \item If $l$ is $\Delta$-colored: $\PI(C) = [\PI(C_1) \land \PI(C_2)]\sigma$
        \item If $l$ is grey: $\PI(C) = [(l \land \PI(C_2)) \lor (\lnot l' \land \PI(C_1)) ]\sigma $
      \end{enumerate}

    \item[Factorisation.]
      If the clause $C$ is the result of a factorisation of $C_1: l \lor l' \lor D$ using a unifier $\sigma$ such that $l\sigma = l'\sigma$, then $\PI(C) = \PI(C_1)\sigma$.

    \item[Paramodulation.]
  \label{def:PI_paramod}
      If the clause $C$ is the result of a paramodulation of $C_1: s=t \lor C$ and $C_2: D\occur{r}$ using a unifier $\sigma$ such that $r\sigma = s\sigma$, then $\PI(C)$ is defined according to the following  case distinction:
      \begin{enumerate}
				\item If $r$ occurs in a maximal $\Delta$-term $h\occur{r}$ in $D\occur{r}$: \comm{Huang's version has another restriction here}
          \label{def:PI_paramod_1}
          \newline
          $\PI(C) = [ ( s=t \land \PI(C_2) ) \lor (s\neq t \land \PI(C_1)) ]\sigma \lor (s=t \land h\occur{s} \neq h\occur{t})\sigma$
        \item If $r$ occurs in a maximal $\Gamma$-term $h\occur{r}$ in $D\occur{r}$: \comm{Huang's version has another restriction here}
          \label{def:PI_paramod_2}
          \newline
          $\PI(C) = [ ( s=t \land \PI(C_2) ) \lor (s\neq t \land \PI(C_1)) ]\sigma \land (s\neq t \lor h\occur{s} = h\occur{t})\sigma$
        \item Otherwise:
          \label{def:PI_paramod_3}
          \newline
          $\PI(C) = [ ( s=t \land (\PI(C_2) \lor h[s] \neq h[t] ) \lor (s\neq t \land \PI(C_1)) ]\sigma$ \qedhere

      \end{enumerate}
  \end{itemize}
\end{defi}


Now we show the ``main'' lemma of Huang's proof without using a propositional deduction $P_P$.
The remaining part of his proof after this lemma does not use the restriction to propositional deductions and hence goes through.

\begin{lemma}[corresponds to Lemma 12 in Huang and Lemma 4.9 in the thesis]
	Let $\pi$ be a resolution refutation of $\Gamma \cup \Delta$.
	Then for $C \in \pi$,
	$ \Gamma \entails \lif{\PI(C) \lor C} $.
	\label{lemma:gamma_entails_interpolant}
\end{lemma}

\begin{proof}
	By induction on the resolution refutation of the strengthening: $\Gamma \entails \lif{\PI(C) \lor C_\Gamma}$, i.e.\ we only consider literals of $C$ which are contained in $\Lang(\Gamma)$.

	Base case:
	Either $C \in \Gamma$, then it does not contain $\Delta$-terms.
	Otherwise $C \in \Delta$ and $\PI(C) = \top$.

	Induction step:
	\begin{description}
		\item{Resolution.}
			\begin{prooftree}
				\AxiomCm{C_1: D \lor l}
				\AxiomCm{C_2: E \lor \lnot l'}
				\RightLabelm{\quad l\sigma = l'\sigma}
				\BinaryInfCm{C: (D\lor E)\sigma}
			\end{prooftree}

			By the induction hypothesis, we can assume that:

			$\Gamma \entails \lif{\PI(C_1) \lor (D\lor l)_\Gamma}$ and $\Gamma \entails \lif{\PI(C_2) \lor (E\lor \lnot l')_\Gamma}$

			which by Lemma \ref{lemma:lift_commute} implies that

			$\Gamma \stackrel{(*)}\entails \lif{\PI(C_1)} \lor \lif{D_\Gamma} \lor \lif {l_\Gamma}$ and $\Gamma \stackrel{(\circ)}\entails \lif{\PI(C_2)} \lor \lif{E_\Gamma} \lor \lnot \lif{l'_\Gamma}$

			Let $\sigma'$ be defined as in Lemma \ref{lemma:lif} with $t_1, \ldots, t_n$ all $\Delta$-terms in this context (we need that every maximal $\Delta$-term has a distinct index, so take all occurring in $C_1$, $C_2$, $\PI(C_1)$, $\PI(C_2)$, with and without $\sigma$ applied to them).

			Case distinction:

			\begin{enumerate}
				\item $l$ is $\Gamma$-colored.
					Then $\PI(C) = [\PI(C_1) \lor \PI(C_2)]\sigma$. 

					We show that $\Gamma \entails \lif{ (\PI(C_1) \lor \PI(C_2))\sigma \lor (D \lor E)_\Gamma\sigma}$,
					\newline 
					i.e.~$\Gamma \entails \lif{ \Big(\PI(C_1) \lor \PI(C_2) \lor D_\Gamma \lor E_\Gamma\Big)\sigma} $.


					Hence by Lemma \ref{lemma:lif},
					$\Gamma \entails \lif{(\PI(C_1) \lor \PI(C_2) \lor D_\Gamma \lor E_\Gamma)}\sigma' $.

					Since $\sigma = \mgu(l, l')$, $l\sigma$ and $l'\sigma$ are syntactically equal and so $\lif{l\sigma} = \lif{l'\sigma}$.
					
					As by Lemma \ref{lemma:lif} $\lif{l\sigma} = \lif{l}\sigma'$ and $\lif{l'\sigma} = \lif{l'}\sigma'$,
					we get $\lif{l}\sigma' = \lif{l'}\sigma'$.\label{aou5jklah}

					So by applying $\sigma'$ to $(*)$ and $(\circ)$ (note that $l_\Gamma = l$ and $l'_\Gamma = l'$ as they are $\Gamma$-colored), we can perform a resolution step on $\lif{l}\sigma'$ and get

					$\Gamma \entails \lif{\PI(C_1)}\sigma' \lor \lif{D_\Gamma} \sigma' \lor \lif{\PI(C_2)}\sigma' \lor \lif {E_\Gamma} \sigma'$.

					and consequently
				$\Gamma \entails \lif{ \PI(C_1) \lor \PI(C_2) \lor D_\Gamma \lor E_\Gamma}\sigma' $.

				So by Lemma \ref{lemma:lif},

				$\Gamma \entails \lif{ \Big(\PI(C_1) \lor \PI(C_2) \lor D_\Gamma \lor E_\Gamma \Big) \sigma } $.


				\item $l$ is $\Delta$-colored.
					Then $\PI(C) = (\PI(C_1) \land \PI(C_2))\sigma$. 

					We show that $\Gamma \entails \lif{(\PI(C_1) \land \PI(C_2))\sigma \lor (D_\Gamma \lor E_\Gamma)\sigma}$

					which by Lemma \ref{lemma:lift_commute} is equivalent to\newline
					$\Gamma \entails \Big(\lif{\PI(C_1)\sigma} \land \lif{\PI(C_2)\sigma}\Big) \lor \lif{D_\Gamma\sigma} \lor \lif{E_\Gamma\sigma}$

					and by Lemma \ref{lemma:lif} is equivalent to\newline
					$\Gamma \stackrel{\markC}\entails \Big(\lif{\PI(C_1)}\sigma' \land \lif{\PI(C_2)}\sigma'\Big) \lor \lif{D_\Gamma}\sigma' \lor \lif{E_\Gamma}\sigma'$

					As $l$ and $l'$ are $\Delta$-colored, we can simplify $(*)$ and $(\circ)$ as follows and apply $\sigma'$:

					$\Gamma \entails \lif{\PI(C_1)}\sigma' \lor \lif{D_\Gamma}\sigma' $ and $\Gamma \entails \lif{\PI(C_2)}\sigma' \lor \lif{E_\Gamma}\sigma'$

					These clearly imply \markC.

				\item $l$ is grey. Then $\PI(C) = [(l \land \PI(C_2) ) \lor (\lnot l' \land \PI(C_2))]\sigma$.

					We show that $\Gamma \entails \lif{ \Big((l \land \PI(C_2) ) \lor (\lnot l' \land \PI(C_2)) \lor D_\Gamma \lor E_\Gamma\Big)\sigma}$, which by Lemma~\ref{lemma:lift_commute} and Lemma~\ref{lemma:lif} is equivalent to

					$\Gamma \entails \Big(\lif{l}\sigma' \land \lif{\PI(C_2)}\sigma'\Big)\lor\Big(\lnot \lif{l'}\sigma' \land \lif{\PI(C_2)}\sigma'\Big)\lor\lif{D_\Gamma}\sigma' \lor \lif{E_\Gamma}\sigma'$.

					Suppose for a model $M$ of $\Gamma$ that  $M \notentails \lif{D_\Gamma}\sigma'$ and $M\notentails \lif{E_\Gamma}\sigma'$ as otherwise we would be done.
					But then by $(*)$ and $(\circ)$,
					$M \entails \lif{\PI(C_1)}\sigma' \lor \lif{l}\sigma'$ and
					$M \entails\nolinebreak \lif{\PI(C_2)}\sigma' \lor \lnot\lif{l'}\sigma'$.

					As observed in case \ref{aou5jklah}, $\lif{l}\sigma' = \lif{l'}\sigma'$. By a case distinction on the truth value of $\lif{l}\sigma'$, we obtain the result.



			\end{enumerate}

		\item{Factorisation.}
			\begin{prooftree}
				\AxiomCm{C_1: l \lor l' \lor D}
				\RightLabelm{\quad \sigma = \mgu(l, l')}
				\UnaryInfCm{C: (l \lor D)\sigma}
			\end{prooftree}
			Then $\PI(C) = \PI(C_1)\sigma$.

			The induction hypothesis gives that
			$\Gamma \entails \lif{\PI(C_1) \lor l \lor l' \lor D}$.
			Let $\sigma'$ be as in Lemma \ref{lemma:lif}.

			Then $\Gamma \entails \lif{\PI(C_1) \lor l \lor l' \lor D}\sigma'$ and by Lemma \ref{lemma:lif},
			$\Gamma \entails \lif{\PI(C_1)\sigma \lor l\sigma \lor l'\sigma \lor D\sigma}$.

			By Lemma \ref{lemma:lift_commute},
			$\Gamma \entails \lif{\PI(C_1)\sigma} \lor \lif{l\sigma} \lor \lif{l'\sigma} \lor \lif{D\sigma}$.

			As $\sigma = \mgu(l, l')$, $l\sigma$ and $l'\sigma$ are syntactically equal, hence $\lif{l\sigma} = \lif{l'\sigma}$.%\todo[noline,size=\tiny]{syntactically equal? does ``equal'' suffice?. see also $s\sigma=r\sigma$ below}

			But then we can apply a factorisation step and get
			$\Gamma \entails \lif{\PI(C_1)\sigma} \lor \lif{l\sigma} \lor \lif{D\sigma}$ and by Lemma \ref{lemma:lif} and Lemma \ref{lemma:lift_commute}, 
			$\Gamma \entails\nolinebreak \lif{\PI(C_1)\sigma \lor l\sigma \lor D\sigma}$.



		\item{Paramodulation.}
			\begin{prooftree}
				\AxiomCm{C_1: D \lor s=t}
				\AxiomCm{C_2: E\occurat{r}{p}}
				\RightLabel{$\quad \sigma = \mgu(s, r)$}
				\BinaryInfCm{C: (D \lor E\occurat{t}{p})\sigma}
			\end{prooftree}
			By the induction hypothesis, we have:

			$\Gamma \entails \lif{\PI(C_1) \lor (D\lor s=t)_\Gamma}$

			$\Gamma \entails \lif{\PI(C_2) \lor (E\occurat{r}{p})_\Gamma}$

			By Lemma~\ref{lemma:lif} and Lemma~\ref{lemma:lift_commute}, we get that:

			$\Gamma \stackrel{\markA}\entails \lif{\PI(C_1)} \lor \lif{D_\Gamma} \lor \lif{s} = \lif{t}$

			$\Gamma \stackrel{\markB}\entails \lif{\PI(C_2)} \lor \lif{(E\occurat{r}{p})_\Gamma}$
\bigskip
		

	
			We distinguish two cases:\nopagebreak
			\begin{enumerate}
				\item Suppose $s$ does not occur in a maximal $\Delta$-term $h\occur{s}$ in $E\occurat{s}{p}$

					We show that $\Gamma \entails \lif{ \Big((s=t \land \PI(C_2)) \lor (s\neq t \land \PI(C_1))\Big)\sigma \lor \Big((D \lor E\occurat{t}{p})_\Gamma\Big)\sigma}$, which subsumes the cases 2 and 3 of the definition of $\PI$ for paramodulation.
					By Lemma~\ref{lemma:lift_commute}, we can pull the liftings inwards and by Lemma~\ref{lemma:lif}, we can commute substitution and lifting by employing $\sigma'$ to arrive at

				$\Gamma \entails
				\Big((\lif{s}\sigma')=(\lif{t}\sigma') \land \lif{\PI(C_2)}\sigma'\Big) \lor
				\Big((\lif{s}\sigma')\neq(\lif{t}\sigma') \land \lif{\PI(C_1)}\sigma'\Big) \lor
				\Big(\lif{D_\Gamma}\sigma' \lor \lif{(E\occurat{t}{p})_\Gamma}\sigma'\Big)$

				Let $M$ be a model of $\Gamma$. Let $M \notentails \lif{D_\Gamma}\sigma' \lor \lif{(E\occurat{t}{p})_\Gamma}\sigma'$ as otherwise we would be done. We show that depending on the truth value of  $(\lif{s}) = (\lif{t})$ in $M$, either the first or second conjunct of the above formula holds.

				Suppose that $M \entails (\lif{s}) \neq (\lif{t})$. 
				Then by~\markA, $M \entails \lif{\PI(C_1)}$ and hence $M \entails \lif{\PI(C_1)}\sigma'$.

				On the other hand, suppose that $M \entails (\lif{s}) = (\lif{t})$.
				Then by Lemma~\ref{aga5tg5ba}, as $s$ at $p$ in $E\occurat{s}{p}$ does not occur in a maximal $\Delta$-term, 
				$M \notentails \lif{E\occurat{s}{p}}$.

				Hence also $M \notentails \lif{E\occurat{s}{p}}\sigma'$ and
				by Lemma \ref{lemma:lif}, $M\notentails \lif{(E\occurat{s}{p})\sigma}$. 

				Due to $\sigma=\mgu(s, r)$, both $s\sigma$ and $r\sigma$ are syntactically equal.
				Suppose they are both not $\Delta$-colored.
				Then the lifting does not affect them and 
				$\lif{(E\occurat{s}{p})\sigma} = \lif{(E\occurat{r}{p})\sigma}$.
				Otherwise the lifting will replace them with the same variable and we as well get that
				$\lif{(E\occurat{s}{p})\sigma} = \lif{(E\occurat{r}{p})\sigma}$.

				By Lemma $\ref{lemma:lif}$, 
				$\lif{(E\occurat{s}{p})}\sigma' = \lif{(E\occurat{r}{p})}\sigma'$,
				so from $M \notentails \lif{E\occurat{s}{p}}\sigma'$, it follows that $M \notentails \lif{(E\occurat{r}{p})}\sigma'$

				Then by~\markB{}, we arrive at $M\entails \lif{\PI(C_2)}\sigma'$


				%The following two lemmas show that $M \notentails \lif{E\occurat{r}{p}}\sigma'$, which by~\markB{} implies that $M\entails \lif{\PI(C_2)}\sigma'$, or $M \entails s=t \;\land\;(\lif{h\occur{s}}) \neq (\lif{h\occur{t}})$.

				%\begin{lemma} $\sigma=\mgu(s, r)$ and $M\notentails \lif{E\occurat{s}{p}}\sigma'$ imply that $M\notentails \lif{E\occurat{r}{p}}\sigma'$.  \end{lemma}\nopagebreak 



			\item Otherwise $s$ occurs in a maximal $\Delta$-term $h\occurat{s}{q}$ in $E\occurat{s}{p}$.

				Then a similar line of argument can be employed, with the difference that the application of Lemma~\ref{aga5tg5ba} yields the extra case that 
			$M\entails s=t \;\land\;(\lif{h\occur{s}}) \neq (\lif{h\occur{t}})$.
				\begin{comment}
				\newenvironment{lemmaCustomNo}[1]
				{\renewcommand{\thelemma}{\ref{#1}$'$}%
					\addtocounter{lemma}{-1}%
				\begin{lemma}}
				{\end{lemma}}


				Then we have to replace Lemma \ref{aga5tg5ba} by:
				\bigskip

				\begin{lemmaCustomNo}{aga5tg5ba}
					$M \entails (\lif{s}) = (\lif{t})$ and $M\notentails \lif{E\occurat{t}{p}}\sigma'$ imply that $M\notentails \lif{E\occurat{s}{p}}\sigma'$ or that $\lif{h\occurat{s}{q}} \neq \lif{h\occurat{t}{q}}$.
				\end{lemmaCustomNo}
				\begin{proof}
					If $\lif{E\occurat{t}{p}}$ and $\lif{E\occurat{s}{p}}$ differ only at position $p$, then the proof of Lemma \ref{aga5tg5ba} applies.
					
					Otherwise position $p$ is in a maximal $\Delta$-term $h\occurat{t}{q}$, such that $h\occurat{t}{q}$ and $h\occurat{s}{q}$ are replaced with distinct variables.
					But then clearly $\lif{h\occurat{s}{q}} \neq \lif{h\occurat{t}{q}}$.
				\end{proof}

				\end{comment}

 Hence the following holds:

				$\Gamma \entails
				\Big((\lif{s}\sigma')=(\lif{t}\sigma') \land \lif{\PI(C_2)}\sigma'\Big) \lor
				\Big((\lif{s}\sigma')\neq(\lif{t}\sigma') \land \lif{\PI(C_1)}\sigma'\Big) \lor
				\Big((\lif{s}\sigma')=(\lif{t}\sigma') \land (\lif{h\occurat{s}{q}}) \neq (\lif{h\occurat{t}{q}} )\Big) \lor
				\Big(\lif{D_\Gamma}\sigma' \lor \lif{(E\occurat{t}{p})_\Gamma}\sigma'\Big)$
				\qedhere
		\end{enumerate}


		\begin{comment}



			easy case:
			$\PI(C) = [ ( s=t \land \PI(C_2) ) \lor (s\neq t \land \PI(C_1)) ]\sigma$

			to show:
			$\Gamma \entails \lif{ [ (( s=t \land \PI(C_2) ) \lor (s\neq t \land \PI(C_1))) \lor (D \lor E[t]) ]\sigma} $

			proof idea: either $s=t$, then also $\PI(C_2)$, or else $s\neq t$, but then also $\PI(C_1)$

			by lemma \ref{lemma:lif} for $\sigma'$ as in lemma, 
			$\Gamma \entails \lif{ (( s=t \land \PI(C_2) ) \lor (s\neq t \land \PI(C_1))) \lor (D \lor E[t]) }\sigma' $

			by lemma 11 (huang)
			$\Gamma \entails [((\lif{s}=\lif{t} \land \lif{\PI(C_2)} ) \lor (\lif{s\neq t} \land \lif{\PI(C_1)})) \lor (\lif{D} \lor \lif{E[t]}) ]\sigma' $

			reformulate:
			$\Gamma \entails ((\lif{s}\sigma'=\lif{t}\sigma' \land \lif{\PI(C_2)}\sigma' ) \lor (\lif{s}\sigma'\neq \lif{t}\sigma' \land \lif{\PI(C_1)}\sigma')) \lor (\lif{D}\sigma' \lor \lif{E[t]}\sigma') $

			By the rule: $s\sigma = r\sigma$, hence also $\lif{s\sigma} = \lif{r\sigma}$ and $\lif{s}\sigma' = \lif{r}\sigma'$ REALLY TRUE? -- think so\dots

			Suppose $M \entails \Gamma$ and $M \not \entails (\lif{D}\sigma' \lor \lif{E[t]}\sigma') $.

			Suppose $M \entails \lif{s}\sigma' = \lif{t}\sigma'$.

			By induction hypothesis (and lemma 11 (huang) and adding the substitution $\sigma'$), 
			$\Gamma \entails \lif{\PI(C_2)}\sigma' \lor \lif{(E[r])}\sigma'$.

			However by assumption $\Gamma \not \entails \lif{E[t]}\sigma'$.

			Hence $\Gamma \not \entails \lif{E[s]}\sigma'$, and
			$\Gamma \not \entails \lif{E[r]}\sigma'$. Therefore $\Gamma \entails \lif{\PI(C_2)}\sigma'$.


			Suppose on the other hand $M \entails \lif{s}\sigma' \neq \lif{t}\sigma'$.

			By the induction hypothesis, 
			$M \entails \lif{\PI(C_1)}\sigma' \lor (\lif{D}\sigma'\lor (\lif{s}=\lif{t})\sigma')$,
			hence then $M \entails \lif{\PI(C_1)}\sigma'$.

			Consequently, 
			$M \entails (\lif{s}\sigma' \neq \lif{t}\sigma' \land \lif{\PI(C_1)}\sigma') \lor (\lif{s}\sigma' = \lif{t}\sigma' \land \lif{\PI(C_2)}\sigma')$.

			By lemma 11 (huang), 
			$M \entails \lif{s \neq {t} \land {\PI(C_1)} \lor ({s} = {t} \land \PI(C_2))}\sigma'$.

			Hence 
			$\Gamma \entails \lif{(s \neq {t} \land {\PI(C_1)} \lor ({s} = {t} \land \PI(C_2))}\sigma' \lor (\lif{D} \lor \lif{E[t]})\sigma') $.

			is this really what i need to show?
		\end{comment}
\end{description}
\end{proof}



Then the following from the thesis (also same in Huang) seem to go through:

Lemma 4.10: swap $\Gamma$ and $\Delta$ and obtain logical negation as interpolant 

Corollary 4.11: $\Delta \entails \lifgamma{ \lnot \PI(C) \lor C}$ 

Lemma 4.12: not important if lifting delta or gamma terms first 

Thm 4.13: ordering 

\end{document}

	
\section{Number of quantifier alternations in the extracted interpolant}

In this section, we examine interpolants produced in Theorem~\ref{thm:two_phases} with respect to the number of quantifier alternations.
We arrive at the conclusion that there is a tight connection between the number of color alternations in the terms produced by the substitutions of the resolution refutation and the number of quantifier alternations in the resulting interpolant.  

We first formally define these notions:

\subsection{Color and quantifier alternations}

In the following, we assume that the maximum $\max$ of an empty sequence is defined to be $0$ and constants are treated as function symbols of arity $0$.
Furthermore $\bot$ is used to denote a color which is not possessed by any symbol.
\begin{defi}[Color alternation $\ca$]
	Let $\Gamma$ and $\Delta$ be sets of formulas and $t$ be a term.

	\medskip

	\noindent
	$\ca(t) \defeq \ca_\bot(t)$
	\medskip

	\begin{adjustwidth}{-1.7em}{}
	\noindent
	$
	\ca_\Phi(t) \defeq 
	\begin{cases}
		0 & \text{if $t$ is a variable} \\
		\max(\ca_\Phi(t_1), \dots, \ca_\Phi(t_n)) & \text{if $t = f(t_1, \dots, t_n)$ is gray} \\
		\max(\ca_\Phi(t_1), \dots, \ca_\Phi(t_n)) & \parbox[t]{0.4\textwidth}{if $t = f(t_1, \dots, t_n)$ is of color $\Phi$} \\
		1 + \max(\ca_\Psi(t_1), \dots, \ca_\Psi(t_n)) & \parbox[t]{0.38\textwidth}{if $t = f(t_1, \dots, t_n)$ is of color $\Psi$, $\Phi \neq \Psi$} \\
	\end{cases}
	$
\end{adjustwidth}
\end{defi}


\begin{defi}[Quantifier alternation $\qa$]
	Let $A$ be a formula.\nopagebreak
	\medskip

	\noindent
	$\qa(A) \defeq \qa_\bot(A)$
	\nopagebreak
	\medskip

	\noindent
	$
	\qa_Q(A) \defeq 
	\begin{cases}
		0 & \text{if $A$ is an atom} \\
		\qa_Q(B) & \text{if $A \equiv \lnot B$} \\
		\max(\parbox[t]{0.22\textwidth}{$\qa_Q(B),$\newline$ \qa_Q(C))$} & \text{if $A \equiv B \circ C$, $\circ \in \{\land, \lor, \limpl\}$} \\
		\qa_Q(B) & \text{if $A \equiv Q x B$} \\
		1+\qa_{Q'}(B) & \text{if $A \equiv Q' x B$, $Q \neq Q'$}  \\
	\end{cases}
	$
	\nopagebreak

	\qedhere
\end{defi}
%Note that this definition of quantifier alternations handles formulas in prenex and non-prenex form.


\subsection{Preliminary considerations}


First, we define the auxiliary procedure $\PI^*$:


\begin{defi}[$\PI^*$]
	$\PI^*$ is defined as $\PI$ with the difference that in $\PI^*$, all literals are considered to be gray.
	$\PIinit^*$ and $\PIstep^*$ are defined analogously.
\end{defi}

Hence $\PIinit^*$ coincides with $\PIinit$.
$\PIstep^*$ coincides with $\PIstep$ in case of factorization and paramodulation inferences.
For resolution inferences, the first two cases in the definition of $\PIstep$ do not occur for $\PIstep^*$.

$\PI^*$ enjoys the convenient property that it absorbs every literal which occurs in some clause:

\begin{prop}
	\label{prop:every_lit_in_pi_star}
	For every literal which occurs in a clause of a resolution refutation $\pi$, a respective successor occurs in $\PI^*(\pi)$.
\end{prop}
\begin{proof}
	By structural induction.
\end{proof}

Note that in $\PI^*$, we can conveniently reason about the occurrence of terms as no terms are lost throughout the extraction.
However Lemma~\ref{lemma:gray_lits_of_pi_star_in_pi} allows us to transfer results about gray literals to $\PI$:

\begin{lemma}
	\label{lemma:gray_lits_of_pi_star_in_pi}
	For every clause $C$ of a resolution refutation,
	the literals and equalities of $\PI(C)$ are exactly the gray literals and equalities of $\PI^*(C)$.
\end{lemma}
\begin{proof}
	Note that $\PIinit$ and $\PIinit^*$ coincide and $\PIstep$ and $\PIstep^*$ only differ for resolution inferences.
	More specifically, they only differ on resolution inferences, where the resolved literal is colored.
	%However here, no gray literals are removed but only colored ones.
	Hence $\PI(C)$ and $\PI^*(C)$ contain the same gray literals and equalities.
	The colored resolved literals however are not added to $\PI(C)$ as desired.
\end{proof}


\begin{lemma}
	\label{lemma:Ot8Gie7y}
	Let $\inference$ be an inference of a resolution refutation using the clauses $C_1, \dots, C_n$ which creates the clause $C$.
	If there is a literal $\lambda$ or an equality $s=t$ in $\PI(C_i)$ or a gray literal $\lambda$ or an equality $s=t$ in $C_i$ for $1 \varleq i \varleq n$, 
	then a successor of $\lambda$ or $s=t$ respectively occurs in $\PIstep(\inference, \PI(C_1), \dots, \PI(C_n)) \lor C$.
\end{lemma}
\begin{proof}
	Immediate by the definition of $\PI$.
\end{proof}

\begin{corr}
	\label{lemma:gray_lits_and_eq_all_in_PI}
	If there is a literal $\lambda$ or an equality $s=t$ in $\PI(C)$ or a gray literal $\lambda$ or an equality $s=t$ in $C$ for a clause $C$ of a resolution refutation $\pi$,
	then a successor of $\lambda$ or $s=t$ respectively occurs in $\PI(\pi)$.
\end{corr}
\begin{proof}
	This is a direct consequence of Lemma~\ref{lemma:Ot8Gie7y}.
\end{proof}


\subsection{Analysis of the occurrences of crucial terms in $\PI$}

We now make some considerations about the construction of certain terms in the context of interpolant extraction.
Thereby we employ the following definition:

\begin{defi}
In a literal or term $\varphi$ containing a subterm $t$, $t$ is said to occur \defiemph{below} a $\Phi$-symbol $s$ if in the syntax tree representation of $\varphi$, there is a node labeled $s$ on the path from the root to $t$. Note that the colored symbol may also be the predicate symbol.
Moreover, $t$ is said to occur \defiemph{directly below} the $\Phi$-symbol $s$ if it occurs below the $\Phi$-symbol $s$ and in the syntax tree representation of $\varphi$ on the path from $s$ to $t$, no nodes with labels with colored symbol occur.
\end{defi}



Moreover, we frequently reason over the stepwise application of the respective unifiers, for which we make use of the following definition:

\begin{defi}
	We define $\PIstepstarnosigma$ to coincide with $\PIstep^*$ but without applying the substitution $\sigma$ in each of the cases.
	Furthermore, $\PIstarnosigma(C)$ is an abbreviation of $\PIstepstarnosigma(\inference, \PI^*(C_1), \dots, \PI^*(C_m))$ if $C$ is created by an inference $\inference$ from the clauses $C_1, \dots, C_n$,
	and $\PIstarnosigma(C)$ coincides with $\PI^*(C)$ if $C\in \Gamma\cup\Delta$.

	Analogously, if $C \equiv D\sigma$, we use $\nosigma{C}$ to denote\nolinebreak{} $D$.
\end{defi}

In the context of an inference $\inference$ using the clauses $C_1, \dots, C_m$ to infer $C$, it holds that:
\begin{align*}
	 \PI^*(C) \lor C 
	 =\,& \PIstep^*(\inference, \PI^*(C_1), \dots, \PI^*(C_m)) \lor C  \\
	=\, & \Big(\PIstepstarnosigma(\inference, \PI^*(C_1), \dots, \PI^*(C_m)) \lor \nosigma{C} \Big) \sigma  \\
	=\, & \Big(\PIstarnosigma(C)\lor \nosigma{C} \Big) \sigma  \\
	=\,& \Big(\PIstarnosigma(C)\lor \nosigma{C} \Big) \sigmarange{0}{|\dom(\sigma)|}
\end{align*}


\newcommand{\inv}{\ensuremath{\PIstarnosigma(C)\lor \nosigma{C}}}
\newcommand{\invp}{\ensuremath{(\inv)}}


Note that if we are able to show that the application of a substitution $\sigma_i$ to $\invp\sigmazmi$ maintains an invariant and the invariant holds for $\inv$, then it immediately follows that it holds for $\PI^*(C) \lor C$. 



\begin{lemma}
	\label{lemma:var_below_phi_symbol}
	Let $\inference$ be an inference in a refutation of $\Gamma\cup\Delta$.
	Suppose that a variable $u$ occurs directly below a $\Phi$-symbol in $\invp\sigmazi$ for $i\vargeq 1$.
	Then at least one of the following statements holds:
	\begin{enumerate}
		\item
			\label{14_1}
			\label{15_1}
			The variable $u$ occurs directly below a $\Phi$-symbol in $\invp\sigmazmi$.

		\item
			\label{14_5}
			\label{15_5}
			The variable $u$ occurs at a gray position in a gray literal or at a gray position in an equality in $\invp\sigmazi$.

		\item 
			\label{14_2}
			\label{15_2}
			There is a variable $v$ such that 
			{
				\renewcommand{\labelitemi}{\textendash}
				\begin{itemize}
					\item $u$ occurs gray in $v\sigma_i$ and
					\item $v$ occurs in $\invp\sigmazmi$ directly below a $\Phi$-symbol as well as directly below a $\Psi$-symbol
				\end{itemize}
			}

	\end{enumerate}
\end{lemma}
\begin{proof}
	We consider all different situations under which the situation in question arises.
	Irrespective of the type of the inference $\inference$, one of these cases can apply:

	\begin{itemize}
		\item
			There is already a literal in $\invp\sigmazmi$ where $u$ occurs directly below a $\Phi$-symbol and $\sigma_i$ does not change this.
			Then clearly \ref{14_1} is the case.

		\item
			There is a variable $v$ in $\invp\sigmazmi$ such that $v\sigma_i$ contains $u$ directly below a $\Phi$-symbol.
			As $v$ is unified with the term $v\sigma_i$, $v\sigma_i$ must occur in $\invp\sigmazmi$, which implies that \ref{14_1} is the case.

	\end{itemize}


	\noindent
	In the case that $\inference$ is a resolution or factorization inference, 
	the following situations can apply:

	\begin{itemize}
		\item
			There is a variable $v$ which occurs directly below a $\Phi$-symbol such that $u$ occurs gray in $v\sigma_i$.

			Hence in the resolved or factorized literals $\lambda$ and $\lambda'$ in $\invp\sigmazmi$, there is a position $p$ such that without loss of generality $\lambda\atp = v$ and $u$ occurs gray in $\lambda'\atp$. 
			Note that due to the definition of the unification algorithm, $\lambda$ and $\lambda'$ must coincide on the path to $p$.

			By Proposition \ref{prop:every_lit_in_pi_star}, $\lambda$ and $\lambda'$ occur in $\inv$ irrespective of their coloring.

			We distinguish cases based on the position $p$:

			\begin{itemize}
				\item Suppose that $p$ occurs directly below a $\Phi$-symbol.
					Then as $u$ occurs gray in  $\lambda'\atp$, $u$ occurs directly below a $\Phi$-symbol in $\invp\sigmazmi$ and \ref{14_1} is the case.

				\item Suppose that $p$ occurs directly below a $\Psi$-symbol.
					Then $v$ occurs directly below a $\Psi$-symbol in $\lambda\atp$ and  \ref{14_2} holds.

				\item
					Suppose that $p$ does not occur directly below a colored symbol.
					Then $p$ does not occur below any colored symbol, hence $u$ is contained in a gray literal in a gray position in $\invp\sigmazmi$. 
					As $\sigma_i$ is trivial on $u$, this occurrence of $u$ also is present in $\invp\sigmazi$ and hence \ref{14_5} is the case.

			\end{itemize}
	\end{itemize}

	Now we consider the case that $\inference$ is a paramodulation inference of the clauses $C_1: r_1=r_2 \lor D$ and $C_2: E\occatp{r}$ with $\sigma=\mgu(\inference)=\mgu(r_1, r)$ yielding $C: (D\lor E\occatp{r_2})\sigma$.
	We again consider the different situations under which the situation in question arises:

	\begin{itemize}

		\item
			The variable $u$ occurs gray in $r_2$ and $p$ in $E$ is directly below a $\Phi$-symbol. 
			But then $u$ occurs gray in an equality in $\invp\sigmazmi$ and as $\sigma_i$ is trivial on $u$ also in $\invp\sigmazi$, hence \ref{15_5} holds.

		\item
			Suppose that some variable $v$ occurs directly below a $\Phi$-symbol in $\invp\sigmazmi$ such that $u$ occurs gray in $v\sigma_i$.
			Then by the definition of the unification algorithm, there exists a position $q$ such that one of $r_1\atq$ and $r\atq$ is $v$ and the other one contains a gray occurrence of $u$.

			We distinguish cases based on the position $q$:

			\begin{itemize}
				\item
					Suppose that $q$ occurs directly below a $\Phi$-symbol. Then clearly \ref{15_1} is the case.

				\item
					Suppose that $q$ occurs directly below a $\Psi$-symbol. Then as the variable $v$ also occurs directly below a $\Phi$-symbol and $u$ occurs gray in $v\sigma_i$, \ref{15_2} is the case.

				\item
					Suppose that $q$ is a gray position.
					Then \ref{15_5} is the case: 
					Either $u$ occurs gray in $r_1$ in $\invp\sigmazmi$ and then also in $\invp\sigmazi$, 
					or otherwise $v$ occurs gray in $r_1$ in $\invp\sigmazmi$, but as $v\sigma_i$ contains $u$ gray, $u$ occurs gray in of $r_1\sigma_i$ in $\invp\sigmazi$.
					\qedhere
			\end{itemize}

	\end{itemize}

\end{proof}


\begin{lemma}
	\label{lemma:col_change}
	Let $\inference$ be an inference of a resolution refutation of $\Gamma \cup \Delta$.
	Suppose that a variable $u$ occurs directly below a $\Phi$-symbol as well as directly below a $\Psi$-symbol in $\invp\sigmazi$.
	Then $u$ occurs gray in a gray literal or gray in an equality in $\invp\sigmazi$.
\end{lemma}
\begin{proof}
	We proceed by induction over the refutation.
	As the original clauses each contain symbols of at most one color, the base case is trivially true.

	For the induction step, suppose that an inference makes use of the clauses $C_1, \dots, C_n$ and that the lemma holds for $\PI^*(C_j) \lor C_j$ for $1\varleq j \varleq n$. 

	Note that then, the lemma holds for $\PIstepstarnosigma(\inference, \PI^*(C_1), \dots, \PI^*(C_n)) \lor \nosigma{C} = \inv$.
	This is because as all clauses are variable-disjoint, 
	if a variable occurs in $\inv$ both directly below a $\Phi$-symbol as well as directly below a $\Psi$-symbol, then this must be the case also in 
	$\PI^*(C_j) \lor C_j$ for some $j$, for which the lemma by assumption holds.
	Furthermore, by the definition of $\PI^*$, every literal which occurs in $\PI^*(C_j) \lor C_i$ for some $j$ occurs in $\inv$.

	Hence it remains to show that the lemma holds for $\invp\sigma = \invp\sigma_0\quantifierdots\sigma_m$, which we do by induction over $i$ for $1\varleq i \varleq m$.
	Suppose that the lemma holds for $\invp\sigmazmi$ and in $\invp\sigmazi$, the variable $u$ occurs directly below a $\Phi$-symbol as well as directly below a $\Psi$-term.

	Then by Lemma~\ref{lemma:var_below_phi_symbol}, we can deduce that one of the following statements holds for $\Phi = \Gamma$ as well as $\Phi = \Delta$. We denote case $j$ for $\Phi = \Gamma$ by $j^\Gamma$ and for $\Phi = \Delta$ by $j^\Delta$.

	\begin{enumerate}
		\item
			\label{16_1}
			The variable $u$ occurs directly below a $\Phi$-symbol in $\invp\sigmazmi$.

		\item
			\label{16_4}
			The variable $u$ occurs at a gray position in a gray literal or at a gray position in an equality in $\invp\sigmazi$.

		\item 
			\label{16_2}
			There is a variable $v$ such that 
			{
				\renewcommand{\labelitemi}{\textendash}
				\begin{itemize}
					\item $u$ occurs gray in $v\sigma_i$ and
					\item $v$ occurs in $\invp\sigmazmi$ directly below a $\Phi$-symbol as well as directly below a $\Psi$-symbol
				\end{itemize}
			}
	\end{enumerate}

	If \ref{16_4}$^\Gamma$ or \ref{16_4}$^\Delta$ is the case, we clearly are done.
	On the other hand if \ref{16_2}$^\Gamma$ or \ref{16_2}$^\Delta$ is the case, then by the induction hypothesis, $v$ occurs gray in a gray literal or gray in an equality in $\invp\sigmazmi$. 
	As $u$ occurs gray in $v\sigma_i$, we obtain that then, $u$ occurs gray in a gray literal or gray in an equality in $\invp\sigmazi$.

	Hence the only remaining possibility is that both \ref{16_1}$^\Gamma$
	and \ref{16_1}$^\Delta$ hold.
	But then $u$ occurs directly below a $\Phi$-symbol as well as below a $\Psi$-symbol in $\invp\sigmazmi$ and again by the induction hypothesis, we obtain that $u$ occurs gray in a gray literal or gray in an equality in $\invp\sigmazmi$, and as $\sigma_i$ is trivial on $u$, the same occurrence of $u$ is present in $\invp\sigmazi$.
\end{proof}



\begin{lemma}
	\label{lemma:subterm_in_gray_lit}
	Let $C$ be a clause in a resolution refutation of $\Gamma \cup \Delta$.
	If $\PI^*(C) \lor C$ contains a maximal colored occurrence of a $\Phi$-term $t\occ{s}$, which contains a maximal $\Psi$-colored term $s$, then $s$ occurs gray in $\PI(C) \lor C$.
\end{lemma}
\begin{proof}
	Note that it suffices to show that the desired term occurs in a gray literal or equality in $\PI^*(C) \lor C$
	since by Lemma~\ref{lemma:gray_lits_of_pi_star_in_pi}, all gray literals and equalities of $\PI^*(C)$ also occur in $\PI(C)$.
	We do so by induction over the resolution refutation.
	
	As the original clauses each contain symbols of at most one color, the base case is vacuously true.

	The induction step is laid out similarly as in the proof of Lemma~\ref{lemma:col_change}. 
	We suppose that an inference makes use of the clauses $C_1, \dots, C_n$ and that the lemma holds for $\PI^*(C_j) \lor C_j$ for $1\varleq j \varleq n$. 
	Then the lemma holds for $\inv = \PIstepstarnosigma(\inference, \PI^*(C_1), \dots, \PI^*(C_n)) \lor \nosigma{C})$ as no new terms are introduced in $\inv$ and all literals from $\PI^*(C_j) \lor C_j)$ for $1\varleq j \varleq n$ occur in~$\inv$.

	It remains to show that the lemma holds for $\invp\sigma = \invp\sigma_0 \quantifierdots \sigma_m$, which we do by induction over $i$ for $0 \varleq i \varleq m$.
	We distinguish based on the situation under which a unification leads to the term $t\occ{s}$.

	\begin{itemize}
		\item 
			Suppose for some variable $u$ that $u\sigma_i$ contains $t\occ{s}$. 
			Then $u$ is unified with a term which contains $t\occ{s}$ and which occurs in $\invp\sigmazmi$.
			Hence by the induction hypothesis, $s$ occurs gray in a gray literal or gray in an equality in $\invp\sigmazmi$ and, as $\sigma_i$ does not change this, also in $\invp\sigmazi$.

		\item 
			Otherwise there is a variable $u$ which occurs directly below a $\Phi$-symbol and $v\sigma_i$ contains a gray occurrence of $s$.
			We distinguish based on the occurrences of $u$ in $\invp\sigmazmi$:

			\begin{itemize}
				\item Suppose that $u$ occurs somewhere in $\invp\sigmazmi$ gray in a gray literal or gray in an equality. Then clearly we are done.
				\item Suppose that $u$ occurs somewhere in $\invp\sigmazmi$ directly below a $\Psi$-symbol.
					Then by Lemma~\ref{lemma:col_change}, $u$ occurs gray in a gray literal or gray in an equality in $\invp\sigmazmi$, whose successor in $\invp\sigmazi$ is an occurrence of $s$ of the same coloring. Hence we are done a well.
				\item Suppose that $u$ occurs in $\invp\sigmazmi$ only directly below a $\Phi$-symbol.
					Here, we differentiate between the types of inference of the current induction step:

					\begin{itemize}
						\item
							Suppose that the inference of the current induction step is a resolution or a factorization inference.
							As $u$ occurs gray in $v\sigma_i$, there is a position $p$ such that for the resolved or factorized literals $\lambda$ and $\lambda'$ it holds without loss of generality that $\lambda\atp = u$ and $s$ occurs gray in $\lambda'\atp$.
							Note that $\lambda$ and $\lambda'$ agree on the path to $p$, including the predicate symbol..

							Now as by assumption $u$ only occurs directly below a $\Phi$-symbol, so must $s$.
							But then $s$ occurs directly below a $\Phi$-symbol in $\invp\sigmazmi$ and we get the result by the induction hypothesis.

						\item
							Suppose that the inference of the current induction step is a paramodulation inference.
							Assume it uses the the clauses $C_1: r_1=r_2 \lor D$ and $C_2: E\occatp{r}$ with $\sigma=\mgu(\inference)=\mgu(r_1, r)$ to yield $C: (D\lor E\occatp{r_2})\sigma$.

							As $u$ is affected by $\sigma_i$, it must occur in $r_1$ or $r$. Let $\bhat u$ refer to this occurrence.
							%We distinguish cases based on the symbol below which $\bhat u$ occurs.

							\begin{itemize}
								\item
									Suppose that $\bhat u$ occurs directly below a $\Phi$-colored function symbol. 

									If $\bhat u$ is contained in $r_1$, then $s$ must be contained in $r$ directly below a $\Phi$-colored function symbol as $r_1$ and $r$ are unifiable. We then get the result by the induction hypothesis.

									If otherwise $\bhat u$ is contained in $r$, 
									then there are two possibilities for the occurrence of $s$ in $r_1$:

									Either $\bhat u$ occurs in a $\Phi$-colored function symbol in $r$. Then $s$ occurs in a $\Phi$-colored function symbol in $r_1$ and we get the result by the induction hypothesis.

									Otherwise $\bhat u$ occurs gray in $r$, but $r$ occurs directly below a $\Phi$-colored function symbol in $E$.
									Then however, as $r$ and $r_1$ are unifiable, $s$ must occur gray in $r_1$ and hence gray in an equality.

								\item
									Suppose that $\bhat u$ occurs directly below a $\Phi$-colored predicate symbol. 

									Then as the equality predicate is not considered to be colored, $u$ must occur gray in $r$.
									But then as $r_1$ and $r$ are unifiable, $s$ must occur gray in $r_1$ and hence gray in an equality.
									\qedhere
							\end{itemize}

					\end{itemize}

			\end{itemize}

	\end{itemize}


	%Let $C$ be a clause of a resolution refutation of $\Gamma \cup \Delta$.
	%As the literals occurring in $\PI(C)$ are a subset of the literals occurring in $\PI^*(C)$, the lemma prerequisites hold true only for terms for which they also hold in $\PI^*(C)$.
	%Therefore we can deduce that if a maximal colored $\Phi$-term $t\occ{s}$ containing a maximal $\Psi$-colored term $s$ occurs in $\PI(C)\lor C$, then $t\occ{s}$ also occurs in $\PI^*(C) \lor C$ and by Lemma~\ref{lemma:subterm_in_gray_lit_star},
	%the term $s$ occurs gray in a gray literal or gray in an equality in $\PI^*(C) \lor C$.
\end{proof}

\subsection{Lower bound}

The lemmas of the previous section are now employed to derive a lower bound on the number of quantifier alternations in the interpolant:

\begin{lemma}
	%\label{lemma:col_alt_in_gray_lit_then_quant_alt}
	\label{lemma:quant_alt_lower_bound}
	If a term with $n$ color alternations occurs in $\PI(C)$ or in a gray literal or equality in $C$ for a clause $C$, then the interpolant $I$ produced in Theorem~\ref{thm:two_phases} contains at least $n$ quantifier alternations.
\end{lemma}
\begin{proof}
	We perform an induction on $n$
	and show the strengthening that
	the quantification of the lifting variable which replaces a term with $n$ color alternations is required to be in the scope of the quantification of $n-1$ alternating quantifiers.

	Note that by Corollary~\ref{lemma:gray_lits_and_eq_all_in_PI}, a successor of every literal and equality of $\PI(C)$ and a successor every gray literal or equality of $C$ occurs in $\PI(\pi)$.

	For $n=0$, no colored terms occur in $I$ and hence also no quantifiers.
	Moreover for $n=1$, there are terms of one color which evidently require at least one quantifier.

	Suppose that the statement holds for $n-1$ for $n>1$ and that a term $t$ with $\ca(t) = n$ occurs in $\PI(C) \lor C$.
	We assume without loss of generality that $t$ is a $\Phi$-term.
	Then $t$ contains some $\Psi$-colored term $s$ with $\ca(s) = n-1$ and
	by Lemma~\ref{lemma:subterm_in_gray_lit}, $s$ occurs gray in $\PI(C) \lor C$.
	By Corollary~\ref{lemma:gray_lits_and_eq_all_in_PI}, a successor of $s$ occurs in $\PI(\pi)$. Note that as $s$ occurs in a gray position, any successor of $s$ also occurs in a gray position.

	By the induction hypothesis, the quantification of the lifting variable for $s$ requires $n-1$ alternated quantifiers.
	As $s$ is a subterm of $t$ and $t$ is lifted, $t$ must be quantified in the scope of the quantification of $s$, and as $t$ and $s$ are of different color, their quantifier type is different. 
	Hence the quantification of the lifting variable for $t$ requires $n$ quantifier alternations.
\end{proof}

\begin{comment}
\begin{lemma}
	%\label{prop:color_alt_eq_quant_alt}
	\label{lemma:quant_alt_lower_bound_old}
	If a term with $n$ color alternations occurs in $\PI^*(C) \lor C$ for a clause $C$, then the interpolant $I$ produced in Theorem~\ref{thm:two_phases} contains at least $n-1$ quantifier alternations.
\end{lemma}
\begin{proof}
	By Lemma~\ref{lemma:subterm_in_gray_lit}, a term with $n-1$ color alternations occurs in a gray literal or an equality in $\PI(C) \lor C$.
	Lemma~\ref{lemma:col_alt_in_gray_lit_then_quant_alt} gives the result.
\end{proof}
\end{comment}

We present an example which illustrates that terms in colored literals may contain more color alternations than the term with the maximal number of color alternations in gray literals or equalities.
Still, the latter determines the minimum number of quantifier alternations in the interpolant.
Note that it is a consequence of Lemma~\ref{lemma:subterm_in_gray_lit} that if for some clause $C$ a term with $n$ color alternations occurs in a colored literal in $\PI^*(C) \lor C$ (which contains all literals, i.e.\ also the colored ones), then $\PI(C)\lor C$ contains a term with at least $n-1$ color alternations.

%We present an example which illustrates that the occurrence of a term with $n$ color alternations in $\PI(C) \lor C$ for a clause $C$ can lead to an interpolant with $n-1$ quantifier alternations (but no less as Lemma~\ref{lemma:quant_alt_lower_bound} shows).
\begin{exa}
	Let $\Gamma = \{ \lnot P(a) \}$ and $\Delta = \{ P(x) \lor Q(f(x)), \lnot Q(y) \}$.
	We consider the following refutation of $\Gamma \cup \Delta$, which we annotate by the interpolation extraction by appending $\PI(C)$ to each clause $C$, separated by ``$|$''.
	For the sake of brevity, we sometimes give simplified but logically equivalent versions of $\PI(C)$.
	This notational convention will be used throughout this thesis for examples of a similar form.
	\begin{prooftree}
		\AxiomCm{ \lnot P(a) \mid \bot }
		\AxiomCm{ P(x) \lor Q(f(x)) \mid \top }

		\RightLabelm{\resrule{\resruleres}{x\mapsto a}}
		\BinaryInfCm{ Q(f(a)) \mid \lnot P(a) }

		\AxiomCm{ \lnot Q(y) \mid \top }
		\RightLabelm{\resrule{\resruleres}{y\mapsto f(a)}}
		\BinaryInfCm{ \square \mid \lnot P(a) }
	\end{prooftree}

	In this example, Theorem~\ref{thm:two_phases} yields the interpolant $I \equiv \exists y_a \lnot P(y_a)$ with $\qa(I) =\nolinebreak 1$.
	The existence of the term $a$ with $\ca(a) = 1$ in a clause of the refutation by Lemma~\ref{lemma:quant_alt_lower_bound} implies that $\qa(I) \vargeq 1$.
	The occurrence of the term $f(a)$ with $\ca(f(a)) = 2$ in the colored literal $Q(f(a))$ is not relevant.
\end{exa}

\subsection{Upper bound and conclusion}

We now also determine an upper bound for the number of quantifier alternations in the interpolant.

Note that as the following example shows, 
an upper bound of $n$ quantifier alternations in the interpolant is not sufficient even if $n$ is the maximal number of color alternations for any term in $\PI(C) \lor C$ for any clause $C$:

\begin{exa}
	Let $\Gamma = \{ P(a) \lor Q(u) \}$ and $\Delta = \{ \lnot P(v), \lnot Q(b) \}$.
	Consider the following refutation of $\Gamma \cup \Delta$:
	\begin{prooftree}
		\AxiomCm{ P(a) \lor Q(u) \mid \bot }
		\AxiomCm{ \lnot P(v) \mid \top }

		\RightLabelm{\resrule{\resruleres}{v\mapsto a}}
		\BinaryInfCm{ Q(u) \mid P(a) }

		\AxiomCm{ \lnot Q(b) \mid \top }
		\RightLabelm{\resrule{\resruleres}{u\mapsto b}}
		\BinaryInfCm{ \square \mid Q(b) \lor P(a) }
	\end{prooftree}

	Given this refutation, Theorem~\ref{thm:two_phases} produces either the interpolant
	$I_1 \equiv \exists y_a \forall x_b ( Q(x_b) \lor P(y_a) )$
	or 
	$I_2 \equiv \forall x_b \exists y_a ( Q(x_b) \lor P(y_a) )$.
	Note that the maximal number of color alternations of a term in $\PI(C) \lor C$ for any clause $C$ is $1$, but the number of quantifier alternations is $2$ for both $I_1$ and $I_2$.
\end{exa}

However the following bound holds in general:

\begin{lemma}
	\label{lemma:quant_alt_upper_bound}
	Let $t$ be a term with the maximal number of color alternations in $\PI(C)$ or a gray literal or equality in $C$ for any clause $C$.
	Then there is an arrangement of the quantifier prefix in Theorem~\ref{thm:two_phases} which gives rise to an interpolant 
	with at most $\ca(t)+\nolinebreak{}1$ quantifier alternations.
\end{lemma}
\begin{proof}
	By Corollary~\ref{lemma:gray_lits_and_eq_all_in_PI}, a successor of $t$ occurs in $\PI(\pi)$.
	Let $T_i^\Phi$ be the set of maximal $\Phi$-colored terms in $\PI(\pi)$ with $i$ color alternations for $1\varleq i \varleq n$, where $n=\ca(t)$.
	Note that every maximal colored term of $\PI(\pi)$ is contained in one of these sets.
	%We use $Q T_i^\Phi$ to denote $ Q_1 z_{t_1} \dots Q_n z_{t_n}$ where $t_1, \dots, t_n$ is an arrangement of the elements of $T_i^\Phi$ in ascending subterm order  and $Q_i z_{t_i}$ for $1\varleq i \varleq n$ is $\forall x_{t_i}$ or $\exists y_{t_i}$ depending on the color of $t_i$.
	We use $\exists T_i^\Gamma$ $(\forall T_i^\Delta)$ to denote $ \exists y_{t_1} \dots \exists y_{t_m}$ $(\forall x_{t_1}\dots\forall x_{t_m})$ where $t_1, \dots, t_m$ is an arrangement of the elements of $T_i^\Gamma$ ($T_i^\Delta$) in ascending subterm order. 


	Now we construct the interpolant
	\[
		I \equiv 
		\forall T_1^\Delta \exists T_1^\Gamma \;
		\exists T_2^\Gamma \forall T_2^\Delta \;
		\forall T_3^\Delta \exists T_3^\Gamma
		\dots 
		Q^\Phi T_n^\Phi \Q^\Psi T_n^\Psi
		\;
		\lifgamma{\lifdelta{\PI(\pi)}},
	\]
	where $ Q^\Phi T_n^\Phi \Q^\Psi T_n^\Psi $ is
	$\forall T_n^\Delta \exists T_n^\Gamma $ if $n$ is odd and 
	$\exists T_n^\Gamma \forall T_n^\Delta $ if $n$ is even.
	Clearly, $I$ has at most $n+1$ color alternations.

	In order to show the result, it remains to show that $I$ is a valid interpolant with respect to Theorem~\ref{thm:two_phases}. 
	Note that the quantifier prefix binds all lifting variables occurring in  
	$\lifgamma{\lifdelta{\PI(\pi)}}$.
	We conclude by showing that the order of the quantifiers is admissible.

	Let $t$ be a maximal colored term in $\lifgamma{\lifdelta{\PI(\pi)}}$. 
	We prove that the quantifier for the lifting variable of every subterm $s$ of $t$ precedes the quantifier for the lifting variable for $t$ in $I$.
	Suppose that $\ca(t) = k$. Then we can deduce that $\ca(s) \varleq\nolinebreak{} k$.
	\begin{itemize}
		\item
			If $\ca(s) = k$, then $t$ and $s$ are of the same color and hence the quantifiers for their respective lifting variables are contained in the same block. 
			However the quantifiers of each block are ordered as desired.
			%their in the same quantifier block
		\item
			Otherwise $\ca(s) = l$ for some $l$ such that $l < k$.
			Then the lifting variable replacing $s$ is quantified in
			$\exists T_{l}^\Gamma$ or
			$\forall T_{l}^\Delta$.
			In any case, it precedes the quantifier for the lifting variable replacing $t$ which is contained in 
			$\exists T_{k}^\Gamma$ or
			$\forall T_{k}^\Delta$.
			%$Then by the arrangement of the quantifiers in~$I$ the quantifier for the lifting variable replacing $s$ precedes the quantifier for the lifting variable replacing $t$.
			\qedhere
	\end{itemize}
\end{proof}

The previous results can be summarized by the following theorem:\nopagebreak

\begin{thm}
	Let $n$ be the maximal number of color alternations of any term in $\PI(C)$ or in a gray literal or equality in $C$ for any clause $C$ of a resolution refutation of $\Gamma \cup \Delta$.
	Then by arranging the quantifiers in a quantifier alternation minimizing fashion the interpolant of Theorem~\ref{thm:two_phases} has at least $n$ and at most $n+1$ quantifier alternations.
\end{thm}
\begin{proof}
	Immediate by Lemma~\ref{lemma:quant_alt_lower_bound} and Lemma~\ref{lemma:quant_alt_upper_bound}.
\end{proof}





}


\newcommand{\content}{
	\chapter{Introduction}
	

The notion of interpolation has been introduced by Craig in \cite{Craig57linear}.
Loosely speaking, given two formulas such that one implies the other, an interpolant is implied by the first and itself implies the latter.
Hence it in some sense captures the logical content of the first formula which necessarily makes the latter true and therefore acts as a link between the original formulas.


\begin{figure}[htbp]
	\centering
	\begin{tikzpicture}[
			implies/.style={double,double equal sign distance,-implies},
			mynode/.style={draw,circle,outer sep=3pt}
		]
		\node[mynode] (A) at (0,0) {$A$};
		\node[mynode] (B) at (4,0) {$B$};
		\node[mynode] (I) at (2,-1.5) {$I$};

		%\draw[->,implies] (A) to (B);
		\draw (A) edge[implies]  (B);
		\draw (A) edge[implies]  (I);
		\draw (I) edge[implies]  (B);

	\end{tikzpicture}
	\caption{Given two formulas $A$ and $B$ such that $A$ implies $B$, an interpolant is a formula $I$ which is implied by $A$ and implies $B$.}
\end{figure}

\mytodo{use A, B in the text?}

Moreover, interpolants are not arbitrary formulas, but their language is restricted to those symbols, which are common to both original formulas.
Thus they represent the logical connection solely by statements on notions, which are relevant to both original formulas.

As Craig has shown that interpolants always exist, they represent a justification for material implication in classical logic:
If under any circumstance an implication in classical logic holds, then there is a formula which contains the logical content explaining this implication.
Or conversely, if such a summary of a potential implication does not exist, the implication does not hold in general.
Furthermore, if formulas are concerned with different matters (such that their language is disjoint), there certainly can not be a logical relation among them, as for such formulas, no interpolant can be found.

Craig interpolation has been and still is studied with respect to a wide variety of logics.
Most notably, it holds for propositional and first-order logic.
This fact can be proven by different means:
Interpolants can be directly extracted from proofs of logical relations of formulas thus showing their existence in a constructive manner.
Alternatively, also semantic arguments for the existence of interpolants can be brought up:
Assuming the non-existence of interpolants, one can build a model contradicting an assumed logical relation of the original formulas.

The applications of Craig interpolation are manifold:
As a theoretic tool, it can for instance be employed to proof Beth's definability theorem or also to show lower bounds on the length of proofs of propositional proof systems (\cite{Pudlak97,krajivcek1997interpolation}).
In recent years, it has been discovered that interpolants serve well in the area of model checking as a means to find formulas overapproximating reachable states of a program (\cite{McMillan03}), which is now an active area of research.
Furthermore, in the field of program analysis, there are also approaches making use of interpolation to extract information about the changes of program state inflicted by loop iterations in order to detect loop invariants  (\cite{weissenbacher2010}).







~

~

verschiedene logiken, insbes prop + fol

konstruktive beweistheoretische ansätze aber auch modelltheoretisch

korollary: beth; andere anwendungen: invariant generation, etc
description logic (talk von workshop)
uniform interpolation?

\url{http://en.wikipedia.org/wiki/Craig_interpolation}

\url{http://math.stanford.edu/~feferman/papers/craigtransps.pdf}

auch was von otto paper: an interpolation theorem


	\chapter{Interpolation and proof theory}
	\label{chap:interpolation_and_proof_theory}
	In this chapter, we introduce basic technical notions (\ref{sec:preliminaries}) in order to then formulate the interpolation theorem (\ref{sec:interpolation}).
	We furthermore present strengthenings of the theorem\nolinebreak{} (\ref{sec:strengthenings}) as well as an application in the form of Beth's definability theorem\nolinebreak{} (\ref{sec:beth}).
	This result is used in discussing the failure of interpolation in higher-order logic\nolinebreak{} (\ref{sec:interpol_hol}).
	We then continue to define the calculi, which will be used throughout this thesis (\ref{sec:resolution} and \ref{sec:lk}) including considerations on the applicability of interpolation to them (\ref{sec:resolution_and_interpolation}). 

	\section{Preliminaries}
\todo[inline]{this section contains all the required notation but will just be written up nicely at the end}

The language of a first-order formula $A$ is denoted by $\Lang(A)$ and contains all predicate, constant and function symbols that occur in $A$.
These are also referred to as the \emph{\mbox{non-logical} symbols} of $A$.
The \emph{logical symbols} on the other hand include all logical connectives, quantifiers, the equality symbol ($=$) as well as symbols denoting truth ($\top$) and falsity ($\bot$).

For formulas $A_1, \ldots, A_n$, $\Lang(A_1, \ldots, A_n) = \bigcup_{1\leq i \leq n} \Lang(A_i)$.

A term $s$ is a subterm of a term $t$ if $s$ occurs in $t$. $s$ is a strict subterm of $t$ if $s$ is a subterm of $t$ and $s \neq t$.

An occurrence of $\Phi$-term is called \emph{maximal} if it does not occur as subterm of another $\Phi$-term.
An occurrence of a colored term $t$ is a maximal colored term if it does not occur as subterm of another colored term.
\todo{colors are only defined later}


We denote $x_1, \ldots, x_n$ by $\bar x$.

For a set of formulas $\Phi$, $\lnot \Phi$ denotes $\{\lnot A \mid A \in \Phi\}$.

A substitution is a mapping of variables to terms. It is denoted by $\phi\subst{x/t}$, where $\phi$ is a formula or term where each occurrence of the variable $x$ is replaced by the term $t$.
A substitution $\sigma$ is called trivial on $x$ if $x\sigma = x$. Otherwise it is called non-trivial.

An \defiemph{abstraction} on the other hand is a mapping of terms to variables. It is denoted by $\phi\abstractionOp{t/x}$, where $\phi$ is a formula or term where each occurrence of the term $t$ is replaced by the variable $x$.

A term $s$ is an \defiemph{instance} of a term $t$ if there exists a substitution $\sigma$ such that $t\sigma = s$.
If $s$ is an instance of $t$, then $t$ is an \defiemph{abstraction} of $s$. Note that the abstraction- and instance-relation are reflexive. 
$s$ is a \defiemph{proper} instance (abstraction) of $t$ if $s$ is an instance (abstraction) of $t$ and $s\neq t$.

The length of a term or formula $\phi$ is the number of logical and non-logical symbols in $\phi$.

$A\occurat{s}{p}$ denotes $A$ with an occurrence of $s$ at position $p$.

$A\occur{s}$ denotes $A$ where the term $s$ occurs on some set of positions $\Phi$. $A\occur{t}$ denotes $A\occur{s}$ where on each position in $\Phi$, $s$ has been replaced by $t$. Due to its vagueness, this notation is mostly used in order to emphasis that the term $s$ does occur in $A$ in some way.

TODO: define $\Sigma$ as subformula set; possibly remove definition in chapter 2

TODO: define $\equiv$ as syntactic equality. Define also $\semiff$, $\liff$.

TODO: define what we mean by model and free variables.
(need universal quantification of free vars)

TODO: define ground, non-ground

TODO: define set-notation of unifiers

TODO: define infinite-domain unifiers

TODO: define range and domain of substitutions

TODO: define prenex formulas with matrix and prefix

TODO: define prefix of term position: e.g. u in f(c, g(u, x, h(a))) has the prefix f(., g(., ., .)), or possible written as sequence of symbols (algo: always go to parent starting at u)

TODO: abstraction has 2 contradictory definitions above

TODO: dei


\subsection{Unification}


\begin{defi}[Unification algorithm]
	Let $\id$ denote the identity function and $\textbf{fail}$ be returned by $\mgu$ in case the arguments are not unifiable to signify that the $\mgu$ of the given arguments is not defined. We treat constants as $0$-ary functions.
	Let $s$ and $t$ denote terms and $x$ a variable.

	The most general unifier $\mgu$ of two literals $l = A(s_1,\dotsc, s_n)$ and $l' = A(t_1,\dots, t_n)$ is defined to be $\mgu(\{ (s_1, t_1), \dotsc, (s_n, t_n)\})$.


	The $\mgu$ for a set of pairs of terms $T$ is defined as follows:

	$
	\mgu(\emptyset) = \id
	$

	\newcommand{\aatahfdgasdfg}{.4\textwidth}
	$
	\mbox{$\mgu(\{t\} \cup T)$} =
	%\mgu(\{t\} \cup T) =$
	\begin{cases}
		\mathbf{fail} 				& \parbox[t]{\aatahfdgasdfg}{if $t = (x, s)$ or $t=(s,x)$ and $x$ occurs in $s$ but $x\neq s$ } \\
		\mgu(T\subst{x/s})\subst{x/s} \cup \{x\mapsto s\} 		& \parbox[t]{\aatahfdgasdfg}{if $t = (x, s)$ or $t=(s,x)$ and $x$ does not occur in $s$ or $x=x$} \\
		\mathbf{fail} 				& \parbox[t]{\aatahfdgasdfg}{if $t = (f(s_1,\dotsc,s_n), g(s_1,\dotsc,s_n))$ with $f\neq g$} \\
		\mgu(T \cup \{(s_1, t_1), \dotsc, (t_n, s_n)\})		& \parbox[t]{\aatahfdgasdfg}{if $t = (f(s_1,\dotsc,s_n), f(t_1,\dotsc,t_n))$} \\
		\mgu(T) 							& \text{if $t=(s, s)$} \\
	\end{cases}
	$
	\qedhere
\end{defi}


\largered{ FIX UP DEF BELOW depending on conj }

\begin{defi}[$\mguarr$]
	Let $l$ and $l'$ be literals such that $\mgu(l, l')$ is defined.
	We write $x\mguarr t$ for a variable $x$ and a term $t$ if in the course of the application of the unification algorithm, the substitution $x\mapsto s$ is added to the result. 
\end{defi}
Note that $s$ may still contain variables which are substituted at a later stage, such that with $\sigma = \mgu(l, l')$, possibly $x\sigma \neq s$.
This notation is convenient for reasoning about the 




	


\section{Craig Interpolation}

\todo[inline]{TODO: write some text about what interpolation means and that we prove more or less only reverse interpolation, but that's fine by the proposition }

\begin{defi}
	\label{def:interpolant}
	Let $\Gamma$ and $\Delta$ be sets of first-order formulas.
	An \defiemph{interpolant} of $\Gamma$ and $\Delta$ is a first-order formula $I$ such that 
	\begin{enumerate}
		\item $ \Gamma \entails I$ \label{int_1}
		\item $ I \entails \Delta $  \label{int_2}
		\item $ \Lang(I) \subseteq \Lang(\Gamma) \intersect \Lang(\Delta)$.  \label{int_3}
	\end{enumerate}

	\begin{samepage}
		A \defiemph{reverse interpolant} of $\Gamma$ and $\Delta$ is a first-order formula $I$ such that $I$ meets conditions \ref{int_1} and \ref{int_3} of an interpolant as well as:
		\begin{enumerate}[\quad\:1'.]
				\setcounter{enumi}{1}
			\item $ \Delta \entails \lnot I $  \label{int_2prime}
				\qedhere
		\end{enumerate}
	\end{samepage}
\end{defi}

%\begin{thm}[Interpolation]
\begin{restatable}[Interpolation]{thm}{interpolationThm}
	\label{thm:interpolation_original}
	Let $\Gamma$ and $\Delta$ be sets of first-order formulas such that $ \Gamma \entails \Delta $.
	Then there exists an interpolant for $\Gamma$ and $\Delta$.
%\end{thm}
\end{restatable}
%
%\begin{thm}[Reverse Interpolation]
\begin{restatable}[Reverse Interpolation]{thm}{interpolationRevThm}
		\label{thm:interpolation}
		Let $\Gamma$ and $\Delta$ be sets of first-order formulas such that $ \Gamma \cup \Delta $ is unsatisfiable.
		Then there exists a reverse interpolant for $\Gamma$ and\nolinebreak{} $\Delta$.
\end{restatable}



\begin{prop}
	Theorem \ref{thm:interpolation_original} and \ref{thm:interpolation} are equivalent.
	\label{prop:interpolations_equivalent}
\end{prop}
\begin{proof}
	Let $\Gamma$ and $\Delta$ be sets of first-order formulas such that $ \Gamma \entails \Delta$.
	Then $\Gamma \cup \lnot \Delta$ is unsatisfiable.
	By Theorem \ref{thm:interpolation}, there exists a reverse interpolant $I$ for $\Gamma$ and $\lnot \Delta$.
	As $\lnot \Delta \entails \lnot I$, we get by contraposition that $I \entails \Delta$, hence $I$ is an interpolant for $\Gamma$ and $\Delta$

	For the other direction,
	let $\Gamma$ and $\Delta$ be sets of first-order formulas such that $ \Gamma \cup \Delta$ is unsatisfiable.
	Then $\Gamma \entails \lnot \Delta$, hence by Theorem \ref{thm:interpolation_original}, there exists an interpolant $I$ of $\Gamma$ and $\lnot \Delta$.
	But as thus $ I\entails \lnot \Delta$, we get by contraposition that $\Delta \entails \lnot I$, so $I$ is a reverse interpolant for $\Gamma$ and $\Delta$.
\end{proof}

As the notions of interpolation and reverse interpolation in this sense coincide, we will in the following only speak of interpolation where  will be clear from the context which definition applies.

\begin{lemma}
	\label{lemma:logically_equivalent_sets}
	Let $\Gamma, \Gamma', \Delta, \Delta'$ be sets of first order formulas such that $\Gamma \semiff \Gamma'$ and $\Delta \semiff \Delta'$ and $\Lang(\Gamma) \cap \Lang(\Delta) = \Lang(\Gamma') \cap \Lang(\Delta')$.
	Then $I$ is an interpolant for $\Gamma$ and $\Delta$ if and only if $I$ is an interpolant for $\Gamma'$ and $\Delta'$.
\end{lemma}
\begin{proof}
	Clearly $\Gamma \entails I$ holds if and only if $\Gamma' \entails I$ and similarly
	$\Delta \entails \lnot I$ holds if and only if $\Delta' \entails \lnot I$.
	As the intersections of the respective languages coincide, the language condition on $I$ is satisfied in both directions.
\end{proof}

\begin{remark}
	In Lemma \ref{lemma:logically_equivalent_sets}, it is not sufficient to require that $\Gamma \semiff \Gamma'$ and $\Delta \semiff \Delta'$. 
	Consider the example where
	$\Gamma = \Delta = \{ \forall x (x=c)\}$
	and 
	$\Gamma' = \Delta' = \{ \forall x (x=d)\}$.
	Then even though $\Gamma$ and $\Gamma'$ as well as $\Delta$ and $\Delta'$ have the same models,
	$\Lang(\Gamma) \cap \Lang(\Delta) = \{c\}$
	whereas
	$\Lang(\Gamma') \cap \Lang(\Delta') = \{d\}$.
	Therefore $\forall x (x=c)$ is an interpolant for $\Gamma$ and $\Delta$ but not for $\Gamma'$ and $\Delta'$.
\end{remark}

In the context of interpolation, every non-logical symbol is assigned a color which indicates its origin(s). 
\begin{defi}[Coloring]
A non-logical symbol is said to be \defiemph{$\Gamma$ ($\Delta$)-colored} if it only occurs in $\Gamma$ ($\Delta$) and \defiemph{grey} in case it occurs in both $\Gamma$ and $\Delta$. A symbol is \defiemph{colored} if it is $\Gamma$- or $\Delta$-colored.
A term is a \defiemph{$\Phi$-term} if its outermost symbol is $\Phi$-colored.

A term $t$ is \defiemph{single-colored} if $t$ is $\Phi$-colored for some $\Phi$ and all colored symbols in $t$ are $\Phi$-colored.
A term $t$ is \defiemph{multi-colored} if $t$ is $\Phi$-colored for some $\Phi$ and $t$ contains a term which is colored but not $\Phi$-colored.\todo{check if multi-colored is even used}
Note that a multi-colored $\Phi$-term consequently is a term whose outermost symbol is $\Phi$-colored and contains a colored but not $\Phi$-colored subterm.

  An occurrence of a term $t$ is called \defiemph{$\Phi$-colored} if $t$ is contained in a maximal $\Phi$-colored term.
	$t$ is called \defiemph{colored} if $t$ is of any color and \defiemph{grey} otherwise.

	A variable is a \defiemph{color-changing} variable if it occurs both in a single-colored $\Phi$-term and a single-colored $\Psi$-term in a given context.
\end{defi}

We sometimes use $\Phi$ and $\Psi$ as colors to emphasise that the reasoning at hand is valid irrespective of the actual color assignment and solely assuming that $\Phi \neq \Psi$. 
%both if $\Phi = \Gamma$ and $\Psi = \Delta  $ as well as if $\Phi = \Delta$ and $\Psi = \Delta$.


\subsection{Degenerate cases}
In this thesis, the equality symbol as well as the symbols for truth and falsity are regarded as a logical symbol. 
This is justified by the following examples, which are referred to in \cite[Example 20.2 and 20.4]{boolos2007computability} as ``failure of interpolation'' and ``degenerate cases'' respectively:

\begin{exa}
	Let $\Gamma = \{ a=b \} $ and $\Delta = \{P(a), \lnot P(b)\}$.
	Note that here, the intersection of $\Lang(\Gamma)$ and $\Lang(\Delta)$ does not contain a predicate symbol.
	By regarding $=$ as logical symbol and therefore permitting it to occur in an interpolant despite the fact that it does not occur in $\Delta$ allows for the interpolant $a=b$.
	If we would not permit $=$ in the interpolant, there would be no interpolant for $\Gamma$ and $\Delta$, even though $\Gamma \cup \Delta$ clearly is unsatisfiable.
\end{exa}

\begin{exa}
	Let $\Gamma = \{ P(a) \land \lnot P(a) \} $ and $\Delta = \emptyset$.
	As clearly the intersection of $\Lang(\Gamma)$ and $\Lang(\Delta)$ is empty, we may form an interpolant only of logical symbols.
	In this instance, the interpolant must be either $\bot$ or a formula logically equivalent to $\bot$.
	By merely swapping $\Gamma$ and $\Delta$, we arrive at a situation where the interpolant must be equivalent to $\top$.

	Note that as we can express a formulas, which are logically equivalent to $\bot$ and $\top$ respectively by employing the equality symbol\footnote{$\forall x\,x\neq x$ and $\forall x\,x=x$ are suitable examples for $\bot$ and $\top$ respectively.}, the symbols for truth and falsity are not strictly required to be regarded as logical symbols for the interpolation theorem to hold.
\end{exa}


\section{Beth's definability theorem}
\label{sec:beth}

In this section, we illustrate the interpolation theorem by presenting Beth's definability theorem, which admits a straightforward proof by means of the interpolation theorem. 
The definability theorem deals with definitions of predicates by means of formulas and bridges the gap between implicit definitions, where predicates are defined by its use, and explicit definitions, which define a formula by means of another formula, by even showing their equivalence.
This is given significance by the circumstance that implicit definitions occur in mathematics, but are by this theorem in no sense weaker than explicit defintions.

Its original publication in \cite{beth1953} precedes Craig's papers on interpolation (\cite{Craig57linear,Craig57three}) by four years and hence relies on a direct proof.
 

\begin{defi}[Implicit and explicit definition]
	Let $\LangSym$ be a first-order language and
	$P$ and $P'$ be two fresh predicate symbols of arity $n$.
	Let $\Gamma_P$ be a set of first-order formulas\todo{open or closed?}{}
	in the language $\LangSym\cup\{P\}$ 
	and $\Gamma_{P'}$ the same set of formulas with every occurrence of $P$ in $\Gamma_P$ replaced by\nolinebreak{} $P'$, such that the language of $\Gamma_{P'}$ is $\LangSym \cup\{P'\}$.

	$\Gamma_P$ defines $P$ implicitly iff
	\[\Gamma_P \cup \Gamma_{P'} \entails \forall x_1\quantifierdots \forall x_n \left(  P(x_1, \dots, x_n) \liff P'(x_1, \dots, x_n)\right).\]
	On the other hand $\Gamma_P$ defined $P$ explicitly iff there is formula $\varphi$ in $\LangSym$ with $\FV(\varphi) = \{x_1, \dots, x_n\}$ such that 
	\[\Gamma_P\entails \forall x_1\quantifierdots \forall x_n \left(  P(x_1, \dots, x_n) \liff \varphi\right).\qedhere\]
\end{defi}

Note that the definition of implicit definitions is essentially second-order 
and 
can be expressed by the second-order sentence
%\[ \forall P\,\forall P' \left(  \left(\bigwedge_{\varphi \in \Gamma^*_P \cup \Gamma^*_{P'}} \varphi \right) \limpl P=P'\right),\] 
\begin{samepage}
\[ \forall P\,\forall P' \left(  \left(\Gamma^*_P \land \Gamma^*_{P'} \right) \limpl P=P'\right),\] 
where $\Gamma^*_P$ and $\Gamma^*_{P'}$ are conjunctions of the formulas of 
respective reductions of $\Gamma_P$ and $\Gamma_{P'}$ 
to finite sets, which exist by the compactness theorem.
%by reducing $\Gamma_P$ and $\Gamma_{P'}$ to finite sets $\Gamma^*_P$ and $\Gamma^*_{P'}$ by a suitable application of the compactness theorem
% cite chang/keisler page 324
\end{samepage}

\begin{thm}[Beth's definability theorem]
	$\Gamma_P$ defines $P$ explicitly if and only if $\Gamma_P$ defines $P$ implicitly.
\end{thm}
\begin{proof}
	Suppose that $\Gamma_P$ defines $P$ explicitly. 
	Then there exists some formula $\varphi$ such that 
	$\Gamma_P\entails \forall x_1\quantifierdots \forall x_n (  P(x_1, \dots, x_n) \liff \varphi)$.
	But then it also holds that 
	$\Gamma_{P'}\entails \forall x_1\quantifierdots \forall x_n (  P'(x_1, \dots, x_n) \liff \varphi)$, hence
	$\Gamma_{P} \cup \Gamma_{P'} \entails \forall x_1\quantifierdots \forall x_n (P(x_1, \dots, x_n) \liff P'(x_1, \dots, x_n))$.
	Therefore $\Gamma_P$ is an implicit definition of $P$.

	For the other direction, suppose that $\Gamma_P$ defines $P$ implicitly. 
	Then
	$\Gamma_P \cup \Gamma_{P'} \entails\allowbreak \forall x_1\quantifierdots \forall x_n (  P(x_1, \dots, x_n) \liff P'(x_1, \dots, x_n))$.
	It follows from the compactness theorem that
	we can find a conjunction $\Gamma^*_{P'}$ of formulas of a finite subset of $\Gamma_{P'}$ such that  
	$\Gamma_P \cup \{\Gamma^*_{P'}\} \entails \forall x_1\quantifierdots \forall x_n (  P(x_1, \dots, x_n) \liff P'(x_1, \dots, x_n))$.
	%Let $\gamma_{P'}$ be the conjunction of all formulas in $\Gamma^*_{P'}$ and 
	Let $y_1, \dots, y_n$ be fresh variables.
	Then we obtain by the deduction theorem that  
	$\Gamma_P \cup \{P(y_1, \dots, y_n)\} \entails \Gamma^*_{P'} \limpl  P'(y_1, \dots, y_n)$.

	Note that $P$ only occurs in the antecedent and $P'$ only occurs in the consequent.
	Hence we can apply the Interpolation Theorem~\ref{thm:interpolation_original} in order to obtain a formula $\chi$
	such that
	$\Gamma_P \cup \{P(y_1, \dots, y_n)\} \entails \chi$ and
	$\chi \entails \Gamma^*_{P'} \limpl  P'(y_1, \dots, y_n)$,
	while additionally $\Lang(\chi) = \Lang(\Gamma_P) \cap \Lang(\Gamma_{P'})$. This implies that neither $P$ nor $P'$ occur in\nolinebreak{} $\chi$.

	Now we apply the deduction theorem another time and get that
	\markA{} $\Gamma_P \entails P(y_1, \dots, y_n) \limpl \chi$ and
	$\Gamma^*_{P'} \entails \chi \limpl  P'(y_1, \dots, y_n)$.
	As $\Gamma^*_{P'}$ implies $\Gamma_{P'}$, we also have that
	$\Gamma_{P'} \entails \chi \limpl  P'(y_1, \dots, y_n)$.
	Since $P$ does not occur in this entailment, it remains valid if we replace every occurrence of the symbol $P'$ by $P$
	and obtain that
	\markB{} $\Gamma_{P} \entails \chi \limpl  P(y_1, \dots, y_n)$.

	But then \markA{} and \markB{} imply that 
	$\Gamma_{P} \entails \chi \liff  P(y_1, \dots, y_n)$, which is equivalent to
	$\Gamma_{P} \entails \forall y_1\quantifierdots\forall y_n \left( \chi \liff  P(y_1, \dots, y_n)\right)$.
	So clearly $\Gamma_P$ defines $P$ explicitly.
\end{proof}




	\section{Strengthenings of the interpolation theorem}

After Craig's initial result, several stronger versions of the theorem have been published.
\cite{Craig57three} can already be counted among those,
as it defines interpolants equivalently to our Definition~\ref{def:interpolant}, 
but the first publication in \cite{Craig57linear} restricts interpolants only with regard to their predicate symbols, but allows non-common function and constant symbols to occur in it.
This is relevant as later results on the interpolation theorem are only based on \cite{Craig57linear}, which usually is not to be understood as proper restriction of the result as it goes through with the extension of \cite{Craig57three} as well.

%\hl{in lit, first version is used sometimes (\cite{lyndon59}, \cite{Henkin63})} 

Arguably one of the most important strenghtenings is due Lyndon. In \cite{lyndon59}, he showed that the interpolation theorem holds for the following definition of interpolant:

\begin{defi}[Lyndon interpolant]
Let $\Gamma$ and $\Delta$ be sets of first-order formulas. A \defiemph{Lyndon interpolant} of $\Gamma$ and $\Delta$ is a first-order formula I such that the conditions \ref{int_1} and \ref{int_2} of Definition~\ref{def:interpolant} as well as the following:
\begin{enumerate}[\quad\:1'.]
	\setcounter{enumi}{2}
	\item Each predicate symbol occurring positively (negatively) in $I$ occurs positively (negatively) in both $\Gamma$ and $\Delta$.
\end{enumerate}
\end{defi}

proof of this:

in \cite{lyndon59}, based on cut eliminiation (just like craig) and also herbrand's theorem.

in \cite{Henkin63} with a proof that extends \cite{sec:joint_consistency}

~



erroneous treatment of equality in lyndon (cf.\ \cite{motohashi84})

but 

In \cite{oberschelp68}, a related restriction is proved for the equality predicate.

provides information about  degenerate cases (see \cite{})

also corollary about interpolants with no equality


TODO: martin otto ; slagle proof of lyndon




\section*{notes}

henkin gives nice proof basically just like our semantic proof, but with polarity, then extends it to equality.
he shows that interpolation holds with equality, but adds axioms (doesn't say where), so equality has to be allowed in any interpolant.
also further strenghtenings based on nicely writing the formula in some nnf-like style and then defining a form of an interpolant.


	

\section{Beth's definability theorem}
\label{sec:beth}

In this section, we illustrate the interpolation theorem by presenting Beth's definability theorem, which admits a straightforward proof by means of the interpolation theorem. 
The definability theorem deals with definitions of predicates by means of formulas and bridges the gap between implicit definitions, where predicates are defined by its use, and explicit definitions, which define a formula by means of another formula, by even showing their equivalence.
This is given significance by the circumstance that implicit definitions occur in mathematics, but are by this theorem in no sense weaker than explicit definitions.

Its original publication in \cite{beth1953} precedes Craig's papers on interpolation (\cite{Craig57linear,Craig57three}) by four years and relies on a direct proof.
 

\begin{defi}[Implicit and explicit definition]
	Let $\LangSym$ be a first-order language and
	$P$ and $P'$ be two fresh predicate symbols of arity $n$.
	Let $\Gamma_P$ be a set of first-order sentences
	in the language $\LangSym\cup\{P\}$ 
	and $\Gamma_{P'}$ the same set of formulas with every occurrence of $P$ in $\Gamma_P$ replaced by\nolinebreak{} $P'$, such that the language of $\Gamma_{P'}$ is $\LangSym \cup\{P'\}$.

	$\Gamma_P$ defines $P$ implicitly iff
	\[\Gamma_P \cup \Gamma_{P'} \entails \forall x_1\quantifierdots \forall x_n \left(  P(x_1, \dots, x_n) \liff P'(x_1, \dots, x_n)\right).\]
	On the other hand $\Gamma_P$ defined $P$ explicitly iff there is formula $\varphi$ in $\LangSym$ with $\FV(\varphi) = \{x_1, \dots, x_n\}$ such that 
	\[\Gamma_P\entails \forall x_1\quantifierdots \forall x_n \left(  P(x_1, \dots, x_n) \liff \varphi\right).\qedhere\]
\end{defi}

Note that the definition of implicit definitions is essentially second-order 
and 
can be expressed by the second-order sentence
%\[ \forall P\,\forall P' \left(  \left(\bigwedge_{\varphi \in \Gamma^*_P \cup \Gamma^*_{P'}} \varphi \right) \limpl P=P'\right),\] 
\begin{samepage}
\[ \forall P\,\forall P' \left(  \left(\Gamma^*_P \land \Gamma^*_{P'} \right) \limpl P=P'\right),\] 
where $\Gamma^*_P$ and $\Gamma^*_{P'}$ are conjunctions of the formulas of 
respective reductions of $\Gamma_P$ and $\Gamma_{P'}$ 
to finite sets, which exist by the compactness theorem.
%by reducing $\Gamma_P$ and $\Gamma_{P'}$ to finite sets $\Gamma^*_P$ and $\Gamma^*_{P'}$ by a suitable application of the compactness theorem
% cite chang/keisler page 324
\end{samepage}

\begin{thm}[Beth's definability theorem]
	\label{thm:beth}
	$\Gamma_P$ defines $P$ explicitly if and only if $\Gamma_P$ defines $P$ implicitly.
\end{thm}
\begin{proof}
	Suppose that $\Gamma_P$ defines $P$ explicitly. 
	Then there exists some formula $\varphi$ such that 
	$\Gamma_P\entails \forall x_1\quantifierdots \forall x_n (  P(x_1, \dots, x_n) \liff \varphi)$.
	But then it clearly also holds that 
	$\Gamma_{P'}\entails \forall x_1\quantifierdots \forall x_n (  P'(x_1, \dots, x_n) \liff \varphi)$,
	hence
	\[
	\Gamma_{P} \cup \Gamma_{P'} \entails \forall x_1\quantifierdots \forall x_n (P(x_1, \dots, x_n) \liff P'(x_1, \dots, x_n)).\]
	Therefore $\Gamma_P$ is an implicit definition of $P$.

	For the other direction, suppose that $\Gamma_P$ defines $P$ implicitly. 
	Then
	$\Gamma_P \cup \Gamma_{P'} \entails\allowbreak \forall x_1\quantifierdots \forall x_n (  P(x_1, \dots, x_n) \liff P'(x_1, \dots, x_n))$.
	It follows from the compactness theorem that
	we can find a conjunction $\Gamma^*_{P'}$ of formulas of a finite subset of $\Gamma_{P'}$ such that  
	$\Gamma_P \cup \{\Gamma^*_{P'}\} \entails \forall x_1\quantifierdots \forall x_n (  P(x_1, \dots, x_n) \liff P'(x_1, \dots, x_n))$.
	%Let $\gamma_{P'}$ be the conjunction of all formulas in $\Gamma^*_{P'}$ and 
	Let $y_1, \dots, y_n$ be fresh variables.
	Then we obtain by the deduction theorem that  
	$\Gamma_P \cup \{P(y_1, \dots, y_n)\} \entails \Gamma^*_{P'} \limpl  P'(y_1, \dots, y_n)$.

	Note that $P$ only occurs in the antecedent and $P'$ only occurs in the consequent.
	Hence we can apply the Interpolation Theorem~\ref{thm:interpolation_original} in order to obtain a formula $\chi$
	such that
	$\Gamma_P \cup \{P(y_1, \dots, y_n)\} \entails \chi$ and
	$\chi \entails \Gamma^*_{P'} \limpl  P'(y_1, \dots, y_n)$,
	while additionally $\Lang(\chi) = \Lang(\Gamma_P) \cap \Lang(\Gamma_{P'})$. This implies that neither $P$ nor $P'$ occur in\nolinebreak{} $\chi$.

	Now we apply the deduction theorem another time and get that
	\markA{} $\Gamma_P \entails P(y_1, \dots, y_n) \limpl \chi$ and
	$\Gamma^*_{P'} \entails \chi \limpl  P'(y_1, \dots, y_n)$.
	As $\Gamma^*_{P'}$ implies $\Gamma_{P'}$, we also have that
	$\Gamma_{P'} \entails \chi \limpl  P'(y_1, \dots, y_n)$.
	Since $P$ does not occur in this entailment, it remains valid if we replace every occurrence of the symbol $P'$ by $P$
	and obtain that
	\markB{} $\Gamma_{P} \entails \chi \limpl  P(y_1, \dots, y_n)$.

	But then \markA{} and \markB{} imply that 
	$\Gamma_{P} \entails \chi \liff  P(y_1, \dots, y_n)$, which is equivalent to
	$\Gamma_{P} \entails \forall y_1\quantifierdots\forall y_n \left( \chi \liff  P(y_1, \dots, y_n)\right)$.
	So clearly $\Gamma_P$ defines $P$ explicitly.
\end{proof}

\section{Interpolation in higher-order logic}
\label{sec:interpol_hol}
In this thesis, we restrict our attention to first-order logic.
This is not only a matter of reasonable scope, but justified by the fact that the interpolation theorem does not hold even in second-order logic as discovered by Craig in \cite{Craig65}.
There, a second-order formula is presented and shown to be implicitly, but not explicitly definable.
This failure of Beth definability directly leads to a failure of interpolation in this logic, which can easily be seen by the proof of Theorem~\ref{thm:beth}.
%This immediately leads to a failure of Beth definability, and consequently also interpolation, in this logic.





	%\section{Calculi}
	%In this chapter, we introduce the calculi that are used subsequently. These are resolution and sequent calculus.

	

\section{Resolution}

Resolution calculus, in the formulation as given here, is a sound and complete calculus for first-order logic with equality.
Due to the simplicity of its rules, it is widely used in the area of automated deduction.

\begin{defi}
	A \defiemph{clause} is a finite set of literals. The empty clause will be denoted by $\square$.
	A \defiemph{resolution refutation} of a set of clauses~$\Gamma$ is a derivation of $\square$ consisting of applications of resolution rules (cf.~figure~\ref{fig:resolution}) starting from clauses in $\Gamma$.
\end{defi}


\begin{thm}
	A clause set $\Gamma$ is unsatisfiable if and only if there is resolution refutation of $\Gamma$.
\end{thm}
\begin{proof}
	See \cite{Rob65}.
\end{proof}

Clauses will usually be denoted by $C$ or $D$, literals by $l$.

\begin{figure}[htbp]
	\begin{prooftree}
		\LeftLabel{\textit{Resolution:}\quad}
		\AxiomCm{ C \lor l }
		\AxiomCm{ D \lor \lnot l' }
		\RightLabelm{\quad \sigma = \mgu(l, l')}
		\BinaryInfCm{ (C \lor D)\sigma }
	\end{prooftree}

	\begin{prooftree}
		\LeftLabel{\textit{Factorisation:}\quad}
		\AxiomCm{ C \lor l \lor l' }
		\RightLabelm{\quad \sigma = \mgu(l, l')}
		\UnaryInfCm{ (C \lor l)\sigma }
	\end{prooftree}

	\begin{prooftree}
		\LeftLabel{\textit{Paramodulation:}\quad}
		\AxiomCm{ C \lor s=t }
		\AxiomCm{ D[r] }
		\RightLabel{$\quad \sigma = \mgu(s, r)$}
		\BinaryInfCm{ (C \lor D[t])\sigma }
	\end{prooftree}

	\caption{The rules of resolution calculus}
	\label{fig:resolution}
\end{figure}


\section{Resolution and Interpolation}


In order to apply resolution to arbitrary first-order formulas, they have to be converted to clauses first.
This usually makes use of intermediate normal forms which are defined as follows:

\begin{defi}
	A formula is in \defiemph{Negation Normal Form (NNF)} if negations only occur directly before of atoms.
	A formula is in \defiemph{Conjunctive Normal Form (CNF)} if it is a conjunction of disjunctions of literals.
\end{defi}

In this context, the conjuncts of a CNF-formula are interpreted as clauses.
A well-established procedure for the translation to CNF is comprised of the following steps:

\begin{enumerate}
		\item NNF-Transformation \label{step_nnf_trans}
		\item Skolemisation \label{step_skolem_trans}
		\item CNF-Transformation \label{step_cnf_trans}
\end{enumerate}

Step \ref{step_nnf_trans} can be achieved by solely pushing the negation inwards.
As this transformation yields an equivalent formula, it clearly has no effect on the interpolants.
Step \ref{step_skolem_trans} and \ref{step_cnf_trans} on the other hand do not produce equivalent formulas since they introduce new symbols.
In this section, we will show that they nonetheless do preserve the set of interpolants.
This fact is vital for the use of resolution-based methods for interpolant computation of arbitrary formulas.


\subsection{Interpolation and Skolemisation}

Skolemisation is a procedure for replacing existential quantifiers with Skolem terms:

\begin{defi}
	Let $V_{\exists x}$ be the set of universally bound variables in the scope of the occurrence of $\exists x$ in a formula.
	The skolemisation of a formula $A$ in NNF, denoted by $\sk(A)$, is the result of replacing every occurrence of an existential quantifier $\exists x$ in $A$ by a term $f(y_1, \ldots, y_n)$ where $f$ is a new Skolem function symbol and $V_{\exists x} = \{y_1, \ldots, y_n\}$.
	In case $V_{\exists x}$ is empty, the occurrence of $\exists x$ is replaced by a new Skolem constant symbol $c$.

	The skolemisation of a set of formulas $\Phi$ is defined to be $\sk(\Phi) = \{ \sk(A) \mid A \in \Phi \}$.
\end{defi}


\begin{prop}
	Let $\Gamma \cup \Delta$ be unsatisfiable.
	Then $I$ is an interpolant for $\Gamma \cup \Delta$ if and only if it is an interpolant for $\sk(\Gamma) \cup \sk(\Delta)$. 
\end{prop}

\begin{proof}
	Since $\sk(\cdot)$ adds fresh symbols to both $\Gamma$ and $\Delta$ individually,
	none of them are contained in $\Lang(\sk(\Gamma)) \intersect \Lang(\sk(\Delta))$.
	Therefore condition \refsub{def:interpolant}{int_3} is satisfied in both directions.

	As for any set of formulas $\Phi$, each model of $\Phi$ can be extended to a model of $\sk(\Phi)$ and every model of $\sk(\Phi)$ is a witness for the satisfiability of $\Phi$, $\Phi \entails I$ iff $\sk(\Phi) \entails I$.
	Hence conditions \refsub{def:interpolant}{int_1} and \refsub{def:interpolant}{int_2} remain satisfied for $I$ as well.
\end{proof}


\subsection{Interpolation and structure-preserving Normal Form Transformation}

A common method for transforming a skolemised formula $A$ into CNF while preserving their structure is defined as follows:

\begin{defi}
For every occurrence of a subformula $B$ of $A$, introduce a new atom $L_B$ which acts as a label for the subformula. 
For each of them, create a defining clause $D_B$:

\begin{itemize}
	\item[If $B$ is atomic:]~

	$D_B\equiv (\lnot B \lor L_B) \land (B \lor \lnot L_B)  $
	\item[If $B$ is of the form $\lnot G$:]~

	$D_B\equiv (L_B \lor L_G) \land (\lnot L_B \lor \lnot L_G)  $
	\item[If $B$ is of the form $G \land H$:]~

		$D_B\equiv (\lnot L_B \lor L_G) \land (\lnot L_B \lor L_H) \land (L_B \lor \lnot L_G \lor \lnot L_H)  $
	\item[If $B$ is of the form $G \lor H$:]~

		$D_B\equiv (L_B \lor \lnot L_G) \land (L_B \lor \lnot L_H) \land (\lnot L_B \lor L_G \lor L_H)  $
	\item[If $B$ is of the form $G \limpl H$:]~

		$D_B\equiv (L_B \lor L_G) \land (L_B \lor \lnot L_H) \land (\lnot L_B \lor \lnot L_G \lor L_H)  $
	\item[If $B$ is of the form $\forall x G$:]~

		$D_B\equiv (\lnot L_B \lor L_G) \land (L_B \lor \lnot L_G)$
\end{itemize}

Let \defiemph{$\delta(A)$} be defined as $\bigwedge_{B \in \Sigma(A)} D_B \land L_A$, where $\Sigma(A)$ denotes the set of occurrences of subformulas of $A$.
\end{defi}

Note that each of the $D_B$ is in CNF, hence also $\delta(A)$ for any skolemised formula $A$.

\begin{prop}
	\label{prop:definitional_form}
	Let $A$ be a formula. Then $\sk(A)$ is unsatisfiable if and only if $\delta(\sk(A))$ is unsatisfiable.
\end{prop}

\begin{prop}
	Let $\sk(\Gamma) \cup \sk(\Delta)$ be unsatisfiable.
	Then $I$ is an interpolant for \mbox{$\sk(\Gamma) \cup \sk(\Delta)$} if and only if 
	$I$ is an interpolant for $\delta(\sk(\Gamma)) \cup \delta(\sk(\Delta))$.
\end{prop}
\begin{proof}
	As $\delta$ introduces fresh symbols for each $\sk(\Gamma)$ and $\sk(\Delta)$, they must not occur in any interpolant of $\sk(\Gamma)$ and $\sk(\Delta)$. 
	This establishes condition \refsub{def:interpolant}{int_3} in both directions.

Using proposition \ref{prop:definitional_form}, condition \refsub{def:interpolant}{int_1} and \refsub{def:interpolant}{int_2} are immediate.
\end{proof}


	\section{Sequent Calculus}
\label{sec:lk}

The famous sequent calculus was introduced in \cite{Gentzen}.
Its use of sequents in lieu of plain formulas allows for a natural mapping of the logical relations expressed by the connectives to the structure of proofs.

\begin{defi} 
	For multisets of first-order formulas $\Gamma$ and $\Delta$, $\Gamma \proves \Delta$ is called a \defiemph{sequent}. 
	In this context $\Gamma$ forms the \defiemph{antecedent}, whereas $\Delta$ is referred to as \defiemph{succedent}.

	A sequent calculus proof of a sequent $\Gamma \proves \Delta$ is a tree such that the root is the sequent $\Gamma \proves \Delta$, the leaves are axioms and each edge is labelled by a rule of sequent calculus as given in Figure~\ref{fig:lk}, such that the nodes connected by the edge match the given form. 

	A sequent $\Gamma \proves \Delta$ is called \defiemph{provable} if there exists a sequent calculus proof of $\Gamma \proves\nolinebreak \Delta$.
\end{defi}

The rules of sequent calculus are as follows:
%\medskip

{ % multicol length scope
\setlength{\multicolsep}{0.5em plus 2.0pt minus 1.5pt}% http://tex.stackexchange.com/questions/37863/reduce-white-space-before-and-after-multicols-environment
%http://tex.stackexchange.com/questions/51312/reduce-vertical-spacing-itemize-and-multicols

\newcommand{\calculussec}[1]{\textbf{#1}\nopagebreak}
%\newcommand{\calculussec}[1]{\subsubsection*{#1}\nopagebreak}
\newenvironment{lkdefsec}{
	%\begin{figure}[H]
	%\begin{samepage}
	%\ContinuedFloat
	%	\begin{adjustwidth}{0.05\textwidth}{0.05\textwidth}
}{
	%	\end{adjustwidth}
	%\end{figure}
	%\end{samepage}
	%\vspace*{-0.5em}
} 


\newenvironment{lkdefitemize}{
}{
	\vspace{0.6em}
}

\newenvironment{lkdefsamepage}{
	\begin{figure}[H]
		\begin{adjustwidth}{0.05\textwidth}{0.05\textwidth}
		}{
		\end{adjustwidth}
	\end{figure}
	\vspace{-1.1em}
}

\newenvironment{lkdefitem}{
	\begin{lkdefsamepage}
		\begin{itemize}
			}{
		\end{itemize}
	\end{lkdefsamepage}
}

\begin{lkdefsec}
	\begin{lkdefsamepage}
	\calculussec{Axioms}
	\begin{multicols}{2}
		\begin{prooftree}
			\AxiomCm{A \fCenter A}
		\end{prooftree}
		\begin{prooftree}
			\AxiomCm{\fCenter t=t}
		\end{prooftree}
	\end{multicols}
	\end{lkdefsamepage}
\end{lkdefsec}


\begin{lkdefsec}
	\calculussec{Cut}
	\begin{prooftree}
		\Axiomm{\Gamma \fCenter \Delta, A}
		\Axiomm{A, \Sigma \fCenter \Pi}
		\BinaryInfm{\Gamma, \Sigma \fCenter \Delta, \Pi}
	\end{prooftree}
\end{lkdefsec}

\begin{lkdefsec}
	\calculussec{Structural rules}
	\begin{lkdefitemize}
			\begin{lkdefitem}
			\item Contraction
				\begin{multicols}{2}
					\begin{prooftree}
						\Axiom$\Gamma, A, A \fCenter \Delta$
						\RightLabelm{\lkrule{c}{l}}
						\UnaryInf$\Gamma, A \fCenter \Delta$
					\end{prooftree}

					\begin{prooftree}
						\Axiomm{\Gamma \fCenter \Delta, A, A}
						\RightLabelm{\lkrule{c}{r}}
						\UnaryInfm{\Gamma \fCenter \Delta, A }
					\end{prooftree}

				\end{multicols}

			\end{lkdefitem}
			\begin{lkdefitem}

			\item Weakening
				\begin{multicols}{2}
					\begin{prooftree}
						\Axiomm{\Gamma \fCenter \Delta}
						\RightLabelm{\lkrule{w}{l}}
						\UnaryInfm{\Gamma, A \fCenter \Delta}
					\end{prooftree}

					\begin{prooftree}
						\Axiomm{\Gamma \fCenter \Delta}
						\RightLabelm{\lkrule{w}{r}}
						\UnaryInfm{\Gamma \fCenter \Delta,  A }
					\end{prooftree}

				\end{multicols}
			\end{lkdefitem}
	\end{lkdefitemize}
\end{lkdefsec}

\begin{lkdefsec}

	\calculussec{Propositional rules}

	\begin{lkdefitemize}
			\begin{lkdefitem}
			\item Negation

				\begin{multicols}{2}
					\begin{prooftree}
						\Axiomm{\Gamma \fCenter \Delta,  A}
						\RightLabelm{\lkrule{\lnot}{l}}
						\UnaryInfm{\lnot A, \Gamma \fCenter \Delta }
					\end{prooftree}
					\begin{prooftree}
						\Axiomm{A, \Gamma \fCenter \Delta}
						\RightLabelm{\lkrule{\lnot}{r}}
						\UnaryInfm{\Gamma \fCenter \Delta, \lnot  A }
					\end{prooftree}
				\end{multicols}

			\end{lkdefitem}
			\begin{lkdefitem}
			\item Conjunction
				\begin{multicols}{2}
					\begin{prooftree}
						\Axiomm{\Gamma, A, B \fCenter \Delta}
						\RightLabelm{\lkrule{\land}{l}}
						\UnaryInfm{\Gamma, A\land B \fCenter \Delta }
					\end{prooftree}

					\begin{prooftree}
						\Axiomm{\Gamma \fCenter \Delta, A}
						\Axiomm{\Sigma \fCenter \Pi, B}
						\RightLabelm{\lkrule{\land}{r}}
						\BinaryInfCm{\Gamma, \Sigma \fCenter \Delta, \Pi, A \land B }
					\end{prooftree}
				\end{multicols}
				%	\end{itemize} \end{lkdefsec}
				%\begin{lkdefsec} \begin{itemize}


			\end{lkdefitem}
			\begin{lkdefitem}
			\item Disjunction\nopagebreak
				\begin{multicols}{2}
					\begin{prooftree}
						\Axiomm{\Gamma, A \fCenter \Delta}
						\Axiomm{\Sigma, B \fCenter \Pi}
						\RightLabelm{\lkrule{\lor}{l}}
						\BinaryInfm{\Gamma, \Sigma, A \lor B \fCenter \Delta, \Pi }
					\end{prooftree}

					\begin{prooftree}
						\Axiomm{\Gamma \fCenter \Delta, A, B}
						\RightLabelm{\lkrule{\lor}{r}}
						\UnaryInfm{\Gamma \fCenter \Delta, A \lor B  }
					\end{prooftree}

				\end{multicols}

			\end{lkdefitem}
			\begin{lkdefitem}
			\item Implication
				\begin{multicols}{2}
					\begin{prooftree}
						\Axiomm{\Gamma \fCenter A, \Delta}
						\Axiomm{\Sigma, B \fCenter \Pi}
						\RightLabelm{\lkrule{\limpl}{l}}
						\BinaryInfm{\Gamma, \Sigma, A \limpl B \fCenter \Delta, \Pi }
					\end{prooftree}

					\begin{prooftree}
						\Axiomm{\Gamma, A \fCenter \Delta, B}
						\RightLabelm{\lkrule{\limpl}{r}}
						\UnaryInfm{\Gamma \fCenter \Delta, A \limpl B}
					\end{prooftree}

				\end{multicols}
			\end{lkdefitem}

	\end{lkdefitemize}
\end{lkdefsec}

\begin{lkdefsec}
	\calculussec{Quantifier rules}

	\begin{lkdefitemize}
			\begin{lkdefitem}
			\item Universal

				\begin{multicols}{2}
					\begin{prooftree}
						\Axiomm{\Gamma, A[x/t] \fCenter \Delta}
						\RightLabelm{\lkrule{\forall}{l}}
						\UnaryInfm{\Gamma, \forall x A \fCenter \Delta }
					\end{prooftree}

					\begin{prooftree}
						\Axiomm{\Gamma \fCenter \Delta, A[x/y]}
						\RightLabelm{\lkrule{\forall}{r}}
						\UnaryInfm{\Gamma\fCenter \Delta, \forall x A  }
					\end{prooftree}

				\end{multicols}

			\end{lkdefitem}
			\begin{lkdefitem}
			\item Existential

				\begin{multicols}{2}
					\begin{prooftree}
						\Axiomm{\Gamma, A[x/y] \fCenter \Delta}
						\RightLabelm{\lkrule{\exists}{l}}
						\UnaryInfm{\Gamma, \exists x A \fCenter \Delta }
					\end{prooftree}

					\begin{prooftree}
						\Axiomm{\Gamma \fCenter \Delta, A[x/t]}
						\RightLabelm{\lkrule{\exists}{r}}
						\UnaryInfm{\Gamma\fCenter \Delta, \exists x A  }
					\end{prooftree}


				\end{multicols}
			\end{lkdefitem}
			\vspace{0.6em}

	(provided no free variable of $t$ becomes bound in $A[x/t]$ and
	$y$ does not occur free in $\Gamma$, $\Delta$ or $A$)
			\vspace{0.3em}

	\end{lkdefitemize}



\end{lkdefsec}
%\begin{figure}[H]
	\begin{lkdefsec}
		\calculussec{Equality rules}
		\begin{lkdefitemize}
			\begin{lkdefitem}
			\item Left rules
				%\begin{multicols}{2}
				\begin{prooftree}
					\Axiomm{\Gamma, A\occurat{t}{p} \fCenter \Delta}
					\Axiomm{\Sigma \fCenter \Pi, s=t}
					\RightLabelm{\lkrule{=}{l_1}}
					\BinaryInfm{\Gamma, \Sigma, A\occurat{s}{p} \fCenter \Delta, \Pi }
				\end{prooftree}
				\begin{prooftree}
					\Axiomm{\Gamma, A\occurat{s}{p} \fCenter \Delta}
					\Axiomm{\Sigma \fCenter \Pi, s=t}
					\RightLabelm{\lkrule{=}{l_2}}
					\BinaryInfm{\Gamma, \Sigma, A\occurat{t}{p} \fCenter \Delta, \Pi }
				\end{prooftree}
				%\end{multicols}

			\end{lkdefitem}
			\begin{lkdefitem}

			\item Right rules
				%\begin{multicols}{2}
				\begin{prooftree}
					\Axiomm{\Gamma\fCenter \Delta, A\occurat{t}{p} }
					\Axiomm{\Sigma \fCenter \Pi, s=t}
					\RightLabelm{\lkrule{=}{r_1}}
					\BinaryInfm{\Gamma, \Sigma\fCenter \Delta, \Pi, A\occurat{s}{p}  }
				\end{prooftree}
				\begin{prooftree}
					\Axiomm{\Gamma\fCenter \Delta, A\occurat{s}{p} }
					\Axiomm{\Sigma \fCenter \Pi, s=t}
					\RightLabelm{\lkrule{=}{r_2}}
					\BinaryInfm{\Gamma, \Sigma\fCenter \Delta, \Pi, A\occurat{t}{p} }
				\end{prooftree}
				%\end{multicols}
			\end{lkdefitem}

		(provided no free variable of $s$ or $t$ becomes bound in $A\occurat{t}{p}$ or $A\occurat{s}{p}$)
			\end{lkdefitemize}

			\vspace{-1em}
		\begin{figure}[H]
		\caption{The rules of sequent calculus}
		\label{fig:lk}
		\end{figure}
	\end{lkdefsec}
%\end{figure}
}
			\vspace{-0.5em}
For the purposes of this thesis, we usually consider the cut-free fragment of sequent calculus.

\begin{thm}
	Cut-free sequent calculus is sound and complete.
\end{thm}
\begin{proof}
	See \cite{Gentzen}.
\end{proof}







	\chapter{Reduction to First-Order Logic without Equality}
\label{chap:reduction}

A common theme of proofs is to avoid the tedious effort of proving the result from first principles by reducing the problem to one that is easier to solve.
In this instance, we are able to give a reduction for finding interpolants in first-order logic \emph{with} equality to first-order logic \emph{without} equality, where it is simpler to give an appropriate algorithm.
This method is due to Craig (\cite{Craig57linear,Craig57three}).

In order to simplify notation, we shall consider constant symbols to be function symbols of arity $0$ in this section.
The general layout of this approach is the following:
From two sets $\Gamma$ and $\Delta$, where $\Gamma \cup \Delta$ is unsatisfiable, we compute two sets $\Gamma'$ and $\Delta'$ which do not make use of equality but simulate the effects of equality in $\Gamma$ and $\Delta$ via axioms.
In the process of this transformation, also function symbols are replaced by predicate symbols with appropriate axioms to make sure that the behavior of these function-representing predicates is compatible to the one of actual functions.
Now an interpolant for $\Gamma'$ and $\Delta'$ can be derived using an algorithm that is only capable of handling predicate symbols as all other non-logical symbols have been removed.
Since the additional axioms ensure that the newly added predicate symbols mimic equality and functions respectively, we will see that the occurrences of these predicates in the interpolant can be translated back to occurrences of equality and function symbols in first-order logic with equality in the language of $\Gamma$ and $\Delta$, thereby yielding the originally desired interpolant.


\section{Translation of formulas}

As we shall see in this section, first-order formulas with equality can be transformed into first-order formulas without equality in a way that is satisfiability-preserving, which is sufficient for our purposes.

First, we define axioms in a language with fresh symbols which allows for simulation of equality and functions in first-order logic without equality and function symbols:

\begin{defi}[Translation of languages]
	For a first-order language $\LangSym$ and fresh predicate symbols $E$ and $F_f$ for $f\in \FS(\LangSym)$, 
	\defiemph{$\Trans(\LangSym)$} denotes $(\LangSym \cup \{E\}\cup\{F_f\mid f \in\nolinebreak \FS(\LangSym)\}) \setminus\allowbreak (\{=\nolinebreak \} \cup\nolinebreak \FS(\LangSym))$.
\end{defi}

\begin{defi}[Equality and function axioms]
	For a first-order language $\LangSym$ we define the following axioms in $\Trans(\LangSym)$:\nopagebreak
	\begin{align*}
		\FAX(\LangSym) \defeq{}& \smashoperator{\bigcup_{f \in \FS(\LangSym)}}  \forall \bar x \exists y (F_f(\bar x, y) \land (\forall z (F_f(\bar x, z) \limpl E(y, z)))) \\
%		\EAX(A) \defeq{} & \forall x \; x=x~\land  \\
%																		 & \begin{aligned} \bigwedge_{\substack{P \in\PS(A)\cup\\ \{F_i\mid f_i\in \FS(A)\}}} &\forall x_1 \ldots \forall x_{\ar(P)} \forall y_1 \ldots \forall y_{\ar(P)} \\
%																 & (( x_1~=~y_1 \land \ldots \land x_{\ar(P)} = y_{\ar(P)}) \limpl  \\
%																 &  (P(x_1, \ldots, x_{\ar(P)}) \Lra P(y_1, \ldots, y_{\ar(P)}) ) ) \end{aligned} 
		\Refl(P) \defeq{}& \forall x P(x,x)  \\ 
		\Congr(P) \defeq{}& 
%\forall x_1 \ldots \forall x_{\ar(P)} \forall y_1 \ldots \forall y_{\ar(P)} \\
\forall x_1 \forall y_1 \ldots \forall x_{\ar(P)} \forall y_{\ar(P)} 
(( E(x_1,y_1) \land \ldots \land E(x_{\ar(P)},  y_{\ar(P)})) \limpl  \\
 & (P(x_1, \ldots, x_{\ar(P)}) \limpl P(y_1, \ldots, y_{\ar(P)}) )) \\
		\EAX(\LangSym) \defeq{}& \Refl(E) \cup \;\smashoperator{\bigcup_{\substack{P \in\PS(\LangSym)\cup\{E\}\cup\\ \{F_f\mid f\in \FS(\LangSym)\}}}}\; \Congr(P)
	%	\EAX(A) \defeq{} & \forall x E(x,x)\land  \\
	%																	 & \begin{aligned} \bigwedge_{\substack{P \in\PS(A)\cup\{E\}\cup\\ \{F_f\mid f\in \FS(A)\}}} &\forall x_1 \ldots \forall x_{\ar(P)} \forall y_1 \ldots \forall y_{\ar(P)} \\
	%																	 & (( E(x_1,y_1) \land \ldots \land E(x_{\ar(P)},  y_{\ar(P)})) \limpl  \\
	%															 &  (P(x_1, \ldots, x_{\ar(P)}) \Lra P(y_1, \ldots, y_{\ar(P)}) ) )
	%\end{aligned} 
\qedhere
\end{align*}
%For sets of first-order formulas $\Phi$ and $h \in \{\FAX, \EAX\}$, $h(\Phi) \defeq \bigcup_{A\in \Phi} h(A)$ .
%$\FAX$ and $\EAX$ generalise to sets of formulas by elementwise application.
\end{defi}

$\Refl(P)$ will be referred to as reflexivity axiom of $P$, $\Congr(P)$ as congruence axiom of $P$.
As any model of $\EAX(\LangSym)$ requires $\Refl(E)$ and $\Congr(E)$, $E$ is also symmetric and transitive in the model:

\begin{prop}
	\label{prop:equivalence_relation}
	In every model of $\Refl(E)$ and $\Congr(E)$,
	$E$ is an equivalence relation.
\end{prop}
\begin{proof}
	Let $M$ be a model of $\Refl(E)$ and $\Congr(E)$.
	Then $M$ clearly is reflexive.
	Due to $M \entails \Congr(E)$,
	$M \entails \forall x \forall y (E(x, y) \land E(x, x)) \limpl ( E(x, x) \limpl\nolinebreak E(y, x))$.
	As we know that $E$ is reflexive, this simplifies to
	$M \entails \forall x \forall y (E(x, y)\limpl\nolinebreak E(y, x))$, i.e.~$E$ is symmetric in $M$.
	We show the transitivity of $E$ by another instance of $\Congr(E)$: 
	$M \entails \forall x \forall y \forall z ((E(y, x) \land E(y, z)) \limpl ( E(y, y) \limpl\nolinebreak E(x, z)))$,
	As $E$ is reflexive and symmetric, we get that 
	$M \entails \forall x \forall y \forall z ((E(x, y) \land E(y, z)) \limpl\nolinebreak E(x, z))$.
\end{proof}


We continue by defining the translation procedure for formulas:


\begin{defi}[Translation and inverse translation of formulas]
	\label{def:trans}
	Let $A$ be a first-order formula and $E$ and $F_f$ for $f \in \FS(A)$ be fresh predicate symbols.
	Then $\Trans(A)$ is the result of applying the following algorithm to $A$:

	\begin{compactenum}
	\item Replace every occurrence of $s=t$ in $A$ by $E(s, t)$
	\label{def:trans_step1}
	\item As long as there is an occurrence of a function symbol $f$ in $A$:
	\label{def:trans_step2}

		Let $B$ be the atom in which $f$ occurs as outermost symbol of a term.
		Then $B$ is of the form $P(s_1, \ldots, s_{j-1}, f(\bar t),\allowbreak s_{j+1}, \ldots s_m)$.
		Replace $B$ in $A$ by $\exists y (F_f(\bar t, y) \land P(s_1, \ldots, s_{j-1}, y, s_{j+1}, \ldots s_m))$ for a fresh variable $y$.
	\end{compactenum}
	\medskip

	Moreover, let the inverse operation $\TransInv(B)$ for formulas $B$ in the language $\Trans(L(A))$ be defined as the result of applying the following algorithm to $B$:
	\begin{compactenum}
	\item Replace every occurrence of $E(s, t)$ in $B$ by $s=t$.
	\item For every $f \in \FS(A)$, replace every occurrence of 
		$\exists y (F_f(\bar t, y) \land P(s_1, \ldots, \allowbreak s_{j-1},\allowbreak y,\allowbreak s_{j+1}, \ldots s_m))$
		in $B$ by $P(s_1, \ldots, s_{j-1},\allowbreak f(\bar t),\allowbreak s_{j+1}, \ldots s_m)$.
		\label{t_inverse_2}

	\item 
		\label{t_inverse_3}
		For every $f \in \FS(A)$, replace every occurrence of 
		$F_f(\bar t, s)$ by $f(\bar t) = s$.
	\end{compactenum}

	For sets of first-order formulas $\Phi$, we define $\Trans(\Phi) \defeq \bigcup_{A\in\Phi} \Trans(A)$ and 
$\TransInv(\Phi) \defeq \bigcup_{A\in\Phi} \TransInv(A)$.
\end{defi}

\begin{remark}
	Let $\LangSym$ be a language.
	Step \ref{t_inverse_2} and \ref{t_inverse_3} of $\TransInv$ are both concerned with replacing occurrences of $F_f$ by occurrences of $f$ for $f \in \FS(\LangSym)$, but are relevant in different contexts.

	Step \ref{t_inverse_2} of $\TransInv$ is the precise inverse of step \ref{def:trans_step2} of $\Trans$ in the sense that for any formula $A$, $\TransInv(\Trans(A)) = A$ as we will show in Lemma \ref{lemma:tinv}.
	In this context, step \ref{t_inverse_3} has no effect, as all occurrences of $F_f$ have been introduced by $\Trans(\cdot)$ and are consequently of exactly the form that is handled by step \ref{t_inverse_2}. 
	So the algorithm is in this regard complete even without step \ref{t_inverse_3}.

	On the other hand, if arbitrary formulas in the language $\Trans(\LangSym)$ are given, 
	they in general do not match that pattern and are only translated to $\LangSym$ in step \ref{t_inverse_3}.
	Note that $\TransInv$ without step \ref{t_inverse_2} yields a complete algorithm, as any formula that is handled there can also be processed in step \ref{t_inverse_3}.
	In such a procedure, $\TransInv(\Trans(A))$ and $A$ are in general not syntactically equal for formulas $A$ but only logically equivalent. 
\end{remark}

\begin{lemma}
	\label{lemma:tinv}
	Let $A$ be a first-order formula and $\Phi$ be a set of first-order formulas.
	Then 
	$\TransInv(\Trans(A)) = A$
	and
	$\TransInv(\Trans(\Phi)) = \Phi$
	.
\end{lemma}
\begin{proof}
	Step 1 and 2 in the algorithms $\Trans$ and $\TransInv$ are each concerned with a different set of symbols and therefore do not interfere with each other.
	Moreover, the respective steps in both algorithms are the inverse of each other.
	For step 1, this is immediate and for step 2, consider that all occurrences of $F_f$ for $f \in \FS(A)$ in $\Trans(A)$ have been introduced by $\Trans$ and are consequently of the form
	$\exists y (F_f(\bar t, y) \land P(s_1, \ldots, s_{j-1}, y, s_{j+1}, \ldots s_m))$, which is replaced by 
	$P(s_1, \ldots, s_{j-1},\allowbreak f(\bar t),\allowbreak s_{j+1}, \ldots s_m)$ by $\TransInv$.
	As no occurrences of $F_f$ remain, step 3 of $\TransInv$ leaves the formula unchanged. 
\end{proof}

\begin{defi}[Translation of formulas including axioms]
	For first-order formulas $A$, let $\TransAll(A) \defeq
	\left(\bigwedge_{B\in\FAX(\Lang(A))} B \right) \land
	\left(\bigwedge_{B\in\EAX(\Lang(A))} B \right) \land
	\Trans(A)$ and for sets of first-order formulas $\Phi$, let $\TransAll(\Phi) \defeq \FAX(\Lang(\Phi)) \cup  \EAX(\Lang(\Phi)) \cup \Trans(\Phi)$.
\end{defi}


Note that $\TransAll(A)$ contains neither the equality predicate nor function symbols but additional predicate symbols instead. More formally:

%\begin{defi}[continues=exa:cont]
%	Let $\LangSym$ be a first-order language. 
%	Then $\Trans(\LangSym)$ denotes $(\LangSym\cup \{E\}\cup\{F_f\mid f \in \FS(\LangSym)\}) \setminus(\{=\nolinebreak \} \cup\nolinebreak \FS(\LangSym))$.
%\end{defi}



\begin{samepage}
	\begin{lemma}~
		\label{lemma:transLang}
		\begin{compactenum}
		\item
			Let $\Phi$ be a set of first-order formulas. Then $\TransAll(\Phi)$ is in the language~$\Trans(\Lang(\Phi))$.
			\label{lemma:transLang1}

		\item 
			If $\Psi$ is in the language $\Trans(\LangSym)$, then $\TransInv(\Psi)$ is in the language~$\LangSym$.
			\label{lemma:transLang2}
		\end{compactenum}
	\end{lemma}
\end{samepage}

\begin{prop}
	\label{prop:transSatEquiv}
	Let $\Phi$ be a set of first-order formulas.
	\begin{compactenum}
	\item If $\Phi$ is satisfiable, then so is $\TransAll(\Phi)$.
			\label{prop:transSatEquiv1}
		\item Let $\LangSym$ be a first-order language and $\Phi$ a set of first-order formulas in the language~$\Trans(\LangSym)$.
			If $\FAX(\LangSym) \cup \EAX(\LangSym) \cup \Phi $ is satisfiable, then so is $\TransInv(\Phi)$.
			\label{prop:transSatEquiv2}
	\end{compactenum}
\end{prop}
\begin{proof}
	Suppose $\Phi$ is satisfiable.
	Let $M$ be a model of $\Phi$.
	We show that $\TransAll(\Phi)$ is satisfiable by extending $M$ to the language $\Lang(\Phi)\cup\{E\}\cup\{F_f\mid f \in \FS(A)\}$ and proving that the extended model satisfies $\TransAll(\Phi)$.

	First, let $M \entails E(s, t)$ if and only if $M \entails s = t$.
	By reflexivity of equality, it follows that $M \entails \Refl(E)$.
	As any predicate, in particular $E$ and $F_f$ for every $f \in \FS(\Phi)$, satisfy the congruence axiom with respect to $=$, by the definition of $E$ in $M$, they satisfy the congruence axiom with respect to $E$.
	Therefore $M$ is a model of $\EAX(\Lang(\Phi))$.

	Second, let $M \entails F_f(\bar x, y)$ if and only if $M \entails f(\bar x) = y$ for all $f \in \FS(\Phi)$. 
	Since $M$ is a model of $\Phi$, it maps every function symbol $f$ to a function, which by definition returns a unique result for every combination of parameters.
	This however is precisely the logical requirement on $F_f$ stated by $\FAX(\Lang(\Phi))$,   
	hence $M$ is a model of $\FAX(\Lang(\Phi))$.

	Lastly, we show that $M \entails \Trans(A)$ for all $A \in \Phi$.
	By the above definition of $E$ in $M$, step $\ref{def:trans_step1}$ of the algorithm in Definition \ref{def:trans} yields a formula that is satisfied by $M$ as it satisfies every formula of $\Phi$.
	For step \ref{def:trans_step2}, suppose $P(s_1, \ldots, s_{j-1}, f(\bar t),\allowbreak s_{j+1}, \ldots s_m)$ does (not) hold under $M$.
	Let $y$ be such that $M \entails f(\bar t)=y$.
	By our definition of $F_f$ under $M$, $M\entails F_f(\bar t, y)$ with this unique $y$.
	Hence $\exists y (F_f(\bar t, y) \land P(s_1, \ldots, s_{j-1}, y, \allowbreak s_{j+1}, \ldots s_m))$ does (not) hold under $M$.


	For 2, suppose $\FAX(\LangSym) \cup \EAX(\LangSym) \cup \Phi$ is satisfiable and let $M$ be a model of it.

	First, note that as $M \entails \EAX(\LangSym)$, by Proposition \ref{prop:equivalence_relation}, $\calI_M(E)$ is an equivalence relation.
	Let $D$ be the domain of $M$.
	We build a model $M'$ whose domain $\domainofmodel{M'}$ is the congruence relation of $\domainofmodel{M}$ modulo $\interpretation{M}(E)$.
	The interpretation $\interpretation{M'}$ of $M'$ is obtained from $\interpretation{M}$ by 
	replacing every occurrence of a domain element $d$ by its respective congruence class with respect to $\interpretation{M}(E)$.
	As $M \entails \EAX(\LangSym)$, $\interpretation{M'}$ satisfies the congruence axioms with respect to every function and predicate symbol, and is therefore well-defined.
	Due to this construction, $M' \entails s = t$ if and only if $M \entails E(s, t)$ for all terms $s$ and $t$.

	%We extend a model $M$ of this set of formulas to a model of $\TransInv(\Phi)$ by extending it from the language $\Trans(\LangSym)$ to include $\FS(\LangSym)$.

	%First, let $M\entails s = t$ if and only if $M\entails E(s, t)$.
	%By Proposition \ref{prop:equivalence_relation} and as $M \entails \EAX(\LangSym)$, $E$ is an equivalence relation in $M$.
	%As these properties directly also apply to $=$ in $M$, equality is adheres to the required axioms in $M$.

	Second, let $M\entails f(\bar t) = s$ if and only if $M\entails F_f(\bar t, s)$ for all $f \in \FS(\LangSym)$.
	As by assumption $M$ is a model of $\FAX(A)$, we know that for every $\bar t$, some $s$ with $M\entails F(\bar t, s)$ exists and is uniquely defined.
	Hence $f$ in $M$ refers to a well-defined function.

	Lastly, to show that $M \entails \TransInv(\Phi)$, 
	consider that the interpretations of the predicates $E$ and $=$ coincide in $M$.
	Furthermore, let $B$ be an occurrence of $\exists y (F_f(\bar t, y) \land P(s_1, \ldots, s_{j-1}, y, s_{j+1}, \ldots s_m))$ for some $f \in \FS(\LangSym)$ in $\Phi$.
	Then by the above definition of $f$ in $M$, we have that $B$ is in $M$ equivalent to $\exists y f(\bar t) = y) \land P(s_1, \ldots, s_{j-1}, y, s_{j+1}, \ldots s_m))$, which due to $f$ being a function is equivalent to 
	$M \entails P(s_1, \ldots, s_{j-1}, f(\bar t), s_{j+1}, \ldots s_m))$.

	Similarly, let $B$ be an occurrence of $F_f(\bar t, s)$ in $\Phi$.
	Then by our above definition of $f$ in $M$, we have that $M \entails f(\bar t) = s$ iff $M \entails B$.
\end{proof}

\begin{corr}
	Let $\Phi$ be a set of first-order formulas.
	Then $\Phi$ is satisfiable if and only if $\TransAll(\Phi)$ is satisfiable.
\end{corr}
\begin{proof}
	The left-to-right direction is directly given in Proposition \ref{prop:transSatEquiv}.
	For the other direction, consider that by Proposition \ref{prop:transSatEquiv}, $\TransInv(\Trans(\Phi))$ is satisfiable, which by Lemma \ref{lemma:tinv} is nothing else than $\Phi$.
\end{proof}



\section{Computation of interpolants}

For the proof of the interpolation theorem by reduction we require an algorithm that operates in first-order logic without equality and function symbols, which we describe in this section.

\begin{remark}
	As the idea of this reduction is to simplify the problem by amongst others not considering function symbols,
	resolution-based methods can not be employed in a direct manner.
	This is because function symbols appear naturally in them as they usually handle existential quantification by means of Skolemization, i.e.~a new function symbol is introduced for every occurrence of an existential quantifier in the scope of a universal quantifier.
	Translating the skolemized formulas to a language without function symbols as described in Definition \ref{def:trans} is of no avail since this translation introduces new existential quantifiers for every function symbol it encounters, necessitating Skolemization yet again.
\end{remark}


\begin{lemma}
	\label{lemma:no_equality_in_proof}
	 %Let $\Gamma$ and $\Delta$ be sets of first-order formulas without the equality symbol and $\Gamma \proves \Delta$ be provable in sequent calculus. 
	 Let $\Gamma$ and $\Delta$ be sets of first-order formulas such that the equality symbol does not occur in them and $\Gamma \proves \Delta$ is provable in sequent calculus. 
	 Then there exists a proof of $\Gamma \proves \Delta$ that does not contain the equality symbol.
\end{lemma}
\begin{proof}
	By the soundness of sequent calculus, we obtain that $\Gamma \entails A$ for some $A \in \Delta$.
	But as sequent calculus without equality rules is complete for first-order logic without equality, there is a proof $\pi$ of $\Gamma \proves A$ in this calculus.
	We extend $\pi$ by a series of weakenings to a proof $\pi'$ of $\Gamma \proves \Delta$.
	However $\pi'$ is obviously also a proof in sequent calculus with equality rules.
\end{proof}

\begin{comment}
	\largered{supported idea: there is a proof in LK with equality, so it is valid. it is true in every model. a calculus without equality is complete for every theorem without equality. hence this calculus without equality has a proof for this formula.}
	Let $\pi$ be a proof of $\Gamma \proves \Delta$.

	Suppose $\pi$ contains an instance of the equality axiom $\proves t = t$ for a term $t$.
	As no equality symbol is contained in the end sequent, there has to be a rule application in $\pi$ which removes either $t=t$.
	Only instances of equality rules or cut are capable of this. 

	Consider the case that an equality rule removed $t=t$.
	As the cases for
	$\lkrule{=}{l_1}$,
	$\lkrule{=}{l_2}$,
	$\lkrule{=}{r_1}$ and
	$\lkrule{=}{r_2}$ are similar, we only consider the case of $\lkrule{=}{l_1}$.
	The proof $\pi'$ leading up to the rule application that removes of the occurrence of $t=t$ is of the form:
	\begin{prooftree}
		\AxiomCm{\varphi}
		\noLine
		\UnaryInfCm{\Gamma, A\occurat{t}{p} \proves \Delta}
		\AxiomCm{\psi}
		\noLine
		\UnaryInfCm{\Sigma \proves \Pi, t=t}
		\RightLabelm{\lkrule{=}{l_1}}
		\BinaryInfm{\Gamma, \Sigma, A\occurat{t}{p} \fCenter \Delta, \Pi }
	\end{prooftree}
	We can replace $\pi'$ in $\pi$ by the following to obtain a proof without an occurrence of the equality symbol:
	\begin{prooftree}
		\AxiomCm{\varphi}
		\noLine
		\UnaryInfCm{\Gamma, A\occurat{t}{p} \proves \Delta}
		\RightLabelm{\lkrule{w}{l}}
		\UnaryInfCm{\Gamma, \Sigma, A\occurat{t}{p} \proves \Delta}
		\RightLabelm{\lkrule{w}{r}}
		\UnaryInfCm{\Gamma, \Sigma, A\occurat{t}{p} \proves \Delta, \Pi}
	\end{prooftree}


	Suppose $\pi$ contains an instance of the axiom $A \proves A$ such that the equality symbol occurs in $A$.
	Then $A$ is of the form $s=t$ for terms $s$ and $t$.
	While the occurrence in the consequent might be eliminated by an equality rule application, due to the subformula property, there is no rule in cut-free sequent calculus such that the occurrence in the antecedent is removed.
	Hence it appears in the final sequent, which contradicts the assumption.

	Suppose $\pi$ contains an instance of \lkrule{w}{l} such that the equality symbol occurs in the principal formula $A$. 
	This case can be argued similarly as for occurrences of $A$ as antecedent of an axiom $A \proves A$.

	Suppose $\pi$ contains an instance of \lkrule{w}{r} such that the equality symbol occurs in the principal formula $A$. 
	Then as it does not occur in the end sequent, it is removed by either an instance of an equality rule or the cut rule. 
	Suppose it is removed via an equality rule. We consider the case of  $\lkrule{=}{l_1}$.
	\begin{prooftree}

		\AxiomCm{\Lambda, A\occurat{t}{p} \proves \Theta}

		\AxiomCm{\varphi}
		\noLine
		\UnaryInfCm{\Gamma \proves \Delta}
		\RightLabelm{\lkrule{w}{r}}
		\UnaryInfCm{\Gamma \proves \Delta, s=t}
		\noLine
		\UnaryInfCm{\vdots}
		\noLine
		\UnaryInfCm{\Sigma \proves \Pi, s=t}

		\RightLabelm{\lkrule{=}{l_1}}                        
		\BinaryInfm{\Lambda, \Sigma, A\occurat{s}{p} \fCenter \Theta, \Pi }
	\end{prooftree}



	not finished


	(proof might become longer, but how does that work? don't inessential cuts remain?)

\end{proof}
\end{comment}


We now show that interpolants can be computed by means of a sequent calculus based procedure by Maehara as described in \cite[Lemma\ 6.5]{takeuti87}.
It is slightly stronger than the required statement as it allows for interpolants of partitions of sequents:

\begin{defi}[Partition of sequents]
	A {partition} of a sequent $\Gamma \proves \Delta$ is denoted by $\parti{\Gamma_1}{\Delta_1}{\Gamma_2}{\Delta_2}$, where
	$\Gamma_1 \uplus \Gamma_2 = \Gamma$ and 
	$\Delta_1 \uplus \Delta_2 = \Delta$.
\end{defi}



\begin{lemma}[Maehara]
	\label{lemma:maehara}
	Let $\Gamma$ and $\Delta$ be sets of first-order formulas without equality and function symbols such that $\Gamma \vdash \Delta$ is provable in cut-free sequent calculus.
	Then for any partition \parti{\Gamma_1}{\Delta_1}{\Gamma_2}{\Delta_2}
	there is an interpolant $I$ such that
	\begin{compactenum}
	\item $\Gamma_1 \proves \Delta_1, I$ is provable 
		\label{maehcond1}
	\item $\Gamma_2, I \proves \Delta_2$ is provable 
		\label{maehcond2}
	\item $\Lang(I) \subseteq \Lang(\Gamma_1, \Delta_1) \cap \Lang(\Gamma_2, \Delta_2)$
		\label{maehcond3}
	\end{compactenum}
\end{lemma}
\begin{proof}
	We prove this lemma by induction on the number of inferences in a cut-free proof of $\Gamma \proves \Delta$.
	By Lemma \ref{lemma:no_equality_in_proof}, we can assume that no equality symbol occurs in the proof, so equality rules need not be considered. 
	\begin{description}
		\item[\normalfont Base case.]
			Suppose no rules were applied.
			Then $C \vdash D$ is of one of the form
				$A \vdash A$. We give interpolants for any of the four possible partitions:
					\begin{enumerate}
						\item \parti{A}{A}{}{}: $I=\bot$
						\item \parti{}{}{A}{A}: $I=\top$
						\item \parti{}{A}{A}{}: $I=\lnot A$\nopagebreak
						\item \parti{A}{}{}{A}: $I=A$
					\end{enumerate}

		\item[\normalfont Structural rules.]
			Suppose the property holds for $n$ rule applications and the $(n+1)$th rule application is a structural one.

			\begin{itemize}
				\item The last rule application is an instance of \lkrule{c}{l}. Then it is of the form:
					\begin{prooftree}
						\Axiomm{\Gamma, A, A \fCenter \Delta}
						\RightLabelm{\lkrule{c}{l}}
						\UnaryInfm{\Gamma, A \fCenter \Delta}
					\end{prooftree}

					There are two possible partition schemes: of $\Gamma, A \proves \Delta$:
					\begin{enumerate}
						\item $\partisym = \parti{\Gamma_1, A}{\Delta_1}{\Gamma_2}{\Delta_2}$.
							By the induction hypothesis, we know that there is an interpolant $I$ for the partition \parti{\Gamma_1, A, A}{\Delta_1}{\Gamma_2}{\Delta_2} of the upper sequent.
							$I$ serves as interpolant for $\partisym$ as well.

						\item $\partisym = \parti{\Gamma_1}{\Delta_1}{\Gamma_2, A}{\Delta_2}$.
							By a similar argument, we get that there is an interpolant $I$ for 
							\parti{\Gamma_1}{\Delta_1}{\Gamma_2, A, A}{\Delta_2}, which again is also an interpolant for $\partisym$.

					\end{enumerate}

					The case of \lkrule{c}{r} is analogous.

				\item The last rule application is an instance of \lkrule{w}{r}. Then it is of the form:
					\begin{prooftree}
						\Axiomm{\Gamma \fCenter \Delta}
						\RightLabelm{\lkrule{w}{r}}
						\UnaryInfm{\Gamma \fCenter \Delta, A}
					\end{prooftree}

					By the induction hypothesis, there exists an interpolant $I$ for any partition \parti{\Gamma_1}{\Delta_1}{\Gamma_2}{\Delta_2} of $\Gamma \vdash \Delta$.
					Clearly $I$ remains an interpolant when adding $A$ to either $\Delta_1$ or $\Delta_2$.

					The case of \lkrule{w}{l} is analogous.

			\end{itemize}

		\item[\normalfont Propositional rules.]
			Suppose the property holds for $n$ rule applications and the $(n+\nolinebreak 1)$th rule application is a propositional one.

			\begin{itemize}
				\item The last rule application is an instance of \lkrule{\lnot}{l}. Then it is of the form:
					\begin{prooftree}
						\Axiomm{\Gamma \fCenter \Delta,  A}
						\RightLabelm{\lkrule{\lnot}{l}}
						\UnaryInfm{\lnot A, \Gamma \fCenter \Delta }
					\end{prooftree}

					There are two possible partition schemes of $\Gamma, \lnot A \vdash \Delta$:
					\begin{enumerate}
						\item $\partisym = \parti{\Gamma_1, \lnot A}{\Delta_1}{\Gamma_2}{\Delta_2}$.
							By the induction hypothesis, there exists an interpolant $I$ for the partition \parti{\Gamma_1}{\Delta_1, A}{\Gamma_2}{\Delta_2} of the upper sequent.
							Clearly $I$ is an interpolant for $\partisym$ as well.

						\item $\partisym = \parti{\Gamma_1}{\Delta_1}{\Gamma_2, \lnot A}{\Delta_2}$. A similar argument goes through. 
					\end{enumerate}

					The case of \lkrule{\lnot}{r} is analogous.

				\item The last rule application is an instance of \lkrule{\limpl}{l}. Then it is of the form:
					\begin{prooftree}
						\Axiomm{\Gamma \fCenter \Delta,  A}
						\Axiomm{\Sigma, B \fCenter \Pi}
						\RightLabelm{\lkrule{\limpl}{l}}
						\BinaryInfm{\Gamma, \Sigma, A \limpl B \fCenter \Delta, \Pi }
					\end{prooftree}

					There are two possible partition schemes of $\Gamma, A\limpl B \vdash \Delta$:
					\begin{enumerate}
						\item $\partisym = \parti{\Gamma_1, \Sigma_1, A\limpl B}{\Delta_1, \Pi_1}{\Gamma_2, \Sigma_2}{\Delta_2, \Pi_2}$.
							By the induction hypothesis, there is an interpolant $I_1$ for the partition $\parti{\Gamma_1}{\Delta_1, A}{\Gamma_2}{\Delta_2}$ of the left upper sequent.
							Hence for $I_1$, we have that $\Gamma_1 \fCenter \Delta_1, A, I_1$ and 
							$I_1, \Gamma_2 \fCenter \Delta_2$ are provable.

							Moreover, we also get by the induction hypothesis that there is an interpolant $I_2$ for the partition $\parti{\Sigma_1, B}{\Pi_1}{\Sigma_2}{\Pi_2}$ of the right upper sequent.
							Therefore $\Sigma_1, B \fCenter \Pi_1, I_2$ and $I_2, \Sigma_2 \fCenter \Pi_2$ are provable.

							Using these prerequisites, we first establish that $I_1 \lor I_2$ fulfills conditions \ref{maehcond1} and \ref{maehcond2} of an interpolant for $\partisym$:
							\medskip

							\begin{prooftree}
								\Axiomm{\Gamma_1 \fCenter \Delta_1, A, I_1}
								\Axiomm{\Sigma_1, B \fCenter \Pi_1, I_2}
								\RightLabelm{\lkrule{\limpl}{l}}
								\BinaryInfm{\Gamma_1, \Sigma_1, A\limpl B \fCenter \Delta_1, \Pi_1, I_1, I_2}
								\RightLabelm{\lkrule{\lor}{r}}
								\UnaryInfm{\Gamma_1, \Sigma_1, A\limpl B \fCenter \Delta_1, \Pi_1, I_1 \lor I_2}
							\end{prooftree}
							\medskip

							\begin{prooftree}
								\Axiomm{I_1, \Gamma_2 \fCenter \Delta_2}
								\Axiomm{I_2, \Sigma_2 \fCenter \Pi_2}
								\RightLabelm{\lkrule{\lor}{l}}
								\BinaryInfm{I_1 \lor I_2, \Gamma_2, \Sigma_2 \fCenter \Delta_2, \Pi_2}
							\end{prooftree}
							\medskip

							{
								%\setlength{\abovedisplayskip}{0pt}
								%\setlength{\belowdisplayskip}{0pt}
								%\setlength{\abovedisplayshortskip}{0pt}
								%\setlength{\belowdisplayshortskip}{0pt}


								To show that also condition \ref{maehcond3} is satisfied, consider that by the induction hypothesis, it holds that:
								\begin{align*}
									\Lang(I_1) &\subseteq \Lang(\Gamma_1, \Delta_1, A) \cap \Lang(\Gamma_2, \Delta_2) \\
									\Lang(I_2) &\subseteq \Lang(\Sigma_1, B, \Pi_1) \cap \Lang(\Sigma_2, \Pi_2)
								\end{align*}\nopagebreak
								Therefore
								\begin{align*}
									\Lang(I_1) \cup \Lang(I_2) &\subseteq
									(\Lang(\Gamma_1, \Delta_1, A) \cap \Lang(\Gamma_2, \Delta_2)) \cup ( \Lang(\Sigma_1, B, \Pi_1) \cap \Lang(\Sigma_2, \Pi_2))  \\
									&\Downarrow \\
									\Lang(I_1) \cup \Lang(I_2) &\subseteq
									(\Lang(\Gamma_1, \Delta_1, A) \cup \Lang(\Sigma_1, B, \Pi_1)) \cap (\Lang(\Gamma_2, \Delta_2) \cup \Lang(\Sigma_2, \Pi_2)) \\
									&\Updownarrow \\
									\Lang(I_1 \lor I_2) &\subseteq \Lang(\Gamma_1, \Sigma_1, A\limpl B, \Delta_1, \Pi_1) \cap \Lang(\Gamma_2, \Sigma_2, \Delta_2, \Pi_2)
								\end{align*}

							}

						\item $\partisym = \parti{\Gamma_1, \Sigma_1}{\Delta_1, \Pi_1}{\Gamma_2, \Sigma_2, A\limpl B}{\Delta_2, \Pi_2}$.
							The argument for this case is similar using $I_1 \land I_2$ as interpolant.
					\end{enumerate}


					For the other binary connectives \lkrule{\land}{l}, \lkrule{\land}{r}, \lkrule{\lor}{l}, \lkrule{\lor}{r} and \lkrule{\limpl}{r}, similar arguments go through, where the interpolant is always either the conjunction or the disjunction of the interpolants of partitions of the preceding sequents.

			\end{itemize}

		\item[\normalfont Quantifier rules.]
			Suppose the property holds for $n$ rule applications and the $(n+1)$th rule application is a quantifier rule.

			\begin{itemize}
				\item The last rule application is an instance of $\lkrule{\forall}{l}$. Then it is of the form:
					\begin{prooftree}
						\Axiomm{\Gamma, A\subst{x/y} \fCenter \Delta}
						\RightLabelm{\lkrule{\forall}{l}}
						\UnaryInfm{\Gamma, \forall x A \fCenter \Delta}
					\end{prooftree}
					Note that since we have excluded function symbols from occurring in the final sequent (and constant symbols are treated as function symbols of arity 0) and
					by completeness there is a proof of the sequent in the language of the sequent, we can assume that no function or constant symbols occur in this proof.
					Hence quantifiers are only instantiated by variables.

					There are two possible partition schemes of $\Gamma, \forall x A \vdash \Delta$:
					\begin{enumerate}
						\item \parti{\Gamma_1, \forall x A}{\Delta_1}{\Gamma_2}{\Delta_2}.
							By the induction hypothesis, there is an interpolant $I$ of the partition $\parti{\Gamma_1, A\subst{x/y}}{\Delta_1}{\Gamma_2}{\Delta_2}$.
							Hence for $I$, 
							$\Gamma_1, A\subst{x/y} \fCenter \Delta_1, I$ and  
							$I, \Gamma_2 \fCenter \Delta_2$ are provable.
							By an application of $\lkrule{\forall}{l}$ to the first sequent we get $\Gamma_1, \forall x A\fCenter \Delta_1, I$, so $I$ satisfies conditions \ref{maehcond1} and \ref{maehcond2} of being an interpolant for $\partisym$.

							In order to show that also $\Lang(I) \subseteq \Lang(\Gamma_1, \forall x A, \Delta_1) \cap \Lang(\Gamma_2, \Delta_2)$, consider that by the induction hypothesis, 
							$\Lang(I) \subseteq \Lang(\Gamma_1, A\subst{x/y}, \Delta_1) \cap \Lang(\Gamma_2, \Delta_2)$.

							As free variables are not considered to be part of the language, $L(\forall x A) = L(A\subst{x/y})$.


						\item \parti{\Gamma_1}{\Delta_1}{\Gamma_2, \forall x A}{\Delta_2}.
							This case can be argued analogously.
					\end{enumerate}

					In the case of \lkrule{\exists}{r}, a similar argument goes through.

				\item The last rule application is an instance of $\lkrule{\forall}{r}$. Then it is of the form:\nopagebreak
					\begin{prooftree}
						\Axiomm{\Gamma\fCenter \Delta, A\subst{x/y} }
						\RightLabelm{\lkrule{\forall}{r}}
						\UnaryInfm{\Gamma\fCenter \Delta, \forall x A }
					\end{prooftree}
					where $y$ does not appear in $\Gamma$, $\Delta$ or $A$.

					There are two possible partition schemes of $\Gamma\vdash \Delta, \forall x A $:
					\begin{enumerate}
						\item $\partisym = \parti{\Gamma_1}{\Delta_1, \forall x A}{\Gamma_2}{\Delta_2}$.
							By the induction hypothesis, there exists an interpolant I of the partition 
							\parti{\Gamma_1}{\Delta_1, A\subst{x/y}}{\Gamma_2}{\Delta_2} of the upper sequent.
							Hence for $I$, 
							$\Gamma_1 \fCenter \Delta_1, A\subst{x/y}, I$ and
							$I, \Gamma_2 \fCenter \Delta_2$ are provable.

						As $y$ does not occur in $\Gamma$ or $\Delta$ and consequently by condition \ref{maehcond3} does not occur in $I$, we may apply the $\lkrule{\forall}{r}$ rule to the former sequent to obtain $\Gamma_1 \fCenter \Delta_1, \forall x A, I$.
							Hence $I$ is an interpolant for $\partisym$ as well.

						\item \parti{\Gamma_1}{\Delta_1}{\Gamma_2}{\Delta_2, \forall x A}.
							This case can be argued analogously.
					\end{enumerate}

					In the case of \lkrule{\exists}{l}, a similar argument goes through.
					\qedhere
			\end{itemize}
			\begin{comment} % i do not explain why this need not be here
			\item[\normalfont Equality rules.]
				Suppose the property holds for $n$ rule applications and the $(n+1)$th rule is an equality rule.

				\begin{itemize}
					\item The last rule application is an instance of $\lkrule{=}{r_1}$. Then it is of the form:
						\begin{prooftree}
							\Axiomm{\Gamma\fCenter \Delta, A\subst{T/t} }
							\Axiomm{\Sigma \fCenter \Pi, s=t}
							\RightLabelm{\lkrule{=}{r_1}}
							\BinaryInfm{\Gamma, \Sigma\fCenter \Delta, \Pi, A\subst{T/s}  }
						\end{prooftree}

						There are two possible partition schemes of $\Gamma, \Sigma \vdash \Delta, \Pi A\subst{T/s} $:
						\begin{enumerate}
							\item $\partisym = \parti{\Gamma_1, \Sigma_1}{\Delta_1, \Pi_1, A\subst{T/s}}{\Gamma_2, \Sigma_2}{\Delta_2, \Pi_2}$.  

								By the induction hypothesis, there is an interpolant $I_1$ for the partition $\parti{\Gamma_1}{\Delta_1, A\subst{T/t}}{\Gamma_2}{\Delta_2}$ of the left upper sequent.
								Hence $\Gamma_1 \fCenter \Delta_1, A\subst{T/t}, I_1$ and $I_1, \Gamma_2 \fCenter \Delta_2$.

								We also get by the induction hypothesis that there is an interpolant $I_2$ for the partition $\parti{\Sigma_1}{\Pi_1, s=t}{\Sigma_2}{\Pi_2}$ of the right upper sequent. Here, we have that 
								$\Sigma_1 \fCenter \Pi_1, s=t, I_2$ and $I_2, \Sigma_2 \fCenter \Pi_2$.

								Now we establish that $I_1 \lor I_2$ is an interpolant for $\partisym$.

								\begin{prooftree}
									\Axiomm{\Gamma_1 \fCenter \Delta_1, A\subst{T/t}, I_1}
									\Axiomm{\Sigma_1 \fCenter \Pi_1, s=t, I_2}
									\RightLabelm{\lkrule{=}{r_2}}
									\BinaryInfm{\Gamma_1, \Sigma_1 \fCenter \Delta_1, \Pi_1, A\subst{T/s}, I_1, I_2}
									\RightLabelm{\lkrule{\lor}{r}}
									\UnaryInfm{\Gamma_1, \Sigma_1 \fCenter \Delta_1, \Pi_1, A\subst{T/s}, I_1 \lor I_2}
								\end{prooftree}

								\begin{prooftree}
									\Axiomm{I_1, \Gamma_2 \fCenter \Delta_2}
									\Axiomm{I_2, \Sigma_2 \fCenter \Pi_2}
									\RightLabelm{\lkrule{\lor}{l}}
									\BinaryInfm{I_1\lor I_2, \Gamma_2, \Sigma_2 \fCenter \Delta_2, \Pi_2}
								\end{prooftree}


								We furthermore get by the induction hypothesis that

								$\Lang(I_1) \subseteq \Lang(\Gamma_1, \Delta_1, A\subst{T/t}) \cap (\Gamma_2, \Delta_2)$

								$\Lang(I_2) \subseteq \Lang(\Sigma_1, \Pi_1, s=t) \cap (\Sigma_2, \Pi_2)$

								$\Lang(I_1 \lor I_2) \subseteq \Lang(\Gamma_1, \Sigma_1, \Delta_1, \Pi_1, A\subst{T/s}) \cap (\Gamma_2 \Sigma_2, \Delta_2, \Pi_2)$
						\end{enumerate}
				\end{itemize}
			\end{comment}
	\end{description}
\end{proof}





This allows us to state the central theorem of this section:
\begin{thm}
	\label{thm:prop_interpol}
	Let $\Gamma$ and $\Delta$ be sets of closed first-order formulas without equality and function symbols such that $\Gamma \cup \Delta$ is unsatisfiable. Then there is an interpolant for $\Gamma$ and~$\Delta$.
\end{thm}
%\begin{proof}
%	We show that there is an interpolant for $\Gamma \entails \lnot \Delta$, which
%	by Proposition \ref{prop:interpolations_equivalent} proves the theorem.
%	By the completeness of cut-free sequent calculus, there is a cut-free proof of $\Gamma \vdash \lnot \Delta$.
%	By Lemma \ref{lemma:maehara}, there is an interpolant $I$ for the partition \parti{\Gamma}{}{}{\lnot \Delta}.
%	$I$ is the desired interpolant for $\Gamma\entails\lnot\Delta$.
%\end{proof}
\begin{proof}
	As $\Gamma \cup \Delta$ are unsatisfiable, by the compactness theorem, there exists a finite conjunction $\Gamma^*$ of formulas of $\Gamma$ as well as a finite conjunction $\Delta^*$ of formulas of $\Delta$ such that $\Gamma^* \land \Delta^*$ are unsatisfiable.
	We may also write this as $\Gamma^* \entails \lnot \Delta^*$.

	By the completeness of cut-free sequent calculus, there is a cut-free proof of $\Gamma^* \proves \lnot \Delta^*$.
	So by Lemma~\ref{lemma:maehara}, there is an interpolant $I$ for the partition \parti{\Gamma^*}{}{}{\lnot \Delta^*}
	such that $\Gamma^* \proves I$, $I \proves \lnot \Delta^*$ and $\Lang(I) \subseteq \Lang(\Gamma^*) \intersect \Lang(\Delta^*)$.
	Clearly then also $\Delta^* \proves \lnot I$ holds.

	As $\Gamma^*$ and $\Delta^*$ are merely conjunctions of formulas of $\Gamma$ and $\Delta$ respectively, we get that
	$\Gamma \entails I$, $\Delta \entails \lnot I$ as well as
	$\Lang(I) \subseteq \Lang(\Gamma) \intersect \Lang(\Delta)$, which by Proposition~\ref{prop:interpolations_equivalent} gives the result.
\end{proof}



		\section{Proof by reduction}

		Using the results of the previous sections, we can now give a proof of the interpolation theorem:

		\interpolationRevThm*
		\begin{proof}%[Proof of Theorem \ref{thm:interpolation} (Interpolation)]

			Since $\Gamma \cup \Delta$ is unsatisfiable,
			by Proposition \ref{prop:transSatEquiv}, $\TransAll(\Gamma \cup \Delta)$ is unsatisfiable.
			\begin{align*}
				\TransAll(\Gamma \cup \Delta)\,\semiff~&\{\FAX(\Lang(\Gamma\cup\Delta)), \EAX(\Lang(\Gamma \cup\Delta))\} \cup \Trans(\Gamma \cup \Delta) \\
				\semiff~&\{\FAX(\Lang(\Gamma)\cup\Lang(\Delta)), \EAX(\Lang(\Gamma)\cup\Lang(\Delta))\} \cup \Trans(\Gamma )\cup \Trans(\Delta) \\
				\semiff~&\{\FAX(\Lang(\Gamma)) \land \FAX(\Lang(\Delta)), \EAX(\Lang(\Gamma)) \land \EAX(\Lang(\Delta))\} \cup \Trans(\Gamma) \cup \Trans(\Delta) \\
				\semiff~&\{\FAX(\Lang(\Gamma)),\EAX(\Lang(\Gamma))\} \cup \Trans(\Gamma) \cup \{\FAX(\Lang(\Delta)), \EAX(\Lang(\Delta))\} \cup \Trans(\Delta) \\
				\semiff~&\TransAll(\Gamma) \cup \TransAll(\Delta)
			\end{align*}
			%It follows from Lemma \ref{lemma:trans_transform}, that $\TransAll(\Gamma) \cup \TransAll(\Delta)$is unsatisfiable as well.
			Hence  $\TransAll(\Gamma) \cup \TransAll(\Delta)$ is unsatisfiable as well.
			By Lemma \refsub{lemma:transLang}{lemma:transLang1} $\TransAll(\Gamma)$ and $\TransAll(\Delta)$ contain neither function symbols nor the equality symbol.
			Hence by Theorem \ref{thm:prop_interpol}, there is an interpolant $I$ such that
			\begin{enumerate}
				\item $\TransAll(\Gamma) \entails I$
				\item $\TransAll(\Delta) \entails \lnot I$ 
				\item $\Lang(I) \subseteq \Lang(\TransAll(\Gamma)) \cap \Lang(\TransAll(\Delta))$
					\label{proof:interpolation1_3}
			\end{enumerate}

			We now show that $\TransInv(I)$ is an interpolant for $\Gamma$ and $\Delta$.

			$\TransAll(\Gamma) \entails I$ is equivalent to $\TransAll(\Gamma) \cup \{\lnot I\}$ being unsatisfiable.
			Through the unfolding of $\TransAll(\Gamma)$, we get that 
			$\{\FAX(\Lang(\Gamma)), \EAX(\Lang(\Gamma))\} \cup \Trans(\Gamma) \cup \{\lnot I\}$ is unsatisfiable.
			This set of formulas can now be translated back to the original language with the equality symbol and function symbols. 
			More formally, 
			since $\Lang(\lnot I) \subseteq \Lang(\TransAll(\Gamma))$, we can apply Proposition
			\refsub{prop:transSatEquiv}{prop:transSatEquiv2}
			by considering $\Trans(\Gamma) \cup \{\lnot I\}$ as $\Phi$ to conclude that $\TransInv(\Trans(\Gamma) \cup \{\lnot I\})$ is unsatisfiable. By pulling $\TransInv$ inward and an application of Lemma \ref{lemma:tinv}, we get that $\Gamma \cup \{\TransInv(\lnot I)\} = \Gamma \cup \{\lnot \TransInv(I)\}$ is unsatisfiable. 
			Therefore $\Gamma \entails \Trans^{-1}(I)$.

			For $\Delta$, an analogous argument goes through and so from $\TransAll(\Gamma) \entails \lnot I$ we can deduce that $\Delta \entails \lnot \Trans^{-1}(I)$.

			By item \ref{proof:interpolation1_3}, $I$ is in the language $\Lang(\TransAll(\Gamma)) \cap \Lang(\TransAll(\Delta))$, which by Lemma \refsub{lemma:transLang}{lemma:transLang1} is $\Trans(\Lang(\Gamma)) \cap\nolinebreak \Trans(\Lang(\Delta))$. 
			\vspace{-\baselineskip}
				\newcommand{\somespace}{\;}
				\newcommand{\impconn}[1]{\somespace#1\somespace}
				%\medskip
				%\begin{figure}[H]
				\begin{adjustwidth}{-3em}{}
			\begin{align*}
				\noalign{\hspace{-4em} $\Trans(\Lang(\Gamma))\cap\nolinebreak \Trans(\Lang(\Delta)) =$ } 
				& \Big(\Lang(\Gamma)\impconn\cup \{E\}\impconn\cup\{F_f\mid f \in \FS(\Gamma)\}\Big) \impconn\setminus\Big(\{=\nolinebreak \} \impconn\cup \FS(\Gamma)\Big)\,\impconn\cap \\
				& \Big(\Lang(\Delta)\impconn\cup \{E\}\impconn\cup\{F_f\mid f \in \FS(\Delta)\}\Big) \impconn\setminus\Big(\{=\nolinebreak \} \impconn\cup \FS(\Delta)\Big)\\
				=\,& \Big((\Lang(\Gamma) \cap \Lang(\Delta)) \impconn\cup \{E\} \impconn\cup\{F_f\mid f \in \FS(\Gamma)\cap\FS(\Delta)\}\Big)  \impconn\setminus\Big(\{=\} \impconn\cup \FS(\Gamma) \impconn\cup \FS(\Delta)\Big) \\
				=\,& \Big((\Lang(\Gamma) \cap \Lang(\Delta)) \impconn\cup \{E\} \impconn\cup \{F_f\mid f\in \FS( \Lang(\Gamma) \cap \Lang(\Delta))\} \Big) \impconn\setminus \Big( \{=\} \impconn\cup \FS(\Lang(\Gamma) \cap \Lang(\Delta)) \Big)  \\ 
				=\,& \Trans(\Lang(\Gamma) \cap \Lang(\Delta))
			\end{align*}
		\end{adjustwidth}
				%\end{figure}
				%\vspace{-1em}
			As $I$ is in the language $\Trans(\Lang(\Gamma) \cap \Lang(\Delta))$, by Lemma \refsub{lemma:transLang}{lemma:transLang2}, $\TransInv(I)$ is in the language $\Lang(\Gamma) \cap \Lang(\Delta)$.
		\end{proof}




	\proofcontent

	\chapter{Interpolant extraction from resolution proofs in one phase}
\label{sec:one_phase}
\label{chap:one_phase}

In contrast to the approach described in chapter \ref{sec:two_phases}, where propositional interpolants are extracted first and colored terms lifted just in a second, separate phase, 
we now present a method which is based on the former but merges the two phases.

The motivation for the separation in two phases lies in the fact that just after the formation of the propositional interpolant, all terms and their logical relation can be known.
This however neglects the fact that proofs are frequently structured in a way such that the occurrence of certain symbols and variables are restricted to certain areas of the proof.
By lifting these and prefixing the entire interpolant with their respective quantifier, the resulting formula is not optimal in the sense that the quantifier scope can be minimised.

Consider the following example:

\begin{exa}
	\label{exa:one_phase_motivation}
	Let $\Gamma = \{ P(x) \lor Q(y) \}$ and $\Delta = \{\lnot P(a), \lnot Q(a)\}$.
	We consider the following refutation of $\Gamma \cup \Delta$, which we annotate by the interpolation extraction by appending $\PI(C)$ to each clause $C$, separated by ``$|$''.
	For the sake of brevity, we sometimes give simplified by logically equivalent versions of $\PI(C)$.
	This notational convention will be used throughout this thesis for examples of a similar form.

	\begin{prooftree}
		\AxiomCm{ P(x) \lor Q(y) \mid \bot}
		\AxiomCm{ \lnot P(a)  \mid \top}
		\BinaryInfCm{ Q(y) \mid P(a) }
		\AxiomCm{ \lnot Q(a)  \mid \top}
		\BinaryInfCm{ \square \mid Q(a) \lor P(a) }
	\end{prooftree}

	Lifting and quantification of this propositional interpolant according to Theorem~\ref{thm:two_phases} gives the interpolant $\forall x_a (Q(x_a) \lor P(x_a))$.
	Note however that the more general formula $(\forall x_a Q(x_a) ) \lor (\forall x_a P(x_a))$ is an interpolant as well, but can not be constructed by this method.
	Consider yet that $\Delta$ entails the negated interpolant, so by generalising the interpolant, the formula entailed by $\Delta$ becomes more specialised.
\end{exa}

%Colored terms satisfying certain restrictions which allow for determining the order of the quantifier of the their corresponding lifting variables are lifted and bound during the extractions of the interpolants.
%The resulting interpolants are therefore in general not in prenex form.

%The key idea which enables this early lifting of colored terms is that proofs often consist of several parts which are independent of each other. 


\section{Interpolant extraction with simultaneous lifting}

We now define the lifted interpolant $\LI$.
Note that the structure of the resulting formula coincides the ones from $\PI$ as defined in Definition~\ref{def:PI} except for quantifiers and, of course, the colored terms.

\begin{defi}[Incrementally lifted interpolant $\LI$]
	Let $\pi$ be a resolution refutation of $\Gamma \cup \Delta$.
	We define $\LI(\pi)$ to be $\LI(\square)$, where $\square$ is the empty clause derived in $\pi$.

	Let $C$ be a clause in $\pi$. 
	%For a literal $\lambda$ in $C$, we denote the corresponding literal in $\LIcl(C)$ by $\lambda\cll$, whose existence is ensured Lemma~\ref{lemma:li_vs_clause_plus_literals_equal}.
	We define the intermediary formula $\LIpre(C)$ as follows:
	\begin{description}
		\item{} Base case.
			If $C \in \Gamma\cup \Delta$, $\LIpre(C) \defeq \PIinit(C)$.

		\item{} Induction step.
			If $C$ is the result of an inference $\inference$ using the clauses $\bar C$, then $\LIpre(C) \defeq \PIstep(\inference, \LI(C_1), \dots, \LI(C_n))$.

	\end{description}

	\noindent
	$\LI(C)$ is built from $\LIpre(C)$ according to the following lifting procedure:

	\begin{enumerate}
		\item Lift all maximal colored occurrences of a term $t$ in $\LIpre(C)$ for which at least one of the following conditions, referred to as \defiemph{lifting conditions}, applies:
			\begin{itemize} 
				\item The term $t$ contains some variable $x$ such that $x$ does not occur in\nolinebreak{} $C$.
				\item The term $t$ is ground and $C$ does not contain $t$.
			\end{itemize} 
			Denote the resulting formula by $\lifsym_\mathrm{part}(\LIpre(C))$.

		\item 
Let $\lifsym_\mathrm{part}^*(\LIpre(C))$ be 
$\lifsym_\mathrm{part}(\LIpre(C))$  where every lifting variable $z_t$, which occurs free, is substituted by a fresh lifting variable $z'_{t}$.\footnote{See Example~\ref{exa:lemma_part_renaming} for an illustration.} 
\label{lemma_part_renaming}

		\item Let $X$ ($Y$) be the set of $\Delta$-($\Gamma$-)lifting variables which occur free in  
			$\lifsym_\mathrm{part}^*(\LIpre(C))$.
			Form an arrangement $\Q(C)$ of the elements of $\{\forall x_t \mid x_t \in X\}\cup\allowbreak\{\exists y_t \mid y_t \in Y\}$ such that if $s$ and $r$ are terms such that $s$ is a subterm of $r$, then $z_s$ precedes\nolinebreak{} $z_r$.
			Finally, let $\LI(C) \defeq Q(C) \lifsym_\mathrm{part}^*(\LIpre(C))$.
			\qedhere
	\end{enumerate}
\end{defi}

%\section{Properties of $\LI$ and $\LIcl$}



%\begin{remark}
%	As a local optimisation, the quantifiers can be moved inwards such that they exhibit the smallest scope which covers every occurrence of the bound variable.
%	Note that when doing so, non-maximal occurrences of these terms have to be taken as being lifted 
%\end{remark}


\section{Main lemma}
Note that the lifting conditions ensure that only terms are lifted,
which do not exhibit a direct logical relation with any term in the remaining clause.
More precisely, they do not influence the subsequent resolution derivation: 
If a variable $x$ occurs in $\LI(C)$ but not in $C$, then as all clauses in a resolution refutation are pairwise variable-disjoint, the variable $x$ does not occur in any other clause.
For ground terms $r$ however which occur in $\LI(C)$ but not in $C$,
it is possible for them to cooccur in a subsequent clause. Let $p$ be the occurrence of $r$ in $\LI(C)$ and $q$ the occurrence of $r$ in a successor-clause of $C$.
Then due to the fact that $p$ is not used in any unification, 
$q$ must be created or originate from other occurrences of the same function and/or constant symbols.
Note that the lifting conditions ensure that for these, the order of the quantifiers of their respective lifting variables is established in a fashion appropriate to ensuring the logical validity of the interpolant, but despite the syntactic equality between $p$ and $q$, there is no logical relation between them.

We now show more formally that the lifting conditions ensure that if a term contains another term, the subterm is not lifted before the superterm:

\begin{lemma}
	\label{lemma:lifting_conditions}
	Let $C$ be a clause of a resolution refutation such that $\lifdeltanovar{\LIpre(C)}$ contains a maximal colored $\Gamma$-term $t$ which is lifted in $\lifdeltanovar{\LI(C)}$.
	Suppose furthermore that $t$ contains a $\Delta$-lifting variable $x_s$.
	Then $x_s$ occurs free as a subterm of $t$ in $\lifdeltanovar{\LIpre(C)}$.
\end{lemma}
\begin{proof}
	By the construction of $\LI$, the lemma is violated only if the term $s$ or a respective predecessor is lifted and bound due to fulfilling one of the lifting conditions.

	For the sake of contradiction suppose that this is the case in the inference creating the clause $C'$.
	Let $s'$ and $t'$ be the respective predecessors of $s$ and $t$ in $C'$.

	\begin{itemize}
		\item Suppose that $s'$ is lifted due to containing a variable which does not occur in\nolinebreak{} $C'$.
			Then as $s'$ is a subterm of $t'$, $t'$ contains this variable as well and therefore is lifted in $\LI(C')$, contradicting the assumption.

		\item Suppose that $s'$ is lifted due to being a ground term which does not occur in\nolinebreak{} $C'$.
			Then $t'$ does not occur in $C'$ either as any occurrence of $t'$ contains $s'$. 
			Hence $t'$ is lifted in $\LI(C')$, contradicting the assumption.
			\qedhere
	\end{itemize}
\end{proof}

Now, we proceed to the main lemma:

\begin{lemma}
	\label{lemma:gamma_entails_delta_lifted_invariant}
	Let $C$ be a clause in a resolution refutation of $\Gamma \cup \Delta$.
	Then
	$\Gamma \entails \lifdeltanovar{ \LI(C) } \lor \lifdeltanovar{C} $
\end{lemma}
\begin{proof}
	We show the strengthening
	$\Gamma \entails \lifdeltanovar{ \LI(C) } \lor \lifdeltanovar{C_\Gamma}$\footnote{Recall that $D_\Phi$ denotes the clause created from the clause $D$ by removing all literals which are not contained $\Lang(\Phi)$.}.

	As a first step, 
	we prove by induction that
	$\Gamma \entails \lifdeltanovar{ \LIpre(C) } \lor \lifdeltanovar{C_\Gamma}$.

	If $C\in \Gamma\cup\Delta$, then Lemma~\ref{lemma:gamma_entails_init} shows that $\Gamma \entails \lifdeltanovar{\PIinit(C) \lor C_\Gamma}$, which is the unfolded definition of $\lifdeltanovar{\LIpre(C) \lor C_\Gamma}$.

	For the induction step, suppose the clause $C$ is the result of an inference $\inference$ using the clauses $C_1, \dots, C_n$.
	By induction hypothesis, $\Gamma \entails \lifdeltanovar{\PI(C_i) \lor\nolinebreak (C_i)_\Gamma}$ for $1\varleq i\varleq n$, hence
	by Lemma~\ref{lemma:gamma_entails_step}, we obtain that 
	$\Gamma \entails \lifdeltanovar{\PIstep(\inference, \bar I) \lor C_\Gamma}$.
	This however is nothing else than $\Gamma \entails \lifdeltanovar{\LIpre(C) \lor C_\Gamma}$.


	As we have now established that
	$\Gamma \entails \lifdeltanovar{ \LIpre(C) } \lor \lifdeltanovar{C_\Gamma}$,
	we show that also
	$\Gamma \entails \lifdeltanovar{ \LI(C) } \lor \lifdeltanovar{C_\Gamma}$ holds.


	The difference between $\lifdeltanovar{\LIpre(C)}$ and $\lifdeltanovar{\LI(C)}$ lies only in certain maximal colored terms which are lifted and the resulting lifting variable bound in $\lifdeltanovar{\LI(C)}$, hence it suffices to consider these.
	Let $t$ be a colored term in $\LIpre(C)$ at position $p$ such that $\LI(C)\atp = \lifboth{t}$.
	Then $t$ is a maximal colored term. % and contains a variable which does not occur in\nolinebreak{} $C$.

	If $t$ is $\Delta$-colored, then $\lifdeltanovar{\LIpre(C)}\atp = \LI(C)\atp = x_t$.
	Note that as $t$ occurs at $p$ in $\LIpre(C)$, $x_t$ occurs free at $\lifdeltanovar{\LIpre(C)}\atp$.
	The renaming of lifting variables in step \ref{lemma_part_renaming} of the lifting procedure
	ensures that $x_t$ is a fresh lifting variable and hence is not bound by quantifiers introduced to to other occurrences of the term $t$, which would otherwise also be lifted by the same lifting variable and bound by the same quantifier\footnote{See Example~\ref{exa:lemma_part_renaming} for an illustration.}.
	Hence $x_t$ is implicitly universally quantified and therefore entails that an explicit universal quantification in $\LI(C)$ is valid with an arbitrarily placed universal quantifier. 

	If otherwise $t$ is a $\Gamma$-term, then $\lifdeltanovar{\LIpre(C)}\atp = \lifdeltanovar{t}$.
	Therefore $\lifdeltanovar{t}$ represents a witness term for the existentially quantified lifting variable $y_t$ at $\LI(C)\atp$.
	In general, $\lifdeltanovar{t}$ however contains $\Delta$-lifting variables, hence for $\lifdeltanovar{t}$ to be a valid witness term, these have to be bound such that the existential quantifier of $y_t$ is in their scope.
	Note that occurrences of colored terms which are not maximal colored terms are not lifted in $\LI$.

	Let $x_s$ be a $\Delta$-lifting variable which occurs in $\lifdeltanovar{t}$. 
	We show that $y_t$ is quantified in the scope of the quantification of $x_s$ by discussing the different possibilities for quantification of $x_s$:

	\begin{itemize}
		\item
			Clearly if $s$ or a respective successor is never bound due to not occurring at a maximal colored position, it is implicitly universally quantified.

		\item
			If $s$ or a respective successor does occur at a maximal colored position but does not satisfy any of the lifting conditions up to the stage where $t$ is lifted, it is bound at some later stage of the interpolant extraction, but as for any successor $C'$ of $C$, $\LI(C)$ is contained in $\LI(C')$, 
			the scope of its quantifier encompasses the quantifier for $y_t$.

		\item
			In the case that $s$ and $t$ are lifted at the same stage of the interpolant extraction, by the definition of the quantifier prefix, the quantification of $x_s$ precedes the quantification for $x_t$ as $s$ is a subterm of $t$.


		\item
			It is furthermore essential to see that neither $s$ nor a respective predecessor is lifted in a previous step of the interpolant extraction, which is shown by Lemma~\ref{lemma:lifting_conditions}.
			\qedhere
	\end{itemize}
\end{proof}

We now present an example which demonstrates that $\LI$ does produce formulas realising the idea presented in Example~\ref{exa:one_phase_motivation}.

\begin{exa}
	\label{exa:lemma_part_renaming}
	Let $\Gamma = \{ P(u, v) \lor Q(u) \lor R(v) \}$
	and $\Delta = \{ \lnot P(w, z), \lnot Q(a), \lnot R(a)\}$.
	We consider a resolution refutation of $\Gamma\cup\Delta$ combined with the interpolant extraction.
	In order to emphasise the lifting steps,
	we do not just write $C\mid \LI(C)$ in the derivation as usual for a clause $C$ but $C\mid\LIpre(C)$ above $C\mid \LI(C)$ without a separating line 
	in case $\LIpre(C)$ is different from $\LI(C)$.
	The primed variables make the renaming of lifting variables in step \ref{lemma_part_renaming} of the lifting procedure explicit.
	\begin{prooftree}
		\AxiomCm{ P(u, v) \lor Q(u) \lor R(v) \mid \bot}
		\AxiomCm{ \lnot P(w, z) \mid \top }

		\RightLabelm{\resrule{\resruleres}{w\mapsto u, v\mapsto z}}
		\BinaryInfCm{ Q(u) \lor R(v) \mid P(u, v) }

		\AxiomCm{ \lnot Q(a) \mid \top }

		\RightLabelm{\resrule{\resruleres}{u\mapsto a}}
		\BinaryInfCm{ R(v) \mid Q(a) \lor P(a, v) }
		\noLine
		\UnaryInfCm{ R(v) \mid \forall x_a( Q(x_a) \lor P(x_a, v) ) }

		\AxiomCm{ \lnot R(a) \mid \top }

		\insertBetweenHyps{\hskip -1cm}
		\RightLabelm{\resrule{\resruleres}{v\mapsto a}}
		\BinaryInfCm{ \square \mid R(a) \lor  \forall x_a( Q(x_a) \lor P(x_a, a) ) }
		\noLine
		\UnaryInfCm{ \square \mid \forall x'_a \big( R(x'_a) \lor  \forall x_a( Q(x_a) \lor P(x_a, x'_a) ) \big) }
	\end{prooftree}

	Hence we obtain here a non-prenex interpolant which reflects the logical expressiveness of $\Gamma$, in contrast to 
	the interpolant which is produced by the two phase approach described in chapter~\ref{sec:two_phases}, which in fact is
	$\forall x_a \big( R(x_a) \lor Q(x_a) \lor P(x_a, x_a) \big)$.

	Note that without the renaming of the lifting variables, the result of the extraction would be
	$\forall x_a \big( R(x_a) \lor  \forall x_a( Q(x_a) \lor P(x_a, x_a) ) \big) $.
	In order to emphasise the binding, we alpha-rename this formula to
	$\forall x \big( R(x) \lor  \forall y( Q(y) \lor P(y, y) ) \big) $.
	This is not an interpolant, as this formula is not entailed by $\Gamma$:

	Consider a model $M$ of $\Gamma$ with domain $\domainofmodel{M} = \{0, 1\}$ and an interpretation $\interpretation{M}$ such that
	$\interpretation{M}(R) = \{0\}$,
	$\interpretation{M}(Q) = \emptyset$ and 
	$\interpretation{M}(P) = \{ (0, 1), (1, 1) \}$.
	Then clearly $M \entails P(u, v) \lor Q(y) \lor R(v) $ as depending on the value of $v$, either $R(v)$ or $P(u, v)$ holds.
	But at the same time $M \notentails \forall x \big( R(x) \lor  \forall y( Q(y) \lor P(y, y) ) \big)$ since the instantiation of the bound variables $x$ to $1$ and $y$ to $0$ results in a formula which does not hold in $M$.

\end{exa}


\section{Towards an interpolant}

In a similar fashion as in Lemma~\ref{lemma:symmetry} for $\PI$, we can also show a symmetry-property for $\LI$:

\begin{lemma}
	\label{lemma:li_symmetry}
	Let $\pi$ be a refutation of $\Gamma\cup\Delta$ and $\bhat \pi$ be $\pi$ with $\bhat \Gamma = \Delta$ and $\bhat \Delta = \Gamma$.
	Then for a clause $C$ in $\pi$ and its corresponding clause $\bhat C$ in $\bhat \pi$, $\LI(C) \spas\semiff \lnot \LI(\bhat C)$.
\end{lemma}
\begin{proof}
	We proceed by induction to show that $\LIpre(C) \semiff \lnot \LIpre(\bhat C)$:

	If $C \in \Gamma\cup \Delta$, we obtain the result by Lemma~\ref{lemma:symmetry_base}.

	For the induction step, suppose that the clause $C$ is the result of an inference $\inference$ of the clauses $\bar C = C_1, \dots, C_n$.
	Then by the induction hypothesis, $\LI(C_i) \semiff \lnot \LI(\primex C_i)$ for $1 \varleq i \varleq n$. 
	Hence we can apply Lemma~\ref{lemma:symmetry_step} to obtain that $\PIstep(\inference, \LI(C_1), \dots, \LI(C_n)) \semiff \lnot \PIstep(\primex \inference, \LI(\primex C_1), \dots, \LI(\primex C_n))$.
	But this is nothing else than $\LIpre(C) \spas\semiff \lnot \LIpre(\primex C)$.


	We conclude by showing that 
	$\LIpre(C) \semiff \lnot \LIpre(\bhat C)$ 
	implies that 
	$\LI(C) \semiff \lnot \LI(\bhat C)$:
	Clearly the terms to be lifted in $\LIpre(C)$ and $\LIpre(\bhat C)$ are the same and differ only in their color.
	Even though this results in different lifting variables, that is of no relevance as all lifted variables are bound, which makes the formulas alpha-equivalent.
	Additionally, the quantifier type of any given lifting variable in $\Q(C)$ is dual to the respective one in $\Q(\bhat C)$.
	Furthermore note that the subterm-relation is not affected by the coloring, so the ordering of the quantifiers in $\Q(C)$ and $\Q(\bhat C)$ is identical.
	Hence 
	$\LI(C) \semiff \lnot \LI(\bhat C)$.
\end{proof}


\begin{lemma}
	\label{lemma:delta_entails_li}
	Let $C$ be a clause in a resolution refutation of $\Gamma \cup \Delta$.
	Then
	$\Delta \entails \lnot\lifgammanovar{\LI(C)} \lor \lifgammanovar{C}$.
\end{lemma}
\begin{proof}
	Construct $\bhat \pi$ with $\bhat \Gamma = \Delta$ and $\bhat \Delta = \Gamma$. 
	Then by Lemma~\ref{lemma:gamma_entails_delta_lifted_invariant}, $\bhat \Gamma \entails \liftnovar{\bhat \Delta}{\LI(\bhat C)} \lor \liftnovar{\bhat \Delta}{\bhat C}$, 
	which by Lemma~\ref{lemma:li_symmetry} is nothing else than
	$\Delta \entails \lnot \liftnovar{\Gamma}{\LI(C)} \lor \liftnovar{\Gamma}{C}$.
\end{proof}

\begin{thm}
	Let $\pi$ be a resolution refutation of $\Gamma \cup \Delta$.
	Then $\LI(\pi)$ is an interpolant for $\Gamma$ and $\Delta$.
\end{thm}
\begin{proof}
	We obtain by Lemma~\ref{lemma:gamma_entails_delta_lifted_invariant} that  $\Gamma \entails \liftnovar{\Delta}{\LI(\pi)}$ and
	by Lemma~\ref{lemma:delta_entails_li} that
	$\Delta \entails \lnot\lifgammanovar{\LI(\pi)}$.
	As the empty clause derived in $\pi$ trivially contains neither variables nor ground terms and as any colored term either contains variables or is ground, at least one lifting condition holds for any term in $\LIpre(\pi)$ and hence all colored terms are lifted in $\LI(\pi)$.
	Therefore $\lifdeltanovar{\LI(\pi)} = \LI(\pi)$ and $\lifgammanovar{\LI(\pi)} = \LI(\pi)$.
\end{proof}

We finish this chapter by demonstrating the application of the interpolant extraction procedure $\LI$ on a larger example:

\begin{exa}
	\newcommand{\var}[1]{\ensuremath{v_{#1}}}
	Let $\Gamma = \{R(f(\var{1}, \var{6})), P(f(\var{2}, g(\var{3}, \var{4}))) \lor Q(g(\var{3}, b)), \lnot S(b) \}$
	and $\Delta = \{ S(\var{8}) \lor \lnot P(\var{9}) \lor \lnot R(\var{5}), \lnot Q(g(a, \var{7})) \}$.
	Hence $\Lang(\Gamma) \cap \Lang(\Delta) = \{ R, P, Q, S, g\}$, $\Lang(\Gamma) \setminus \Lang(\Delta) = \{f, b\}$ and $\Lang(\Delta)\setminus\Lang(\Gamma) = \{ a \}$.
	We can produce an interpolant for $\Gamma$ and $\Delta$ using the following refutation and extraction in the same notation as Example~\ref{exa:lemma_part_renaming}.
	We emphasise liftings of terms justified by being a ground term not occurring in the clause by \markA{}, and those justified by occurrences of variables which do not occur in the clause by \markB.


	%\tiny


	\begin{landscape}
		\vspace*{\fill}
		\begin{center}
		%\small
		\begin{adjustwidth}{-4cm}{}
			\begin{prooftree}
				\AxiomCm{ P(f(\var{2}, g(\var{3}, \var{4}))) \lor Q(g(\var{3}, b)) \mid \bot }
				\AxiomCm{ \lnot Q(g(a, \var{7})) \mid \top}

				\RightLabelm{\resrule{\resruleres}{\var{3}\mapsto a, \var{7}\mapsto b}}
				\BinaryInfCm{ P(f(\var{2}, g(a, \var{4}))) \mid Q(g(a, b))  }
				\noLine
				\LeftLabelm{\markA_1}
				\UnaryInfCm{  P(f(\var{2}, g(a, \var{4}))) \mid \exists y_b Q(g(a, y_b)) }

				\AxiomCm{  S(\var{8}) \lor \lnot P(\var{9}) \lor \lnot R(\var{5}) \mid \top }
				\AxiomCm{  R(f(\var{1}, \var{6})) \mid \bot}
				\RightLabelm{\resrule{\resruleres}{\var{5} \mapsto f(\var{1}, \var{6})}}
				\BinaryInfCm{ S(\var{8}) \lor \lnot P(\var{9}) \mid R(f(\var{1}, \var{6}))}
				\noLine
				\LeftLabelm{\markB_2}
				\UnaryInfCm{ S(\var{8}) \lor \lnot P(\var{9}) \mid \exists y_{f(\var{1}, \var{6})} R(y_{f(\var{1}, \var{6})})}

				\RightLabelm{\resrule{\resruleres}{\var{9}\mapsto f(v_2, g(a, \var{4}))}}
				\BinaryInfCm{ S(\var{8}) \mid  P(f(\var{2}, g(a, \var{4}))) \land \exists y_{f(\var{1}, \var{6})} R(y_{f(\var{1}, \var{6})}) \spam \lor \lnot P(f(\var{2}, g(a, \var{4}))) \land \exists y_b Q(g(a, y_b))  }
				\noLine
				\LeftLabelm{\markA\markB_3}
				\UnaryInfCm{ S(\var{8}) \mid \forall x_a \exists y_{f(\var{2}, g(a, \var{4}))} \big(  P(y_{f(\var{2}, g(a, \var{4}))}) \land \exists y_{f(\var{1}, \var{6})} R(y_{f(\var{1}, \var{6})}) \spam \lor \lnot P(y_{f(\var{2}, g(a, \var{4}))}) \land \exists y_b Q(g(x_a, y_b)) \big) }

				\AxiomCm{\lnot S(b) \mid \top}
				\RightLabelm{\resrule{\resruleres}{\var{8} \mapsto b}}

				\insertBetweenHyps{\hskip -2cm}
				\BinaryInfCm{ \square \mid S(b) \land  \forall x_a \exists y_{f(\var{2}, g(a, \var{4}))} \big(  P(y_{f(\var{2}, g(a, \var{4}))}) \land \exists y_{f(\var{1}, \var{6})} R(y_{f(\var{1}, \var{6})}) \spam \lor \lnot P(y_{f(\var{2}, g(a, \var{4}))}) \land \exists y_b Q(g(x_a, y_b)) \big) }
				\noLine
				\LeftLabelm{\markA_4}
				\UnaryInfCm{ \square \mid \exists y'_b \big( S(y'_b) \land  \forall x_a \exists y_{f(\var{2}, g(a, \var{4}))} \big(  P(y_{f(\var{2}, g(a, \var{4}))}) \land \exists y_{f(\var{1}, \var{6})} R(y_{f(\var{1}, \var{6})}) \spam \lor \lnot P(y_{f(\var{2}, g(a, \var{4}))}) \land \exists y_b Q(g(x_a, y_b)) \big) \big) }

			\end{prooftree}
		\end{adjustwidth}
		\end{center}
		\medskip

		\begin{adjustwidth}{-1cm}{}
		\begin{multicols}{2}
		$\markA_1$: The maximal colored term $b$ is lifted as it does not occur in the clause. On the other hand, the maximal colored term $a$ can not be lifted due to the opposite reason.

		$\markB_2$: The maximal colored term $f(v_1, v_6)$ contains the variables $v_1$ and $v_6$, which are not present in the clause.
		Due to the variable-disjointness restriction on clauses, these variables do not occur in any subsequent clause.

		$\markA\markB_3$: Clearly, the term $a$ is a subterm of $f(v_2, g(a, v_4))$, hence we must quantify $x_a$ before $y_{f(v_2, g(a, v_4))}$.

		$\markA_4$: We encounter another occurrence of the maximal colored term $b$ (cf.\ $\markA_1$).
		The lifting conditions however ensure that different lifting variables ($y_b$ and $y'_b$ respectively) are justified.
		\qedhere
	\end{multicols}
		\end{adjustwidth}

		\vspace*{\fill}
	\end{landscape}
\end{exa}


	\semanticcontent

}

\newcommand{\semanticcontent}{
	
\chapter{The semantic perspective on interpolation}

A curious feature of the interpolant theorem is that it admits a proof, which is distinct from the proof-theoretic ones discussed in the foregoing chapters, as it is a purely model-theoretic.
It is based on the joint consistency theorem by Robinson (\cite{robinson1956result}), which we show to be equivalent to the interpolation theorem.
The joint consistency theorem itself was presented as a proof of Beth's definability theorem, which is discussed in section~\ref{sec:beth}.

\section{Joint consistentcy}
\label{sec:joint_consistency}

The notion of joint consistency is based on separability of sets of formulas:

\begin{defi}[Separability]
	Let $\Gamma$ and $\Delta$ be sets of first-order formulas.
	A formula $A$ in the language $\Lang(\Gamma)\cap \Lang(\Delta)$ is said to \defiemph{separate} $\Gamma$ and $\Delta$ if $\Gamma \entails A$ and $\Delta \entails \lnot A$.
	$\Gamma$ and $\Delta$ are \defiemph{separable} if there exists a formula in the language $\Lang(\Gamma)\cap \Lang(\Delta)$ which separates $\Gamma$ and $\Delta$ and \defiemph{inseparable} otherwise.
\end{defi}

Note that for joint consistency, it is not necessary to require the original sets to be consistent as this is implied by separability:

\begin{lemma}
	\label{lemma:insep_consistent}
	Let $\Gamma$ and $\Delta$ be inseparable sets of first-order formulas. Then $\Gamma$ and $\Delta$ are each consistent.
\end{lemma}
\begin{proof}
	Suppose w.l.o.g.\ that $\Gamma$ is inconsistent. Then $\Gamma \entails \bot$, and as $\Delta \entails \top$, $\bot$ separates $\Gamma$ and $\Delta$.
\end{proof}

The joint consistency theorem shows that if there exists no formula in the language $\Lang(\Gamma)\cap \Lang(\Delta)$ which separates $\Gamma$ and $\Delta$, then there exists no formula in any language which separate $\Gamma$ and $\Delta$ as then, $\Gamma \cup \Delta$ is consistent:

\begin{thm}[Robinson's joint consistency theorem]
	\label{thm:robinson}
	Let $\Gamma$ and $\Delta$ be sets of first-order formulas.
	Then $\Gamma \cup \Delta$ is consistent if and only if $\Gamma$ and $\Delta$ are inseparable.
\end{thm}
The following proof essentially follows \cite{Henkin63} and \cite{chang1990model}.
\begin{proof}
	Suppose that $\Gamma\cup\Delta$ is consistent and let $M$ be a model of it.
	Then clearly for every formula $A$, if $\Gamma \entails A$, then $M \entails A$ as $M \entails \Gamma$.
	But $M \entails \Delta$, hence it can not be the case that $\Delta \entails \lnot A$.

	For the other direction, suppose that $\Gamma$ and $\Delta$ are inseparable.
	We proceed by iteratively constructing two maximal consistent sets of formulas $T$ and $T'$ such that $\Gamma \subseteq T$ and $\Delta \subset T'$ where $T \cup T'$ is consistent in order to then derive a model of this union, thus establishing the consistency of $\Gamma$ and $\Delta$.

	Let $C = \{c_0, c'_0, c_1, c'_1, \dots\}$ be
	a countably infinite set of fresh constant symbols.
	Let $\mathcal{A}_0, \mathcal{A}_1, \dots$ be an enumeration of all sentences in the language $\Lang(\Gamma) \cup C$
	and $\mathcal{B}_0, \mathcal{B}_1, \dots$ an enumeration of all sentences in the language $\Lang(\Delta) \cup C$.

	Let $T_0 = \Gamma$ and $T'_0 = \Delta$. 
	We construct
	$T_{i+1}$ 
	from
	$T_{i}$
	by means of the following formation rules:
	%\begin{enumerate}[(1)]
	\begin{enumerate}[~~(1)]
		\item
			\label{theory_construction_1}
			If $T_{i} \cup \{\mathcal{A}_i\}$ and $T'_{i}$ are separable, then $T_{i+1} \defeq{}\, T_i$.
		\item Otherwise:
			\label{theory_construction_2}
			\begin{enumerate}[(2a)]
			\label{theory_construction_2a}
				\item If $\mathcal{A}_i$ is of the form $\exists x A$, then $T_{i+1} \defeq{}\, T_i \cup \{ \mathcal{A}_i, A\subst{x/c_i} \}$.
			\label{theory_construction_2b}
				\item Otherwise $T_{i+1} \defeq{}\, T_i \cup \{ \mathcal{A}_i \}$.
			\end{enumerate}
	\end{enumerate}

	\noindent
	$T'_{i+1}$ is formed in a similar fashion:
	\begin{enumerate}[~~(1$'$)]
		\item
			If $T'_{i} \cup \{\mathcal{B}_i\}$ and $T_{i+1}$ are separable, then $T'_{i+1} \defeq{}\, T'_i$.
		\item
			\begin{samepage}
				Otherwise: 
			\begin{enumerate}[~(2$'$a)]
				\item If $\mathcal{B}_i$ is of the form $\exists x A$, then $T'_{i+1} \defeq{}\, T'_i \cup \{ \mathcal{B}_i, A\subst{x/c'_i} \}$.
				\item Otherwise $T'_{i+1} \defeq{}\, T'_i \cup \{ \mathcal{B}_i \}$.
			\end{enumerate}
			\end{samepage}
	\end{enumerate}

	Now let
	$T = \bigcup_{i\vargeq 0} T_i$
	and
	$T' = \bigcup_{i\vargeq 0} T'_i$.
	We prove properties on $T$ and $T'$ which will be vital for the construction of a model of $T\cup T'$:

	\begin{enumerate}[I.]
		\item
			\label{enum:theories_insep}
			$T_i$ and $T'_i$ are inseparable.

			$\Gamma$ and $\Delta$ are inseparable by assumption and clearly the construction of the subsequent elements of the sequence do not violate this invariant.

			\item 
			\label{enum:theories_consistent}
			$T_i$ and $T'_i$ are consistent.

			Immediate by \ref{enum:theories_insep} and Lemma~\ref{lemma:insep_consistent}.

		\item
			\label{enum:each_max_consistent}
			$T$ and $T'$ are each maximal consistent with respect to $\Lang(\Gamma) \cup C$ and $\Lang(\Delta) \cup C$ respectively.

			We show the result for $T$.
			By~\ref{enum:theories_consistent}, $T$ is consistent.
			Suppose that for some $i$, $\mathcal{A}_i \not\in T$ and $\lnot\mathcal{A}_i \not\in T$.
			Then by the construction of $T$, we can derive that
			$T_i \cup \{\mathcal{A}_i\}$ and $T'_i$ are separable.
			Hence also
			$T \cup \{\mathcal{A}_i\}$ and $T'$ are, i.e.\ there exists a formula $B_1$ in the language $\Lang(T\cup\{\mathcal{A}_i\}) \cap \Lang(T') = (\Lang(\Gamma) \cap \Lang(\Delta)) \cup C$ such that
			$T \cup \{\mathcal{A}_i\} \entails B_1$ and $T' \entails \lnot B_1$.
			By the deduction theorem, we also have that \markA{} $T \entails \mathcal{A}_i \limpl B_1$.

			As we also assume that $\lnot \mathcal{A}_i \not\in T$, by a similar argument, there exists a formula $B_2$ in the language  $(\Lang(\Gamma) \cap \Lang(\Delta)) \cup C$ such that 
			\markB{} $T \entails \lnot \mathcal{A}_i \limpl B_2$ and $T' \entails \lnot B_2$.

			Then however \markA{} and \markB{} entail that in any model, depending on whether $\mathcal{A}_i$ holds in the model, at least one of $B_1$ and $B_2$ holds, i.e.\ $T \entails B_1 \lor B_2$.
			But as neither $B_1$ nor $B_2$ hold in $T'$, we obtain that $T' \entails \lnot (B_1 \lor B_2)$, in effect establishing that $B_1 \lor B_2$ separates $T$ and $T'$, a contradiction to \ref{enum:theories_insep}.


		\item
			\label{enum:intersection_consistent}
			$T \cap T'$ is maximal consistent with respect to $(\Lang(\Gamma) \cap \Lang(\Delta)) \cup C$.

			By~\ref{enum:each_max_consistent}, for every formula $A$ in $(\Lang(\Gamma) \cap \Lang(\Delta)) \cup C$ it holds that either 
			$A \in T$ or $\lnot A \in T$ as well as
			$A \in T'$ or $\lnot A \in T'$. As $T$ and $T'$ are inseparable, either $A \in T$ and $A\in T'$ or otherwise $\lnot A \in T$ and $\lnot A \in T'$.

	\end{enumerate}


	As $T$ is consistent, let $M$ be a model of $T$.
	Due to~\ref{enum:each_max_consistent}, for each term $t$ in $\Lang(\Gamma)\cup C$, $\exists x\, (t = x) \in T$ and hence by~\ref{theory_construction_2a}, there is some $c_i \in C$ such that $t=c_i \in T$.
	Therefore we can find a submodel of $N$ of $M$ which as $M$ is in the language $\Lang(\Gamma)\cup C$ such that
	every domain element in $N$ corresponds to a constant symbol in $C$.
	%the domain of $N$ is $\{ \interpretation{M}(c) \mid c \in C\}$, where $\interpretation{M}$ is the interpretation of $M$.
	Models $M'$ of $T'$ allow by a similar reasoning for finding submodels $N'$ of $M'$.

	As by~\ref{enum:intersection_consistent}, $T$ and $T'$ agree on all formulas of $(\Lang(\Gamma) \cap \Lang(\Delta))\cup C$, 
	we are able to find an isomorphism between the reducts $N$ and $N'$ to their common language.
	Hence we may build a common model $K$ based on $N$ and extending it to $\Lang(\Delta)$ by copying the respective interpretation of $N'$ with regard to the isomorphism.
	%Hence we may build $N'_c$ from $N'$ by exchanging every $c'_i \in C$ by its corresponding $c_j \in C$.
	%Now we see that by extending the $N$ to $\Lang(\Delta)$ by copying the interpretation of $N''$,
	Thus as $N\entails T$ and $N'\entails T'$, $K\entails T \cup T'$, which implies that $\Gamma\cup\Delta$ is consistent.
\end{proof}

\section{Joint consistency and interpolation}

Despite the fact that the proof given in the previous section is of a different nature than the ones given in the previous chapters, it is easy to see that it expresses an equivalent notion.
To that end, let us recall the Interpolation Theorem~\ref{thm:interpolation} in the reverse formulation:

\interpolationRevThm*

\begin{prop}
	Theorem~\ref{thm:robinson} and Theorem~\ref{thm:interpolation} are equivalent.
\end{prop}
\begin{proof}
	It is easy to see that the notion of reverse interpolant and separating formulas coincide.
\end{proof}






}

\newcommand{\appendixcontent}{

	\appendix

	
\chapter{Interpolant extraction from resolution proofs due to Huang}  
\label{sec:huang}
\label{chap:huang}

This section essentially presents the original proof of \cite{Huang95} in a modern format.
It forms the base for our work in chapter~\ref{sec:two_phases} and \ref{sec:one_phase}, and we refer to these chapters for lemmas and definitions which also apply here.
Section~\ref{sec:huang_commentary} features comments on the original publication. 

\section{Propositional interpolants}


Let $\Gamma \cup \Delta$ be unsatisfiable and $\pi$ be a proof of the empty clause from $\Gamma \cup \Delta$. Then $\PI$ is a function that returns a interpolant with respect to the current clause. 

\begin{defi}[Propositional interpolant]
	Let $\pi$ be a resolution refutation of $\Gamma \cup \Delta$.
	A formula $A$ is a \defiemph{propositional interpolant} if
	\label{def:huang_orig_rel_prop_interpol}
	\begin{enumerate}
		\item $\Gamma \entails A$
			\label{huang_orig_rel_prop_interpol_cond1}
		\item $\Delta \entails \lnot A$
			\label{huang_orig_rel_prop_interpol_cond2}
		\item $\Pred(A) \subseteq (\Pred(\Gamma) \intersect \Pred(\Delta)) \cup \{\top, \bot\} $.
			\label{huang_orig_rel_prop_interpol_cond_lang}
	\end{enumerate}


	For a clause $C$ in $\pi$, a formula $A_C$ is a \defiemph{propositional interpolant relative to $C$} if
	\begin{enumerate}
		\item $\Gamma \entails A_C \lor C$
			%\label{huang_orig_rel_prop_interpol_cond1}
		\item $\Delta \entails \lnot A_C \lor C$
			%\label{huang_orig_rel_prop_interpol_cond2}
		\item $\Pred(A_C) \subseteq (\Pred(\Gamma) \intersect \Pred(\Delta)) \cup \{\top, \bot\} $.
			%\label{huang_orig_rel_prop_interpol_cond_lang}
	\end{enumerate}

	The propositional interpolant for the empty clause derived in $\pi$ is denoted by $\PI(\pi)$.\qedhere
\end{defi}

The third condition of a propositional interpolant will sometimes be referred to as \emph{language restriction}.
It is easy to see that the propositional interpolant relative to the empty clause of a resolution refutation is a propositional interpolant.

We refer to Definition~\ref{def:PI} for the definition of $\PI$.

\begin{prop}
	\label{prop:prop_interpol}
	Let $C$ be a clause of a resolution refutation of $\Gamma \cup \Delta$.
	Then $\PI(C)$ is a propositional interpolant with respect to $C$. 
\end{prop}
\begin{proof}
	Proof by induction on the number of rule applications including the following strengthenings:
	$\Gamma \entails \PI(C) \lor C_\Gamma$ and
	$\Delta \entails \lnot \PI(C) \lor C_\Delta$, where $D_\Phi$ denotes the clause D with only the literals which are contained in $\Lang(\Phi)$. They clearly imply conditions \ref{huang_orig_rel_prop_interpol_cond1} and \ref{huang_orig_rel_prop_interpol_cond2} of definition \ref{def:huang_orig_rel_prop_interpol}. 

	\begin{indproof}
		\indproofitem{Base case}
			Suppose no rules were applied. We distinguish two possible cases:
			\begin{enumerate}
				\item $C \in \Gamma$.
					Then $\PI(C) = \bot$. Clearly $\Gamma \entails \bot \lor C_\Gamma$ as $C_\Gamma = C \in \Gamma$, $\Delta \entails \lnot \bot \lor C_\Delta$ and $\bot$ satisfies the restriction on the language.

				\item $C \in \Delta$.
					Then $\PI(C) = \top$. Clearly $\Gamma \entails \top \lor C_\Gamma$, $\Delta \entails \lnot \top \lor C_\Delta$ as $C_\Delta = C \in \Delta$ and $\top$ satisfies the restriction on the language.
			\end{enumerate}

			Suppose the property holds for $n$ rule applications.
			We show that it holds for $n+1$ applications by considering the last one:

		\indproofitem{Resolution}
			Suppose the last rule application is an instance of resolution. Then it is of the form:
			\begin{prooftree}
				\AxiomCm{C_1: D \lor l}
				\AxiomCm{C_2: E \lor \lnot l'}
				\RightLabelm{\quad l\sigma = l'\sigma}
				\BinaryInfCm{C: (D\lor E)\sigma}
			\end{prooftree}

			By the induction hypothesis, we can assume that:

			$\Gamma \entails \PI(C_1) \lor (D\lor l)_\Gamma$

			$\Delta \entails \lnot \PI(C_1) \lor (D\lor l)_\Delta$

			$\Gamma \entails \PI(C_2) \lor (E\lor \lnot l')_\Gamma$

			$\Delta \entails \lnot \PI(C_2) \lor (E\lor \lnot l')_\Delta$

			We consider the respective cases from definition \ref{def:PI_resolution}:

			\begin{enumerate}
				\item $l$ is $\Gamma$-colored.
					\label{huang_proof_prop_case_1}
					Then $\PI(C) = [\PI(C_1) \lor \PI(C_2)]\sigma$. 

					As $\Pred(l) \in \Lang(\Gamma)$,
					$\Gamma \entails (\PI(C_1) \lor D_\Gamma\lor l)\sigma$
					as well as $\Gamma \entails (\PI(C_2) \lor E_\Gamma\lor \lnot l')\sigma$.
					By a resolution step, we get $\Gamma \entails (\PI(C_1) \lor \PI(C_2))\sigma \lor ((D \lor E)\sigma)_\Gamma$.

					Furthermore, as $\Pred(l) \not\in \Lang(\Delta)$, 
					$\Delta \entails (\lnot\PI(C_1) \lor D_\Delta)\sigma$
					as well as $\Delta \entails (\lnot\PI(C_2) \lor E_\Delta)\sigma$.
					Hence it certainly holds that $\Delta \entails (\lnot \PI(C_1) \lor \lnot\PI(C_2))\sigma \lor (D \lor E)\sigma_\Delta$.

					The language restriction clearly remains satisfied as no non-logical symbols are added.

				\item $l$ is $\Delta$-colored.
					\label{huang_proof_prop_case_2}
					Then $\PI(C) = [\PI(C_1) \land \PI(C_2)]\sigma$. 

					As $\Pred(l) \not\in \Lang(\Gamma)$,
					$\Gamma \entails (\PI(C_1) \lor D_\Gamma)\sigma$
					as well as $\Gamma \entails (\PI(C_2) \lor E_\Gamma)\sigma$.
					Suppose that in a model $M$ of $\Gamma$, $M \notentails D_\Gamma$ and $M \notentails E_\Gamma$. Then $M \entails \PI(C_1) \land \PI(C_2)$.
					Hence 
					$\Gamma \entails (\PI(C_1) \land \PI(C_2))\sigma \lor ((D \lor E)\sigma)_\Gamma$.

					Furthermore due to $\Pred(l) \in \Lang(\Delta)$,
					$\Delta \entails (\lnot\PI(C_1) \lor D_\Delta \lor l)\sigma$
					as well as $\Delta \entails (\lnot\PI(C_2) \lor E_\Delta \lor \lnot l')\sigma$.
					By a resolution step, we get $\Delta \entails (\lnot\PI(C_1) \lor \lnot\PI(C_2))\sigma \lor (D_\Delta \lor E_\Delta)\sigma $
					and hence 
					$\Delta \entails \lnot (\PI(C_1) \land \PI(C_2))\sigma \lor (D_\Delta \lor E_\Delta)\sigma $.

					The language restriction again remains intact.

				\item $l$ is gray.
					Then $\PI(C) = [(l \land \PI(C_2)) \lor (\lnot l' \land \PI(C_1)) ]\sigma $

					First, we have to show that 
					$ \Gamma \entails [(l \land \PI(C_2)) \lor (l' \land \PI(C_1)) ]\sigma \lor ((D \lor E)\sigma)_\Gamma$.
					Suppose that in a model $M$ of $\Gamma$, $M \notentails D_\Gamma$ and $\Gamma \notentails E$. Otherwise we are done.
					The induction assumption hence simplifies to $M \entails \PI(C_1) \lor l$ and $M \entails \PI(C_2) \lor \lnot l'$ respectively.
					As $l\sigma = l'\sigma$, by a case distinction argument on the truth value of $l\sigma$, we get that either $M \entails (l \land \PI(C_2))\sigma$ or $M \entails  (\lnot l' \land \PI(C_1))\sigma$.


					Second, we show that 
					$\Delta \entails ((l \lor \lnot \PI(C_1)) \land (\lnot l' \lor \lnot \PI(C_2)))\sigma \lor ((D \lor E)\sigma)_\Delta$.
					Suppose again that in a model $M$ of $\Delta$, $M \notentails D_\Delta$ and $\Gamma \notentails E_\Delta$. 
					Then the required statement follows from the induction hypothesis.

					The language condition remains satisfied as only the common literal $l$ is added to the interpolant.


			\end{enumerate}

			\indproofitem{Factorization}
			Suppose the last rule application is an instance of factorization. Then it is of the form:
			\begin{prooftree}
				\AxiomCm{C_1: l \lor l' \lor D}
				\RightLabelm{\quad \sigma = \mgu(l, l')}
				\UnaryInfCm{C: (l \lor D)\sigma}
			\end{prooftree}

			Then the propositional interpolant $\PI(C)$ is defined as $\PI(C_1)$.
			By the induction hypothesis, we have:

			$\Gamma \entails \PI(C_1) \lor (l \lor l' \lor D)_\Gamma$

			$\Delta \entails \PI(C_1) \lor (l \lor l' \lor D)_\Delta$

			It is easy to see that then also:

			$\Gamma \entails (\PI(C_1)\lor (l \lor D)_\Gamma)\sigma$

			$\Delta \entails (\PI(C_1)\sigma \lor (l \lor D)_\Delta)\sigma$

			The restriction on the language trivially remains intact.


		\indproofitem{Paramodulation}	
			Suppose the last rule application is an instance of paramodulation. Then it is of the form:
			\begin{prooftree}
				\AxiomCm{C_1: D \lor s=t}
				\AxiomCm{C_2: E\occurat{s}{p}}
				\RightLabel{$\quad \sigma = \mgu(s, r)$}
				\BinaryInfCm{C: D \lor E\occurat{t}{p}}
			\end{prooftree}


			By the induction hypothesis, we have:

			$\Gamma \entails \PI(C_1) \lor (D\lor s=t)_\Gamma$

			$\Delta \entails \lnot \PI(C_1) \lor (D\lor s=t)_\Delta$

			$\Gamma \entails \PI(C_2) \lor (E[r])_\Gamma$

			$\Delta \entails \lnot \PI(C_2) \lor (E[r])_\Delta$

			First, we show that $\PI(C)$ as constructed in case \ref{def:PI_paramod_3} of the definition is a propositional interpolant in any of these cases:

			$\PI(C) = (s=t \land \PI(C_2)) \lor (s\neq t \land \PI(C_1)) $

			Suppose that in a model $M$ of $\Gamma$, $M \notentails D\sigma$ and $M \notentails E\occurat{t}{p}\sigma$. Otherwise we are done.
			Furthermore, assume that $M \entails (s=t)\sigma$. Then $M \notentails E\occurat{r}{p}\sigma$, but then necessarily $M \entails \PI(C_2)\sigma$. \\
			On the other hand, suppose $M \entails (s\neq t)\sigma$. As also $M \notentails D\sigma$, $M \entails \PI(C_1)\sigma$.
			Consequently, $M \entails [(s=t \land \PI(C_2)) \lor (s\neq t \land \PI(C_1))]\sigma \lor [(D \lor E)_\Gamma]\sigma$

			By an analogous argument, we get $\Delta \entails [(s=t \land \lnot \PI(C_2)) \lor (s\neq t \land \lnot \PI(C_1))]\sigma \lor [(D \lor E)_\Delta]\sigma$,
			which implies
			$\Delta \entails [( s\neq t \lor \lnot \PI(C_2)) \land (s = t \lor \lnot \PI(C_1))]\sigma \lor ((D \lor E)_\Delta)\sigma $

			%By a similar case distinction for a model $M$ of $\Delta$ and assuming that $M \notentails D_\Delta$ and $M \notentails E_\Delta$, we get that if $M \entails (s=t)\sigma$, $M \entails \lnot P$, which implies

			The language restriction again remains satisfied as the only predicate, that is added to the interpolant, is $=$.

			This concludes the argumentation for case \ref{def:PI_paramod_3}. 

			The interpolant for case \ref{def:PI_paramod_1} differs only by an additional formula added via a disjunction and hence condition \ref{huang_orig_rel_prop_interpol_cond1} of definition \ref{def:huang_orig_rel_prop_interpol} holds by the above reasoning.
			As the adjoined formula is a contradiction, its negation is valid which in combination with the above reasoning establishes condition \ref{huang_orig_rel_prop_interpol_cond2}.
			Since no new predicated are added, the language condition remains intact. 

			The situation in case \ref{def:PI_paramod_2} is somewhat symmetric: 
			As a tautology is added to the interpolant with respect to case \ref{def:PI_paramod_1}, condition \ref{huang_orig_rel_prop_interpol_cond1} is satisfied by the above reasoning.
			For condition \ref{huang_orig_rel_prop_interpol_cond2}, consider that the negated interpolant for case \ref{def:PI_paramod_1} implies the negated interpolant for this case.
			The language condition again remains intact.
			\qedhere
	\end{indproof}
\end{proof}

\section{Propositional refutations}
Before we are able to specify a procedure to transform the propositional interpolant generated by $\PI$ into a proper interpolant without any colored terms,
we need to make some observations about tree refutations.

In a tree refutation where the input clauses have a disjoint sets of variables, every variable has a unique ancestor which traces back to an input clause and hence appears only along a certain path.
This insight allows us to push substitutions of the variables upwards along this path and arrive at the following definition and lemma:



%For every unification $\sigma$ in the deduction and for every variable $x$, either $x\sigma = x$ or $x\sigma = t$ where $x$ does not occur in $t$.
%Hence along the path from the input clause to its unification or removal by resolution or factorization, it occurs unchanged.
%Therefore replacing $x$ along the path with $\sigma x$, where $\sigma$ is a non-trivial unifier used on $x$ in the derivation creates still a valid refutation of whatever.

\begin{defi}
	A resolution refutation is a \defiemph{propositional refutation} if no nontrivial substitutions are employed.
\end{defi}

\begin{lemma}
	Let $\Phi$ be unsatisfiable.
	Then there is a propositional refutation of $\Phi$ which starts from instances of $\Phi$.
\end{lemma}
\begin{proof}
	Let $\pi$ be a resolution refutation of $\Phi$.
	By Lemma \ref{lemma:bin_tree_deduction}, we can assume without loss of generality that $\pi$ is a tree refutation where the sets of variables of the input clauses are disjoint.
	Furthermore, we can assume that only most general unifiers are employed in $\pi$.

	Then any unifier in $\pi$ is either trivial on $x$ or there is one unique unifier $\sigma$ in $\pi$ with $x\sigma = t$ where $x$ does not occur in $t$.
	Hence along the path through the deduction where $x$ occurs, it remains unchanged.
	Therefore we can create a new resolution refutation $\pi'$ from $\pi$ where $x$ is replaced by $t$.
	Clearly $\pi'$ is rooted in instances of $\Phi$.

	By application of this procedure to all variable occurring in $\pi$, we obtain a desired resolution refutation.
\end{proof}

Even though propositional refutations have nice properties for theoretical analysis, their use in practise is not desired as its construction involves a considerable blowup of the refutation. 
But its use is still justified in this instance as we can show for arbitrary refutations $\pi$
that the algorithm stated in \ref{def:PI} gives closely related results for both $\pi$ and its corresponding propositional refutation.

\begin{lemma}
	Let $\pi$ be a resolution refutation of $\Phi$ and $\pi'$ a propositional refutation corresponding to $\pi$.
	Then for every clause $C$ in $\pi$ and its corresponding clause $C'$ in $\pi'$, $\PI(C)\sigma = \PI(C')$, where $\sigma$ is the composition of the unifications of $\pi$ which are applied to the variables occurring in $C$ .
\end{lemma}
\begin{proof}
	For the construction of the propositional skeleton of $\PI(\cdot)$ only the coloring of the clauses is relevant and since this is the same in both $\pi$ and $\pi'$, it coincides for $\PI(C)$ and $\PI(C')$.

	Hence $\PI(C)$ and $\PI(C')$ differ only in their term structure. 
	To be more specific, in $\PI(C')$, the composition of substitutions that are applied in $\pi$ have already been applied to the initial clauses of $\pi'$. 
	Note that substitution commutes with the rules of resolution.
	Therefore the only difference between $\PI(C)$ and $\PI(C')$ is that at certain term positions, there are variables in $\PI(C)$ where in $\PI(C')$ by some substitution a different term is located. 
	But these substitutions are certainly applied by $\sigma$, hence $\PI(C)\sigma = \PI(C')$.
\end{proof}

\section{Lifting of colored symbols}

We rely on the same definition of lifting as given in \ref{sec:lifting}.
First, we consider the lifting of the $\Delta$-terms, which corresponds to Lemma~\ref{lemma:gamma_entails_lifted_interpolant}, but differs in the proof by relying on propositional refutations.\nopagebreak[4]

\begin{lemma}
	\label{lemma:gamma_entails_lifted_interpolantHuang}
	Let $\pi$ be a resolution refutation of $\Gamma \cup \Delta$. 
	Then $\Gamma \entails \lift{\Delta}{ \PI(C) \lor C }{x} $ for $C$ in $\pi$.
\end{lemma}
\begin{proof}
	We proof this result by induction on the number of rule applications in the propositional refutation corresponding to $\pi$. 
	Similar to the proof of \ref{prop:prop_interpol}, we show the strengthening:
	$\Gamma \entails \lift{\Delta}{ \PI(C) \lor C_\Gamma }{x} $ for $C$ in $\pi$.

	\begin{indproof}

			\newcommand{\lif}[1]{\lift{\Delta}{#1}{x}}
			\indproofitem{Base case}

			If no rules have been applied, $C$ is an instance of a clause of either $\Gamma$ or $\Delta$.
			In the former case, all $\Delta$-terms of $C$ were added by unification, hence by replacing them with variables, we obtain a clause $C'$ which still is an instance of $C$ and consequently is implied by $\Gamma$. 
			In the latter case, $\PI(C) = \top$. 

		\indproofitem{Resolution} Suppose the last rule application is an instance of resolution. Then it is of the form:
			\begin{prooftree}
				\AxiomCm{C_1: D\lor l}
				\AxiomCm{C_2: E\lor \lnot l}
				\BinaryInfCm{C: D \lor E}
			\end{prooftree}

			By the induction hypothesis,

			$\Gamma \entails \lift{\Delta}{ \PI(C_1) \lor (D \lor l)_\Gamma }{x}$ and

			$\Gamma \entails \lift{\Delta}{ \PI(C_2) \lor (E \lor \lnot l)_\Gamma }{x}$

			which by Lemma \ref{lemma:lift_logic_commute} is equivalent to

			$\Gamma \entails \lift{\Delta}{ \PI(C_1) }{x} \lor
			\lift{\Delta}{ D_\Gamma }{x} \lor
			\lift{\Delta}{ l_\Gamma }{x} \;\; {(\circ)} $
			and

			$\Gamma \entails \lift{\Delta}{ \PI(C_2) }{x} \lor
			\lift{\Delta}{ E_\Gamma }{x} \lor
			\lnot \lift{\Delta}{ l_\Gamma }{x} \;\; {(*)}$ .


			\begin{enumerate}
				\item Suppose $l$ is $\Gamma$-colored.
					Then $\PI(C) = \PI(C_1) \lor \PI(C_2)$.
					By using resolution of ${(*)}$ and ${(\circ)}$ on $\lift{\Delta}{l_\Gamma}{x}$, we get that 
					$$\Gamma \entails\lift{\Delta}{ \PI(C_1) }{x} \lor \lift{\Delta}{ \PI(C_2) }{x} \lor
					\lift{\Delta}{ D_\Gamma }{x} \lor
					\lift{\Delta}{ E_\Gamma }{x}.$$
					Several applications of Lemma \ref{lemma:lift_logic_commute} give
					$\Gamma \entails\lift{\Delta}{ \PI(C_1)  \lor  \PI(C_2) \lor (D \lor E)_\Gamma }{x}$.

				\item Suppose $l$ is $\Delta$-colored.
					Then $\PI(C) = \PI(C_1) \land \PI(C_2)$.

					As $l$ and $\lnot l$ are not contained in $\Lang(\Gamma)$, we get that 

					$\Gamma \entails \lift{\Delta}{ \PI(C_1) }{x} \lor
					\lift{\Delta}{ D_\Gamma }{x}$
					and

					$\Gamma \entails \lift{\Delta}{ \PI(C_2) }{x} \lor
					\lift{\Delta}{ E_\Gamma }{x}$.

					So if in a model $M$ of $\Gamma$ we have that
					$M \notentails \lift{\Delta}{ D_\Gamma }{x}$ and 
					$M \notentails \lift{\Delta}{ E_\Gamma }{x}$, it follows that $M \entails \lift{\Delta}{ \PI(C_1) }{x}$ and $M \entails \lift{\Delta}{ \PI(C_2) }{x}$. Hence by Lemma~\ref{lemma:lift_logic_commute}
					$M \entails \lift{\Delta}{ \PI(C_1) \land \PI(C_2) }{x} \lor
					\lift{\Delta}{ (D \lor E)_\Gamma }{x}$.

				\item Suppose $l$ is gray.
					Then $\PI(C) =  (l \land \PI(C_2)) \lor (\lnot l \land \PI(C_1))$.

					We show that 
					$\Gamma \entails \lift{\Delta}{(l \land \PI(C_2)) \lor (\lnot l \land \PI(C_1)) \lor (D \lor E)_\Gamma  }{x} $. 

					Suppose that for a model $M$ of $\Gamma$ that 
					$M \notentails \lift{\Delta}{ D_\Gamma }{x}$ and 
					$M \notentails \lift{\Delta}{ E_\Gamma }{x}$.
					Then by ${(\circ)}$
					and ${(*)}$, we get that\nopagebreak

					$M \entails \lift{\Delta}{ \PI(C_1) }{x} \lor
					\lift{\Delta}{ l_\Gamma }{x}$ as well as

					$M \entails \lift{\Delta}{ \PI(C_2) }{x} \lor
					\lnot \lift{\Delta}{ l_\Gamma }{x}$.

					So $M \entails \lift{\Delta}{ l_\Gamma }{x}$ implies that 
					$M \entails \lift{\Delta}{\PI(C_2)}{x}$ and 
					$M \entails \lnot \lift{\Delta}{ l_\Gamma }{x}$  implies that 
					$M \entails \lift{\Delta}{\PI(C_1)}{x}$ and 

					Therefore
					$M\entails (\lif{l} \land \lif{\PI(C_2)}) \lor (\lnot \lif{l} \land \lif{\PI(C_1)}) \lor (\lif{D_\Gamma} \lor \lif{E_\Gamma}) $,
					and several applications of Lemma \ref{lemma:lift_logic_commute} give
					$M\entails \lif{(l \land \PI(C_2)) \lor (\lnot {l} \land {\PI(C_1)}) \lor ({D_\Gamma} \lor {E_\Gamma})} $.
			\end{enumerate}


		\indproofitem{Factorization} Suppose the last rule application is an instance of factorization. Then it is of the form:
			\begin{prooftree}
				\AxiomCm{C_1: l \lor l \lor D}
				\UnaryInfCm{C: l \lor D}
			\end{prooftree}

			The propositional interpolant directly carried over from $C_1$, i.e.~$\PI(C) = \PI(C_1)$.

			By the induction hypothesis, we get that $\Gamma \entails \lif{\PI(C_1) \lor (l \lor l \lor D)_\Gamma}$.
			By Lemma \ref{lemma:lift_logic_commute}, 

			$\Gamma \entails \lif{\PI(C_1)} \lor (\lif{l_\Gamma} \lor  \lif{l_\Gamma} \lor \lif{D_\Gamma})$,

			which clearly is equivalent to

			$\Gamma \entails \lif{\PI(C_1)} \lor (\lif{l_\Gamma} \lor \lif{D_\Gamma})$,

			so by again applying Lemma \ref{lemma:lift_logic_commute}, we arrive at

			$\Gamma \entails \lif{\PI(C_1) \lor (l \lor D)_\Gamma}$.



		\indproofitem{Paramodulation} Suppose the last rule application is an instance of paramodulation. Then it is of the form:
			\begin{prooftree}
				\AxiomCm{C_1: D \lor s=t}
				\AxiomCm{C_2: E\occurat{s}{p}}
				\BinaryInfCm{C: D \lor E\occurat{t}{p}}
			\end{prooftree}

			By the induction hypothesis, we have that 

			$\Gamma \entails \lif{\PI(C_1) \lor (D \lor s=t)_\Gamma}$ and 

			$\Gamma \entails \lif{\PI(C_2) \lor (E\occurat{s}{p})_\Gamma}$.

			By Lemma \ref{lemma:lift_logic_commute}, we get that 

			$\Gamma \entails \lif{\PI(C_1)} \lor \lif{D_\Gamma} \lor \lif{s}=\lif{t}$ and 

			$\Gamma \entails \lif{\PI(C_2)} \lor \lif{(E\occurat{s}{p})_\Gamma}$.

			We distinguish two cases:\nopagebreak
			\begin{enumerate}
				\item Suppose $s$ does not occur in a maximal $\Delta$-term $h\occur{s}$ in $E\occurat{s}{p}$ which occurs more than once in $\PI(E(s)) \lor E\occurat{s}{p}$.

					We show that $\Gamma \entails \lif{ (s=t \land \PI(C_2)) \lor (s\neq t \land \PI(C_1)) \lor (D \lor E\occurat{t}{p})_\Gamma}$, which subsumes the cases \ref{def:PI_paramod_2} and \ref{def:PI_paramod_3} of 
					Definition \ref{def:PI_paramod}. By Lemma \ref{lemma:lift_logic_commute}, this is equivalent to

					$\Gamma \entails (\lif{s}=\lif{t} \land \lif{\PI(C_2)}) \lor (\lif s\neq \lif t \land \lif{\PI(C_1)}) \lor (\lif{D_\Gamma} \lor \lif{(E\occurat{t}{p})_\Gamma})$

					Suppose that $M$ is a model and $\alpha$ an assignment to the free variables 
					such that $M_\alpha \entails \Gamma$,
					$M_\alpha \notentails \lift{\Delta}{ D_\Gamma }{x}$ and 
					$M_\alpha \notentails \lift{\Delta}{ (E\occurat{t}{p})_\Gamma }{x}$.
					We show that then, depending on whether $\lif{s} = \lif{t}$ holds in $M_\alpha$, one of the first two disjuncts holds in $M_\alpha$.

					In case $M_\alpha \entails \lif{s} = \lif{t}$ we also get
					$M_\alpha \notentails \lift{\Delta}{ (E\occurat{s}{p})_\Gamma }{x}$ and consequently by the induction hypothesis $M_\alpha\entails \lif{\PI(C_2)}$.

					However in case $M_\alpha \entails \lif{s} \neq \lif{t}$ we get by the induction hypothesis that 
					$M\entails \lif{\PI(C_1)}$.

					\label{njktahjtkhltah}

				\item Otherwise $s$ occurs in a maximal $\Delta$-term $h\occur{s}$ in $E\occurat{s}{p}$ which occurs more than once in $\PI(E(s)) \lor E\occurat{s}{p}$.
				This reflects case \ref{def:PI_paramod_1} of Definition \ref{def:PI_paramod}.

					Then models are possible in which $s=t$ holds, while at the same time $\lif{h\occur{s}} \neq \lif{h\occur{t}}$ does not as $h\occur{s}$ and $h\occur{t}$ are replaced by distinct variables due to being different $\Delta$-terms.

					Therefore we amend the proof of case \ref{njktahjtkhltah} as follows:

					In case $M_\alpha \entails \lif{s} = \lif{t}$ (otherwise proceed as in case \ref{njktahjtkhltah}), 
					one of the following cases holds:

					\begin{itemize}
					\item $M_\alpha\entails \lif{h\occur{s}} = \lif{h\occur{t}}$. From this, it follows that as in the proof of case \ref{njktahjtkhltah}, $M \not\entails \lif{(E\occurat{s}{p})_\Gamma}$ and consequently $M \entails \lif{\PI(C_2)}$ again by the induction hypothesis.

					\item 
						$M_\alpha \entails \lif{h\occur{s}} \neq \lif{h\occur{t}}$.
						However as here $\PI(C)$ contains the with respect to case \ref{njktahjtkhltah} additional disjunct $s=t \land h\occur{s} \neq h\occur{t}$,
						$M_\alpha \entails \lif{PI(C)}$ due to $M_\alpha \entails \lif{s}=\lif{t} \land \lif{h\occur{s}} \neq \lif{h\occur{t}}.$
					\qedhere
					\end{itemize}
			\end{enumerate}

	\end{indproof}
\end{proof}

From this, we can directly proof the theorem by relying on the notion of symmetry already shown in Section~\ref{sec:symmetry}.

\begin{thm}
	Let $\pi$ be a resolution refutation of $\Gamma \cup \Delta$ and
	$t_1, \dots, t_n$ be the maximal colored terms in $\PI(\pi)$ in ascending order.
	Then
	$Q_1 z_{t_1} \ldots Q_n z_{t_n}\,\lifgamma{\lifdelta{\PI(\pi)}}$, where $Q_i$ is $\forall$ $(\exists)$ if $z_{t_i}$ replaces a $\Delta$ $(\Gamma)$-term, is an interpolant.
\end{thm}
\begin{proof}
	By Lemma \ref{lemma:gamma_entails_lifted_interpolantHuang}, $\Gamma \entails \forall x_{s_1} \ldots \forall x_{s_m}\,\lifdelta{\PI(\pi)}$, where $s_1, \dots s_m$ are the maximal colored $\Delta$-terms in $\PI(\pi)$.

	A term in $\lifdelta{\PI(\pi)}$ is either $x_{s_i}$, $1 \varleq i \varleq m$, a gray term or a $\Gamma$-term.
	Let $t$ be a maximal $\Gamma$-term in $\PI(\pi)$ and ${r_1}, \dots, {r_k}$ the maximal $\Delta$-terms in\nolinebreak{} $t$.
	Then in $\lifdelta{\PI(\pi)}$, the terms ${r_1}, \dots, {r_k}$ are replaced by $x_{r_1}, \dots, x_{r_k}$ respectively.
	Note that as all of ${r_1}, \dots, {r_k}$ due to being strict subterms of $t$ are of strictly smaller length than $t$, all of $x_{r_1}, \dots, x_{r_k}$ precede $y_t$ in the arrangement of the lifting variables.
	%Note that all of $r_1, \dots, r_k$ are subterms of $t$.
	%Note that the $\Delta$-terms, which are replaced by $x_{r_1}, \ldots, x_{i_{j_k}}$ respectively, are each of strictly smaller size than $t$ as they are strict subterms of $t$.

	%Then it is of the form $f(x_{i_1}, \ldots, x_{i_{n_x}}, u_1, \ldots, u_{n_u}, v_1, \ldots, v_{n_v})$, where $f$ is $\Gamma$-colored, the $u_j$, $1\varleq j \varleq n_u$ are gray terms and the $v_j$, $1\varleq j\varleq n_v$ are $\Gamma$-terms.

	In $\lifgamma{\lifdelta{\PI(\pi)}}$, $t$ is lifted by $y_t$, which is existentially quantified.
	Hence $t$ is a witness for $y_j$ as due to the quantifier ordering,
	it is bound in the scope of the quantification of the lifting variables $x_{r_1}, \dots, x_{r_k}$.
	Therefore $\Gamma \entails Q_1 z_{t_1} \ldots Q_n z_n\,\lifgamma{\lifdelta{\PI(\pi)}}$.

	By Corollary \ref{cor:delta_entails_lifted_interpolant} $\Delta \entails \forall y_{u_1} \dots \forall y_{u_{k'}}\,\lnot \lift{\Gamma}{\PI(\pi)}{y}$, where $u_1, \dots u_{k'}$ are the maximal colored $\Gamma$-terms in $\PI(\pi)$.

	By a similar line of argumentation as above, we can replace the maximal $\Delta$-\nolinebreak{}terms by variables which are then existentially quantified and arrive at
	$\Delta \entails\nolinebreak{} \overline Q_1 z_{t_1} \dots \overline Q_n z_{t_n}\,\lnot \lft{\Delta}{x}{\lft{\Gamma}{y}{\PI(\pi)}}$ where $\overline Q_i = \exists$ ($\forall$) if $Q_i = \forall$ ($\exists$).
	Therefore also
	$\Delta \entails\nolinebreak{} \lnot Q_1 z_{t_1} \dots Q_n z_{t_n}\,\lft{\Delta}{x}{\lft{\Gamma}{y}{\PI(\pi)}}$ and
	finally by Lemma \ref{lemma:lifting_order_not_relevant},
	$\Delta \entails\nolinebreak{} \lnot Q_1 z_{t_1} \dots Q_n z_{t_n}\,\lft{\Gamma}{y}{\lft{\Delta}{x}{\PI(\pi)}}$.

	As it is now easy to see that $Q_1 z_{t_1} \dots Q_n z_{t_n}\,\lft{\Gamma}{y}{\lft{\Delta}{x}{\PI(\pi)}}$ contains no colored symbol, it is an interpolant.
\end{proof}




\section{Comments on the original publication}
\label{sec:huang_commentary} 


In \cite[Definition 3]{Huang95}, a maximal occurrence of a $\Gamma$ ($\Delta$)-term is defined to be an occurrence of a $\Gamma$ ($\Delta$)-term which is not a subterm of a larger $\Gamma$ ($\Delta$)-term.

Furthermore, in the extension of the ``Interpolation Algorithm'' to include paramodulation inferences in \cite[p.~183]{Huang95}, this notion is used to distinguish between the respective cases.
Translated into our notation in the context of our corresponding Definition~\ref{def:PI_paramod} for the case of paramodulation inferences, the conditions for the three cases can be stated as follows:
\begin{enumerate}
	\item The term $r$ occurs in $E\occ{r}$ as subterm of a maximal $\Gamma$-term, which occurs more than once in $E\occ{r} \lor \PI(E\occ{r})$.
		\label{case_1}
	\item The term $r$ occurs in $E\occ{r}$ as subterm of a maximal $\Delta$-term, which occurs more than once in $E\occ{r} \lor \PI(E\occ{r})$.
	\item Otherwise.
\end{enumerate}

Note that if reading this definition in the strict sense, an ambiguity arises:
It is very well possible for a term to be a subterm of a maximal $\Gamma$-term and a maximal $\Delta$-term at the same time.
Suppose $g$ is a $\Gamma$-colored and $h$ a $\Delta$-colored function symbol.
Then the term $h(g(c))$ contains the maximal $\Delta$-term $h(g(c))$ as well as the maximal $\Gamma$-term $g(c)$ since $g(c)$ is not subterm of a larger $\Gamma$-term in\nolinebreak{} $h(g(c))$.

We present the following example, which illustrates that the definition of the conditions for the cases above is to be read as ``maximal colored term, which is $\Phi$-colored'' (or more concisely: ``maximal colored $\Phi$-term'') in place of ``maximal $\Phi$-term''.

\begin{exa}
	Let 
	$\Gamma = \{ P(x) \lor \lnot Q(x), \lnot P(y) \lor Q(y), c=d, \lnot R(g(d)), \lnot S(g(c))  \}$
	and
	$\Delta = \{ S(v) \lor \lnot Q(h(v)), R(u) \lor  Q(h(u)), T(c, d)\}$.
	Hence $h$ is a $\Delta$-colored function symbol and $g$ a $\Gamma$-colored function symbol, while the constant symbols $c$ and $d$ are gray.
	%\Gamma = \{ P(x) \lor \lnot Q(x), c=d, \lnot R(g(d)), \lnot P(v)  \}$
	%and $\Delta = \{ R(u) \lor  Q(h(u)) \}$.

	We present a resolution refutation of $\Gamma \cup \Delta$ in combination with the interpolant extraction such that each label is of the form $C\mid \PI(C)$, where $C$ is the clause of the refutation and $\PI(C)$ is sometimes given in a simplified but logically equivalent form.
	The presentation of the refutation is split into parts in order to improve readability.
	%\begin{landscape}

	Note that at the paramodulation inference \markC{}, case~\ref{case_1} is erroneously selected due to $d$ occurring in the maximal $\Gamma$-colored term $g(d)$, 
	even though $d$ is also contained in the maximal $\Delta$-colored term $h(g(d))$.
	{ \small
		~

		\begin{adjustwidth}{-1.5em}{0em}
		\begin{prooftree}
			\AxiomCm{{ \lnot R(g(d)) \mid \bot}}
			\AxiomCm{{  R(u) \lor  Q(h(u)) \mid \top}}

			%\RightLabelm{u\mapsto g(t)}
			\RightLabelm{\resrule{\resruleres}{u\mapsto g(d)}}
			%\RightLabelm{\resruleres}
			\BinaryInfCm{ Q(h(g(d))) \mid \lnot R(g(d)) }

			\AxiomCm{P(x) \lor \lnot Q(x) \mid \bot}

			\RightLabelm{\resrule{\resruleres}{x \mapsto h(g(d))}}
			\BinaryInfCm{ P(h(g(d))) \mid \lnot R(g(d)) \land \lnot Q(h(g(d)))}

			\AxiomCm{c=d \mid \bot}
			\RightLabelm{\resruleremark{\resrulepar}{\id}{\markC}}
			\BinaryInfCm{P(h(g(c))) \mid (c=d \land \lnot R(g(d)) \land \lnot Q(h(g(d)))) \lor (c\neq d \land g(c) = g(d))}
		\end{prooftree}
		\vspace{1em}

		\begin{prooftree}
			\AxiomCm{ \lnot S(g(c)) \mid \bot}
			\AxiomCm{  S(v) \lor \lnot Q(h(v)) \mid \top }

			\RightLabelm{\resrule{\resruleres}{v \mapsto g(c)}}
			\BinaryInfCm{ \lnot Q(h(g(c))) \mid \lnot S(g(c)) }

			\AxiomCm{\lnot P(y) \lor Q(y) \mid \bot}

			\RightLabelm{\resrule{\resruleres}{y \mapsto h(g(c))}}
			\BinaryInfCm{ \lnot P(h(g(c)))  \mid {\lnot S(g(c)) \land Q(h(g(c)))} }

		\end{prooftree}

		\end{adjustwidth}
	}
		~

	By combining these two derivation by means of a final resolution inference on the last remaining literal employing a trivial substitution, we obtain the empty clause and the corresponding interpolant $\PI(\square)$:
	\[
	(c=d \land \lnot R(g(d)) \land \lnot Q(h(g(d)))) \spam\lor (c\neq d \land g(c) = g(d))  \spam\lor\allowbreak
{\lnot S(g(c)) \land Q(h(g(c)))}\]
	Lifting $\PI(\square)$ and adding appropriate quantifiers gives the final result $I$ of the interpolant extraction:
	\begin{gather*}
		\exists y_{g(c)} \exists y_{g(d)} \forall x_{h(g(c))} \forall x_{h(g(d))} 
		\Big(
		(c=d \land \lnot R(y_{g(d)}) \land \lnot Q(x_{h(g(d))})) \spam\lor \\
		(c\neq d \land y_{g(c)} = y_{g(d)})  \spam\lor
		{\lnot S(y_{g(c)}) \land Q(x_{h(g(c))})}
		\Big)
	\end{gather*}

	Now we show that $\Gamma \not\entails I$.
	Note that as $\Gamma \entails c=d$, no model of $\Gamma$ satisfies $(c\neq d \land y_{g(c)} = y_{g(d)})$.
	The remaining two disjuncts imply that 
	$\forall x_{h(g(c))} \forall x_{h(g(d))} ( \lnot Q(x_{h(g(d))}) \lor Q(x_{h(g(c))}))$,
	but we can easily find a model of $\Gamma$ where at least one domain element satisfies the predicate $Q$ and another domain element does not.
	Any such model is a countermodel to the proposition $\Gamma \entails I$.
\end{exa}





	%\nocite{*}
	\bibliography{bib}
}



\usepackage{blindtext}

%% somewhat of a compromise, still long lines:
%\settypeblocksize{0.67\stockheight}{0.67\stockwidth}{*}
%%\settypeblocksize{0.65\stockheight}{0.65\stockwidth}{*}
%%\settypeblocksize{0.64\stockheight}{0.64\stockwidth}{*}
%%\settypeblocksize{0.63\stockheight}{0.63\stockwidth}{*}
%\setlrmargins{*}{*}{1.0}
%\setulmargins{*}{*}{1.4}
%\checkandfixthelayout[nearest]

\usepackage{mathabx}


%\usepackage{refcheck}
%\usepackage[inline]{showlabels}
%\usepackage[final]{showlabels}


\begin{document}

\begin{comment}
\selectlanguage{english}
\blindtext
\blindtext
\blindtext
55
\blindtext
33
\blindtext
10
\blindtext
\blindtext
\blindtext
20
\blindtext
\blindtext
\blindtext
\end{comment}

\captionnamefont{\bfseries}


%%%%%%%%%%%%%%%%%%%%%%%%%%%%%%%%%%%%%%%%%
%%%   PARTIAL COMPILATION OPTIONS    %%%%
%%%%%%%%%%%%%%%%%%%%%%%%%%%%%%%%%%%%%%%%%
\ifdefined\secproofsonly 

\selectlanguage{english}
\proofcontent


\else
\ifdefined\semanticonly 

\selectlanguage{english}
\semanticcontent

\else

%\pagenumbering{arabic}
\mysetpagestyle

\ifdefined\contentonly 
\selectlanguage{english}


\tableofcontents

\content

\appendixcontent 

\else 


%%%%%%%%%%%%%%%%%%%%%%%%%%%%%%%%%%%%%%%%%
%%%   FRONTMATTER    %%%%%%%%%%%%%%%%%%%%
%%%%%%%%%%%%%%%%%%%%%%%%%%%%%%%%%%%%%%%%%
\frontmatter
\pagenumbering{roman}

%%%%%%%%%%%%%%%%%%%%%%%%%%%%%%%%%%%%%%%%%
%%%   TITLEPAGES    %%%%%%%%%%%%%%%%%%%%%
%%%%%%%%%%%%%%%%%%%%%%%%%%%%%%%%%%%%%%%%%

% the german title page is required as first page
% $Id: titlepage.tex 1752 2010-03-20 11:07:02Z tkren $
%
% TU Wien - Faculty of Informatics
% thesis titlepage
%
% This titlepage is using the geometry package, see
% <http://www.ctan.org/macros/latex/contrib/geometry/geometry.pdf>
%
% For questions and comments send an email to
% Thomas Krennwallner <tkren@kr.tuwien.ac.at>
% or to Petra Brosch <brosch@big.tuwien.ac.at>
%

\selectlanguage{ngerman}

% setup page dimensions for titlepage
\newgeometry{left=2.4cm,right=2.4cm,bottom=2.5cm,top=2cm}

% force baselineskip and parindent
\newlength{\tmpbaselineskip}
\setlength{\tmpbaselineskip}{\baselineskip}
\setlength{\baselineskip}{13.6pt}
\newlength{\tmpparindent}
\setlength{\tmpparindent}{\parindent}
\setlength{\parindent}{17pt}

% first titlepage
\thispagestyle{tuinftitlepage}

%
% Kludge: for each titlepage set \pagenumbering to a different
% style. This is used to fix a problem with hyperref, because there
% are multiple "page 1" and hyperref hates that
%
\pagenumbering{Alph}

\begin{center}
{\ \vspace{3.4cm}}

\begin{minipage}[t][2.8cm][s]{\textwidth}%
\centering
\thesistitlefontHUGE\sffamily\bfseries\tuinfthesistitle\\
\bigskip
{\thesistitlefonthuge\sffamily\bfseries\tuinfthesissubtitle}
\end{minipage}

\vspace{1.3cm}

{\thesistitlefontLARGE\sffamily \tuinfthesistype}

\vspace{6mm}

{\thesistitlefontlarge\sffamily zur Erlangung des akademischen Grades}

\vspace{6mm}

{\thesistitlefontLARGE\sffamily\bfseries \tuinfthesisdegree}

\vspace{6mm}

{\thesistitlefontlarge\sffamily im Rahmen des Studiums}

\vspace{6mm}

{\thesistitlefontLarge\sffamily\bfseries \tuinfthesiscurriculum}

\vspace{6.5mm}

{\thesistitlefontlarge\sffamily eingereicht von}

\vspace{6mm}

{\thesistitlefontLarge\sffamily\bfseries \tuinfthesisauthor}

\vspace{1.5mm}

{\thesistitlefontlarge\sffamily Matrikelnummer \tuinfthesismatrikelno} 

\vspace{1.4cm}

\vspace{0pt}\raggedright\thesistitlefontnormalsize\sffamily
\begin{minipage}[t][1.6cm][t]{\textwidth}%
  %
  an der

  Fakult\"{a}t f\"{u}r Informatik der Technischen Universit\"{a}t Wien
\end{minipage}

\begin{minipage}[t][4cm][t]{\textwidth}%
  \vspace{0pt}\raggedright\thesistitlefontnormalsize\sffamily
  %
  \begin{tabbing}%
	    \hspace{19mm} \= \hspace{66mm} \kill
	    \tuinfthesisbetreuung: \> \tuinfthesisbetreins\\
	    Mitwirkung: \> \tuinfthesisbetrzwei\\
	                \> \tuinfthesisbetrdrei
  \end{tabbing}
\end{minipage}

\begin{minipage}[t][1.5cm][t]{\textwidth}%
  \vspace{0pt}\sffamily\thesistitlefontnormalsize
  \begin{tabbing}%
    \hspace{45mm} \= \hspace{63mm} \= \hspace{51mm} \kill
    Wien, \tuinfthesisdate \> {\raggedright\rule{51mm}{0.5pt}} \> {\raggedright\rule{51mm}{0.5pt}} \\
    \> \begin{minipage}[t][0.5cm][t]{51mm}\centering (Unterschrift \tuinfthesisverfassung)\end{minipage}
    \> \begin{minipage}[t][0.5cm][t]{51mm}\centering (Unterschrift \tuinfthesisbetreuung)\end{minipage}
    \end{tabbing}
\end{minipage}

\end{center}

% we want an empty page right after first titlepage
\pagestyle{empty}
\cleardoublepage

% we're done with the titlepages, proceed with default pagenumbering
\pagenumbering{roman}

% restore baselineskip
\setlength{\baselineskip}{\tmpbaselineskip}
\setlength{\parindent}{\tmpparindent}

% back to normal geometry
\restoregeometry

\selectlanguage{english}

%%% Local Variables:
%%% TeX-PDF-mode: t
%%% TeX-debug-bad-boxes: t
%%% TeX-parse-self: t
%%% TeX-auto-save: t
%%% reftex-plug-into-AUCTeX: t
%%% End:


% an english translation may follow
% $Id: titlepage.tex 1752 2010-03-20 11:07:02Z tkren $
%
% TU Wien - Faculty of Informatics
% thesis titlepage
%
% This titlepage is using the geometry package, see
% <http://www.ctan.org/macros/latex/contrib/geometry/geometry.pdf>
%
% For questions and comments send an email to
% Thomas Krennwallner <tkren@kr.tuwien.ac.at>
% or to Petra Brosch <brosch@big.tuwien.ac.at>
%

% setup page dimensions for titlepage
\newgeometry{left=2.4cm,right=2.4cm,bottom=2.5cm,top=2cm}

% force baselineskip and parindent
%\newlength{\tmpbaselineskip}
%\setlength{\tmpbaselineskip}{\baselineskip}
%\setlength{\baselineskip}{13.6pt}
%\newlength{\tmpparindent}
%\setlength{\tmpparindent}{\parindent}
%\setlength{\parindent}{17pt}

% first titlepage
\thispagestyle{tuinftitlepage}

%
% Kludge: for each titlepage set \pagenumbering to a different
% style. This is used to fix a problem with hyperref, because there
% are multiple "page 1" and hyperref hates that
%
\pagenumbering{Roman}

\begin{center}
{\ \vspace{3.4cm}}

\begin{minipage}[t][2.8cm][s]{\textwidth}%
\centering
\thesistitlefontHUGE\sffamily\bfseries\tuinfthesistitle\\
\bigskip
{\thesistitlefonthuge\sffamily\bfseries\tuinfthesissubtitle}
\end{minipage}

\vspace{1.3cm}

{\thesistitlefontLARGE\sffamily \tuinfthesistypeen}

\vspace{6mm}

{\thesistitlefontlarge\sffamily submitted in partial fulfillment of the requirements for the degree of}

\vspace{6mm}

{\thesistitlefontLARGE\sffamily\bfseries \tuinfthesisdegreeen}

\vspace{6mm}

{\thesistitlefontlarge\sffamily in}

\vspace{6mm}

{\thesistitlefontLarge\sffamily\bfseries \tuinfthesiscurriculumen}

\vspace{6.5mm}

{\thesistitlefontlarge\sffamily by}

\vspace{6mm}

{\thesistitlefontLarge\sffamily\bfseries \tuinfthesisauthor}

\vspace{1.5mm}

{\thesistitlefontlarge\sffamily Registration Number \tuinfthesismatrikelno} 

\vspace{1.4cm}

\begin{minipage}[t][1.6cm][t]{\textwidth}%
  \vspace{0pt}\raggedright\thesistitlefontnormalsize\sffamily
  %
  to the Faculty of Informatics 

  at the Vienna University of Technology
\end{minipage}

\vspace{0pt}\raggedright\thesistitlefontnormalsize\sffamily
\begin{minipage}[t][4cm][t]{\textwidth}%
  \begin{tabbing}%
	    \hspace{19mm} \= \hspace{66mm} \kill
	    Advisor: \> \tuinfthesisbetreins\\
	    Assistance: \> \tuinfthesisbetrzwei\\
	                \> \tuinfthesisbetrdrei
     \end{tabbing}
\end{minipage}

\begin{minipage}[t][1.5cm][t]{\textwidth}%
  \vspace{0pt}\sffamily\thesistitlefontnormalsize
  \begin{tabbing}%
    \hspace{45mm} \= \hspace{63mm} \= \hspace{51mm} \kill
    Vienna, \tuinfthesisdate \> {\raggedright\rule{51mm}{0.5pt}} \> {\raggedright\rule{51mm}{0.5pt}} \\
    \> \begin{minipage}[t][0.5cm][t]{51mm}\centering (Signature of Author)\end{minipage}
    \> \begin{minipage}[t][0.5cm][t]{51mm}\centering (Signature of Advisor)\end{minipage}
    \end{tabbing}
\end{minipage}

\end{center}

% we want an empty page right after first titlepage
\pagestyle{empty}
\cleardoublepage

% we're done with the titlepages, proceed with default pagenumbering
\pagenumbering{roman}

% restore baselineskip
\setlength{\baselineskip}{\tmpbaselineskip}
\setlength{\parindent}{\tmpparindent}

% back to normal geometry
\restoregeometry


%%% Local Variables:
%%% TeX-PDF-mode: t
%%% TeX-debug-bad-boxes: t
%%% TeX-parse-self: t
%%% TeX-auto-save: t
%%% reftex-plug-into-AUCTeX: t
%%% End:
 % optional

%%%%%%%%%%%%%%%%%%%%%%%%%%%%%%%%%%%%%%%%%
%%%   ERKLAERUNG DER SELBSTAENDIGKEIT   %
%%%%%%%%%%%%%%%%%%%%%%%%%%%%%%%%%%%%%%%%%
\cleardoublepage
\selectlanguage{ngerman}
\chapter*{Erkl"arung zur Verfassung der Arbeit}

\tuinfthesisauthor\\
\tuinfthesisauthoraddress

\vspace*{1.2cm}

Hiermit erkl"are ich, dass ich diese Arbeit selbst"andig verfasst habe, 
dass ich die verwendeten Quellen und Hilfsmittel vollst"andig angegeben 
habe und dass ich die Stellen der Arbeit - einschlie\ss{}lich Tabellen, 
Karten und Abbildungen -, die anderen Werken oder dem Internet im 
Wortlaut oder dem Sinn nach entnommen sind, auf jeden Fall unter Angabe 
der Quelle als Entlehnung kenntlich gemacht habe.\\

\vspace*{2cm}
\begin{tabbing}%
    \hspace{58mm} \= \hspace{28mm} \= \hspace{58mm} \kill
    {\raggedright\rule{58mm}{0.5pt}} \> \> {\raggedright\rule{58mm}{0.5pt}} \\
    \begin{minipage}[t][0.5cm][t]{58mm}
	\vspace{0pt}\sffamily\thesistitlefontnormalsize
	\centering (Ort, Datum)
    \end{minipage}
    \> \>
    \begin{minipage}[t][0.5cm][t]{58mm}
	\vspace{0pt}\sffamily\thesistitlefontnormalsize
	\centering (Unterschrift \tuinfthesisverfassung)
    \end{minipage}
\end{tabbing}


\selectlanguage{english}

%%%%%%%%%%%%%%%%%%%%%%%%%%%%%%%%%%%%%%%%%
%%%   ACKNOWLEDGEMENTS    %%%%%%%%%%%%%%%
%%%%%%%%%%%%%%%%%%%%%%%%%%%%%%%%%%%%%%%%%

% optional acknowledgements may be included in german or in english
%%\chapter*{Danksagung}

%Hier fügen Sie optional eine Danksagung ein.
 		% optional
\chapter*{Acknowledgements}

Optional acknowledgements may be inserted here.	% optional

%%%%%%%%%%%%%%%%%%%%%%%%%%%%%%%%%%%%%%%%%
%%%   ABSTRACT    %%%%%%%%%%%%%%%%%%%%%%%
%%%%%%%%%%%%%%%%%%%%%%%%%%%%%%%%%%%%%%%%%

\chapter*{Abstract}

Craig's interpolation theorem is a long known basic result of mathematical logic.
Interpolants lay bare certain logical relations between formulas or sets of formulas in a concise way and can often be calculated efficiently from proofs of these relations.
Leveraging the tremendous
progress of automatic deduction systems in the last decades, obtaining the required proofs
is feasible. 
This enables real world applications for instance in the area of software verification.

For practical applicability, interpolation is often studied in relatively weak formalisms such as propositional logic.
This thesis however aims at giving a comprehensive account of existing techniques and results with respect to unrestricted classical first-order logic with equality in three parts:

First, we present Craig's initial proof of the interpolation theorem by reduction to first-order logic without equality and function symbols.
Due to the inherent overhead,
this approach only gives rise to an impractical algorithm for interpolant extraction.

Second, a constructive proof by Huang is introduced in slightly improved form.
It employs direct interpolant extraction from resolution proofs in two phases 
and thereby
shows that even in full first-order logic with equality, interpolants can efficiently be calculated.
Moreover, we present an analysis of the number of quantifier alternations of the interpolants produced by this algorithm.
We additionally propose a novel approach which combines the two phases of Huang's algorithm and thereby allows for creating non-prenex interpolants.

Third, we give a semantic perspective on interpolation in the form of a model-theoretic proof based on Robinson's joint consistency theorem.
This emphasizes the close relation between the proof-theoretic and the model-theoretic view on interpolation.

\cleardoublepage
\selectlanguage{ngerman}
\chapter*{Kurzfassung}

% ATTENTION: TILDE IN TEXT!!!!
Der Interpolationssatz von Craig stellt ein grundlegendes Ergebnis der mathematischen Logik dar. Interpolanten fassen gewisse logische Beziehungen zwischen Formeln präzise zusammen und können oftmals effizient aus Beweisen dieser Beziehungen extrahiert werden. Der immense Fortschritt von Inferenzsystemen der letzten Jahrzehnte ermöglicht die Berechnung der erforderlichen Beweise, was den Grundstein für Anwendungen etwa im Bereich der Softwareverifikation~legt.
% ATTENTION: TILDE IN TEXT!!!!

Aufgrund der besseren praktischen Anwendbarkeit wird Interpolation häufig in relativ schwachen logischen Formalismen wie etwa der Aussagenlogik untersucht. Diese Arbeit setzt sich hingegen zum Ziel, einen umfassenden Überblick über bestehende Techniken und Resultate im Bereich der uneingeschränkten Prädikatenlogik mit Gleichheit zu geben. Dies geschieht in drei Abschnitten:

Zuerst gehen wir auf den ursprünglichen Beweis des Interpolationssatzes von Craig ein, welcher eine Reduktion auf Prädikatenlogik ohne Gleichheit und Funktionssymbole durchführt.
Aufgrund des dadurch entstehenden Mehraufwandes ergibt sich daraus nur ein ineffizienter Algorithmus zur Interpolantenextraktion.

Danach stellen wir einen konstruktiven Beweis von Huang in einer etwas verbesserten Form vor. Hier werden Interpolanten direkt aus Resolutionsbeweisen in zwei Phasen extrahiert, was somit zeigt, dass auch in uneingeschränkter Prädikatenlogik mit Gleichheit eine effiziente Interpolantenberechnung möglich ist. Desweiteren analysieren wir die Anzahl der Quantorenalternationen in den daraus resultierenden Interpolanten und stellen einen neuen Ansatz vor, welcher beide Phasen von Huangs Algorithmus kombiniert und dadurch nicht prenexe Interpolanten liefert.

Im letzten Abschnitt beschäftigen wir uns mit einer semantischen Sichtweise auf Interpolation in Form eines modelltheoretischen Beweises basierend auf dem Joint Consistency Satz von Robinson, was sowohl Ähnlichkeiten als auch Unterschiede zur beweistheoretischen Betrachtungsweise illustriert.

\selectlanguage{english}

%%%%%%%%%%%%%%%%%%%%%%%%%%%%%%%%%%%%%%%%%
%%%   CONTENTS    %%%%%%%%%%%%%%%%%%%%%%%
%%%%%%%%%%%%%%%%%%%%%%%%%%%%%%%%%%%%%%%%%
% uncomment to set document language to german (results in "Inhaltsverzeichnis", "Kapitel", "Abbildung", etc. instead of "Contents", "Chapter", and "Figure"), otherwise the document's language is english
%\selectlanguage{ngerman}

\cleardoublepage
\mysetpagestyle
\tableofcontents*

%%%%%%%%%%%%%%%%%%%%%%%%%%%%%%%%%%%%%%%%%
%%%   MAINMATTER    %%%%%%%%%%%%%%%%%%%%%
%%%%%%%%%%%%%%%%%%%%%%%%%%%%%%%%%%%%%%%%%

\mainmatter
\pagenumbering{arabic}
\mysetpagestyle

%%%%%%%%%%%%%%%%%%%%%%%%%%%%%%%%%%%%%%%%%

\content

%\chapter{Notes}
%

\newcommand{\seq}{\vdash} % the sequent sign
\newcommand{\impl}{\supset} %logical connectives: implies, not, and, or
\renewcommand{\lnot}{\neg}
\renewcommand{\land}{\wedge}
\renewcommand{\lor}{\vee}


\subsubsection*{Axioms}

\begin{prooftree}
\AxiomCm{}
\RightLabelm{(\mt{Identity Axiom})}
\UnaryInfCm{\Gamma, A \seq \Delta, A}
\end{prooftree}
Interpolant: $A$

\begin{prooftree}
\AxiomCm{}
\RightLabelm{(\mt{Reflexivity Axiom})}
\UnaryInfCm{\Gamma \seq \Delta, t = t}
\end{prooftree}
Interpolant: $\top$

\subsubsection*{Cut}

  \begin{prooftree}
  \AxiomCm{\Gamma \seq \Delta, A}
  \AxiomCm{\Sigma, A \seq \Pi}
  \RightLabelm{(\mt{cut})}
  \BinaryInfCm{\Gamma, \Sigma \seq \Delta, \Pi}
  \end{prooftree}
Interpolant: TODO

\subsubsection*{Structural rules}

EASY

\begin{multicols}{2}

  \subsubsection*{Left rules}

  \begin{prooftree}
  \AxiomCm{\Gamma \seq \Delta}
  \RightLabelm{(\mt{w:l})}
  \UnaryInfCm{\Gamma, A \seq \Delta}
  \end{prooftree}

  \begin{prooftree}
  \AxiomCm{\Gamma, A, A \seq \Delta}
  \RightLabelm{(\mt{c:l})}
  \UnaryInfCm{\Gamma, A \seq \Delta}
  \end{prooftree}

  \subsubsection*{Right rules}

  \begin{prooftree}
  \AxiomCm{\Gamma \seq \Delta}
  \RightLabelm{(\mt{w:r})}
  \UnaryInfCm{\Gamma \seq \Delta, A}
  \end{prooftree}

  \begin{prooftree}
  \AxiomCm{\Gamma \seq \Delta, A, A}
  \RightLabelm{(\mt{c:r})}
  \UnaryInfCm{\Gamma \seq \Delta, A}
  \end{prooftree}

\end{multicols}



\subsubsection*{Propositional rules}

\begin{multicols}{2}

  \subsubsection*{Left rules}

  \begin{prooftree}
  \AxiomCm{\Gamma, A \seq \Delta}
  \RightLabelm{(\land\mt{l}_1)}
  \UnaryInfCm{\Gamma, A \land B \seq \Delta}
  \end{prooftree}
Interpolant: $I_1$

  \begin{prooftree}
  \AxiomCm{\Gamma, B \seq \Delta}
  \RightLabelm{(\land\mt{l}_2)}
  \UnaryInfCm{\Gamma, A \land B \seq \Delta}
  \end{prooftree}
Interpolant: $I_1$

  \begin{prooftree}
  \AxiomCm{\Gamma, A \seq \Delta}
  \AxiomCm{\Sigma, B \seq \Pi}
  \RightLabelm{(\lor\mt{:l})}
  \BinaryInfCm{\Gamma, \Sigma, A \lor B \seq \Delta, \Pi}
  \end{prooftree}
Interpolant: $I_1 \lor I_2$

  \begin{prooftree}
  \AxiomCm{\Gamma \seq \Delta, A}
  \RightLabelm{(\lnot\mt{:l})}
  \UnaryInfCm{\Gamma, \neg A \seq \Delta}
  \end{prooftree}
Interpolant: $I_1$ (considering proper coloring (global view))

  \begin{prooftree}
  \AxiomCm{\Gamma \seq \Delta, A}
  \AxiomCm{\Sigma, B \seq \Pi}
  \RightLabelm{(\impl\mt{:l})}
  \BinaryInfCm{\Gamma, \Sigma, A \impl B \seq \Delta, \Pi}
  \end{prooftree}
Interpolant: $I_1 \lor I_2$ (again with global view)



  \subsubsection*{Right rules}

  \begin{prooftree}
  \AxiomCm{\Gamma \seq \Delta, A}
  \AxiomCm{\Sigma \seq \Pi, B}
  \RightLabelm{(\land\mt{:r})}
  \BinaryInfCm{\Gamma, \Sigma \seq \Delta, \Pi, A \land B}
  \end{prooftree}
Interpolant: $I_1 \land I_2$

  \begin{prooftree}
  \AxiomCm{\Gamma \seq \Delta, A}
  \RightLabelm{(\lor\mt{:r}_1)}
  \UnaryInfCm{\Gamma \seq \Delta, A \lor B}
  \end{prooftree}
Interpolant: $I_1$

  \begin{prooftree}
  \AxiomCm{\Gamma \seq \Delta, B}
  \RightLabelm{(\lor\mt{:r}_2)}
  \UnaryInfCm{\Gamma \seq \Delta, A \lor B}
  \end{prooftree}
Interpolant: $I_1$

  \begin{prooftree}
  \AxiomCm{\Gamma, A\seq \Delta}
  \RightLabelm{(\lnot\mt{:r})}
  \UnaryInfCm{\Gamma \seq \Delta, \neg A}
  \end{prooftree}
Interpolant: $I_1$

  \begin{prooftree}
  \AxiomCm{\Gamma, A \seq B, \Delta}
  \RightLabelm{(\impl\mt{:r})}
  \UnaryInfCm{\Gamma \seq A \impl B, \Delta}
  \end{prooftree}
Interpolant: $I_1 (global view)$

\end{multicols}

\subsubsection*{Quantification roles}

\begin{multicols}{2}

  \subsubsection*{Left rules}

  \begin{prooftree}
  \AxiomCm{\Gamma, A[t/x] \seq \Delta}
  \RightLabelm{(\forall\mt{:l})}
  \UnaryInfCm{\Gamma, \forall x A \seq \Delta}
  \end{prooftree}
Interpolant: $I_1$

  \begin{prooftree}
  \AxiomCm{\Gamma, A[y/x] \seq \Delta}
  \RightLabelm{(\exists\mt{:l})}
  \UnaryInfCm{\Gamma, \exists x A \seq \Delta}
  \end{prooftree}
Interpolant: $I_1$, possibly overbinding eigenvar

  \subsubsection*{Right rules}

  \begin{prooftree}
  \AxiomCm{\Gamma \seq A[y/x] \Delta}
  \RightLabelm{(\forall\mt{:r})}
  \UnaryInfCm{\Gamma \seq \forall x A, \Delta}
  \end{prooftree}
Interpolant: $I_1$, possibly overbinding eigenvar

  \begin{prooftree}
  \AxiomCm{\Gamma \seq A[t/x] \Delta}
  \RightLabelm{(\exists\mt{:r})}
  \UnaryInfCm{\Gamma \seq \exists x A, \Delta}
  \end{prooftree}
Interpolant: $I_1$

\end{multicols}

The variable $y$ must not occur free in $\Gamma$ or $\Delta$. The term $t$ must avoid variable capture, i.e. it must not contain free occurrences of variables bound in $A$.


\subsubsection*{Equational rules}

\begin{multicols}{2}

  \subsubsection*{Left rules}

  \begin{prooftree}
  \AxiomCm{\Gamma \seq \Delta, s=t}
  \AxiomCm{\Sigma, A[T/s] \seq \Pi}
  \RightLabelm{(\mt{=:l}_1)}
  \BinaryInfCm{\Gamma, \Sigma, A[T/t] \seq \Delta, \Pi}
  \end{prooftree}
Interpolant: $I = I_12$.
Proof of first implication:
Supp $M \models LHS$. Then $M \models I_1$

Proof of 2nd implication:
Supp $M \models I$.
As $M \models I_1$, $M \models \Delta \lor s=t$. If $M \models \Delta$, we are done. Otw $M \models s=t$. 

  \begin{prooftree}
  \AxiomCm{\Gamma \seq \Delta, s=t}
  \AxiomCm{\Sigma, A[T/t] \seq \Pi}
  \RightLabelm{(\mt{=:l}_2)}
  \BinaryInfCm{\Gamma, \Sigma, A[T/s] \seq \Delta, \Pi}
  \end{prooftree}
	symmetric

  \subsubsection*{Right rules}

  \begin{prooftree}
  \AxiomCm{\Gamma \seq \Delta, s=t}
  \AxiomCm{\Sigma \seq \Pi, A[T/s]}
  \RightLabelm{(\mt{=:r}_1)}
  \BinaryInfCm{\Gamma, \Sigma \seq \Delta, \Pi, A[T/t]}
  \end{prooftree}
Interpolant: $I = I_1 \land I_2$.
Proof of 2nd implications:
Supp $M \models I$. If $M \models \Pi$, we are done. Otherwise $M \models A[T/s]$. 
If $M \models \Delta$, we are done. Otherwise $M \models s=t$. But then $M \models A[T/t]$.

  \begin{prooftree}
  \AxiomCm{\Gamma \seq \Delta, s=t}
  \AxiomCm{\Sigma, \seq \Pi, A[T/t]}
  \RightLabelm{(\mt{=:r}_2)}
  \BinaryInfCm{\Gamma, \Sigma \seq \Delta, \Pi, A[T/s]}
  \end{prooftree}
	symmetric

\end{multicols}





\appendixcontent


\fi
\fi
\fi

\end{document}
