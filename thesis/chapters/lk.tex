\section{Sequent Calculus}

The famous sequent calculus was introduced in \cite{Gentzen}.
Its use of sequents in lieu of plain formulas allows for a natural mapping of the logical relations expressed by the connectives to the structure of proofs.

\begin{defi} 
	For multisets of first-order formulas $\Gamma$ and $\Delta$,  $\Gamma \proves \Delta$ is called a \defiemph{sequent}. 
	In this context $\Gamma$ forms the \defiemph{antecedent}, whereas $\Delta$ is referred to as \defiemph{succedent}.

	A sequent $\Gamma \proves \Delta$ is called \defiemph{provable} if there is a sequent calculus proof of $\Gamma \proves \Delta$.
\end{defi}

The rules of sequent calculus are as follows:
%\newcommand{\calculussec}[1]{\textbf{#1}\nopagebreak}
\newcommand{\calculussec}[1]{\subsubsection*{#1}\nopagebreak}
\newenvironment{lkdefsec}{
	%\begin{figure}[h!b]
	%\ContinuedFloat
		\begin{adjustwidth}{0.05\textwidth}{0.05\textwidth}
		}{
		\end{adjustwidth}
	%\end{figure}
}
\begin{lkdefsec}
	\calculussec{Axioms}
	\begin{multicols}{2}
		\begin{prooftree}
			\AxiomCm{A \fCenter A}
		\end{prooftree}

		\begin{prooftree}
			\AxiomCm{\fCenter t=t}
		\end{prooftree}
	\end{multicols}

\end{lkdefsec}

\begin{lkdefsec}
	\calculussec{Cut}
	\begin{prooftree}
		\Axiomm{\Gamma \fCenter \Delta, A}
		\Axiomm{A, \Sigma \fCenter \Pi}
		\BinaryInfm{\Gamma, \Sigma \fCenter \Delta, \Pi}
	\end{prooftree}

\end{lkdefsec}
\begin{lkdefsec}

	\calculussec{Structural rules}
	\begin{itemize}
		\item Contraction
			\begin{multicols}{2}
				\begin{prooftree}
					\Axiom$\Gamma, A, A \fCenter \Delta$
					\RightLabelm{\lkrule{c}{l}}
					\UnaryInf$\Gamma, A \fCenter \Delta$
				\end{prooftree}

				\begin{prooftree}
					\Axiomm{\Gamma \fCenter \Delta, A, A}
					\RightLabelm{\lkrule{c}{r}}
					\UnaryInfm{\Gamma \fCenter \Delta, A }
				\end{prooftree}

			\end{multicols}

		\item Weakening
			\begin{multicols}{2}
				\begin{prooftree}
					\Axiomm{\Gamma \fCenter \Delta}
					\RightLabelm{\lkrule{w}{l}}
					\UnaryInfm{\Gamma, A \fCenter \Delta}
				\end{prooftree}

				\begin{prooftree}
					\Axiomm{\Gamma \fCenter \Delta}
					\RightLabelm{\lkrule{w}{r}}
					\UnaryInfm{\Gamma \fCenter \Delta,  A }
				\end{prooftree}

			\end{multicols}


	\end{itemize}

\end{lkdefsec}
\begin{lkdefsec}

	\calculussec{Propositional rules}

	\begin{itemize}
		\item Negation

			\begin{multicols}{2}
				\begin{prooftree}
					\Axiomm{\Gamma \fCenter \Delta,  A}
					\RightLabelm{\lkrule{\lnot}{l}}
					\UnaryInfm{\lnot A, \Gamma \fCenter \Delta }
				\end{prooftree}

				\begin{prooftree}
					\Axiomm{A, \Gamma \fCenter \Delta}
					\RightLabelm{\lkrule{\lnot}{r}}
					\UnaryInfm{\Gamma \fCenter \Delta, \lnot  A }
				\end{prooftree}
			\end{multicols}

	\end{itemize}
\end{lkdefsec}
\begin{lkdefsec}

	\begin{itemize}

		\item Conjunction
			\begin{multicols}{2}
				\begin{prooftree}
					\Axiomm{\Gamma, A, B \fCenter \Delta}
					\RightLabelm{\lkrule{\land}{l}}
					\UnaryInfm{\Gamma, A\land B \fCenter \Delta }
				\end{prooftree}

				\begin{prooftree}
					\Axiomm{\Gamma \fCenter \Delta, A}
					\Axiomm{\Sigma \fCenter \Pi, B}
					\RightLabelm{\lkrule{\land}{r}}
					\BinaryInfCm{\Gamma, \Sigma \fCenter \Delta, \Pi, A \land B }
				\end{prooftree}
			\end{multicols}

		\item Disjunction\nopagebreak
			\begin{multicols}{2}
				\begin{prooftree}
					\Axiomm{\Gamma, A \fCenter \Delta}
					\Axiomm{\Sigma, B \fCenter \Pi}
					\RightLabelm{\lkrule{\lor}{l}}
					\BinaryInfm{\Gamma, \Sigma, A \lor B \fCenter \Delta, \Pi }
				\end{prooftree}

				\begin{prooftree}
					\Axiomm{\Gamma \fCenter \Delta, A, B}
					\RightLabelm{\lkrule{\lor}{r}}
					\UnaryInfm{\Gamma \fCenter \Delta, A \lor B  }
				\end{prooftree}

			\end{multicols}

		\item Implication
			\begin{multicols}{2}
				\begin{prooftree}
					\Axiomm{\Gamma \fCenter A, \Delta}
					\Axiomm{\Sigma, B \fCenter \Pi}
					\RightLabelm{\lkrule{\limpl}{l}}
					\BinaryInfm{\Gamma, \Sigma A \limpl B \fCenter \Delta, \Pi }
				\end{prooftree}

				\begin{prooftree}
					\Axiomm{\Gamma, A \fCenter \Delta, B}
					\RightLabelm{\lkrule{\limpl}{r}}
					\UnaryInfm{\Gamma \fCenter \Delta, A \limpl B}
				\end{prooftree}

			\end{multicols}

	\end{itemize}
\end{lkdefsec}
\begin{lkdefsec}

	\calculussec{Quantifier rules}

	\begin{itemize}


		\item Universal

			\begin{multicols}{2}
				\begin{prooftree}
					\Axiomm{\Gamma, A[x/t] \fCenter \Delta}
					\RightLabelm{\lkrule{\forall}{l}}
					\UnaryInfm{\Gamma, \forall x A \fCenter \Delta }
				\end{prooftree}

				\begin{prooftree}
					\Axiomm{\Gamma \fCenter \Delta, A[x/y]}
					\RightLabelm{\lkrule{\forall}{r}}
					\UnaryInfm{\Gamma\fCenter \Delta, \forall x A  }
				\end{prooftree}

			\end{multicols}

		\item Existential

			\begin{multicols}{2}
				\begin{prooftree}
					\Axiomm{\Gamma, A[x/y] \fCenter \Delta}
					\RightLabelm{\lkrule{\exists}{l}}
					\UnaryInfm{\Gamma, \exists x A \fCenter \Delta }
				\end{prooftree}

				\begin{prooftree}
					\Axiomm{\Gamma \fCenter \Delta, A[x/t]}
					\RightLabelm{\lkrule{\exists}{r}}
					\UnaryInfm{\Gamma\fCenter \Delta, \exists x A  }
				\end{prooftree}

			\end{multicols}

	\end{itemize}

	(provided no free variable of $t$ becomes bound in $A[x/t]$ and
	$y$ does not occur free in $\Gamma$, $\Delta$ or $A$)

\end{lkdefsec}
	\begin{figure}[H]
\begin{lkdefsec}

	\calculussec{Equality rules}
	\begin{itemize}
			\item Left rules
	\begin{multicols}{2}
		\begin{prooftree}
			\Axiomm{\Gamma, A\occurat{t}{p} \fCenter \Delta}
			\Axiomm{\Sigma \fCenter \Pi, s=t}
			\RightLabelm{\lkrule{=}{l_1}}
			\BinaryInfm{\Gamma, \Sigma, A\occurat{s}{p} \fCenter \Delta, \Pi }
		\end{prooftree}
		\begin{prooftree}
			\Axiomm{\Gamma, A\occurat{s}{p} \fCenter \Delta}
			\Axiomm{\Sigma \fCenter \Pi, s=t}
			\RightLabelm{\lkrule{=}{l_2}}
			\BinaryInfm{\Gamma, \Sigma, A\occurat{t}{p} \fCenter \Delta, \Pi }
		\end{prooftree}
	\end{multicols}

			\item Right rules
	\begin{multicols}{2}
		\begin{prooftree}
			\Axiomm{\Gamma\fCenter \Delta, A\occurat{t}{p} }
			\Axiomm{\Sigma \fCenter \Pi, s=t}
			\RightLabelm{\lkrule{=}{r_1}}
			\BinaryInfm{\Gamma, \Sigma\fCenter \Delta, \Pi, A\occurat{s}{p}  }
		\end{prooftree}
		\begin{prooftree}
			\Axiomm{\Gamma\fCenter \Delta, A\occurat{s}{p} }
			\Axiomm{\Sigma \fCenter \Pi, s=t}
			\RightLabelm{\lkrule{=}{r_2}}
			\BinaryInfm{\Gamma, \Sigma\fCenter \Delta, \Pi, A\occurat{t}{p} }
		\end{prooftree}
	\end{multicols}
	\end{itemize}

	(provided no free variable of $s$ or $t$ becomes bound in $A\occurat{t}{p}$ or $A\occurat{s}{p}$)

	\caption{The rules of sequent calculus}
	\label{fig:lk}

\end{lkdefsec}
	\end{figure}

For the purposes of this thesis, we consider the cut-free fragment of sequent calculus.

\begin{thm}
	Cut-free sequent calculus is sound and complete.
\end{thm}





