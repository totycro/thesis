\chapter{Conclusion}

This thesis gives a comprehensive account of results and techniques with respect to interpolation in full first-order logic with equality.
The notion of interpolation enjoys applicability in many areas:

Among the most notable practical uses of interpolation we can certainly count the application in model checking introduced in \cite{McMillan03}.
Here, interpolants represent concise formulas describing an overapproximation of the set of reachable states of a program, which can then be used to prove the unreachability of error states.
Moreover, interpolants can be employed to construct loop invariants (\cite{weissenbacher2010}) which is a major challenge for program verification.
In the realm of theory, for instance Beth's definability theorem can very easily be proven using the interpolation theorem.

Even though the interpolation theorem holds in first-order logic with equality, % full first-order with equality, 
a multitude of applications in fact mostly deal only with weaker logics such as propositional logic or equational logic with uninterpreted function symbols.

In order to facilitate future applications in full first-order logic with equality, the focus of this work is geared towards constructive proofs which give rise to concrete algorithms for calculating interpolants.
We present the first such in Chapter~\ref{chap:reduction}, which is also historically the first one:
In \cite{Craig57linear,Craig57three}, where Craig introduces the notion of interpolation, he already gives a constructive proof 
%accompanies it with an algorithm 
which however is not intended for practical use.
By a reduction to first-order logic without equality and function symbols, which allows for a simpler constructive proof, interpolants can effectively be calculated, but only at the cost of dealing with additional axioms for equality and for every function symbol.

Arguably the most significant subsequent contribution for interpolant construction in the logic at hand is due to Huang.
In \cite{Huang95}, a two-phase approach is introduced which is capable of efficiently extracting interpolants from resolution refutations which include paramodulation inferences.
Here, a preliminary structure in the form of a propositional interpolant is extracted directly from the refutation, where colored constant and function symbols are then in the second stage replaced by appropriately quantified lifting variables.
This leads to interpolants in prenex form.
We present this algorithm in detail in Chapter~\ref{chap:two_phases} in a slightly improved form and in Appendix~\ref{chap:huang} in a version following \cite{Huang95} more closely. 
An analysis of the construction of terms with color alternations lead us to a result on quantifier alternations:
%on the number of quantifier alternations in the interpolants created by this algorithm%.
We show that the number of color alternations in terms of the resolution essentially coincides with the number of quantifier alternations in the interpolant created by this algorithm.

%We show that they are very closely related to the number of color alternations which are created by the substitutions of the resolution refutation. 


As a variations of Huang's work, we propose an approach which combines the two phases into one 
by lifting and quantifying colored terms during the extraction phase. 
Consequently, the resulting interpolants are not in prenex form but the scope of quantifiers is limited to the subformula where the lifted term is of relevance.
This algorithm is dealt with in Chapter~\ref{chap:one_phase}.

Complementary to these algorithms, we also present a non-constructive, model-theoretic approach to interpolation.
Assuming the non-existence of an interpolant, a maximal consistent intersection of two theories is constructed, where the theories are each based on the sets of formulas to interpolate. The details of this method are laid out in Chapter~\ref{chap:semantic}.


~

~

The two main algorithmic approaches towards interpolant calculation which we present in this thesis differ in their practical applicability. 
Craig's proof in \cite{Craig57linear,Craig57three} gives rise to a procedure which in its run introduces next to basic axioms for the equality predicate also congruence axioms for every predicate symbol and functional axioms for every function symbol.
Furthermore, the complexity of nested terms in the initial formulas is translated into a formula structure without nested terms.
Once this translation is established, the actual interpolant calculation in first-order logic without equality and function symbols can be done in a straightforward manner by a direct extraction from a proof.

Hence the question of whether it is possible to perfom an interpolant extraction from a proof of formulas in full first-order logic with equality arises naturally, and as Huang has shown in \cite{Huang95}, the answer is positive if we consider proofs in resolution calculus.

The first phase of Huang's approach is similar to other approaches for propositional logic (\cite{krajivcek1997interpolation,Pudlak97,McMillan03}),
but after fixing the propositional structure, he introduces a lifting phase in order to handle colored function and constant symbols.
It is interesting to see that despite that resolution in first-order logic has the with respect to resolution in propositional calculus additional rule of paramodulation, the same strategy of inductive propositional interpolant extraction can be applied as for resolution and factorization.

Our analysis of the number of quantifier alternations in interpolants produced by this procedure is based on an analysis of the lifting phase of Huang's proof.
We show that a $\Phi$-colored term $t$ with color-alternations can only occur if every maximal $\Psi$-colored subterm $s$ of $t$ occurs in a grey literal and therefore $s$ is contained in the final interpolant.
Hence underlying resolution refutation also shapes the lifting variables and their quantifiers in a somewhat immediate manner.

Our one-phase approach for interpolant extraction can be seen as a variation of especially the lifting procecure of Huang's algorithm, which so far has only received relatively little attention.











\clearpage

constructive/non-constructive

craig is an algorithm, but huang is an actual algorithm

-> unterschiedlicher syntaktischer overhead

interpolant extraction with equality works similar as without equality

quantifiers are introduced in a ``deterministic'' way with respect to the resolution refutation


