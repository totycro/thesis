
\chapter{The semantic perspective on interpolation}

An interesting feature of the interpolant theorem is that it admits a proof, which is distinct from the proof-theoretic ones discussed in the foregoing chapters, as it is a purely model-theoretic.
It is based on the joint consistency theorem by Robinson (\cite{robinson1956result}), which we show to be equivalent to the interpolation theorem.
The joint consistency theorem itself was originally presented as a proof of Beth's definability theorem, which is discuss in section~\ref{sec:beth}.

\section{Joint consistentcy}
\label{sec:joint_consistency}

The notion of joint consistency is based on separability of sets of formulas:

\begin{defi}[Separability]
	Let $\Gamma$ and $\Delta$ be sets of first-order formulas.
	A formula $A$ in the language $\Lang(\Gamma)\cap \Lang(\Delta)$ is said to \defiemph{separate} $\Gamma$ and $\Delta$ if $\Gamma \entails A$ and $\Delta \entails \lnot A$.
	$\Gamma$ and $\Delta$ are \defiemph{separable} if there exists a formula in the language $\Lang(\Gamma)\cap \Lang(\Delta)$ which separates $\Gamma$ and $\Delta$ and \defiemph{inseparable} otherwise.
\end{defi}

Note that for joint consistency, it is not necessary to require the original sets to be consistent as this is implied by separability:

\begin{lemma}
	\label{lemma:insep_consistent}
	Let $\Gamma$ and $\Delta$ be inseparable sets of first-order formulas. Then $\Gamma$ and $\Delta$ are each consistent.
\end{lemma}
\begin{proof}
	Suppose w.l.o.g.\ that $\Gamma$ is inconsistent. Then $\Gamma \entails \bot$, and as $\Delta \entails \top$, $\bot$ separates $\Gamma$ and $\Delta$.
\end{proof}

The joint consistency theorem shows that if there exists no formula in the language $\Lang(\Gamma)\cap \Lang(\Delta)$ which separates $\Gamma$ and $\Delta$, then there exists no formula in any language which separate $\Gamma$ and $\Delta$ as then, $\Gamma \cup \Delta$ is consistent:

\begin{thm}[Robinson's joint consistency theorem]
	\label{thm:robinson}
	Let $\Gamma$ and $\Delta$ be sets of first-order formulas.
	Then $\Gamma \cup \Delta$ is consistent if and only if $\Gamma$ and $\Delta$ are inseparable.
\end{thm}
The following proof essentially follows \cite{Henkin63} and \cite{chang1990model}.
\begin{proof}
	Suppose that $\Gamma\cup\Delta$ is consistent and let $M$ be a model of it.
	Then clearly for every formula $A$, if $\Gamma \entails A$, then $M \entails A$ as $M \entails \Gamma$.
	But $M \entails \Delta$, hence it can not be the case that $\Delta \entails \lnot A$.

	For the other direction, suppose that $\Gamma$ and $\Delta$ are inseparable.
	We proceed by iteratively constructing two maximal consistent sets of formulas $T$ and $T'$ such that $\Gamma \subseteq T$ and $\Delta \subset T'$ where $T \cup T'$ is consistent in order to then derive a model of this union, thus establishing the consistency of $\Gamma$ and $\Delta$.

	Let $C = \{c_0, c'_0, c_1, c'_1, \dots\}$ be
	a countably infinite set of fresh constant symbols.
	Let $\mathcal{A}_0, \mathcal{A}_1, \dots$ be an enumeration of all sentences in the language $\Lang(\Gamma) \cup C$
	and $\mathcal{B}_0, \mathcal{B}_1, \dots$ an enumeration of all sentences in the language $\Lang(\Delta) \cup C$.

	Let $T_0 = \Gamma$ and $T'_0 = \Delta$. 
	We construct
	$T_{i+1}$ 
	from
	$T_{i}$
	by means of the following formation rules:
	%\begin{enumerate}[(1)]
	\begin{enumerate}[~~(1)]
		\item
			\label{theory_construction_1}
			If $T_{i} \cup \{\mathcal{A}_i\}$ and $T'_{i}$ are separable, then $T_{i+1} \defeq{}\, T_i$.
		\item Otherwise:
			\label{theory_construction_2}
			\begin{enumerate}[(2a)]
			\label{theory_construction_2a}
				\item If $\mathcal{A}_i$ is of the form $\exists x A$, then $T_{i+1} \defeq{}\, T_i \cup \{ \mathcal{A}_i, A\subst{x/c_i} \}$.
			\label{theory_construction_2b}
				\item Otherwise $T_{i+1} \defeq{}\, T_i \cup \{ \mathcal{A}_i \}$.
			\end{enumerate}
	\end{enumerate}

	\noindent
	$T'_{i+1}$ is formed in a similar fashion:
	\begin{enumerate}[~~(1$'$)]
		\item
			If $T'_{i} \cup \{\mathcal{B}_i\}$ and $T_{i+1}$ are separable, then $T'_{i+1} \defeq{}\, T'_i$.
		\item
			\begin{samepage}
				Otherwise: 
			\begin{enumerate}[~(2$'$a)]
				\item If $\mathcal{B}_i$ is of the form $\exists x A$, then $T'_{i+1} \defeq{}\, T'_i \cup \{ \mathcal{B}_i, A\subst{x/c'_i} \}$.
				\item Otherwise $T'_{i+1} \defeq{}\, T'_i \cup \{ \mathcal{B}_i \}$.
			\end{enumerate}
			\end{samepage}
	\end{enumerate}

	Now let
	$T = \bigcup_{i\vargeq 0} T_i$
	and
	$T' = \bigcup_{i\vargeq 0} T'_i$.
	We prove properties on $T$ and $T'$ which will be vital for the construction of a model of $T\cup T'$:

	\begin{enumerate}[I.]
		\item
			\label{enum:theories_insep}
			$T_i$ and $T'_i$ are inseparable.

			$\Gamma$ and $\Delta$ are inseparable by assumption and clearly the construction of the subsequent elements of the sequence do not violate this invariant.

			\item 
			\label{enum:theories_consistent}
			$T_i$ and $T'_i$ are consistent.

			Immediate by \ref{enum:theories_insep} and Lemma~\ref{lemma:insep_consistent}.

		\item
			\label{enum:each_max_consistent}
			$T$ and $T'$ are each maximal consistent with respect to $\Lang(\Gamma) \cup C$ and $\Lang(\Delta) \cup C$ respectively.

			We show the result for $T$.
			By~\ref{enum:theories_consistent}, $T$ is consistent.
			Suppose that for some $i$, $\mathcal{A}_i \not\in T$ and $\lnot\mathcal{A}_i \not\in T$.
			Then by the construction of $T$, we can derive that
			$T_i \cup \{\mathcal{A}_i\}$ and $T'_i$ are separable.
			Hence also
			$T \cup \{\mathcal{A}_i\}$ and $T'$ are, i.e.\ there exists a formula $B_1$ in the language $\Lang(T\cup\{\mathcal{A}_i\}) \cap \Lang(T') = (\Lang(\Gamma) \cap \Lang(\Delta)) \cup C$ such that
			$T \cup \{\mathcal{A}_i\} \entails B_1$ and $T' \entails \lnot B_1$.
			By the deduction theorem, we also have that \markA{} $T \entails \mathcal{A}_i \limpl B_1$.

			As we also assume that $\lnot \mathcal{A}_i \not\in T$, by a similar argument, there exists a formula $B_2$ in the language  $(\Lang(\Gamma) \cap \Lang(\Delta)) \cup C$ such that 
			\markB{} $T \entails \lnot \mathcal{A}_i \limpl B_2$ and $T' \entails \lnot B_2$.

			Then however \markA{} and \markB{} entail that in any model, depending on whether $\mathcal{A}_i$ holds in the model, at least one of $B_1$ and $B_2$ holds, i.e.\ $T \entails B_1 \lor B_2$.
			But as neither $B_1$ nor $B_2$ hold in $T'$, we obtain that $T' \entails \lnot (B_1 \lor B_2)$, in effect establishing that $B_1 \lor B_2$ separates $T$ and $T'$, a contradiction to \ref{enum:theories_insep}.


		\item
			\label{enum:intersection_consistent}
			$T \cap T'$ is maximal consistent with respect to $(\Lang(\Gamma) \cap \Lang(\Delta)) \cup C$.

			By~\ref{enum:each_max_consistent}, for every formula $A$ in $(\Lang(\Gamma) \cap \Lang(\Delta)) \cup C$ it holds that either 
			$A \in T$ or $\lnot A \in T$ as well as
			$A \in T'$ or $\lnot A \in T'$. As $T$ and $T'$ are inseparable, either $A \in T$ and $A\in T'$ or otherwise $\lnot A \in T$ and $\lnot A \in T'$.

	\end{enumerate}


	As $T$ is consistent, let $M$ be a model of $T$.
	Due to~\ref{enum:each_max_consistent}, for each term $t$ in $\Lang(\Gamma)\cup C$, $\exists x\, (t = x) \in T$ and hence by~\ref{theory_construction_2a}, there is some $c_i \in C$ such that $t=c_i \in T$.
	Therefore we can find a submodel of $N$ of $M$ which as $M$ is in the language $\Lang(\Gamma)\cup C$ such that
	every domain element in $N$ corresponds to a constant symbol in $C$.
	%the domain of $N$ is $\{ \interpretation{M}(c) \mid c \in C\}$, where $\interpretation{M}$ is the interpretation of $M$.
	Models $M'$ of $T'$ allow by a similar reasoning for finding submodels $N'$ of $M'$.

	As by~\ref{enum:intersection_consistent}, $T$ and $T'$ agree on all formulas of $(\Lang(\Gamma) \cap \Lang(\Delta))\cup C$, 
	we are able to find an isomorphism between the reducts $N$ and $N'$ to their common language.
	Hence we may build a common model $K$ based on $N$ and extending it to $\Lang(\Delta)$ by copying the respective interpretation of $N'$ with regard to the isomorphism.
	%Hence we may build $N'_c$ from $N'$ by exchanging every $c'_i \in C$ by its corresponding $c_j \in C$.
	%Now we see that by extending the $N$ to $\Lang(\Delta)$ by copying the interpretation of $N''$,
	Thus as $N\entails T$ and $N'\entails T'$, $K\entails T \cup T'$, which implies that $\Gamma\cup\Delta$ is consistent.
\end{proof}

\section{Joint consistency and interpolation}

Despite the fact that the proof given in the previous section is of a different nature than the ones given in the previous chapters, it is easy to see that it expresses an equivalent notion.
To that end, let us recall the Interpolation Theorem~\ref{thm:interpolation} in the reverse formulation:

\interpolationRevThm*

\begin{prop}
	Theorem~\ref{thm:robinson} and Theorem~\ref{thm:interpolation} are equivalent.
\end{prop}
\begin{proof}
	It is easy to see that the notion of reverse interpolant and separating formulas coincide.
\end{proof}





