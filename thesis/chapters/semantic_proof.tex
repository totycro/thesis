
\chapter{The semantic perspective on interpolation}

The proofs of the interpolation theorem in the preceding chapters have a common feature:
They are proof-theoretic approaches.

\mytodo{ finish this when the chapter is done content-wise }

\begin{thm}
	interpolation again
\end{thm}
\begin{proof}
	Suppose that there is no interpolant for $\Gamma$ and $\Delta$
	We show that then, $\Gamma\cup\Delta$ is satisfiable by constructing a model.

	Let $L_0 = \Lang(\Gamma) \cap \Lang(\Delta)$

	Note that by assumption there is no formula $I$ such that $\Gamma\entails I$ and $\Delta \entails \lnot I$.

	Let $\Phi = \varphi_1, \varphi_2, \dots$ be an enumeration of all formulas in the language $\Lang(\Gamma)$
	and
	$\Psi = \psi_1, \psi_2, \dots$ be an enumeration of all formulas in the language $\Lang(\Delta)$.

	Let $\Gamma_0 = \Gamma$
	and $\Delta_0 = \Delta$.
	Let $\Gamma_{i+1} = \Gamma_i \cup \{\varphi_i\}$ if there is no interpolant for $\Gamma_i \cup \{\varphi_i\}$ and $\Delta_i$, and $\Gamma_{i+1} = \Gamma_i$ otherwise.
	Let $\Delta_{i+1} = \Delta_i \cup \{\psi_i\}$ if there is no interpolant for $\Delta_i \cup \{\psi_i\}$ and $\Gamma_i$, and $\Delta_{i+1} = \Delta_i$ otherwise.

	Let $\Gamma_\omega$ be the limit of the sequence of the $\Gamma_i$ and
	Let $\Delta_\omega$ be the limit of the sequence of the $\Delta_i$.

	Note that $\Gamma_\omega$ and $\Delta_\omega$ are ??? (complete, maximal consistent).
	Suppose for a formula $\chi$ of language ??? that neither
	$\chi \in \Gamma_\omega$
	nor
	$\lnot\chi \in \Gamma_\omega$.

As then $\chi \not\in \Gamma_\omega$, but the construction considers all formulas of appropriate language, it must hold that there is an interpolant $\Gamma_i \cup \{\chi\}$ and $\Delta_i$ for some $i$.
Let $I$ denote this interpolant.

Hence we obtain that $\Gamma_\omega \entails \chi \limpl I$ 
and $\Delta_\omega \entails \lnot I$.

As also $\lnot\chi \not\in \Gamma_\omega$, by a similar argument, we obtain that 
there must be an interpolant $J$ for $\Gamma_j \cup \{\lnot \chi\}$ and $\Delta_j$ for some $j$ and consequently
$\Gamma_\omega \entails \lnot \chi \limpl J$ 
and $\Delta_\omega \entails \lnot J$.

But then
$\Gamma_\omega \entails (\chi \limpl I) \land (\lnot \chi \limpl J)$, which implies $\Gamma_\omega \entails I\lor J$, and also $\Delta\entails \lnot (I \lor J)$, thus $I\lor J$ is, in contradiction with the assumption, an interpolant for $\Gamma_\omega$ and $\Delta_\omega$.

By analogous argument, $\Delta_\omega$ is ??? as well. 

Note that for each formula $\alpha$ in $\Lang(\Gamma) \cap \Lang(\Delta)$,
both $\Gamma_\omega$ and $\Delta_\omega$ contain either $\alpha$ or $\lnot \alpha$.
Moreover, as there is no interpolant for $\Gamma_\omega$ and $\Delta_\omega$, they either both contain $\alpha$ or both contain $\lnot \alpha$.
Hence they agree on every formula in $\Lang(\Gamma) \cap \Lang(\Delta)$, and as both $\Gamma_\omega$ and $\Delta_\omega$ are consistent, their reduct to this language is consistent.

But as $\Gamma \subseteq \Gamma_\omega$ and 
$\Delta \subseteq \Delta_\omega$, there is a model of $\Gamma \cup \Delta$.






\end{proof}


\clearpage



\section{notes on shoenfield}

\begin{description}
	\item{$T[\Gamma]$:}

$\Gamma$ is a set of formulas in the theory $T$

$T[\Gamma]$: theory obtained from $T$ with $\Gamma$ as additional (nonlogical) axioms


	\item{Extension of theory:}
		may contains additional symbols and new theorems, and every symbol/theorem of the old language/theory is a symbol/theorem of the new language/theory

		elementary: from wikipedia: if all formulas of the language of the larger one with variables interpreted with elements of the smaller language hold in the smaller iff they hold in the larger.

	\item{Conservative extension of theory:}
		$T'$ is conservative extension of $T$ if formulas in $T$ which are theorems of $T'$ are theorems in $T$.

		Hence additional formulas may only be theorems if they involve new symbols.

	\item{Reduction theorem for consistency:}
		if a set of formulas in a theory is inconsistent, then there is a disjunction of negated formulas of the set which is a theorem of $T$

	\item{Chain:}
		a sequence of structure, such that every one is an extension of the former

		a chain is elementary if every element is an elementary extension of the former

	\item{Tarksi's Lemma:}
		the union of an elementary chain is an elementary extension of each member

	\item{Regular set of formulas:} 
		every $x=y$ or $x\neq y$ is contained, and for every formula $A[x_1, \dots, x_n]$ is contained (does not say if $x_i$ is var or term)

\end{description}

