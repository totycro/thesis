
\chapter{The semantic perspective on interpolation}

A curious feature of the interpolant theorem is that it admits a proof, which is distinct from the proof-theoretic ones discussed in the foregoing chapters as it is a purely model-theoretic.
It was initially presented in \cite{robinson1956result} as a proof of Beth's definability theorem, which is discussed in section~\ref{sec:beth}.

\begin{defi}[Separability]
	Let $\Gamma$ and $\Delta$ be sets of first order formulas.
	A formula $\varphi$ in the language $\Lang(\Gamma)\cap \Lang(\Delta)$ is said to \defiemph{separate} $\Gamma$ and $\Delta$ if $\Gamma \entails \varphi$ and $\Delta \entails \lnot \varphi$.
	$\Gamma$ and $\Delta$ are \defiemph{inseparable} if there exists no formula in the language $\Lang(\Gamma)\cap \Lang(\Delta)$ which separates $\Gamma$ and $\Delta$.
\end{defi}

Note that the following theorem shows that if there exists no formula in the language $\Lang(\Gamma)\cap \Lang(\Delta)$ which separates $\Gamma$ and $\Delta$, then there exists no formula in any language which separate $\Gamma$ and $\Delta$ as then, $\Gamma \cup \Delta$ is consistent:

\begin{thm}[Robinson's joint consistency theorem]
	\label{thm:robinson}
	Let $\Gamma$ and $\Delta$ be sets of first-order formulas.
	Then $\Gamma \cup \Delta$ is consistent if and only if $\Gamma$ and $\Delta$ are inseparable.
\end{thm}


\begin{prop}
	Theorem~\ref{thm:robinson} and ??? are equivalent.
	\todo{possibly still move this to the end of the section}
\end{prop}
\begin{proof}
	Suppose that ??? holds.

	Clearly if there is a formula $\varphi$ such that $\Gamma\entails \varphi$ and $\Delta\entails \lnot \varphi$, then $\Gamma \cup \Delta $ is inconsistent.

	On the other hand, if $\Gamma\cup\Delta$ is inconsistent, 

	\mytodo{ can easily use positive or negative form of craig interpolation here, continue proof when this detail is fixed}

\end{proof}

\begin{proof}[Proof of Theorem~\ref{thm:robinson}.]
	Suppose that $\Gamma\cup\Delta$ is consistent and let $M$ be a model of it.
	Then clearly for every formula $\varphi$, if $\Gamma \entails \varphi$, then $M \entails \varphi$ as $M \entails \Gamma$.
	But $M \entails \Delta$, hence it can not be the case that $\Delta \entails \lnot \varphi$.

	For the other direction, suppose that $\Gamma$ and $\Delta$ are inseparable.
	\mytodo{ICI}


	~

	~

	~


	\mytodo{old first proof approach, formulated for interpolation}
	Random note: if theories are inseparable, then each is by itself consistent (else pick some formula implies by the consistent theory, and its negation will be implied by the inconsistent one)


	Suppose that there is no interpolant for $\Gamma$ and $\Delta$
	We show that then, $\Gamma\cup\Delta$ is satisfiable by constructing a model.

	Let $L_0 = \Lang(\Gamma) \cap \Lang(\Delta)$

	Note that by assumption there is no formula $I$ such that $\Gamma\entails I$ and $\Delta \entails \lnot I$.
	\NB{this is a reverse interpolant, not sure which interpolant we would have in the thm statement. possibly write it down explicitly or use "barred" (BBL) or separable (everyone else) }



	Let $\Phi = \varphi_1, \varphi_2, \dots$ be an enumeration of all formulas in the language $\Lang(\Gamma)$
	and
	$\Psi = \psi_1, \psi_2, \dots$ be an enumeration of all formulas in the language $\Lang(\Delta)$.

	Let $\Gamma_0 = \Gamma$
	and $\Delta_0 = \Delta$.
	Let $\Gamma_{i+1} = \Gamma_i \cup \{\varphi_i\}$ if there is no interpolant for $\Gamma_i \cup \{\varphi_i\}$ and $\Delta_i$, and $\Gamma_{i+1} = \Gamma_i$ otherwise.
	Let $\Delta_{i+1} = \Delta_i \cup \{\psi_i\}$ if there is no interpolant for $\Delta_i \cup \{\psi_i\}$ and $\Gamma_i$, and $\Delta_{i+1} = \Delta_i$ otherwise.

	Let $\Gamma_\omega$ be the limit of the sequence of the $\Gamma_i$ and
	Let $\Delta_\omega$ be the limit of the sequence of the $\Delta_i$.

	Note that $\Gamma_\omega$ and $\Delta_\omega$ are ??? (complete, maximal consistent).
	Suppose for a formula $\chi$ of language ??? that neither
	$\chi \in \Gamma_\omega$
	nor
	$\lnot\chi \in \Gamma_\omega$.

As then $\chi \not\in \Gamma_\omega$, but the construction considers all formulas of appropriate language, it must hold that there is an interpolant $\Gamma_i \cup \{\chi\}$ and $\Delta_i$ for some $i$.
Let $I$ denote this interpolant.

Hence we obtain that $\Gamma_\omega \entails \chi \limpl I$ 
and $\Delta_\omega \entails \lnot I$.

As also $\lnot\chi \not\in \Gamma_\omega$, by a similar argument, we obtain that 
there must be an interpolant $J$ for $\Gamma_j \cup \{\lnot \chi\}$ and $\Delta_j$ for some $j$ and consequently
$\Gamma_\omega \entails \lnot \chi \limpl J$ 
and $\Delta_\omega \entails \lnot J$.

But then
$\Gamma_\omega \entails (\chi \limpl I) \land (\lnot \chi \limpl J)$, which implies $\Gamma_\omega \entails I\lor J$, and also $\Delta\entails \lnot (I \lor J)$, thus $I\lor J$ is, in contradiction with the assumption, an interpolant for $\Gamma_\omega$ and $\Delta_\omega$.

By analogous argument, $\Delta_\omega$ is ??? as well. 

Note that for each formula $\alpha$ in $\Lang(\Gamma) \cap \Lang(\Delta)$,
both $\Gamma_\omega$ and $\Delta_\omega$ contain either $\alpha$ or $\lnot \alpha$.
Moreover, as there is no interpolant for $\Gamma_\omega$ and $\Delta_\omega$, they either both contain $\alpha$ or both contain $\lnot \alpha$.
Hence they agree on every formula in $\Lang(\Gamma) \cap \Lang(\Delta)$, and as both $\Gamma_\omega$ and $\Delta_\omega$ are consistent, their reduct to this language is consistent.

But as $\Gamma \subseteq \Gamma_\omega$ and 
$\Delta \subseteq \Delta_\omega$, there is a model of $\Gamma \cup \Delta$.






\end{proof}


\clearpage


\section{shoenfield in my words}

Supp no closed formula $A$ such that $\proves_T A$ and $\proves_{T'} \lnot A$.
Then construct model of $T\cup T'$.

Let $\Gamma$ contain all theorems of $T'$ in the language of $T$ (hence in the language of the intersection).

$T$ with $\Gamma$ is consistent, because else we can construct a forbidden formula $A$ (somehow).
\bigskip

Let $\mathbf{ M_1 }$ model of $\mathbf T$ and $\Gamma$.

Let $N$ be $M_1 | L$ with symbols for $T'$ added (no info on interpretation of these).

All $L$-theorems of $T'$ are valid in $M_1$ and hence in $N$

By model extension theorem, have $M'_1$ such that $ \mathbf{M'_1 \entails T'}$, agrees with $N$ on $L$-formulas.
Hence \allbold{$M'_1$ and $M_1$ agree on the $L$-formulas}, $M'_1$ might have more domain elements.
\bigskip

$\Ra$ can construct this because due to earlier reasoning, there is no contradiction on the $L$-formulas
\bigskip


Let $E$ be $M'_1|L$ with symbols of $T$ added.

$M'_1|L$ is elementary extension of $M_1|L$, so $E$ is elementary extension of $M_1$.

By the corollary,
there is $M_2$ which agrees on the $L$-formulas with $E$, and which is an elementary extension of $M_1$


there is $M_2$ which agrees on the $L$-formulas with $E$ and agrees on all formulas with $M_1$.

$M_2$ is elementary extension of $M_1$, so $M_1\entails T$ implies $M_2 \entails T$.

$M_2|L$ is elementary extension of $E|L = M'_1|L$.

\medskip

similarily:
$M'_2$ is elementary extension of $M'_1$, so $M'_1\entails T'$ implies $M'_2 \entails T'$.

\bigskip


Same formulas hold always for $M_i$, $M'_i$ and both $M_i|L$ and $M'_i|L$.

\bigskip

Question: what is different from the union of all $M_i$ in comparision with a single $M_i$?
The single ones are already proper models of their respective theories.

does the union differn in so far as it has exactly the same elements?

we always take a model of e.g. $T$ and make a model of $T'$ out of it. before we create the model of $T'$, we take the reduct w.r.t.\ $L$, but we do keep the domain elements.

consider aufschaukeln!,:

$T = \{ P(t(a)), \forall x (P(t(x)) \limpl P(s(x))) \}$

$T' = \{  \forall x (P(s(x)) \limpl P(r(x))) ,
\forall x (P(r(x)) \limpl P(t(s((x))))) \}$
\comm{$r$ is just to have something which is not in $L$ to see what the reduct actually does. We could build something similar for $T$.}


~

~




\section{notes on shoenfield}



\begin{description}
	\item{$T[\Gamma]$:}

$\Gamma$ is a set of formulas in the theory $T$

$T[\Gamma]$: theory obtained from $T$ with $\Gamma$ as additional (nonlogical) axioms


	\item{Extension of theory:}
		may contains additional symbols and new theorems, and every symbol/theorem of the old language/theory is a symbol/theorem of the new language/theory

		elementary: from wikipedia: if all formulas in the language of the larger one with variables interpreted with elements of the smaller language hold in the smaller iff they hold in the larger.

	\item{Conservative extension of theory:}
		$T'$ is conservative extension of $T$ if formulas in $T$ which are theorems of $T'$ are theorems in $T$.

		Hence additional formulas may only be theorems if they involve new symbols.

	\item{Reduction theorem for consistency:}
		if a set of formulas in a theory is inconsistent, then there is a disjunction of negated formulas of the set which is a theorem of $T$

	\item{Chain:}
		a sequence of structure, such that every one is an extension of the former

		a chain is elementary if every element is an elementary extension of the former

	\item{Tarksi's Lemma:}
		the union of an elementary chain is an elementary extension of each member

	\item{Regular set of formulas:} 
		every $x=y$ or $x\neq y$ is contained, and for every formula $A[x_1, \dots, x_n]$ is contained (does not say if $x_i$ is var or term)

\end{description}

