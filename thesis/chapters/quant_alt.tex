
\section{Number of quantifier alternations in the extracted interpolant}

In this section, we examine interpolants produced in Theorem~\ref{thm:two_phases} with respect to the number of quantifier alternations.
We arrive at the conclusion that there is a tight connection between the number of color alternations in the terms produced by the substitutions of the resolution refutation and the number of quantifier alternations in the resulting interpolant.  

We first formally define these notions:

\subsection{Color and quantifier alternations}

In the following, we assume that the maximum $\max$ of an empty sequence is defined to be $0$ and constants are treated as function symbols of arity $0$.
Furthermore $\bot$ is used to denote a color which is not possessed by any symbol.
\begin{defi}[Color alternation $\ca$]
	Let $\Gamma$ and $\Delta$ be sets of formulas and $t$ be a term.

	\medskip

	\noindent
	$\ca(t) \defeq \ca_\bot(t)$
	\medskip

	\begin{adjustwidth}{-1.7em}{}
	\noindent
	$
	\ca_\Phi(t) \defeq 
	\begin{cases}
		0 & \text{if $t$ is a variable} \\
		\max(\ca_\Phi(t_1), \dots, \ca_\Phi(t_n)) & \text{if $t = f(t_1, \dots, t_n)$ is gray} \\
		\max(\ca_\Phi(t_1), \dots, \ca_\Phi(t_n)) & \parbox[t]{0.4\textwidth}{if $t = f(t_1, \dots, t_n)$ is of color $\Phi$} \\
		1 + \max(\ca_\Psi(t_1), \dots, \ca_\Psi(t_n)) & \parbox[t]{0.38\textwidth}{if $t = f(t_1, \dots, t_n)$ is of color $\Psi$, $\Phi \neq \Psi$} \\
	\end{cases}
	$
\end{adjustwidth}
\end{defi}


\begin{defi}[Quantifier alternation $\qa$]
	Let $A$ be a formula.\nopagebreak
	\medskip

	\noindent
	$\qa(A) \defeq \qa_\bot(A)$
	\nopagebreak
	\medskip

	\noindent
	$
	\qa_Q(A) \defeq 
	\begin{cases}
		0 & \text{if $A$ is an atom} \\
		\qa_Q(B) & \text{if $A \equiv \lnot B$} \\
		\max(\parbox[t]{0.22\textwidth}{$\qa_Q(B),$\newline$ \qa_Q(C))$} & \text{if $A \equiv B \circ C$, $\circ \in \{\land, \lor, \limpl\}$} \\
		\qa_Q(B) & \text{if $A \equiv Q x B$} \\
		1+\qa_{Q'}(B) & \text{if $A \equiv Q' x B$, $Q \neq Q'$}  \\
	\end{cases}
	$
	\nopagebreak

	\qedhere
\end{defi}
%Note that this definition of quantifier alternations handles formulas in prenex and non-prenex form.


\subsection{Preliminary considerations}


First, we define the auxiliary procedure $\PI^*$:


\begin{defi}[$\PI^*$]
	$\PI^*$ is defined as $\PI$ with the difference that in $\PI^*$, all literals are considered to be gray.
	$\PIinit^*$ and $\PIstep^*$ are defined analogously.
\end{defi}

Hence $\PIinit^*$ coincides with $\PIinit$.
$\PIstep^*$ coincides with $\PIstep$ in case of factorization and paramodulation inferences.
For resolution inferences, the first two cases in the definition of $\PIstep$ do not occur for $\PIstep^*$.

$\PI^*$ enjoys the convenient property that it absorbs every literal which occurs in some clause:

\begin{prop}
	\label{prop:every_lit_in_pi_star}
	For every literal which occurs in a clause of a resolution refutation $\pi$, a respective successor occurs in $\PI^*(\pi)$.
\end{prop}
\begin{proof}
	By structural induction.
\end{proof}

Note that in $\PI^*$, we can conveniently reason about the occurrence of terms as no terms are lost throughout the extraction.
However Lemma~\ref{lemma:gray_lits_of_pi_star_in_pi} allows us to transfer results about gray literals to $\PI$:

\begin{lemma}
	\label{lemma:gray_lits_of_pi_star_in_pi}
	For every clause $C$ of a resolution refutation,
	the literals and equalities of $\PI(C)$ are exactly the gray literals and equalities of $\PI^*(C)$.
\end{lemma}
\begin{proof}
	Note that $\PIinit$ and $\PIinit^*$ coincide and $\PIstep$ and $\PIstep^*$ only differ for resolution inferences.
	More specifically, they only differ on resolution inferences, where the resolved literal is colored.
	%However here, no gray literals are removed but only colored ones.
	Hence $\PI(C)$ and $\PI^*(C)$ contain the same gray literals and equalities.
	The colored resolved literals however are not added to $\PI(C)$ as desired.
\end{proof}


\begin{lemma}
	\label{lemma:Ot8Gie7y}
	Let $\inference$ be an inference of a resolution refutation using the clauses $C_1, \dots, C_n$ which creates the clause $C$.
	If there is a literal $\lambda$ or an equality $s=t$ in $\PI(C_i)$ or a gray literal $\lambda$ or an equality $s=t$ in $C_i$ for $1 \varleq i \varleq n$, 
	then a successor of $\lambda$ or $s=t$ respectively occurs in $\PIstep(\inference, \PI(C_1), \dots, \PI(C_n)) \lor C$.
\end{lemma}
\begin{proof}
	Immediate by the definition of $\PI$.
\end{proof}

\begin{corr}
	\label{lemma:gray_lits_and_eq_all_in_PI}
	If there is a literal $\lambda$ or an equality $s=t$ in $\PI(C)$ or a gray literal $\lambda$ or an equality $s=t$ in $C$ for a clause $C$ of a resolution refutation $\pi$,
	then a successor of $\lambda$ or $s=t$ respectively occurs in $\PI(\pi)$.
\end{corr}
\begin{proof}
	This is a direct consequence of Lemma~\ref{lemma:Ot8Gie7y}.
\end{proof}


\subsection{Analysis of the occurrences of crucial terms in $\PI$}

We now make some considerations about the construction of certain terms in the context of interpolant extraction.
Thereby we employ the following definition:

\begin{defi}
In a literal or term $\varphi$ containing a subterm $t$, $t$ is said to occur \defiemph{below} a $\Phi$-symbol $s$ if in the syntax tree representation of $\varphi$, there is a node labeled $s$ on the path from the root to $t$. Note that the colored symbol may also be the predicate symbol.
Moreover, $t$ is said to occur \defiemph{directly below} the $\Phi$-symbol $s$ if it occurs below the $\Phi$-symbol $s$ and in the syntax tree representation of $\varphi$ on the path from $s$ to $t$, no nodes with labels with colored symbol occur.
\end{defi}



Moreover, we frequently reason over the stepwise application of the respective unifiers, for which we make use of the following definition:

\begin{defi}
	We define $\PIstepstarnosigma$ to coincide with $\PIstep^*$ but without applying the substitution $\sigma$ in each of the cases.
	Furthermore, $\PIstarnosigma(C)$ is an abbreviation of $\PIstepstarnosigma(\inference, \PI^*(C_1), \dots, \PI^*(C_m))$ if $C$ is created by an inference $\inference$ from the clauses $C_1, \dots, C_n$,
	and $\PIstarnosigma(C)$ coincides with $\PI^*(C)$ if $C\in \Gamma\cup\Delta$.

	Analogously, if $C \equiv D\sigma$, we use $\nosigma{C}$ to denote\nolinebreak{} $D$.
\end{defi}

In the context of an inference $\inference$ using the clauses $C_1, \dots, C_m$ to infer $C$, it holds that:
\begin{align*}
	 \PI^*(C) \lor C 
	 =\,& \PIstep^*(\inference, \PI^*(C_1), \dots, \PI^*(C_m)) \lor C  \\
	=\, & \Big(\PIstepstarnosigma(\inference, \PI^*(C_1), \dots, \PI^*(C_m)) \lor \nosigma{C} \Big) \sigma  \\
	=\, & \Big(\PIstarnosigma(C)\lor \nosigma{C} \Big) \sigma  \\
	=\,& \Big(\PIstarnosigma(C)\lor \nosigma{C} \Big) \sigmarange{0}{|\dom(\sigma)|}
\end{align*}


\newcommand{\inv}{\ensuremath{\PIstarnosigma(C)\lor \nosigma{C}}}
\newcommand{\invp}{\ensuremath{(\inv)}}


Note that if we are able to show that the application of a substitution $\sigma_i$ to $\invp\sigmazmi$ maintains an invariant and the invariant holds for $\inv$, then it immediately follows that it holds for $\PI^*(C) \lor C$. 



\begin{lemma}
	\label{lemma:var_below_phi_symbol}
	Let $\inference$ be an inference in a refutation of $\Gamma\cup\Delta$.
	Suppose that a variable $u$ occurs directly below a $\Phi$-symbol in $\invp\sigmazi$ for $i\vargeq 1$.
	Then at least one of the following statements holds:
	\begin{enumerate}
		\item
			\label{14_1}
			\label{15_1}
			The variable $u$ occurs directly below a $\Phi$-symbol in $\invp\sigmazmi$.

		\item
			\label{14_5}
			\label{15_5}
			The variable $u$ occurs at a gray position in a gray literal or at a gray position in an equality in $\invp\sigmazi$.

		\item 
			\label{14_2}
			\label{15_2}
			There is a variable $v$ such that 
			{
				\renewcommand{\labelitemi}{\textendash}
				\begin{itemize}
					\item $u$ occurs gray in $v\sigma_i$ and
					\item $v$ occurs in $\invp\sigmazmi$ directly below a $\Phi$-symbol as well as directly below a $\Psi$-symbol
				\end{itemize}
			}

	\end{enumerate}
\end{lemma}
\begin{proof}
	We consider all different situations under which the situation in question arises.
	Irrespective of the type of the inference $\inference$, one of these cases can apply:

	\begin{itemize}
		\item
			There is already a literal in $\invp\sigmazmi$ where $u$ occurs directly below a $\Phi$-symbol and $\sigma_i$ does not change this.
			Then clearly \ref{14_1} is the case.

		\item
			There is a variable $v$ in $\invp\sigmazmi$ such that $v\sigma_i$ contains $u$ directly below a $\Phi$-symbol.
			As $v$ is unified with the term $v\sigma_i$, $v\sigma_i$ must occur in $\invp\sigmazmi$, which implies that \ref{14_1} is the case.

	\end{itemize}


	\noindent
	In the case that $\inference$ is a resolution or factorization inference, 
	the following situations can apply:

	\begin{itemize}
		\item
			There is a variable $v$ which occurs directly below a $\Phi$-symbol such that $u$ occurs gray in $v\sigma_i$.

			Hence in the resolved or factorized literals $\lambda$ and $\lambda'$ in $\invp\sigmazmi$, there is a position $p$ such that without loss of generality $\lambda\atp = v$ and $u$ occurs gray in $\lambda'\atp$. 
			Note that due to the definition of the unification algorithm, $\lambda$ and $\lambda'$ must coincide on the path to $p$.

			By Proposition \ref{prop:every_lit_in_pi_star}, $\lambda$ and $\lambda'$ occur in $\inv$ irrespective of their coloring.

			We distinguish cases based on the position $p$:

			\begin{itemize}
				\item Suppose that $p$ occurs directly below a $\Phi$-symbol.
					Then as $u$ occurs gray in  $\lambda'\atp$, $u$ occurs directly below a $\Phi$-symbol in $\invp\sigmazmi$ and \ref{14_1} is the case.

				\item Suppose that $p$ occurs directly below a $\Psi$-symbol.
					Then $v$ occurs directly below a $\Psi$-symbol in $\lambda\atp$ and  \ref{14_2} holds.

				\item
					Suppose that $p$ does not occur directly below a colored symbol.
					Then $p$ does not occur below any colored symbol, hence $u$ is contained in a gray literal in a gray position in $\invp\sigmazmi$. 
					As $\sigma_i$ is trivial on $u$, this occurrence of $u$ also is present in $\invp\sigmazi$ and hence \ref{14_5} is the case.

			\end{itemize}
	\end{itemize}

	Now we consider the case that $\inference$ is a paramodulation inference of the clauses $C_1: r_1=r_2 \lor D$ and $C_2: E\occatp{r}$ with $\sigma=\mgu(\inference)=\mgu(r_1, r)$ yielding $C: (D\lor E\occatp{r_2})\sigma$.
	We again consider the different situations under which the situation in question arises:

	\begin{itemize}

		\item
			The variable $u$ occurs gray in $r_2$ and $p$ in $E$ is directly below a $\Phi$-symbol. 
			But then $u$ occurs gray in an equality in $\invp\sigmazmi$ and as $\sigma_i$ is trivial on $u$ also in $\invp\sigmazi$, hence \ref{15_5} holds.

		\item
			Suppose that some variable $v$ occurs directly below a $\Phi$-symbol in $\invp\sigmazmi$ such that $u$ occurs gray in $v\sigma_i$.
			Then by the definition of the unification algorithm, there exists a position $q$ such that one of $r_1\atq$ and $r\atq$ is $v$ and the other one contains a gray occurrence of $u$.

			We distinguish cases based on the position $q$:

			\begin{itemize}
				\item
					Suppose that $q$ occurs directly below a $\Phi$-symbol. Then clearly \ref{15_1} is the case.

				\item
					Suppose that $q$ occurs directly below a $\Psi$-symbol. Then as the variable $v$ also occurs directly below a $\Phi$-symbol and $u$ occurs gray in $v\sigma_i$, \ref{15_2} is the case.

				\item
					Suppose that $q$ is a gray position.
					Then \ref{15_5} is the case: 
					Either $u$ occurs gray in $r_1$ in $\invp\sigmazmi$ and then also in $\invp\sigmazi$, 
					or otherwise $v$ occurs gray in $r_1$ in $\invp\sigmazmi$, but as $v\sigma_i$ contains $u$ gray, $u$ occurs gray in of $r_1\sigma_i$ in $\invp\sigmazi$.
					\qedhere
			\end{itemize}

	\end{itemize}

\end{proof}


\begin{lemma}
	\label{lemma:col_change}
	Let $\inference$ be an inference of a resolution refutation of $\Gamma \cup \Delta$.
	Suppose that a variable $u$ occurs directly below a $\Phi$-symbol as well as directly below a $\Psi$-symbol in $\invp\sigmazi$.
	Then $u$ occurs gray in a gray literal or gray in an equality in $\invp\sigmazi$.
\end{lemma}
\begin{proof}
	We proceed by induction over the refutation.
	As the original clauses each contain symbols of at most one color, the base case is trivially true.

	For the induction step, suppose that an inference makes use of the clauses $C_1, \dots, C_n$ and that the lemma holds for $\PI^*(C_j) \lor C_j$ for $1\varleq j \varleq n$. 

	Note that then, the lemma holds for $\PIstepstarnosigma(\inference, \PI^*(C_1), \dots, \PI^*(C_n)) \lor \nosigma{C} = \inv$.
	This is because as all clauses are variable-disjoint, 
	if a variable occurs in $\inv$ both directly below a $\Phi$-symbol as well as directly below a $\Psi$-symbol, then this must be the case also in 
	$\PI^*(C_j) \lor C_j$ for some $j$, for which the lemma by assumption holds.
	Furthermore, by the definition of $\PI^*$, every literal which occurs in $\PI^*(C_j) \lor C_i$ for some $j$ occurs in $\inv$.

	Hence it remains to show that the lemma holds for $\invp\sigma = \invp\sigma_0\quantifierdots\sigma_m$, which we do by induction over $i$ for $1\varleq i \varleq m$.
	Suppose that the lemma holds for $\invp\sigmazmi$ and in $\invp\sigmazi$, the variable $u$ occurs directly below a $\Phi$-symbol as well as directly below a $\Psi$-term.

	Then by Lemma~\ref{lemma:var_below_phi_symbol}, we can deduce that one of the following statements holds for $\Phi = \Gamma$ as well as $\Phi = \Delta$. We denote case $j$ for $\Phi = \Gamma$ by $j^\Gamma$ and for $\Phi = \Delta$ by $j^\Delta$.

	\begin{enumerate}
		\item
			\label{16_1}
			The variable $u$ occurs directly below a $\Phi$-symbol in $\invp\sigmazmi$.

		\item
			\label{16_4}
			The variable $u$ occurs at a gray position in a gray literal or at a gray position in an equality in $\invp\sigmazi$.

		\item 
			\label{16_2}
			There is a variable $v$ such that 
			{
				\renewcommand{\labelitemi}{\textendash}
				\begin{itemize}
					\item $u$ occurs gray in $v\sigma_i$ and
					\item $v$ occurs in $\invp\sigmazmi$ directly below a $\Phi$-symbol as well as directly below a $\Psi$-symbol
				\end{itemize}
			}
	\end{enumerate}

	If \ref{16_4}$^\Gamma$ or \ref{16_4}$^\Delta$ is the case, we clearly are done.
	On the other hand if \ref{16_2}$^\Gamma$ or \ref{16_2}$^\Delta$ is the case, then by the induction hypothesis, $v$ occurs gray in a gray literal or gray in an equality in $\invp\sigmazmi$. 
	As $u$ occurs gray in $v\sigma_i$, we obtain that then, $u$ occurs gray in a gray literal or gray in an equality in $\invp\sigmazi$.

	Hence the only remaining possibility is that both \ref{16_1}$^\Gamma$
	and \ref{16_1}$^\Delta$ hold.
	But then $u$ occurs directly below a $\Phi$-symbol as well as below a $\Psi$-symbol in $\invp\sigmazmi$ and again by the induction hypothesis, we obtain that $u$ occurs gray in a gray literal or gray in an equality in $\invp\sigmazmi$, and as $\sigma_i$ is trivial on $u$, the same occurrence of $u$ is present in $\invp\sigmazi$.
\end{proof}



\begin{lemma}
	\label{lemma:subterm_in_gray_lit}
	Let $C$ be a clause in a resolution refutation of $\Gamma \cup \Delta$.
	If $\PI^*(C) \lor C$ contains a maximal colored occurrence of a $\Phi$-term $t\occ{s}$, which contains a maximal $\Psi$-colored term $s$, then $s$ occurs gray in $\PI(C) \lor C$.
\end{lemma}
\begin{proof}
	Note that it suffices to show that the desired term occurs in a gray literal or equality in $\PI^*(C) \lor C$
	since by Lemma~\ref{lemma:gray_lits_of_pi_star_in_pi}, all gray literals and equalities of $\PI^*(C)$ also occur in $\PI(C)$.
	We do so by induction over the resolution refutation.
	
	As the original clauses each contain symbols of at most one color, the base case is vacuously true.

	The induction step is laid out similarly as in the proof of Lemma~\ref{lemma:col_change}. 
	We suppose that an inference makes use of the clauses $C_1, \dots, C_n$ and that the lemma holds for $\PI^*(C_j) \lor C_j$ for $1\varleq j \varleq n$. 
	Then the lemma holds for $\inv = \PIstepstarnosigma(\inference, \PI^*(C_1), \dots, \PI^*(C_n)) \lor \nosigma{C})$ as no new terms are introduced in $\inv$ and all literals from $\PI^*(C_j) \lor C_j)$ for $1\varleq j \varleq n$ occur in~$\inv$.

	It remains to show that the lemma holds for $\invp\sigma = \invp\sigma_0 \quantifierdots \sigma_m$, which we do by induction over $i$ for $0 \varleq i \varleq m$.
	We distinguish based on the situation under which a unification leads to the term $t\occ{s}$.

	\begin{itemize}
		\item 
			Suppose for some variable $u$ that $u\sigma_i$ contains $t\occ{s}$. 
			Then $u$ is unified with a term which contains $t\occ{s}$ and which occurs in $\invp\sigmazmi$.
			Hence by the induction hypothesis, $s$ occurs gray in a gray literal or gray in an equality in $\invp\sigmazmi$ and, as $\sigma_i$ does not change this, also in $\invp\sigmazi$.

		\item 
			Otherwise there is a variable $u$ which occurs directly below a $\Phi$-symbol and $v\sigma_i$ contains a gray occurrence of $s$.
			We distinguish based on the occurrences of $u$ in $\invp\sigmazmi$:

			\begin{itemize}
				\item Suppose that $u$ occurs somewhere in $\invp\sigmazmi$ gray in a gray literal or gray in an equality. Then clearly we are done.
				\item Suppose that $u$ occurs somewhere in $\invp\sigmazmi$ directly below a $\Psi$-symbol.
					Then by Lemma~\ref{lemma:col_change}, $u$ occurs gray in a gray literal or gray in an equality in $\invp\sigmazmi$, whose successor in $\invp\sigmazi$ is an occurrence of $s$ of the same coloring. Hence we are done a well.
				\item Suppose that $u$ occurs in $\invp\sigmazmi$ only directly below a $\Phi$-symbol.
					Here, we differentiate between the types of inference of the current induction step:

					\begin{itemize}
						\item
							Suppose that the inference of the current induction step is a resolution or a factorization inference.
							As $u$ occurs gray in $v\sigma_i$, there is a position $p$ such that for the resolved or factorized literals $\lambda$ and $\lambda'$ it holds without loss of generality that $\lambda\atp = u$ and $s$ occurs gray in $\lambda'\atp$.
							Note that $\lambda$ and $\lambda'$ agree on the path to $p$, including the predicate symbol..

							Now as by assumption $u$ only occurs directly below a $\Phi$-symbol, so must $s$.
							But then $s$ occurs directly below a $\Phi$-symbol in $\invp\sigmazmi$ and we get the result by the induction hypothesis.

						\item
							Suppose that the inference of the current induction step is a paramodulation inference.
							Assume it uses the the clauses $C_1: r_1=r_2 \lor D$ and $C_2: E\occatp{r}$ with $\sigma=\mgu(\inference)=\mgu(r_1, r)$ to yield $C: (D\lor E\occatp{r_2})\sigma$.

							As $u$ is affected by $\sigma_i$, it must occur in $r_1$ or $r$. Let $\bhat u$ refer to this occurrence.
							%We distinguish cases based on the symbol below which $\bhat u$ occurs.

							\begin{itemize}
								\item
									Suppose that $\bhat u$ occurs directly below a $\Phi$-colored function symbol. 

									If $\bhat u$ is contained in $r_1$, then $s$ must be contained in $r$ directly below a $\Phi$-colored function symbol as $r_1$ and $r$ are unifiable. We then get the result by the induction hypothesis.

									If otherwise $\bhat u$ is contained in $r$, 
									then there are two possibilities for the occurrence of $s$ in $r_1$:

									Either $\bhat u$ occurs in a $\Phi$-colored function symbol in $r$. Then $s$ occurs in a $\Phi$-colored function symbol in $r_1$ and we get the result by the induction hypothesis.

									Otherwise $\bhat u$ occurs gray in $r$, but $r$ occurs directly below a $\Phi$-colored function symbol in $E$.
									Then however, as $r$ and $r_1$ are unifiable, $s$ must occur gray in $r_1$ and hence gray in an equality.

								\item
									Suppose that $\bhat u$ occurs directly below a $\Phi$-colored predicate symbol. 

									Then as the equality predicate is not considered to be colored, $u$ must occur gray in $r$.
									But then as $r_1$ and $r$ are unifiable, $s$ must occur gray in $r_1$ and hence gray in an equality.
									\qedhere
							\end{itemize}

					\end{itemize}

			\end{itemize}

	\end{itemize}


	%Let $C$ be a clause of a resolution refutation of $\Gamma \cup \Delta$.
	%As the literals occurring in $\PI(C)$ are a subset of the literals occurring in $\PI^*(C)$, the lemma prerequisites hold true only for terms for which they also hold in $\PI^*(C)$.
	%Therefore we can deduce that if a maximal colored $\Phi$-term $t\occ{s}$ containing a maximal $\Psi$-colored term $s$ occurs in $\PI(C)\lor C$, then $t\occ{s}$ also occurs in $\PI^*(C) \lor C$ and by Lemma~\ref{lemma:subterm_in_gray_lit_star},
	%the term $s$ occurs gray in a gray literal or gray in an equality in $\PI^*(C) \lor C$.
\end{proof}

\subsection{Lower bound}

The lemmas of the previous section are now employed to derive a lower bound on the number of quantifier alternations in the interpolant:

\begin{lemma}
	%\label{lemma:col_alt_in_gray_lit_then_quant_alt}
	\label{lemma:quant_alt_lower_bound}
	If a term with $n$ color alternations occurs in $\PI(C)$ or in a gray literal or equality in $C$ for a clause $C$, then the interpolant $I$ produced in Theorem~\ref{thm:two_phases} contains at least $n$ quantifier alternations.
\end{lemma}
\begin{proof}
	We perform an induction on $n$
	and show the strengthening that
	the quantification of the lifting variable which replaces a term with $n$ color alternations is required to be in the scope of the quantification of $n-1$ alternating quantifiers.

	Note that by Corollary~\ref{lemma:gray_lits_and_eq_all_in_PI}, a successor of every literal and equality of $\PI(C)$ and a successor every grey literal or equality of $C$ occurs in $\PI(\pi)$.

	For $n=0$, no colored terms occur in $I$ and hence also no quantifiers.
	Moreover for $n=1$, there are terms of one color which evidently require at least one quantifier.

	Suppose that the statement holds for $n-1$ for $n>1$ and that a term $t$ with $\ca(t) = n$ occurs in $\PI(C) \lor C$.
	We assume without loss of generality that $t$ is a $\Phi$-term.
	Then $t$ contains some $\Psi$-colored term $s$ with $\ca(s) = n-1$ and
	by Lemma~\ref{lemma:subterm_in_gray_lit}, $s$ occurs gray in $\PI(C) \lor C$.
	By Corollary~\ref{lemma:gray_lits_and_eq_all_in_PI}, a successor of $s$ occurs in $\PI(\pi)$. Note that as $s$ occurs in a gray position, any successor of $s$ also occurs in a gray position.

	By the induction hypothesis, the quantification of the lifting variable for $s$ requires $n-1$ alternated quantifiers.
	As $s$ is a subterm of $t$ and $t$ is lifted, $t$ must be quantified in the scope of the quantification of $s$, and as $t$ and $s$ are of different color, their quantifier type is different. 
	Hence the quantification of the lifting variable for $t$ requires $n$ quantifier alternations.
\end{proof}

\begin{comment}
\begin{lemma}
	%\label{prop:color_alt_eq_quant_alt}
	\label{lemma:quant_alt_lower_bound_old}
	If a term with $n$ color alternations occurs in $\PI^*(C) \lor C$ for a clause $C$, then the interpolant $I$ produced in Theorem~\ref{thm:two_phases} contains at least $n-1$ quantifier alternations.
\end{lemma}
\begin{proof}
	By Lemma~\ref{lemma:subterm_in_gray_lit}, a term with $n-1$ color alternations occurs in a gray literal or an equality in $\PI(C) \lor C$.
	Lemma~\ref{lemma:col_alt_in_gray_lit_then_quant_alt} gives the result.
\end{proof}
\end{comment}

We present an example which illustrates that terms in colored literals may contain more color alternations than the term with the maximal number of color alternations in gray literals or equalities.
Still, the latter determines the minimum number of quantifier alternations in the interpolant.
Note that it is a consequence of Lemma~\ref{lemma:subterm_in_gray_lit} that if for some clause $C$ a term with $n$ color alternations occurs in a colored literal in $\PI^*(C) \lor C$ (which contains all literals, i.e.\ also the colored ones), then $\PI(C)\lor C$ contains a term with at least $n-1$ color alternations.

%We present an example which illustrates that the occurrence of a term with $n$ color alternations in $\PI(C) \lor C$ for a clause $C$ can lead to an interpolant with $n-1$ quantifier alternations (but no less as Lemma~\ref{lemma:quant_alt_lower_bound} shows).
\begin{exa}
	Let $\Gamma = \{ \lnot P(a) \}$ and $\Delta = \{ P(x) \lor Q(f(x)), \lnot Q(y) \}$.
	We consider the following refutation of $\Gamma \cup \Delta$, which we annotate by the interpolation extraction by appending $\PI(C)$ to each clause $C$, separated by ``$|$''.
	For the sake of brevity, we sometimes give simplified but logically equivalent versions of $\PI(C)$.
	This notational convention will be used throughout this thesis for examples of a similar form.
	\begin{prooftree}
		\AxiomCm{ \lnot P(a) \mid \bot }
		\AxiomCm{ P(x) \lor Q(f(x)) \mid \top }

		\RightLabelm{\resrule{\resruleres}{x\mapsto a}}
		\BinaryInfCm{ Q(f(a)) \mid \lnot P(a) }

		\AxiomCm{ \lnot Q(y) \mid \top }
		\RightLabelm{\resrule{\resruleres}{y\mapsto f(a)}}
		\BinaryInfCm{ \square \mid \lnot P(a) }
	\end{prooftree}

	In this example, Theorem~\ref{thm:two_phases} yields the interpolant $I \equiv \exists y_a \lnot P(y_a)$ with $\qa(I) =\nolinebreak 1$.
	The existence of the term $a$ with $\ca(a) = 1$ in a clause of the refutation by Lemma~\ref{lemma:quant_alt_lower_bound} implies that $\qa(I) \vargeq 1$.
	The occurrence of the term $f(a)$ with $\ca(f(a)) = 2$ in the colored literal $Q(f(a))$ is not relevant.
\end{exa}

\subsection{Upper bound and conclusion}

We now also determine an upper bound for the number of quantifier alternations in the interpolant.

Note that as the following example shows, 
an upper bound of $n$ quantifier alternations in the interpolant is not sufficient even if $n$ is the maximal number of color alternations for any term in $\PI(C) \lor C$ for any clause $C$:

\begin{exa}
	Let $\Gamma = \{ P(a) \lor Q(u) \}$ and $\Delta = \{ \lnot P(v), \lnot Q(b) \}$.
	Consider the following refutation of $\Gamma \cup \Delta$:
	\begin{prooftree}
		\AxiomCm{ P(a) \lor Q(u) \mid \bot }
		\AxiomCm{ \lnot P(v) \mid \top }

		\RightLabelm{\resrule{\resruleres}{v\mapsto a}}
		\BinaryInfCm{ Q(u) \mid P(a) }

		\AxiomCm{ \lnot Q(b) \mid \top }
		\RightLabelm{\resrule{\resruleres}{u\mapsto b}}
		\BinaryInfCm{ \square \mid Q(b) \lor P(a) }
	\end{prooftree}

	Given this refutation, Theorem~\ref{thm:two_phases} produces either the interpolant
	$I_1 \equiv \exists y_a \forall x_b ( Q(x_b) \lor P(y_a) )$
	or 
	$I_2 \equiv \forall x_b \exists y_a ( Q(x_b) \lor P(y_a) )$.
	Note that the maximal number of color alternations of a term in $\PI(C) \lor C$ for any clause $C$ is $1$, but the number of quantifier alternations is $2$ for both $I_1$ and $I_2$.
\end{exa}

However the following bound holds in general:

\begin{lemma}
	\label{lemma:quant_alt_upper_bound}
	Let $t$ be a term with the maximal number of color alternations in $\PI(C)$ or a grey literal or equality in $C$ for any clause $C$.
	Then there is an arrangement of the quantifier prefix in Theorem~\ref{thm:two_phases} which gives rise to an interpolant 
	with at most $\ca(t)+\nolinebreak{}1$ quantifier alternations.
\end{lemma}
\begin{proof}
	By Corollary~\ref{lemma:gray_lits_and_eq_all_in_PI}, a successor of $t$ occurs in $\PI(\pi)$.
	Let $T_i^\Phi$ be the set of maximal $\Phi$-colored terms in $\PI(\pi)$ with $i$ color alternations for $1\varleq i \varleq n$, where $n=\ca(t)$.
	Note that every maximal colored term of $\PI(\pi)$ is contained in one of these sets.
	%We use $Q T_i^\Phi$ to denote $ Q_1 z_{t_1} \dots Q_n z_{t_n}$ where $t_1, \dots, t_n$ is an arrangement of the elements of $T_i^\Phi$ in ascending subterm order  and $Q_i z_{t_i}$ for $1\varleq i \varleq n$ is $\forall x_{t_i}$ or $\exists y_{t_i}$ depending on the color of $t_i$.
	We use $\exists T_i^\Gamma$ $(\forall T_i^\Delta)$ to denote $ \exists y_{t_1} \dots \exists y_{t_m}$ $(\forall x_{t_1}\dots\forall x_{t_m})$ where $t_1, \dots, t_m$ is an arrangement of the elements of $T_i^\Gamma$ ($T_i^\Delta$) in ascending subterm order. 


	Now we construct the interpolant
	\[
		I \equiv 
		\forall T_1^\Delta \exists T_1^\Gamma \;
		\exists T_2^\Gamma \forall T_2^\Delta \;
		\forall T_3^\Delta \exists T_3^\Gamma
		\dots 
		Q^\Phi T_n^\Phi \Q^\Psi T_n^\Psi
		\;
		\lifgamma{\lifdelta{\PI(\pi)}},
	\]
	where $ Q^\Phi T_n^\Phi \Q^\Psi T_n^\Psi $ is
	$\forall T_n^\Delta \exists T_n^\Gamma $ if $n$ is odd and 
	$\exists T_n^\Gamma \forall T_n^\Delta $ if $n$ is even.
	Clearly, $I$ has at most $n+1$ color alternations.

	In order to show the result, it remains to show that $I$ is a valid interpolant with respect to Theorem~\ref{thm:two_phases}. 
	Note that the quantifier prefix binds all lifting variables occurring in  
	$\lifgamma{\lifdelta{\PI(\pi)}}$.
	We conclude by showing that the order of the quantifiers is admissible.

	Let $t$ be a maximal colored term in $\lifgamma{\lifdelta{\PI(\pi)}}$. 
	We prove that the quantifier for the lifting variable of every subterm $s$ of $t$ precedes the quantifier for the lifting variable for $t$ in $I$.
	Suppose that $\ca(t) = k$. Then we can deduce that $\ca(s) \varleq\nolinebreak{} k$.
	\begin{itemize}
		\item
			If $\ca(s) = k$, then $t$ and $s$ are of the same color and hence the quantifiers for their respective lifting variables are contained in the same block. 
			However the quantifiers of each block are ordered as desired.
			%their in the same quantifier block
		\item
			Otherwise $\ca(s) = l$ for some $l$ such that $l < k$.
			Then the lifting variable replacing $s$ is quantified in
			$\exists T_{l}^\Gamma$ or
			$\forall T_{l}^\Delta$.
			In any case, it precedes the quantifier for the lifting variable replacing $t$ which is contained in 
			$\exists T_{k}^\Gamma$ or
			$\forall T_{k}^\Delta$.
			%$Then by the arrangement of the quantifiers in~$I$ the quantifier for the lifting variable replacing $s$ precedes the quantifier for the lifting variable replacing $t$.
			\qedhere
	\end{itemize}
\end{proof}

The previous results can be summarized by the following theorem:\nopagebreak

\begin{thm}
	Let $n$ be the maximal number of color alternations of any term in $\PI(C)$ or in a grey literal or equality in $C$ for any clause $C$ of a resolution refutation of $\Gamma \cup \Delta$.
	Then by arranging the quantifiers in a quantifier alternation minimizing fashion the interpolant of Theorem~\ref{thm:two_phases} has at least $n$ and at most $n+1$ quantifier alternations.
\end{thm}
\begin{proof}
	Immediate by Lemma~\ref{lemma:quant_alt_lower_bound} and Lemma~\ref{lemma:quant_alt_upper_bound}.
\end{proof}



