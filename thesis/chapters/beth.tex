

\section{Beth's definability theorem}
\label{sec:beth}

In this section, we illustrate the interpolation theorem by presenting Beth's definability theorem, which admits a straightforward proof by means of the interpolation theorem. 
The definability theorem deals with definitions of predicates by means of formulas and bridges the gap between implicit definitions, where predicates are defined by its use, and explicit definitions, which define a predicate by means of another formula, by even showing their equivalence.
This is given significance by the circumstance that implicit definitions occur in mathematics, but by this theorem do not have less expressive power than explicit definitions.

Its original publication in \cite{beth1953} precedes Craig's papers on interpolation (\cite{Craig57linear,Craig57three}) by four years and relies on a direct proof.
 

\begin{defi}[Implicit and explicit definition]
	Let $\LangSym$ be a first-order language and
	$P$ and $P'$ be two fresh predicate symbols of arity $n$.
	Let $\Gamma_P$ be a set of first-order sentences
	in the language $\LangSym\cup\{P\}$ 
	and $\Gamma_{P'}$ the same set of formulas with every occurrence of $P$ in $\Gamma_P$ replaced by\nolinebreak{} $P'$, such that the language of $\Gamma_{P'}$ is $\LangSym \cup\{P'\}$.

	$\Gamma_P$ defines $P$ implicitly iff
	\[\Gamma_P \cup \Gamma_{P'} \entails \forall x_1\quantifierdots \forall x_n \left(  P(x_1, \dots, x_n) \liff P'(x_1, \dots, x_n)\right).\]
	On the other hand $\Gamma_P$ defines $P$ explicitly iff there is formula $\varphi$ in $\LangSym$ with $\FV(\varphi) = \{x_1, \dots, x_n\}$ such that 
	\[\Gamma_P\entails \forall x_1\quantifierdots \forall x_n \left(  P(x_1, \dots, x_n) \liff \varphi\right).\qedhere\]
\end{defi}

Note that the definition of implicit definitions is essentially second-order 
and 
can be expressed by the second-order sentence
%\[ \forall P\,\forall P' \left(  \left(\bigwedge_{\varphi \in \Gamma^*_P \cup \Gamma^*_{P'}} \varphi \right) \limpl P=P'\right),\] 
%\begin{samepage}
\[ \forall P\,\forall P' \left(  \left(\Gamma^*_P \land \Gamma^*_{P'} \right) \limpl P=P'\right),\] 
where $\Gamma^*_P$ and $\Gamma^*_{P'}$ are conjunctions of the formulas of 
respective reductions of $\Gamma_P$ and $\Gamma_{P'}$ 
to finite sets, which exist by the compactness theorem.
%by reducing $\Gamma_P$ and $\Gamma_{P'}$ to finite sets $\Gamma^*_P$ and $\Gamma^*_{P'}$ by a suitable application of the compactness theorem
% cite chang/keisler page 324
%\end{samepage}

\begin{thm}[Beth's definability theorem]
	\label{thm:beth}
	$\Gamma_P$ defines $P$ explicitly if and only if $\Gamma_P$ defines $P$ implicitly.
\end{thm}
\begin{proof}
	Suppose that $\Gamma_P$ defines $P$ explicitly. 
	Then there exists some formula $\varphi$ such that 
	$\Gamma_P\entails \forall x_1\quantifierdots \forall x_n (  P(x_1, \dots, x_n) \liff \varphi)$.
	But then it clearly also holds that 
	$\Gamma_{P'}\entails \forall x_1\quantifierdots \forall x_n (  P'(x_1, \dots, x_n) \liff \varphi)$,
	hence
	\[
	\Gamma_{P} \cup \Gamma_{P'} \entails \forall x_1\quantifierdots \forall x_n (P(x_1, \dots, x_n) \liff P'(x_1, \dots, x_n)).\]
	Therefore $\Gamma_P$ is an implicit definition of $P$.

	For the other direction, suppose that $\Gamma_P$ defines $P$ implicitly. 
	Then
	$\Gamma_P \cup \Gamma_{P'} \entails\allowbreak \forall x_1\quantifierdots \forall x_n (  P(x_1, \dots, x_n) \liff P'(x_1, \dots, x_n))$.
	It follows from the compactness theorem that
	we can find a conjunction $\Gamma^*_{P'}$ of formulas of a finite subset of $\Gamma_{P'}$ such that  
	$\Gamma_P \cup \{\Gamma^*_{P'}\} \entails \forall x_1\quantifierdots \forall x_n (  P(x_1, \dots, x_n) \liff P'(x_1, \dots, x_n))$.
	%Let $\gamma_{P'}$ be the conjunction of all formulas in $\Gamma^*_{P'}$ and 
	Let $y_1, \dots, y_n$ be fresh variables.
	Then we obtain by the deduction theorem that  
	$\Gamma_P \cup \{P(y_1, \dots, y_n)\} \entails \Gamma^*_{P'} \limpl  P'(y_1, \dots, y_n)$.

	Note that $P$ only occurs in the antecedent and $P'$ only occurs in the consequent.
	Hence we can apply the Interpolation Theorem~\ref{thm:interpolation_original} in order to obtain a formula $\chi$
	such that
	$\Gamma_P \cup \{P(y_1, \dots, y_n)\} \entails \chi$ and
	$\chi \entails \Gamma^*_{P'} \limpl  P'(y_1, \dots, y_n)$,
	while additionally $\Lang(\chi) = \Lang(\Gamma_P) \cap \Lang(\Gamma_{P'})$.
	This implies that neither $P$ nor $P'$ occur in\nolinebreak{} $\chi$.
	By interpreting the free variables as constants for the purposes of the application of the interpolation theorem, we can also ensure that the only free variables in $\chi$ are $y_1, \dots, y_n$. 


	Now we apply the deduction theorem another time and get that
	\markA{} $\Gamma_P \entails P(y_1, \dots, y_n) \limpl \chi$ and
	$\Gamma^*_{P'} \entails \chi \limpl  P'(y_1, \dots, y_n)$.
	As $\Gamma_{P'}$ implies $\Gamma^*_{P'}$, we also have that
	$\Gamma_{P'} \entails \chi \limpl  P'(y_1, \dots, y_n)$.
	Since $P$ does not occur in this entailment, it remains valid if we replace every occurrence of the symbol $P'$ by $P$
	and obtain that
	\markB{} $\Gamma_{P} \entails \chi \limpl  P(y_1, \dots, y_n)$.

	But then \markA{} and \markB{} imply that 
	$\Gamma_{P} \entails \chi \liff  P(y_1, \dots, y_n)$, which is equivalent to
	$\Gamma_{P} \entails \forall y_1\quantifierdots\forall y_n \left( \chi \liff  P(y_1, \dots, y_n)\right)$.
	So clearly $\Gamma_P$ defines $P$ explicitly.
\end{proof}

\section{Interpolation in higher-order logic}
\label{sec:interpol_hol}
In this thesis, we restrict our attention to first-order logic.
This is not only a matter of reasonable scope, but justified by the fact that the interpolation theorem does not hold even in second-order logic as discovered by Craig in \cite{Craig65}.
There, a second-order formula is presented and shown to be implicitly, but not explicitly definable.
This failure of Beth definability directly leads to a failure of interpolation in this logic, which can easily be seen by the proof of Theorem~\ref{thm:beth}.
%This immediately leads to a failure of Beth definability, and consequently also interpolation, in this logic.



