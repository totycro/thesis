\chapter{Reduction to First-Order Logic without Equality}

A common theme of proofs in theoretical computer science is to avoid the tedious effort of proving the result from first principles by reducing the problem to one that is easier to solve.
In this instance, we are able to give a reduction for finding interpolants in first-order logic \emph{with} equality to first-order logic \emph{without} equality, where it is simpler to give an appropriate algorithm.

The general layout of this approach is the following:
From two sets $\Gamma$ and $\Delta$, where $\Gamma \cup \Delta$ is unsatisfiable, we compute two sets $\Gamma'$ and $\Delta'$ which do not make use of equality but simulate the effects of equality in $\Gamma$ and $\Delta$ via axioms.
In the process of this transformation, also function symbols are replaced by predicate symbols with appropriate axioms to make sure that the behaviour of these function-representing predicates is compatible to the one of actual functions.
Now an interpolant of $\Gamma'$ and $\Delta'$ can be derived using an algorithm that is only capable of handling predicate symbols as all other non-logical symbols have been removed.
Since the additional axioms ensure that the newly added predicate symbols mimic equality and functions respectively, we will see that the occurrences of these predicates in the interpolant can be translated back to occurrences of equality and function symbols in first-order logic with equality in the language of $\Gamma$ and $\Delta$, thereby yielding the originally desired interpolant.


\section{Translation of formulas}

As we shall see in this section, first-order formulas with equality can be transformed into first-order formulas without equality in a way that is satisfiability-preserving, which is sufficient for our purposes.

In order to simplify notation, we shall consider constant symbols to be function symbols of arity $0$ in this section.

First, we define the axioms which allow for simulation of equality and functions in first order logic without equality and function symbols:

\begin{defi}[Equality and function axioms]
	For a first-order language $\LangSym$ and fresh predicate symbols $E$ and $F_f$ for $f\in \FS(\LangSym)$, we define:\nopagebreak
	\begin{align*}
		\FAX(\LangSym) \defeq{}& \smashoperator{\bigwedge_{f \in \FS(\LangSym)}}  \forall \bar x \exists y (F_f(\bar x, y) \land (\forall z (F_f(\bar x, z) \limpl E(y, z)))) \\
%		\EAX(A) \defeq{} & \forall x \; x=x~\land  \\
%																		 & \begin{aligned} \bigwedge_{\substack{P \in\PS(A)\cup\\ \{F_i\mid f_i\in \FS(A)\}}} &\forall x_1 \ldots \forall x_{\ar(P)} \forall y_1 \ldots \forall y_{\ar(P)} \\
%																 & (( x_1~=~y_1 \land \ldots \land x_{\ar(P)} = y_{\ar(P)}) \limpl  \\
%																 &  (P(x_1, \ldots, x_{\ar(P)}) \Lra P(y_1, \ldots, y_{\ar(P)}) ) ) \end{aligned} 
		\Refl(P) \defeq{}& \forall x P(x,x)  \\ 
		\Congr(P) \defeq{}& 
%\forall x_1 \ldots \forall x_{\ar(P)} \forall y_1 \ldots \forall y_{\ar(P)} \\
\forall x_1 \forall y_1 \ldots \forall x_{\ar(P)} \forall y_{\ar(P)} 
(( E(x_1,y_1) \land \ldots \land E(x_{\ar(P)},  y_{\ar(P)})) \limpl  \\
 & (P(x_1, \ldots, x_{\ar(P)}) \limpl P(y_1, \ldots, y_{\ar(P)}) )) \\
		\EAX(\LangSym) \defeq{}& \Refl(E) \land \;\smashoperator{\bigwedge_{\substack{P \in\PS(\LangSym)\cup\{E\}\cup\\ \{F_f\mid f\in \FS(\LangSym)\}}}}\; \Congr(P)
	%	\EAX(A) \defeq{} & \forall x E(x,x)\land  \\
	%																	 & \begin{aligned} \bigwedge_{\substack{P \in\PS(A)\cup\{E\}\cup\\ \{F_f\mid f\in \FS(A)\}}} &\forall x_1 \ldots \forall x_{\ar(P)} \forall y_1 \ldots \forall y_{\ar(P)} \\
	%																	 & (( E(x_1,y_1) \land \ldots \land E(x_{\ar(P)},  y_{\ar(P)})) \limpl  \\
	%															 &  (P(x_1, \ldots, x_{\ar(P)}) \Lra P(y_1, \ldots, y_{\ar(P)}) ) )
	%\end{aligned} 
\qedhere
\end{align*}
%For sets of first-order formulas $\Phi$ and $h \in \{\FAX, \EAX\}$, $h(\Phi) \defeq \bigcup_{A\in \Phi} h(A)$ .
%$\FAX$ and $\EAX$ generalise to sets of formulas by elementwise application.
\end{defi}

$\Refl(P)$ will be referred to as reflexivity axiom of $P$, $\Congr(P)$ as congruence axiom of $P$.
Now we define the precise language we translate formulas to as well as the translation procedure.

\begin{defi}[Translation of languages]
	Let $\LangSym$ be a language.
	Then $\Trans(\LangSym)$ denotes $(\Lang(\LangSym)\cup \{E\}\cup\{F_f\mid f \in \FS(\LangSym)\}) \setminus(\{=\nolinebreak \} \cup\nolinebreak \FS(\LangSym))$.
\end{defi}

\begin{defi}[Translation and inverse translation of formulas]
	\label{def:trans}
	Let $A$ be a first-order formula and $E$ and $F_f$ for $f \in \FS(A)$ be fresh predicate symbols.
	Then $\Trans(A)$ is the result of applying the following algorithm to $A$:

	\begin{compactenum}
	\item Replace every occurrence of $s=t$ in $A$ by $E(s, t)$
	\label{def:trans_step1}
	\item As long as there is an occurrence of a function symbol $f$ in $A$:
	\label{def:trans_step2}

		Let $B$ be the atom in which $f$ occurs as outermost symbol of a term.
		Then $B$ is of the form $P(s_1, \ldots, s_{j-1}, f(\bar t),\allowbreak s_{j+1}, \ldots s_m)$.
		Replace $B$ in $A$ by $\exists y (F_f(\bar t, y) \land P(s_1, \ldots, s_{j-1}, y, s_{j+1}, \ldots s_m))$ for a fresh variable $y$.
	\end{compactenum}
	\medskip

	Moreover, let the inverse operation $\TransInv(B)$ for formulas $B$ in the language $\Trans(L(A))$ be defined as the result of applying the following algorithm to $B$:
	\begin{compactenum}
	\item Replace every occurrence of $E(s, t)$ in $B$ by $s=t$.
	\item For every $f \in FS(A)$, replace every occurrence of 
		$\exists y (F_f(\bar t, y) \land P(s_1, \ldots, s_{j-1},\allowbreak y,\allowbreak s_{j+1}, \ldots s_m))$
		in $B$ by $P(s_1, \ldots, s_{j-1},\allowbreak f(\bar t),\allowbreak s_{j+1}, \ldots s_m)$ and every remaining occurrence of $F_f(\bar t, s)$ by $f(\bar t) = s$.
	\end{compactenum}

	For sets of first-order formulas $\Phi$, let $\Trans(\Phi) \defeq \bigcup_{A\in\Phi} \Trans(A)$ and 
$\TransInv(\Phi) \defeq \bigcup_{A\in\Phi} \TransInv(A)$.
\end{defi}

\begin{lemma}
	\label{lemma:tinv}
	Let $A$ be a first-order formula and $\Phi$ be a set of first-order formulas.
	Then 
	$\TransInv(\Trans(A)) = A$
	and
	$\TransInv(\Trans(\Phi)) = \Phi$
	.
\end{lemma}
\begin{proof}
	Step 1 and 2 in the transformation algorithm for $\Trans$ and $\TransInv$ are each concerned with a different set of symbols and therefore do not interfere with each other.
	Moreover, the respective steps in both algorithms are the inverse of each other.
	For step 1, this is immediate and for step 2, consider that all occurrences of $F_f$ for $f \in \FS(A)$ in $\Trans(A)$ have been introduced by $\Trans$ and are consequently of the form
	$\exists y (F_f(\bar t, y) \land P(s_1, \ldots, s_{j-1}, y, s_{j+1}, \ldots s_m))$, which is replaced by 
$P(s_1, \ldots, s_{j-1},\allowbreak f(\bar t),\allowbreak s_{j+1}, \ldots s_m)$ by~$\TransInv$.
\end{proof}

\begin{defi}[Translation of formulas including axioms]
	For first-order formulas $A$, let $\TransAll(A) = \FAX(\Lang(A)) \land \EAX(\Lang(A)) \land \Trans(A)$ and for sets of first-order formulas $\Phi$, let $\TransAll(\Phi) = \{\FAX(\Lang(\Phi)), \EAX(\Lang(\Phi))\} \cup \Trans(\Phi)$.
\end{defi}


Note that $\TransAll(A)$ contains neither the equality predicate nor function symbols but additional predicate symbols instead. More formally:

%\begin{defi}[continues=exa:cont]
%	Let $\LangSym$ be a first-order language. 
%	Then $\Trans(\LangSym)$ denotes $(\LangSym\cup \{E\}\cup\{F_f\mid f \in \FS(\LangSym)\}) \setminus(\{=\nolinebreak \} \cup\nolinebreak \FS(\LangSym))$.
%\end{defi}


\begin{lemma}~ 
	\label{lemma:trans_lang}
	\begin{compactenum}
	\item
		Let $\Phi$ be a set of first-order formulas. Then $\TransAll(\Phi)$ is in the language~$\Trans(\Lang(\Phi))$.
		\label{lemma:trans_lang1}

	\item 
		If $\Psi$ is in the language $\Trans(\LangSym)$, then $\TransInv(\Psi)$ is in the language~$\LangSym$.
		\label{lemma:trans_lang2}
	\end{compactenum}
\end{lemma}

\begin{prop}
	\label{prop:trans_sat_equiv}
	Let $\Phi$ be a set of first-order formulas.
	\begin{compactenum}
		\item If $\Phi$ is satisfiable, then so is $\TransAll(\Phi)$.
			\label{prop:trans_sat_equiv1}
		\item Let $\LangSym$ be a first-order language and $\Phi$ a set of first-order formulas in the language~$\Trans(\LangSym)$.
			If $\{\FAX(\LangSym), \EAX(\LangSym)\} \cup \Phi $ is satisfiable, then so is $\TransInv(\Phi)$.
			\label{prop:trans_sat_equiv2}
	\end{compactenum}
\end{prop}
\begin{proof}
	Suppose $\Phi$ is satisfiable.
	Let $M$ be a model of $\Phi$.
	We show that $\TransAll(\Phi)$ is satisfiable by extending $M$ to the language $\Lang(\Phi)\cup\{E\}\cup\{F_f\mid f \in \FS(A)\}$ and proving that the extended model satisfies $\TransAll(\Phi)$.

	First, let $M \entails E(s, t)$ if and only if $M \entails s = t$.
	By reflexivity of equality, it follows that $M \entails \Refl(E)$.
	As any predicate, in particular $E$ and $F_f$ for every $f \in \FS(\Phi)$, satisfy the congruence axiom with respect to $=$, by the definition of $E$ in $M$, they satisfy the congruence axiom with respect to $E$.
	Therefore $M$ is a model of $\EAX(\Lang(\Phi))$.

	Second, let $M \entails F_f(\bar x, y)$ if and only if $M \entails f(\bar x) = y$ for all $f \in \FS(\Phi)$. 
	Since $M$ is a model of $\Phi$, it maps every function symbol $f$ to a function, which by definition returns a unique result for every combination of parameters.
	This however is precisely the logical requirement on $F_f$ stated by $\FAX(\Lang(\Phi))$,   
	hence $M$ is a model of $\FAX(\Lang(\Phi))$.

	Lastly, we show that $M \entails \Trans(A)$ for all $A \in \Phi$.
	By the above definition of $E$ in $M$, step $\ref{def:trans_step1}$ of the algorithm in definition \ref{def:trans} yields a formula that is satisfied by $M$ as it satisfies every formula of $\Phi$.
	For step \ref{def:trans_step2}, suppose $P(s_1, \ldots, s_{j-1}, f(\bar t),\allowbreak s_{j+1}, \ldots s_m)$ does (not) hold under $M$.
	Let $y$ such that $M \entails f(\bar t)=y$.
	By our definition of $F$ under $M$, $M\entails F(\bar t, y)$ with this unique $y$.
	Hence $\exists y (F(\bar t, y) \land P(s_1, \ldots, s_{j-1}, y, \allowbreak s_{j+1}, \ldots s_m))$ does (not) hold under $M$.


	For the other direction, suppose $\{\FAX(\LangSym), \EAX(\LangSym)\} \cup \Phi$ is satisfiable.
	We extend a model $M$ of this set of formulas to a model of $\TransInv(\Phi)$ by extending it from the language $\Trans(\LangSym)$ to include $\{=\}$ and $\FS(\LangSym)$.

	First, let $M\entails s = t$ if and only if $M\entails E(s, t)$.
	As $M$ is a model of $\EAX(\LangSym)$, $E$ is reflexive. 
	Since $M \entails \Congr(E)$,
	$M \entails \forall x \forall y (E(x, y) \land E(x, x)) \limpl ( E(x, x) \limpl\nolinebreak E(y, x))$.
	As we know that $E$ is reflexive, this simplifies to
	$M \entails \forall x \forall y (E(x, y)\limpl\nolinebreak E(y, x))$, i.e.~$E$ is symmetric in $M$.
	We show the transitivity of $E$ by another instance of $\Congr(E)$: 
	$M \entails \forall x \forall y \forall z ((E(y, x) \land E(y, z)) \limpl ( E(y, y) \limpl\nolinebreak E(x, z)))$,
	As $E$ is reflexive and symmetric, we get that 
	$M \entails \forall x \forall y \forall z ((E(x, y) \land E(y, z)) \limpl\nolinebreak E(x, z))$.
	As these properties directly also apply to $=$ in $M$, equality is adheres to the required axioms in $M$.

	Second, let $M\entails f(\bar t) = s$ if and only if $M\entails F_f(\bar t, s)$ for all $f \in \FS(\LangSym)$.
	As by assumption $M$ is a model of $\FAX(A)$, we know that for every $\bar t$, some $s$ with $M\entails F(\bar t, s)$ exists and is uniquely defined.
	Hence $f$ in $M$ refers to a well-defined function.

	Lastly, to show that $M \entails \TransInv(\Phi)$, 
	consider that the interpretations of the predicates $E$ and $=$ coincide in $M$.
	Furthermore, let $B$ be an occurrence of $\exists y (F_f(\bar t, y) \land P(s_1, \ldots, s_{j-1}, y, s_{j+1}, \ldots s_m))$ for some $f \in \FS(\LangSym)$ in $\Phi$.
	Then by the above definition of $f$ in $M$, we have that $B$ is in $M$ equivalent to $\exists y f(\bar t) = y) \land P(s_1, \ldots, s_{j-1}, y, s_{j+1}, \ldots s_m))$, which due to $f$ being a function is equivalent to 
	$M \entails P(s_1, \ldots, s_{j-1}, f(\bar t), s_{j+1}, \ldots s_m))$.

	Similarly, let $B$ be an occurrence of $F_f(\bar t, s)$ in $\Phi$.
	Then by our above definition of $f$ in $M$, we have that $M \entails f(\bar t) = s$ iff $M \entails B$.
\end{proof}

\begin{corr}
	Let $\Phi$ be a set of first-order formulas.
	Then $\Phi$ is satisfiable if and only if $\TransAll(\Phi)$ is satisfiable.
\end{corr}
\begin{proof}
	The left-to-right direction is directly given in Proposition \ref{prop:trans_sat_equiv}.
	For the other direction, consider that by Proposition \ref{prop:trans_sat_equiv}, $\TransInv(\Trans(\Phi))$ is satisfiable, which by Lemma \ref{lemma:tinv} is nothing else than $\Phi$.
\end{proof}



\section{Computation of interpolants in first-order logic without equality and function symbols}


\begin{thm}
	\label{thm:prop_interpol}
	Let $\Gamma$ and $\Delta$ be sets of first-order clauses without equality and function symbols such that $\Gamma \cup \Delta$ is unsatisfiable. Then there is an interpolant of $\Gamma$ and $\Delta$.
\end{thm}
\begin{proof}
	By the Compactness Theorem, there are finite $\Gamma' \subseteq \Gamma$ and $\Delta' \subseteq \Delta$ such that $\Gamma' \cup \Delta'$ is unsatisfiable.
	Borrowing an idea of the proof of \ref{prop:interpolations_equivalent},
	we show that there is an interpolant for $\bigwedge_{A\in \Gamma'}A \entails \bigvee_{B \in \Delta'} \lnot B$.
	Let $C = \bigwedge_{A \in \Gamma'} A$ and $D = \bigvee_{B \in \Delta'} \lnot B$.

	By the completness of cut-free sequent calculus, there is a proof of $C \vdash D$. By Lemma \ref{lemma:maehara}, there is an interpolant of $C$ and $D$, which also is an interpolant of $\Gamma$~and~$\Delta$.
\end{proof}

\begin{lemma}[Maehara]
	\label{lemma:maehara}
	If $C \vdash D$ is provable in cut-free sequent calculus,
	then there is an interpolant  $C$ and $D$.
\end{lemma}
\begin{proof}
	We prove this lemma by induction on the number of inferences.
	As many cases are similar, we prove some examples only.

	\begin{description}
		\item[\normalfont Base case.] Suppose no rules were applied. Then $C \vdash D$ is either of the form $A \vdash A$, in which case $A$ serves as interpolant, or of the form $\vdash t=t$, in which case $\top$ is an interpolant.


		\item[\normalfont Structural rules.]
			Suppose the property holds for $n$ rule applications and the $(n+1)$th rule is a structural one.
			We consider some examples:

			\begin{itemize}
				\item Contraction left
					\begin{prooftree}
						\Axiomm{\Gamma, A \fCenter \Delta}
						\RightLabelm{\lkrule{c}{l}}
						\UnaryInfm{\Gamma, A, A \fCenter \Delta}
					\end{prooftree}


					By the induction hypothesis, there exists an interpolant $I$ of $\Gamma, A \vdash \Delta$. Clearly $I$ is an interpolant for $\Gamma, A, A \vdash \Delta $ as well.

				\item Weakening right
					\begin{prooftree}
						\Axiomm{\Gamma \fCenter \Delta}
						\RightLabelm{\lkrule{w}{r}}
						\UnaryInfm{\Gamma \fCenter \Delta, A}
					\end{prooftree}

					By the induction hypothesis, there exists an interpolant $I$ of $\Gamma \vdash \Delta$.
					Clearly $I$ is also an interpolant for $\Gamma A \vdash \Delta $ as well.

			\end{itemize}

		\item[\normalfont Propositional rules.]
			Suppose the property holds for $n-1$ rule applications and the $n$th rule is a propositional one.
			We consider some examples:

			\begin{itemize}
				\item Negation left
					\begin{prooftree}
						\Axiomm{\Gamma \fCenter \Delta,  A}
						\RightLabelm{\lkrule{\lnot}{l}}
						\UnaryInfm{\Gamma, \lnot A \fCenter \Delta }
					\end{prooftree}

					By the induction hypothesis, there exists an interpolant $I$ of $\Gamma \vdash \Delta, A$.

					\todo[inline]{Need full maehara lemma, not just this restricted form, to have "global coloring"?}


			\end{itemize}

	\end{description}
\end{proof}




\section{Conclusion}

Using the results of the previous sections, we can now give a proof of the interpolation theorem:

\begin{proof}[Proof of Theorem \ref{thm:interpolation} (Interpolation)]

	Since $\Gamma \cup \Delta$ is unsatisfiable,
	by Proposition \ref{prop:trans_sat_equiv}, $\TransAll(\Gamma \cup \Delta)$ is unsatisfiable.
	\begin{align*}
		\TransAll(\Gamma \cup \Delta)\,\liff~&\{\FAX(\Lang(\Gamma\cup\Delta)), \EAX(\Lang(\Gamma \cup\Delta))\} \cup \Trans(\Gamma \cup \Delta) \\
		\liff~&\{\FAX(\Lang(\Gamma)\cup\Lang(\Delta)), \EAX(\Lang(\Gamma)\cup\Lang(\Delta))\} \cup \Trans(\Gamma )\cup \Trans(\Delta) \\
		\liff~&\{\FAX(\Lang(\Gamma)) \land \FAX(\Lang(\Delta)), \EAX(\Lang(\Gamma)) \land \EAX(\Lang(\Delta))\} \cup \Trans(\Gamma) \cup \Trans(\Delta) \\
		\liff~&\{\FAX(\Lang(\Gamma)),\EAX(\Lang(\Gamma))\} \cup \Trans(\Gamma) \cup \{\FAX(\Lang(\Delta)), \EAX(\Lang(\Delta))\} \cup \Trans(\Delta) \\
		\liff~&\TransAll(\Gamma) \cup \TransAll(\Delta)
	\end{align*}
	%It follows from Lemma \ref{lemma:trans_transform}, that $\TransAll(\Gamma) \cup \TransAll(\Delta)$is unsatisfiable as well.
	Hence  $\TransAll(\Gamma) \cup \TransAll(\Delta)$ is unsatisfiable as well.
	By Lemma \refsub{lemma:trans_lang}{lemma:trans_lang1} $\TransAll(\Gamma) \cup \TransAll(\Delta)$ contains neither function symbols nor the equality symbol.
	Hence by Theorem \ref{thm:prop_interpol}, there is an interpolant $I$ such that
	\begin{enumerate}
		\item $\TransAll(\Gamma) \entails I$
		\item $\TransAll(\Delta) \entails \lnot I$ 
		\item $\Lang(I) \subseteq \Lang(\TransAll(\Gamma)) \cap \Lang(\TransAll(\Delta))$
			\label{proof:interpolation1_3}
	\end{enumerate}

	We now show that $\TransInv(I)$ is an interpolant of $\Gamma$ and $\Delta$.

	$\TransAll(\Gamma) \entails I$ is equivalent to $\TransAll(\Gamma) \cup \{\lnot I\}$ being unsatisfiable.
	Through the unfolding of $\TransAll(\Gamma)$, we get that 
	$\{\FAX(\Lang(\Gamma)), \EAX(\Lang(\Gamma))\} \cup \Trans(\Gamma) \cup \{\lnot I\}$ is unsatisfiable.

	Since $\Lang(\lnot I) \subseteq \Lang(\TransAll(\Gamma))$, we can apply Proposition
	\refsub{prop:trans_sat_equiv}{prop:trans_sat_equiv2}
	by considering $\Trans(\Gamma) \cup \{\lnot I\}$ as $\Phi$ to conclude that $\TransInv(\Trans(\Gamma) \cup \{\lnot I\})$ is unsatisfiable. By pulling $\TransInv$ inward and an application of Lemma \ref{lemma:tinv}, we get that $\Gamma \cup \{\TransInv(\lnot I)\} = \Gamma \cup \{\lnot \TransInv(I)\}$ is unsatisfiable. 
	Therefore $\Gamma \entails \Trans^{-1}(I)$.

	For $\Delta$, an analoguous argument goes through and so from $\TransAll(\Gamma) \entails \lnot I$ we can deduce that $\Delta \entails \lnot \Trans^{-1}(I)$.

	By item \ref{proof:interpolation1_3}, $I$ is in the language $\Lang(\TransAll(\Gamma)) \cap \Lang(\TransAll(\Delta))$, which by Lemma \refsub{lemma:trans_lang}{lemma:trans_lang1} is $\Trans(\Lang(\Gamma)) \cap\nolinebreak \Trans(\Lang(\Delta)) $. 
	\begin{align*}
		\Trans(\Lang(\Gamma)) \cap\nolinebreak \Trans(\Lang(\Delta)) =\,&
		(\Lang(\Gamma)\cup \{E\}\cup\{F_f\mid f \in \FS(\Gamma)\}) \setminus(\{=\nolinebreak \} \cup\nolinebreak \FS(\Gamma))~\cap \\
		& (\Lang(\Delta)\cup \{E\}\cup\{F_f\mid f \in \FS(\Delta)\}) \setminus(\{=\nolinebreak \} \cup\nolinebreak \FS(\Delta))\\
		=\,& ((\Lang(\Gamma) \cap \Lang(\Delta)) \cup \{E\} \cup\{F_f\mid f \in \FS(\Gamma)\cap\FS(\Delta\}))  \setminus(\{=\} \cup \FS(\Gamma) \cup \FS(\Delta)) \\
		=\,& ((\Lang(\Gamma) \cap \Lang(\Delta)) \cup \{E\} \cup \{F_f\mid f\in \FS( \Lang(\Gamma) \cap \Lang(\Delta)) ) \setminus ( \{=\} \cup \FS(\Lang(\Gamma) \cap \Lang(\Delta)) )  \\ 
		=\,& \Trans(\Lang(\Gamma) \cap \Lang(\Delta))
	\end{align*}

	As $I$ is in the language $\Trans(\Lang(\Gamma) \cap \Lang(\Delta))$, by Lemma \refsub{lemma:trans_lang}{lemma:trans_lang2}, $\TransInv(I)$ is in the language $\Lang(\Gamma) \cap \Lang(\Delta)$.
\end{proof}


