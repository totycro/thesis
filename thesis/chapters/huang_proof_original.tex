
\chapter{Interpolant extraction from resolution proofs due to Huang}  
\label{sec:huang}
\label{chap:huang}

This section essentially presents the original proof of \cite{Huang95} in a modern format.
It forms the base for our work in chapter~\ref{sec:two_phases} and \ref{sec:one_phase}, and we refer to these chapters for lemmas and definitions which also apply here.
Section~\ref{sec:huang_commentary} features a commentary on the original publication. 

\section{Propositional interpolants}


Let $\Gamma \cup \Delta$ be unsatisfiable and $\pi$ be a proof of the empty clause from $\Gamma \cup \Delta$. Then $\PI$ is a function that returns a interpolant with respect to the current clause. 

\begin{defi}[Propositional interpolant]
	Let $\pi$ be a resolution refutation of $\Gamma \cup \Delta$.
	A formula $A$ is a \defiemph{propositional interpolant} if
	\label{def:huang_orig_rel_prop_interpol}
	\begin{enumerate}
		\item $\Gamma \entails A$
			\label{huang_orig_rel_prop_interpol_cond1}
		\item $\Delta \entails \lnot A$
			\label{huang_orig_rel_prop_interpol_cond2}
		\item $\Pred(A) \subseteq (\Pred(\Gamma) \intersect \Pred(\Delta)) \cup \{\top, \bot\} $.
			\label{huang_orig_rel_prop_interpol_cond_lang}
	\end{enumerate}


	For a clause $C$ in $\pi$, a formula $A_C$ is a \defiemph{propositional interpolant relative to $C$} if
	\begin{enumerate}
		\item $\Gamma \entails A_C \lor C$
			%\label{huang_orig_rel_prop_interpol_cond1}
		\item $\Delta \entails \lnot A_C \lor C$
			%\label{huang_orig_rel_prop_interpol_cond2}
		\item $\Pred(A_C) \subseteq (\Pred(\Gamma) \intersect \Pred(\Delta)) \cup \{\top, \bot\} $.
			%\label{huang_orig_rel_prop_interpol_cond_lang}
	\end{enumerate}

	The propositional interpolant for the empty clause derived in $\pi$ is denoted by $\PI(\pi)$.\qedhere
\end{defi}

The third condition of a propositional interpolant will sometimes be referred to as \emph{language restriction}.
It is easy to see that the propositional interpolant relative to the empty clause of a resolution refutation is a propositional interpolant.

We refer to Definition~\ref{def:PI} for the definition of $\PI$.

\begin{prop}
	\label{prop:prop_interpol}
	Let $C$ be a clause of a resolution refutation of $\Gamma \cup \Delta$.
	Then $\PI(C)$ is a propositional interpolant with respect to $C$. 
\end{prop}
\begin{proof}
	Proof by induction on the number of rule applications including the following strengthenings:
	$\Gamma \entails \PI(C) \lor C_\Gamma$ and
	$\Delta \entails \lnot \PI(C) \lor C_\Delta$, where $D_\Phi$ denotes the clause D with only the literals which are contained in $\Lang(\Phi)$. They clearly imply conditions \ref{huang_orig_rel_prop_interpol_cond1} and \ref{huang_orig_rel_prop_interpol_cond2} of definition \ref{def:huang_orig_rel_prop_interpol}. 

	\begin{indproof}
		\indproofitem{Base case}
			Suppose no rules were applied. We distinguish two possible cases:
			\begin{enumerate}
				\item $C \in \Gamma$.
					Then $\PI(C) = \bot$. Clearly $\Gamma \entails \bot \lor C_\Gamma$ as $C_\Gamma = C \in \Gamma$, $\Delta \entails \lnot \bot \lor C_\Delta$ and $\bot$ satisfies the restriction on the language.

				\item $C \in \Delta$.
					Then $\PI(C) = \top$. Clearly $\Gamma \entails \top \lor C_\Gamma$, $\Delta \entails \lnot \top \lor C_\Delta$ as $C_\Delta = C \in \Delta$ and $\top$ satisfies the restriction on the language.
			\end{enumerate}

			Suppose the property holds for $n$ rule applications.
			We show that it holds for $n+1$ applications by considering the last one:

		\indproofitem{Resolution}
			Suppose the last rule application is an instance of resolution. Then it is of the form:
			\begin{prooftree}
				\AxiomCm{C_1: D \lor l}
				\AxiomCm{C_2: E \lor \lnot l'}
				\RightLabelm{\quad l\sigma = l'\sigma}
				\BinaryInfCm{C: (D\lor E)\sigma}
			\end{prooftree}

			By the induction hypothesis, we can assume that:

			$\Gamma \entails \PI(C_1) \lor (D\lor l)_\Gamma$

			$\Delta \entails \lnot \PI(C_1) \lor (D\lor l)_\Delta$

			$\Gamma \entails \PI(C_2) \lor (E\lor \lnot l')_\Gamma$

			$\Delta \entails \lnot \PI(C_2) \lor (E\lor \lnot l')_\Delta$

			We consider the respective cases from definition \ref{def:PI_resolution}:

			\begin{enumerate}
				\item $l$ is $\Gamma$-colored.
					\label{huang_proof_prop_case_1}
					Then $\PI(C) = [\PI(C_1) \lor \PI(C_2)]\sigma$. 

					As $\Pred(l) \in \Lang(\Gamma)$,
					$\Gamma \entails (\PI(C_1) \lor D_\Gamma\lor l)\sigma$
					as well as $\Gamma \entails (\PI(C_2) \lor E_\Gamma\lor \lnot l')\sigma$.
					By a resolution step, we get $\Gamma \entails (\PI(C_1) \lor \PI(C_2))\sigma \lor ((D \lor E)\sigma)_\Gamma$.

					Furthermore, as $\Pred(l) \not\in \Lang(\Delta)$, 
					$\Delta \entails (\lnot\PI(C_1) \lor D_\Delta)\sigma$
					as well as $\Delta \entails (\lnot\PI(C_2) \lor E_\Delta)\sigma$.
					Hence it certainly holds that $\Delta \entails (\lnot \PI(C_1) \lor \lnot\PI(C_2))\sigma \lor (D \lor E)\sigma_\Delta$.

					The language restriction clearly remains satisfied as no non-logical symbols are added.

				\item $l$ is $\Delta$-colored.
					\label{huang_proof_prop_case_2}
					Then $\PI(C) = [\PI(C_1) \land \PI(C_2)]\sigma$. 

					As $\Pred(l) \not\in \Lang(\Gamma)$,
					$\Gamma \entails (\PI(C_1) \lor D_\Gamma)\sigma$
					as well as $\Gamma \entails (\PI(C_2) \lor E_\Gamma)\sigma$.
					Suppose that in a model $M$ of $\Gamma$, $M \notentails D_\Gamma$ and $M \notentails E_\Gamma$. Then $M \entails \PI(C_1) \land \PI(C_2)$.
					Hence 
					$\Gamma \entails (\PI(C_1) \land \PI(C_2))\sigma \lor ((D \lor E)\sigma)_\Gamma$.

					Furthermore due to $\Pred(l) \in \Lang(\Delta)$,
					$\Delta \entails (\lnot\PI(C_1) \lor D_\Delta \lor l)\sigma$
					as well as $\Delta \entails (\lnot\PI(C_2) \lor E_\Delta \lor \lnot l')\sigma$.
					By a resolution step, we get $\Delta \entails (\lnot\PI(C_1) \lor \lnot\PI(C_2))\sigma \lor (D_\Delta \lor E_\Delta)\sigma $
					and hence 
					$\Delta \entails \lnot (\PI(C_1) \land \PI(C_2))\sigma \lor (D_\Delta \lor E_\Delta)\sigma $.

					The language restriction again remains intact.

				\item $l$ is gray.
					Then $\PI(C) = [(l \land \PI(C_2)) \lor (\lnot l' \land \PI(C_1)) ]\sigma $

					First, we have to show that 
					$ \Gamma \entails [(l \land \PI(C_2)) \lor (l' \land \PI(C_1)) ]\sigma \lor ((D \lor E)\sigma)_\Gamma$.
					Suppose that in a model $M$ of $\Gamma$, $M \notentails D_\Gamma$ and $\Gamma \notentails E$. Otherwise we are done.
					The induction assumption hence simplifies to $M \entails \PI(C_1) \lor l$ and $M \entails \PI(C_2) \lor \lnot l'$ respectively.
					As $l\sigma = l'\sigma$, by a case distinction argument on the truth value of $l\sigma$, we get that either $M \entails (l \land \PI(C_2))\sigma$ or $M \entails  (\lnot l' \land \PI(C_1))\sigma$.


					Second, we show that 
					$\Delta \entails ((l \lor \lnot \PI(C_1)) \land (\lnot l' \lor \lnot \PI(C_2)))\sigma \lor ((D \lor E)\sigma)_\Delta$.
					Suppose again that in a model $M$ of $\Delta$, $M \notentails D_\Delta$ and $\Gamma \notentails E_\Delta$. 
					Then the required statement follows from the induction hypothesis.

					The language condition remains satisfied as only the common literal $l$ is added to the interpolant.


			\end{enumerate}

			\indproofitem{Factorization}
			Suppose the last rule application is an instance of factorization. Then it is of the form:
			\begin{prooftree}
				\AxiomCm{C_1: l \lor l' \lor D}
				\RightLabelm{\quad \sigma = \mgu(l, l')}
				\UnaryInfCm{C: (l \lor D)\sigma}
			\end{prooftree}

			Then the propositional interpolant $\PI(C)$ is defined as $\PI(C_1)$.
			By the induction hypothesis, we have:

			$\Gamma \entails \PI(C_1) \lor (l \lor l' \lor D)_\Gamma$

			$\Delta \entails \PI(C_1) \lor (l \lor l' \lor D)_\Delta$

			It is easy to see that then also:

			$\Gamma \entails (\PI(C_1)\lor (l \lor D)_\Gamma)\sigma$

			$\Delta \entails (\PI(C_1)\sigma \lor (l \lor D)_\Delta)\sigma$

			The restriction on the language trivially remains intact.


		\indproofitem{Paramodulation}	
			Suppose the last rule application is an instance of paramodulation. Then it is of the form:
			\begin{prooftree}
				\AxiomCm{C_1: D \lor s=t}
				\AxiomCm{C_2: E\occurat{s}{p}}
				\RightLabel{$\quad \sigma = \mgu(s, r)$}
				\BinaryInfCm{C: D \lor E\occurat{t}{p}}
			\end{prooftree}


			By the induction hypothesis, we have:

			$\Gamma \entails \PI(C_1) \lor (D\lor s=t)_\Gamma$

			$\Delta \entails \lnot \PI(C_1) \lor (D\lor s=t)_\Delta$

			$\Gamma \entails \PI(C_2) \lor (E[r])_\Gamma$

			$\Delta \entails \lnot \PI(C_2) \lor (E[r])_\Delta$

			First, we show that $\PI(C)$ as constructed in case \ref{def:PI_paramod_3} of the definition is a propositional interpolant in any of these cases:

			$\PI(C) = (s=t \land \PI(C_2)) \lor (s\neq t \land \PI(C_1)) $

			Suppose that in a model $M$ of $\Gamma$, $M \notentails D\sigma$ and $M \notentails E\occurat{t}{p}\sigma$. Otherwise we are done.
			Furthermore, assume that $M \entails (s=t)\sigma$. Then $M \notentails E\occurat{r}{p}\sigma$, but then necessarily $M \entails \PI(C_2)\sigma$. \\
			On the other hand, suppose $M \entails (s\neq t)\sigma$. As also $M \notentails D\sigma$, $M \entails \PI(C_1)\sigma$.
			Consequently, $M \entails [(s=t \land \PI(C_2)) \lor (s\neq t \land \PI(C_1))]\sigma \lor [(D \lor E)_\Gamma]\sigma$

			By an analogous argument, we get $\Delta \entails [(s=t \land \lnot \PI(C_2)) \lor (s\neq t \land \lnot \PI(C_1))]\sigma \lor [(D \lor E)_\Delta]\sigma$,
			which implies
			$\Delta \entails [( s\neq t \lor \lnot \PI(C_2)) \land (s = t \lor \lnot \PI(C_1))]\sigma \lor ((D \lor E)_\Delta)\sigma $

			%By a similar case distinction for a model $M$ of $\Delta$ and assuming that $M \notentails D_\Delta$ and $M \notentails E_\Delta$, we get that if $M \entails (s=t)\sigma$, $M \entails \lnot P$, which implies

			The language restriction again remains satisfied as the only predicate, that is added to the interpolant, is $=$.

			This concludes the argumentation for case \ref{def:PI_paramod_3}. 

			The interpolant for case \ref{def:PI_paramod_1} differs only by an additional formula added via a disjunction and hence condition \ref{huang_orig_rel_prop_interpol_cond1} of definition \ref{def:huang_orig_rel_prop_interpol} holds by the above reasoning.
			As the adjoined formula is a contradiction, its negation is valid which in combination with the above reasoning establishes condition \ref{huang_orig_rel_prop_interpol_cond2}.
			Since no new predicated are added, the language condition remains intact. 

			The situation in case \ref{def:PI_paramod_2} is somewhat symmetric: 
			As a tautology is added to the interpolant with respect to case \ref{def:PI_paramod_1}, condition \ref{huang_orig_rel_prop_interpol_cond1} is satisfied by the above reasoning.
			For condition \ref{huang_orig_rel_prop_interpol_cond2}, consider that the negated interpolant for case \ref{def:PI_paramod_1} implies the negated interpolant for this case.
			The language condition again remains intact.
			\qedhere
	\end{indproof}
\end{proof}

\section{Propositional refutations}
Before we are able to specify a procedure to transform the propositional interpolant generated by $\PI$ into a proper interpolant without any colored terms,
we need to make some observations about tree refutations.

In a tree refutation where the input clauses have a disjoint sets of variables, every variable has a unique ancestor which traces back to an input clause and hence appears only along a certain path.
This insight allows us to push substitutions of the variables upwards along this path and arrive at the following definition and lemma:



%For every unification $\sigma$ in the deduction and for every variable $x$, either $x\sigma = x$ or $x\sigma = t$ where $x$ does not occur in $t$.
%Hence along the path from the input clause to its unification or removal by resolution or factorisation, it occurs unchanged.
%Therefore replacing $x$ along the path with $\sigma x$, where $\sigma$ is a non-trivial unifier used on $x$ in the derivation creates still a valid refutation of whatever.

\begin{defi}
	A resolution refutation is a \defiemph{propositional refutation} if no nontrivial substitutions are employed.
\end{defi}

\begin{lemma}
	Let $\Phi$ be unsatisfiable.
	Then there is a propositional refutation of $\Phi$ which starts from instances of $\Phi$.
\end{lemma}
\begin{proof}
	Let $\pi$ be a resolution refutation of $\Phi$.
	By Lemma \ref{lemma:bin_tree_deduction}, we can assume without loss of generality that $\pi$ is a tree refutation where the sets of variables of the input clauses are disjoint.
	Furthermore, we can assume that only most general unifiers are employed in $\pi$.

	Then any unifier in $\pi$ is either trivial on $x$ or there is one unique unifier $\sigma$ in $\pi$ with $x\sigma = t$ where $x$ does not occur in $t$.
	Hence along the path through the deduction where $x$ occurs, it remains unchanged.
	Therefore we can create a new resolution refutation $\pi'$ from $\pi$ where $x$ is replaced by $t$.
	Clearly $\pi'$ is rooted in instances of $\Phi$.

	By application of this procedure to all variable occurring in $\pi$, we obtain a desired resolution refutation.
\end{proof}

Even though propositional refutations have nice properties for theoretical analysis, their use in practice is not desired as its construction involves a considerable blowup of the refutation. 
But its use is still justified in this instance as we can show for arbitrary refutations $\pi$
that the algorithm stated in \ref{def:PI} gives closely related results for both $\pi$ and its corresponding propositional refutation.

\begin{lemma}
	Let $\pi$ be a resolution refutation of $\Phi$ and $\pi'$ a propositional refutation corresponding to $\pi$.
	Then for every clause $C$ in $\pi$ and its corresponding clause $C'$ in $\pi'$, $\PI(C)\sigma = \PI(C')$, where $\sigma$ is the composition of the unifications of $\pi$ which are applied to the variables occurring in $C$ .
\end{lemma}
\begin{proof}
	For the construction of the propositional skeleton of $\PI(\cdot)$ only the coloring of the clauses is relevant and since this is the same in both $\pi$ and $\pi'$, it coincides for $\PI(C)$ and $\PI(C')$.

	Hence $\PI(C)$ and $\PI(C')$ differ only in their term structure. 
	To be more specific, in $\PI(C')$, the composition of substitutions that are applied in $\pi$ have already been applied to the initial clauses of $\pi'$. 
	Note that substitution commutes with the rules of resolution.
	Therefore the only difference between $\PI(C)$ and $\PI(C')$ is that at certain term positions, there are variables in $\PI(C)$ where in $\PI(C')$ by some substitution a different term is located. 
	But these substitutions are certainly applied by $\sigma$, hence $\PI(C)\sigma = \PI(C')$.
\end{proof}

\section{Lifting of colored symbols}

We rely on the same definition of lifting as given in \ref{sec:lifting}.
First, we consider the lifting of the $\Delta$-terms, which corresponds to Lemma~\ref{lemma:gamma_entails_lifted_interpolant}, but differs in the proof by relying on propositional refutations.\nopagebreak[4]

\begin{lemma}
	\label{lemma:gamma_entails_lifted_interpolantHuang}
	Let $\pi$ be a resolution refutation of $\Gamma \cup \Delta$. 
	Then $\Gamma \entails \lift{\Delta}{ \PI(C) \lor C }{x} $ for $C$ in $\pi$.
\end{lemma}
\begin{proof}
	We proof this result by induction on the number of rule applications in the propositional refutation corresponding to $\pi$. 
	Similar to the proof of \ref{prop:prop_interpol}, we show the strengthening:
	$\Gamma \entails \lift{\Delta}{ \PI(C) \lor C_\Gamma }{x} $ for $C$ in $\pi$.

	\begin{indproof}

			\newcommand{\lif}[1]{\lift{\Delta}{#1}{x}}
			\indproofitem{Base case}

			If no rules have been applied, $C$ is an instance of a clause of either $\Gamma$ or $\Delta$.
			In the former case, all $\Delta$-terms of $C$ were added by unification, hence by replacing them with variables, we obtain a clause $C'$ which still is an instance of $C$ and consequently is implied by $\Gamma$. 
			In the latter case, $\PI(C) = \top$. 

		\indproofitem{Resolution} Suppose the last rule application is an instance of resolution. Then it is of the form:
			\begin{prooftree}
				\AxiomCm{C_1: D\lor l}
				\AxiomCm{C_2: E\lor \lnot l}
				\BinaryInfCm{C: D \lor E}
			\end{prooftree}

			By the induction hypothesis,

			$\Gamma \entails \lift{\Delta}{ \PI(C_1) \lor (D \lor l)_\Gamma }{x}$ and

			$\Gamma \entails \lift{\Delta}{ \PI(C_2) \lor (E \lor \lnot l)_\Gamma }{x}$

			which by Lemma \ref{lemma:lift_logic_commute} is equivalent to

			$\Gamma \entails \lift{\Delta}{ \PI(C_1) }{x} \lor
			\lift{\Delta}{ D_\Gamma }{x} \lor
			\lift{\Delta}{ l_\Gamma }{x} \;\; {(\circ)} $
			and

			$\Gamma \entails \lift{\Delta}{ \PI(C_2) }{x} \lor
			\lift{\Delta}{ E_\Gamma }{x} \lor
			\lnot \lift{\Delta}{ l_\Gamma }{x} \;\; {(*)}$ .


			\begin{enumerate}
				\item Suppose $l$ is $\Gamma$-colored.
					Then $\PI(C) = \PI(C_1) \lor \PI(C_2)$.
					By using resolution of ${(*)}$ and ${(\circ)}$ on $\lift{\Delta}{l_\Gamma}{x}$, we get that 
					$$\Gamma \entails\lift{\Delta}{ \PI(C_1) }{x} \lor \lift{\Delta}{ \PI(C_2) }{x} \lor
					\lift{\Delta}{ D_\Gamma }{x} \lor
					\lift{\Delta}{ E_\Gamma }{x}.$$
					Several applications of Lemma \ref{lemma:lift_logic_commute} give
					$\Gamma \entails\lift{\Delta}{ \PI(C_1)  \lor  \PI(C_2) \lor (D \lor E)_\Gamma }{x}$.

				\item Suppose $l$ is $\Delta$-colored.
					Then $\PI(C) = \PI(C_1) \land \PI(C_2)$.

					As $l$ and $\lnot l$ are not contained in $\Lang(\Gamma)$, we get that 

					$\Gamma \entails \lift{\Delta}{ \PI(C_1) }{x} \lor
					\lift{\Delta}{ D_\Gamma }{x}$
					and

					$\Gamma \entails \lift{\Delta}{ \PI(C_2) }{x} \lor
					\lift{\Delta}{ E_\Gamma }{x}$.

					So if in a model $M$ of $\Gamma$ we have that
					$M \notentails \lift{\Delta}{ D_\Gamma }{x}$ and 
					$M \notentails \lift{\Delta}{ E_\Gamma }{x}$, it follows that $M \entails \lift{\Delta}{ \PI(C_1) }{x}$ and $M \entails \lift{\Delta}{ \PI(C_2) }{x}$. Hence by Lemma~\ref{lemma:lift_logic_commute}
					$M \entails \lift{\Delta}{ \PI(C_1) \land \PI(C_2) }{x} \lor
					\lift{\Delta}{ (D \lor E)_\Gamma }{x}$.

				\item Suppose $l$ is grey.
					Then $\PI(C) =  (l \land \PI(C_2)) \lor (\lnot l \land \PI(C_1))$.

					We show that 
					$\Gamma \entails \lift{\Delta}{(l \land \PI(C_2)) \lor (\lnot l \land \PI(C_1)) \lor (D \lor E)_\Gamma  }{x} $. 

					Suppose that for a model $M$ of $\Gamma$ that 
					$M \notentails \lift{\Delta}{ D_\Gamma }{x}$ and 
					$M \notentails \lift{\Delta}{ E_\Gamma }{x}$.
					Then by ${(\circ)}$
					and ${(*)}$, we get that\nopagebreak

					$M \entails \lift{\Delta}{ \PI(C_1) }{x} \lor
					\lift{\Delta}{ l_\Gamma }{x}$ as well as

					$M \entails \lift{\Delta}{ \PI(C_2) }{x} \lor
					\lnot \lift{\Delta}{ l_\Gamma }{x}$.

					So $M \entails \lift{\Delta}{ l_\Gamma }{x}$ implies that 
					$M \entails \lift{\Delta}{\PI(C_2)}{x}$ and 
					$M \entails \lnot \lift{\Delta}{ l_\Gamma }{x}$  implies that 
					$M \entails \lift{\Delta}{\PI(C_1)}{x}$ and 

					Therefore
					$M\entails (\lif{l} \land \lif{\PI(C_2)}) \lor (\lnot \lif{l} \land \lif{\PI(C_1)}) \lor (\lif{D_\Gamma} \lor \lif{E_\Gamma}) $,
					and several applications of Lemma \ref{lemma:lift_logic_commute} give
					$M\entails \lif{(l \land \PI(C_2)) \lor (\lnot {l} \land {\PI(C_1)}) \lor ({D_\Gamma} \lor {E_\Gamma})} $.
			\end{enumerate}


		\indproofitem{Factorisation} Suppose the last rule application is an instance of factorisation. Then it is of the form:
			\begin{prooftree}
				\AxiomCm{C_1: l \lor l \lor D}
				\UnaryInfCm{C: l \lor D}
			\end{prooftree}

			The propositional interpolant directly carried over from $C_1$, i.e.~$\PI(C) = \PI(C_1)$.

			By the induction hypothesis, we get that $\Gamma \entails \lif{\PI(C_1) \lor (l \lor l \lor D)_\Gamma}$.
			By Lemma \ref{lemma:lift_logic_commute}, 

			$\Gamma \entails \lif{\PI(C_1)} \lor (\lif{l_\Gamma} \lor  \lif{l_\Gamma} \lor \lif{D_\Gamma})$,

			which clearly is equivalent to

			$\Gamma \entails \lif{\PI(C_1)} \lor (\lif{l_\Gamma} \lor \lif{D_\Gamma})$,

			so by again applying Lemma \ref{lemma:lift_logic_commute}, we arrive at

			$\Gamma \entails \lif{\PI(C_1) \lor (l \lor D)_\Gamma}$.



		\indproofitem{Paramodulation} Suppose the last rule application is an instance of paramodulation. Then it is of the form:
			\begin{prooftree}
				\AxiomCm{C_1: D \lor s=t}
				\AxiomCm{C_2: E\occurat{s}{p}}
				\BinaryInfCm{C: D \lor E\occurat{t}{p}}
			\end{prooftree}

			By the induction hypothesis, we have that 

			$\Gamma \entails \lif{\PI(C_1) \lor (D \lor s=t)_\Gamma}$ and 

			$\Gamma \entails \lif{\PI(C_2) \lor (E\occurat{s}{p})_\Gamma}$.

			By Lemma \ref{lemma:lift_logic_commute}, we get that 

			$\Gamma \entails \lif{\PI(C_1)} \lor \lif{D_\Gamma} \lor \lif{s}=\lif{t}$ and 

			$\Gamma \entails \lif{\PI(C_2)} \lor \lif{(E\occurat{s}{p})_\Gamma}$.

			We distinguish two cases:\nopagebreak
			\begin{enumerate}
				\item Suppose $s$ does not occur in a maximal $\Delta$-term $h\occur{s}$ in $E\occurat{s}{p}$ which occurs more than once in $\PI(E(s)) \lor E\occurat{s}{p}$.

					We show that $\Gamma \entails \lif{ (s=t \land \PI(C_2)) \lor (s\neq t \land \PI(C_1)) \lor (D \lor E\occurat{t}{p})_\Gamma}$, which subsumes the cases \ref{def:PI_paramod_2} and \ref{def:PI_paramod_3} of 
					Definition \ref{def:PI_paramod}. By Lemma \ref{lemma:lift_logic_commute}, this is equivalent to

					$\Gamma \entails (\lif{s}=\lif{t} \land \lif{\PI(C_2)}) \lor (\lif s\neq \lif t \land \lif{\PI(C_1)}) \lor (\lif{D_\Gamma} \lor \lif{(E\occurat{t}{p})_\Gamma})$

					Suppose that $M$ is a model and $\alpha$ an assignment to the free variables 
					such that $M_\alpha \entails \Gamma$,
					$M_\alpha \notentails \lift{\Delta}{ D_\Gamma }{x}$ and 
					$M_\alpha \notentails \lift{\Delta}{ (E\occurat{t}{p})_\Gamma }{x}$.
					We show that then, depending on whether $\lif{s} = \lif{t}$ holds in $M_\alpha$, one of the first two disjuncts holds in $M_\alpha$.

					In case $M_\alpha \entails \lif{s} = \lif{t}$ we also get
					$M_\alpha \notentails \lift{\Delta}{ (E\occurat{s}{p})_\Gamma }{x}$ and consequently by the induction hypothesis $M_\alpha\entails \lif{\PI(C_2)}$.

					However in case $M_\alpha \entails \lif{s} \neq \lif{t}$ we get by the induction hypothesis that 
					$M\entails \lif{\PI(C_1)}$.

					\label{njktahjtkhltah}

				\item Otherwise $s$ occurs in a maximal $\Delta$-term $h\occur{s}$ in $E\occurat{s}{p}$ which occurs more than once in $\PI(E(s)) \lor E\occurat{s}{p}$.
				This reflects case \ref{def:PI_paramod_1} of Definition \ref{def:PI_paramod}.

					Then models are possible in which $s=t$ holds, while at the same time $\lif{h\occur{s}} \neq \lif{h\occur{t}}$ does not as $h\occur{s}$ and $h\occur{t}$ are replaced by distinct variables due to being different $\Delta$-terms.

					Therefore we amend the proof of case \ref{njktahjtkhltah} as follows:

					In case $M_\alpha \entails \lif{s} = \lif{t}$ (otherwise proceed as in case \ref{njktahjtkhltah}), 
					one of the following cases holds:

					\begin{itemize}
					\item $M_\alpha\entails \lif{h\occur{s}} = \lif{h\occur{t}}$. From this, it follows that as in the proof of case \ref{njktahjtkhltah}, $M \not\entails \lif{(E\occurat{s}{p})_\Gamma}$ and consequently $M \entails \lif{\PI(C_2)}$ again by the induction hypothesis.

					\item 
						$M_\alpha \entails \lif{h\occur{s}} \neq \lif{h\occur{t}}$.
						However as here $\PI(C)$ contains the with respect to case \ref{njktahjtkhltah} additional disjunct $s=t \land h\occur{s} \neq h\occur{t}$,
						$M_\alpha \entails \lif{PI(C)}$ due to $M_\alpha \entails \lif{s}=\lif{t} \land \lif{h\occur{s}} \neq \lif{h\occur{t}}.$
					\qedhere
					\end{itemize}
			\end{enumerate}

	\end{indproof}
\end{proof}

From this, we can directly proof the theorem by relying on the notion of symmetry already shown in Section~\ref{sec:symmetry}.

\begin{thm}
	Let $\pi$ be a resolution refutation of $\Gamma \cup \Delta$ and
	$t_1, \dots, t_n$ be the maximal colored terms in $\PI(\pi)$ in ascending order.
	Then
	$Q_1 z_{t_1} \ldots Q_n z_{t_n}\,\lifgamma{\lifdelta{\PI(\pi)}}$, where $Q_i$ is $\forall$ $(\exists)$ if $z_{t_i}$ replaces a $\Delta$ $(\Gamma)$-term, is an interpolant.
\end{thm}
\begin{proof}
	By Lemma \ref{lemma:gamma_entails_lifted_interpolantHuang}, $\Gamma \entails \forall x_{s_1} \ldots \forall x_{s_m}\,\lifdelta{\PI(\pi)}$, where $s_1, \dots s_m$ are the maximal colored $\Delta$-terms in $\PI(\pi)$.

	A term in $\lifdelta{\PI(\pi)}$ is either $x_{s_i}$, $1 \varleq i \varleq m$, a grey term or a $\Gamma$-terms.
	Let $t$ be a maximal $\Gamma$-term in $\PI(\pi)$ and ${r_1}, \dots, {r_k}$ the maximal $\Delta$-terms in\nolinebreak{} $t$.
	Then in $\lifdelta{\PI(\pi)}$, the terms ${r_1}, \dots, {r_k}$ are replaced by $x_{r_1}, \dots, x_{r_k}$ respectively.
	Note that as all of ${r_1}, \dots, {r_k}$ due to being strict subterms of $t$ are of strictly smaller length than $t$, all of $x_{r_1}, \dots, x_{r_k}$ precede $y_t$ in the arrangement of the lifting variables.
	%Note that all of $r_1, \dots, r_k$ are subterms of $t$.
	%Note that the $\Delta$-terms, which are replaced by $x_{r_1}, \ldots, x_{i_{j_k}}$ respectively, are each of strictly smaller size than $t$ as they are strict subterms of $t$.

	%Then it is of the form $f(x_{i_1}, \ldots, x_{i_{n_x}}, u_1, \ldots, u_{n_u}, v_1, \ldots, v_{n_v})$, where $f$ is $\Gamma$-colored, the $u_j$, $1\varleq j \varleq n_u$ are grey terms and the $v_j$, $1\varleq j\varleq n_v$ are $\Gamma$-terms.

	In $\lifgamma{\lifdelta{\PI(\pi)}}$, $t$ is lifted by $y_t$, which is existentially quantified.
	Hence $t$ is a witness for $y_j$ as due to the quantifier ordering,
	it is bound in the scope of the quantification of the lifting variables $x_{r_1}, \dots, x_{r_k}$.
	Therefore $\Gamma \entails Q_1 z_{t_1} \ldots Q_n z_n\,\lifgamma{\lifdelta{\PI(\pi)}}$.

	By Corollary \ref{cor:delta_entails_lifted_interpolant} $\Delta \entails \forall y_{u_1} \dots \forall y_{u_{k'}}\,\lnot \lift{\Gamma}{\PI(\pi)}{y}$, where $u_1, \dots u_{k'}$ are the maximal colored $\Gamma$-terms in $\PI(\pi)$.

	By a similar line of argumentation as above, we can replace the maximal $\Delta$-\nolinebreak{}terms by variables which are then existentially quantified and arrive at
	$\Delta \entails\nolinebreak{} \overline Q_1 z_{t_1} \dots \overline Q_n z_{t_n}\,\lnot \lft{\Delta}{x}{\lft{\Gamma}{y}{\PI(\pi)}}$ where $\overline Q_i = \exists$ ($\forall$) if $Q_i = \forall$ ($\exists$).
	Therefore also
	$\Delta \entails\nolinebreak{} \lnot Q_1 z_{t_1} \dots Q_n z_{t_n}\,\lft{\Delta}{x}{\lft{\Gamma}{y}{\PI(\pi)}}$ and
	finally by Lemma \ref{lemma:lifting_order_not_relevant},
	$\Delta \entails\nolinebreak{} \lnot Q_1 z_{t_1} \dots Q_n z_{t_n}\,\lft{\Gamma}{y}{\lft{\Delta}{x}{\PI(\pi)}}$.

	As it is now easy to see that $Q_1 z_{t_1} \dots Q_n z_{t_n}\,\lft{\Gamma}{y}{\lft{\Delta}{x}{\PI(\pi)}}$ contains no colored symbol, it is an interpolant.
\end{proof}




\section{Commentary on the original publication}
\label{sec:huang_commentary} 


In \cite[Definition 3]{Huang95}, a maximal occurrence of a $\Gamma$ ($\Delta$)-term is defined to be an occurrence of a $\Gamma$ ($\Delta$) term which is not a subterm of a larger $\Gamma$ ($\Delta$)-term.

Furthermore, in the extension of the ``Interpolation Algorithm'' to include paramodulation inferences in \cite[p.~183]{Huang95}, this notion is used to distinguish between the respective cases.
Translated into our notation in the context of our corresponding Definition~\ref{def:PI_paramod} for the case of paramodulation inferences, the conditions for the three cases can be stated as follows:
\begin{enumerate}
	\item The term $r$ occurs in $E\occ{r}$ as subterm of a maximal $\Gamma$-term, which occurs more than once in $E\occ{r} \lor \PI(E\occ{r})$.
		\label{case_1}
	\item The term $r$ occurs in $E\occ{r}$ as subterm of a maximal $\Delta$-term, which occurs more than once in $E\occ{r} \lor \PI(E\occ{r})$.
	\item Otherwise.
\end{enumerate}

Note that if reading this definition in the strict sense, an ambiguity arises:
It is very well possible for a term to be a subterm of a maximal $\Gamma$-term and a maximal $\Delta$-term at the same time.
Suppose $g$ is a $\Gamma$-colored and $h$ a $\Delta$-colored function symbol.
Then the term $h(g(c))$ contains the maximal $\Delta$-term $h(g(c))$ as well as the maximal $\Gamma$-term $g(c)$ since $g(c)$ is not subterm of a larger $\Gamma$-term in\nolinebreak{} $h(g(c))$.

We present the following example, which illustrates that the definition of the conditions for the cases above is to be read as ``maximal colored term, which is $\Phi$-colored'' (or more concisely: ``maximal colored $\Phi$-term'') in place of ``maximal $\Phi$-term''.

\begin{exa}
	Let 
	$\Gamma = \{ P(x) \lor \lnot Q(x), \lnot P(y) \lor Q(y), c=d, \lnot R(g(d)), \lnot S(g(c))  \}$
	and
	$\Delta = \{ S(v) \lor \lnot Q(h(v)), R(u) \lor  Q(h(u)), T(c, d)\}$.
	Hence $h$ is a $\Delta$-colored function symbol and $g$ a $\Gamma$-colored function symbol, while the constant symbols $c$ and $d$ are grey.
	%\Gamma = \{ P(x) \lor \lnot Q(x), c=d, \lnot R(g(d)), \lnot P(v)  \}$
	%and $\Delta = \{ R(u) \lor  Q(h(u)) \}$.

	We present a resolution refutation of $\Gamma \cup \Delta$ in combination with the interpolant extraction such that each label is of the form $C\mid \PI(C)$, where $C$ is the clause of the refutation and $\PI(C)$ is sometimes given in a simplified but logically equivalent form.
	The presentation of the refutation is split into parts in order to improve readability.
	%\begin{landscape}

	Note that at the paramodulation inference \markC{}, case~\ref{case_1} is erroneously selected due to $d$ occurring in the maximal $\Gamma$-colored term $g(d)$, 
	even though $d$ is also contained in the maximal $\Delta$-colored term $h(g(d))$.
	{ \small
		~

		\begin{adjustwidth}{-1.5em}{0em}
		\begin{prooftree}
			\AxiomCm{{ \lnot R(g(d)) \mid \bot}}
			\AxiomCm{{  R(u) \lor  Q(h(u)) \mid \top}}

			%\RightLabelm{u\mapsto g(t)}
			\RightLabelm{\resrule{\resruleres}{u\mapsto g(d)}}
			%\RightLabelm{\resruleres}
			\BinaryInfCm{ Q(h(g(d))) \mid \lnot R(g(d)) }

			\AxiomCm{P(x) \lor \lnot Q(x) \mid \bot}

			\RightLabelm{\resrule{\resruleres}{x \mapsto h(g(d))}}
			\BinaryInfCm{ P(h(g(d))) \mid \lnot R(g(d)) \land \lnot Q(h(g(d)))}

			\AxiomCm{c=d \mid \bot}
			\RightLabelm{\resruleremark{\resrulepar}{\id}{\markC}}
			\BinaryInfCm{P(h(g(c))) \mid (c=d \land \lnot R(g(d)) \land \lnot Q(h(g(d)))) \lor (c\neq d \land g(c) = g(d))}
		\end{prooftree}
		\vspace{1em}

		\begin{prooftree}
			\AxiomCm{ \lnot S(g(c)) \mid \bot}
			\AxiomCm{  S(v) \lor \lnot Q(h(v)) \mid \top }

			\RightLabelm{\resrule{\resruleres}{v \mapsto g(c)}}
			\BinaryInfCm{ \lnot Q(h(g(c))) \mid \lnot S(g(c)) }

			\AxiomCm{\lnot P(y) \lor Q(y) \mid \bot}

			\RightLabelm{\resrule{\resruleres}{y \mapsto h(g(c))}}
			\BinaryInfCm{ \lnot P(h(g(c)))  \mid {\lnot S(g(c)) \land Q(h(g(c)))} }

		\end{prooftree}

		\end{adjustwidth}
	}
		~

	By combining these two derivation by means of a final resolution inference on the last remaining literal employing a trivial substitution, we obtain the empty clause and the corresponding interpolant $\PI(\square)$:
	\[
	(c=d \land \lnot R(g(d)) \land \lnot Q(h(g(d)))) \spam\lor (c\neq d \land g(c) = g(d))  \spam\lor\allowbreak
{\lnot S(g(c)) \land Q(h(g(c)))}\]
	Lifting $\PI(\square)$ and adding appropriate quantifiers gives the final result $I$ of the interpolant extraction:
	\begin{gather*}
		\exists y_{g(c)} \exists y_{g(d)} \forall x_{h(g(c))} \forall x_{h(g(d))} 
		\Big(
		(c=d \land \lnot R(y_{g(d)}) \land \lnot Q(x_{h(g(d))})) \spam\lor \\
		(c\neq d \land y_{g(c)} = y_{g(d)})  \spam\lor
		{\lnot S(y_{g(c)}) \land Q(x_{h(g(c))})}
		\Big)
	\end{gather*}

	Now we show that $\Gamma \not\entails I$.
	Note that as $\Gamma \entails c=d$, no model of $\Gamma$ satisfies $(c\neq d \land y_{g(c)} = y_{g(d)})$.
	The remaining two disjuncts imply that 
	$\forall x_{h(g(c))} \forall x_{h(g(d))} ( \lnot Q(x_{h(g(d))}) \lor Q(x_{h(g(c))}))$,
	but we can easily find a model of $\Gamma$ where at least one domain element satisfies the predicate $Q$ and another domain element does not.
	Any such model is a countermodel to the proposition $\Gamma \entails I$.
\end{exa}




