\chapter*{Abstract}

Craig's interpolation theorem is a long known basic result of mathematical logic.
Interpolants lay bare certain logical relations between formulas or sets of formulas in a
concise way.
This process is fully analytic in the sense that interpolants can efficiently
be calculated from proofs of the relations of the formulas. Leveraging the tremendous
progress of automatic deduction systems in the last decades, obtaining the required proofs
is feasible. 
This enables real world applications for instance in the area of software verification.

For practical applicability, it is often studied in relatively weak formalisms such as propositional logic.
This thesis however aims at giving a comprehensive account of existing techniques and results with respect to unrestricted classical first-order logic with equality in three parts:

First, we present Craig's original proof of the interpolation theorem by reduction to first-order logic without equality and function symbols.
This approach succeeds in proving the result but gives rise to only impractical algorithms for interpolant extraction.

Second, a constructive proof by Huang employing interpolant extraction from resolution proofs is introduced in a slightly improved form.
This shows that even in full first-order logic with equality, interpolants can efficiently be calculated.
We present an analysis of the number of quantifier alternations of the interpolants produced by this algorithm.
Moreover, we propose a novel approach which combines the two phases of Huang's algorithm and thereby allows for creating non-prenex interpolants.

Third, we give a semantic perspective on interpolation in the form of a model-theoretic proof based on Robinson's Joint Consistency Theorem.
This emphasizes the close relation between the proof-theoretic and the model-theoretic view.






%~
%
%andere arbeiten:
%
%- worum gehts ungefähr
%
%- in this work, \dots
%
%- was tuen die dinge in dieser arbeit
%
%- kann ganz kurz sein (200 wörter); sollte vllt unter einer seite sein
%
%- soll fürs submitten unter 2000 zeichen sein
%
%~
%
%richtlinien:
%
%- halbe bis eine seite
%
%- kontext der arbeit / aufgabenstellung
%
%- fragestellung der arbeit
%
%- methode / verfahrensweise
%
%- zentrale ergebnisse
%
%
%\url{http://www.informatik.tuwien.ac.at/dekanat/richtlinie_verfassen_diplomarbeit_03122013.pdf}
%
%
