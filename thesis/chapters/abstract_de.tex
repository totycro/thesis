\chapter*{Kurzfassung}

Der Interpolationssatz von Craig stellt eine grundlegendes Ergebnis der mathematischen Logik dar.
Interpolanten sind in der Lage, gewisse logische Beziehungen zwischen Formeln oder Formelmengen kurz und präzise zusammenzufassen und können oftmals effizient aus Beweisen der jeweiligen Beziehungen extrahiert werden.
Der immensen Fortschritt von automatischen Inferenzsystemen der letzten Jahrzehnte ermöglicht die Berechnung der benötigten Beweise.
Dies legt den Grundstein für Anwendungen etwa im Bereich der Softwareverifikation.

Aufgrund der besseren praktischen Anwendbarkeit wird Interpolation häufig in relativ schwachen logischen Formalismen wie etwa der Aussagenlogik untersucht.
Diese Arbeit setzt sich hingegen zum Ziel, einen umfassenden Überblick über bestehende Techniken und Resultate im Bereich der uneingeschränkten Prädikatenlogik mit Gleichheit zu geben.
Dies geschieht in drei Abschnitten:

Zuerst gehen wir auf den ursprünglichen Beweis des Interpolationssatzes von Craig ein, welcher eine Reduktion auf Prädikatenlogik ohne Gleichheit und Funktionssymbole durchführt.
Aufgrund des dadurch entstehenden Mehraufwandes ergibt sich aus dieser Vorgehensweise nur ein ineffizienter Algorithmus zur Interpolantenextraktion.

Danach stellen wir einen konstruktiven Beweis von Huang in einer etwas verbesserten Form vor.
Hier werden Interpolanten direkt aus Resolutionsbeweisen in zwei Phasen extrahiert, was somit zeigt, dass auch in uneingeschränkter Prädikatenlogik mit Gleichheit eine effiziente Interpolantenberechnung möglich ist.
Desweiteren analysieren wir die Anzahl der Quantorenalternationen in den auf diese Weise berechneten Interpolanten und stellen einen neuen Ansatz vor, welcher beide Phasen von Huang's Algorithmus kombiniert und dadurch nicht prenexe Interpolanten liefert.

Im letzten Abschnitt beschäftigen wir uns mit einer semantischen Sichtweise auf Interpolation in Form eines modelltheoretischen Beweises basierend auf dem Joint Consistency Satz von Robinson.
Dies illustriert sowohl Ähnlichkeiten als auch Unterschiede zwischen der beweistheoretischen und der modelltheoretischen Betrachtungsweise.



