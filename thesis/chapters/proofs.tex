
\section{WT: Interpolation extraction in one pass}

easy for constants, just as in huang but in one pass

terms can grow unpredictably, order cannot be determined during pass

\section{WT: Interpolation extraction in two passes}

\subsection{huang proof revisited}

\subsubsection{propositional part}

Let $\Gamma \cup \Delta$ be unsatisfiable. Let $\pi$ be a proof of $\square$ from $\Gamma \cup \Delta$. Then $\PI$ is a function that returns a relative interpolant w.r.t. the current clause. 

\begin{defi}
	$\theta$ is a \defiemph{relative propositional interpolant} with respect to a clause $C$ in a resolution refutation $\pi$ of $\Gamma \cup \Delta$ if 
	\label{def:rel_prop_interpol}
	\begin{compactenum}
		\item $\Gamma \entails \theta \lor C$
			\label{rel_prop_interpol_cond1}
		\item $\Delta \entails \lnot \theta \lor C$
			\label{rel_prop_interpol_cond2}
		\item $\Pred(\theta) \subseteq (\Pred(\Gamma) \intersect \Pred(\Delta)) \cup \{\top, \bot\} $.
			\label{rel_prop_interpol_cond_lang}
			\qedhere
	\end{compactenum}
\end{defi}

The third condition will sometimes be referred to as \emph{language restriction}.
It is easy to see that a relative propositional interpolant with respect to $\square$ is a propositional interpolant\todo{add this to the definition, i.e.~possible define rel prop interpol from prop interpol}, i.e.~it is an interpolant without the language restriction on constant, variable and function symbols.

We proceed by defining a procedure $\PI$ which extracts relative interpolants from a resolution refutation.

\begin{defi}
	\defiemph{$\PI$} is defined as follows:
\begin{itemize}
	\item[Base case.]
		If $C \in \Gamma$, $\PI(C) = \bot$. 
		If otherwise $C \in \Delta$, $\Delta(C) = \top$. 
	\item[Resolution.]
	\label{def:PI_resolution}
		Suppose the clause $C$ is the result of a resolution step. Then it has the following form: 

%	\begin{prooftree}
%		\AxiomCm{C_1: D \lor l}
%		\AxiomCm{C_2: E \lor \lnot l'}
%		\RightLabelm{\quad l\sigma = l'\sigma}
%		\BinaryInfCm{C: (D\lor E)\sigma}
%	\end{prooftree}
		%\todo{write as prooftree? (not necessary, but nicer)}
		If the clause $C$ is the result of a resolution step of $C_1: D \lor l$ and $C_2: E \lor \lnot l'$ using a unifier $\sigma$ such that $l\sigma = l'\sigma$, then $\PI(C)$ is defined as follows:
	%$\PI(C)$ is defined according to this case distinction:
		\begin{enumerate}
			\item If $\Pred(l) \in L(\Gamma) \setminus L(\Delta)$:\todo{change to "is $\Gamma$-colored?"} $\PI(C) = [\PI(C_1) \lor \PI(C_2)]\sigma$
			\item If $\Pred(l) \in L(\Delta) \setminus L(\Gamma)$: $\PI(C) = [\PI(C_1) \land \PI(C_2)]\sigma$
			\item If $\Pred(l) \in L(\Gamma) \intersect L(\Delta)$: $\PI(C) = [(l \land \PI(C_2)) \lor (l' \land \PI(C_1)) ]\sigma $
		\end{enumerate}

	\item[Factorisation.]
		If the clause $C$ is the result of a factorisation of $C_1: l \lor l' \lor D$ using a unifier $\sigma$ such that $l\sigma = l'\sigma$, then $\PI(C) = \PI(C_1)\sigma$.

	\item[Paramodulation.]
		If the clause $C$ is the result of a paramodulation of $C_1: s=t \lor C$ and $C_2: D[r]$ using a unifier $\sigma$ such that $r\sigma = s\sigma$, then $\PI(C)$ is defined according to the following case distinction:
		\begin{enumerate}
			\item If $r$ occurs in a maximal $\Delta$-term $h(r)$ in $D[r]$ and $h(r)$ occurs more than once in $D[r] \lor \PI(D[r])$:
				\label{def:PI_paramod_1}
				\newline
				$\PI(C) = [ ( s=t \land \PI(C_2) ) \lor (s\neq t \land \PI(C_1)) ]\sigma \lor (s=t \land h(s) \neq h(t))$ 
			\item If $r$ occurs in a maximal $\Gamma$-term $h(r)$ in $D[r]$ and $h(r)$ occurs more than once in $D[r] \lor \PI(D[r])$:
				\label{def:PI_paramod_2}
				\newline
				$\PI(C) = [ ( s=t \land \PI(C_2) ) \lor (s\neq t \land \PI(C_1)) ]\sigma \land (s\neq t \lor h(s) = h(t))$ 
			\item Otherwise:
				\label{def:PI_paramod_3}
				\newline
				$\PI(C) = [ ( s=t \land \PI(C_2) ) \lor (s\neq t \land \PI(C_1)) ]\sigma$ \qedhere

		\end{enumerate}
\end{itemize}
\end{defi}


\begin{prop}
	Let $C$ be a clause of a resolution refutation.
	Then $\PI(C)$ is a relative propositional interpolant with respect to $C$. 
\end{prop}
\begin{proof}
	Proof by induction on the number of rule applications including the following strenghtenings:
	$\Gamma \entails \PI(C) \lor C_\Gamma$ and
	$\Delta \entails \lnot \PI(C) \lor C_\Delta$, where $D_\Phi$ denotes the clause D with only the literals which are contained in $L(\Phi)$. They clearly imply conditions \ref{rel_prop_interpol_cond1} and \ref{rel_prop_interpol_cond2} of definition \ref{def:rel_prop_interpol}. 

\begin{itemize}
	\item[Base case.]
	Suppose no rules were applied. We distinguish two possible cases:
	\begin{enumerate}
		\item $C \in \Gamma$.
			Then $\PI(C) = \bot$. Clearly $\Gamma \entails \bot \lor C_\Gamma$ as $C_\Gamma = C \in \Gamma$, $\Delta \entails \lnot \bot \lor C_\Delta$ and $\bot$ satisfies the restriction on the language.

		\item $C \in \Delta$.
			Then $\PI(C) = \top$. Clearly $\Gamma \entails \top \lor C_\Gamma$, $\Delta \entails \lnot \top \lor C_\Delta$ as $C_\Delta = C \in \Delta$ and $\top$ satisfies the restriction on the language.
	\end{enumerate}

	Suppose the property holds for $n$ rule applications.
	We show that it holds for $n+1$ applications by considering the last one:

\item[Resolution.]
	Suppose the last rule application is an instance of resolution. Then it is of the form:
	\begin{prooftree}
		\AxiomCm{C_1: D \lor l}
		\AxiomCm{C_2: E \lor \lnot l'}
		\RightLabelm{\quad l\sigma = l'\sigma}
		\BinaryInfCm{C: (D\lor E)\sigma}
	\end{prooftree}

	By the induction hypothesis, we can assume that:

	$\Gamma \entails \PI(C_1) \lor (D\lor l)_\Gamma$

	$\Delta \entails \lnot \PI(C_1) \lor (D\lor l)_\Delta$

	$\Gamma \entails \PI(C_2) \lor (E\lor \lnot l')_\Gamma$

	$\Delta \entails \lnot \PI(C_2) \lor (E\lor \lnot l')_\Delta$

		We consider the respective cases from definition \ref{def:PI_resolution}:

			\begin{enumerate}
				\item $\Pred(l) \in L(\Gamma) \setminus L(\Delta)$:
					\label{huang_proof_prop_case_1}
					Then $\PI(C) = [\PI(C_1) \lor \PI(C_2)]\sigma$. 

					As $\Pred(l) \in L(\Gamma)$,
					$\Gamma \entails (\PI(C_1) \lor D_\Gamma\lor l)\sigma$
					as well as $\Gamma \entails (\PI(C_2) \lor E_\Gamma\lor \lnot l')\sigma$.
					By a resolution step, we get $\Gamma \entails (\PI(C_1) \lor \PI(C_2))\sigma \lor ((D \lor E)\sigma)_\Gamma$.

					Furthermore, as $\Pred(l) \not\in L(\PI)$, 
					$\Delta \entails (\lnot\PI(C_1) \lor D_\Delta)\sigma$
					as well as $\Delta \entails (\lnot\PI(C_2) \lor E_\Delta)\sigma$.
					Hence it certainly holds that $\Delta \entails (\lnot \PI(C_1) \lor \lnot\PI(C_2))\sigma \lor (D \lor E)\sigma_\Delta$.

					The language restriction clearly remains satisfied as no nonlogical symbols are added.

				\item $\Pred(l) \in L(\Delta) \setminus L(\Gamma)$: 
					\label{huang_proof_prop_case_2}
					Then $\PI(C) = [\PI(C_1) \land \PI(C_2)]\sigma$. 

					As $\Pred(l) \not\in L(\Gamma)$,
					$\Gamma \entails (\PI(C_1) \lor D_\Gamma)\sigma$
					as well as $\Gamma \entails (\PI(C_2) \lor E_\Gamma)\sigma$.
					Suppose that in a model $M$ of $\Gamma$, $M \cancel \entails D_\Gamma$ and $M \cancel \entails E_\Gamma$. Then $M \entails \PI(C_1) \land \PI(C_2)$.
					Hence 
					$\Gamma \entails (\PI(C_1) \land \PI(C_2))\sigma \lor ((D \lor E)\sigma)_\Gamma$.

					Furthermore due to $\Pred(l) \in L(\Delta)$,
					$\Delta \entails (\lnot\PI(C_1) \lor D_\Delta \lor l)\sigma$
					as well as $\Delta \entails (\lnot\PI(C_2) \lor E_\Delta \lor \lnot l')\sigma$.
					By a resolution step, we get $\Delta \entails (\lnot\PI(C_1) \lor \lnot\PI(C_2))\sigma \lor (D_\Delta \lor E_\Delta)\sigma $
					and hence 
					$\Delta \entails \lnot (\PI(C_1) \land \PI(C_2))\sigma \lor (D_\Delta \lor E_\Delta)\sigma $.

					The language restriction again remains intact.

				\item $\Pred(l) \in L(\Delta) \intersect L(\Gamma)$:
					Then $\PI(C) = [(l \land \PI(C_2)) \lor (\lnot l' \land \PI(C_1)) ]\sigma $

					First, we have to show that 
					$ \Gamma \entails [(l \land \PI(C_2)) \lor (l' \land \PI(C_1)) ]\sigma \lor ((D \lor E)\sigma)_\Gamma$.
					Suppose that in a model $M$ of $\Gamma$, $M \cancel \entails D_\Gamma$ and $\Gamma \cancel \entails E$. Otherwise we are done.
					The induction assumtion hence simplifies to $M \entails \PI(C_1) \lor l$ and $M \entails \PI(C_2) \lor \lnot l'$ respectively.
					As $l\sigma = l'\sigma$, by a case distinction argument on the truth value of $l\sigma$, we get that either $M \entails (l \land \PI(C_2))\sigma$ or $M \entails  (\lnot l' \land \PI(C_1))\sigma$.


					Second, we show that 
					$\Delta \entails ((l \lor \lnot \PI(C_1)) \land (\lnot l' \lor \lnot \PI(C_2)))\sigma \lor ((D \lor E)\sigma)_\Delta$.
					Suppose again that in a model $M$ of $\Delta$, $M \cancel \entails D_\Delta$ and $\Gamma \cancel \entails E_\Delta$. 
					Then the required statement follows from the induction hypothesis.
					
					The language condition remains satisfied as only the common literal $l$ is added to the relative interpolant.


			\end{enumerate}

		\item[Factorisation.]	
			Suppose the last rule application is an instance of factorisation. Then it is of the form:
			\begin{prooftree}
				\AxiomCm{C_1: l \lor l' \lor D}
				\RightLabelm{\quad \sigma = \mgu(l, l')}
				\UnaryInfCm{C_1: (l \lor D)\sigma}
			\end{prooftree}

			Then the propositional interpolant $\PI(C)$ is defined as $\PI(C_1)$.
			By the induction hypothesis, we have:

			$\Gamma \entails \PI(C_1) \lor (l \lor l' \lor D)_\Gamma$

			$\Delta \entails \PI(C_1) \lor (l \lor l' \lor D)_\Delta$

			It is easy to see that then also:

			$\Gamma \entails (\PI(C_1)\lor (l \lor D)_\Gamma)\sigma$

			$\Delta \entails (\PI(C_1)\sigma \lor (l \lor D)_\Delta)\sigma$

			The restriction on the language trivially remains intract.
			

		\item[Paramodulation.]	
			Suppose the last rule application is an instance of paramodulation. Then it is of the form:
			\begin{prooftree}
				\AxiomCm{C_1: D \lor s=t}
				\AxiomCm{C_2: E[r]}
				\RightLabel{$\quad \sigma = \mgu(s, r)$}
				\BinaryInfCm{C: (D \lor E[t])\sigma}
			\end{prooftree}

			By the induction hypothesis, we have:

			$\Gamma \entails \PI(C_1) \lor (D\lor s=t)_\Gamma$

			$\Delta \entails \lnot \PI(C_1) \lor (D\lor s=t)_\Delta$

			$\Gamma \entails \PI(C_2) \lor (E[r])_\Gamma$

			$\Delta \entails \lnot \PI(C_2) \lor (E[r])_\Delta$

			First, we show that $\PI(C)$ as constructed in case \ref{def:PI_paramod_3} of the definition is a relative propositional interpolant in any of these cases:

			$\PI(C) = (s=t \land \PI(C_2)) \lor (s\neq t \land \PI(C_1)) $
			
			Suppose that in a model $M$ of $\Gamma$, $M \cancel \entails D\sigma$ and $M \cancel \entails E[t]\sigma$. Otherwise we are done.
			Furthermore, assume that $M \entails (s=t)\sigma$. Then $M \cancel \entails E[r]\sigma$, but then necessarily $M \entails \PI(C_2)\sigma$. \\
			On the other hand, suppose $M \entails (s\neq t)\sigma$. As also $M \cancel \entails D\sigma$, $M \entails \PI(C_1)\sigma$.
			Consequently, $M \entails [(s=t \land \PI(C_2)) \lor (s\neq t \land \PI(C_1))]\sigma \lor [(D \lor E)_\Gamma]\sigma$

			By an analogous argument, we get $\Delta \entails [(s=t \land \lnot \PI(C_2)) \lor (s\neq t \land \lnot \PI(C_1))]\sigma \lor [(D \lor E)_\Delta]\sigma$,
			which implies
			$\Delta \entails [( s\neq t \lor \lnot \PI(C_2)) \land (s = t \lor \lnot \PI(C_1))]\sigma \lor ((D \lor E)_\Delta)\sigma $

			%By a similar case distinction for a model $M$ of $\Delta$ and assuming that $M \cancel \entails D_\Delta$ and $M \cancel \entails E_\Delta$, we get that if $M \entails (s=t)\sigma$, $M \entails \lnot P$, which implies

			The language restriction again remains satisfied as the only predicate, that is added to the interpolant, is $=$.

			This concludes the argumentation for case \ref{def:PI_paramod_3}. 

			The interpolant of case \ref{def:PI_paramod_1} differs only by an additional formula added via a disjunction and hence condition \ref{rel_prop_interpol_cond1} of definition \ref{def:rel_prop_interpol} holds by the above reasoning.
			As the adjoined formula is a contradiction, its negation is valid which in combination with the above reasoning establishes condition \ref{rel_prop_interpol_cond2}.
			Since no new predicated are added, the language condition remains intact. 

			The situation in case \ref{def:PI_paramod_2} is somewhat symmetric: 
			As a tautology is added to the interpolant with respect to case \ref{def:PI_paramod_1}, condition \ref{rel_prop_interpol_cond1} is satisfied by the above reasoning.
			For condition \ref{rel_prop_interpol_cond2}, consider that the negated interpolant of case \ref{def:PI_paramod_1} implies the negated interpolant of this case.
			The language condition again remains intact.
			\qedhere
	\end{itemize}

	proof that we are allowed to overbind

	TODO: define procedure

	TODO: proof

	\end{proof}


	\subsubsection{overbinding}

	Algorithm (input: propositional interpolant $\theta$):
	\begin{enumerate}
		\item Let $t_1, \ldots, t_n$ be the maximal occurrences of noncommon terms in $\theta$. Order $t_i$ ascendingly by term size. 
		\item Let $\theta^*$ be $\theta$ with maximal occurrences of $\Delta$-terms $r_1, \ldots, r_k$ replaced by fresh variables $x_1, \ldots, x_k$ and maximal occurrences of $\Gamma$-terms $s_1, \ldots, s_{n-k}$ by fresh variables $x_{k+1}, \ldots, x_{n}$
		\item Return $Q_1 x_1, \ldots Q_n x_n \theta^*$, where $Q_i$ is $\forall$ if $t_i$ is a $\Delta$-term and $\exists$ otherwise.
	\end{enumerate}

	Language condition easily established. To prove:

	$\Gamma \entails Q_1 x_1, \ldots Q_n x_n \theta^*$

	$\Delta \entails \lnot Q_1 x_1, \ldots Q_n x_n \theta^*$

	We know that $\theta$ works, just the terms are missing.

	\clearpage
	Attempt without $P_P$:


	\begin{defi}
		\label{def:overline}
		Overline as in paper, replace $\Delta$-terms $t_1, \ldots, t_k$ by respective fresh variables in parenthesis
	\end{defi}

	\begin{lemma}
		\label{lemma:overline}
		$(\overline{C\sigma}(x_1, \ldots, x_n))$ reduces to
		$(\overline{C}(x_1, \ldots, x_n))\sigma'$, where $\sigma' = \sigma[t_1 / x_1]\ldots[t_n / x_n]$.

		$(\overline{C}(x_1, \ldots, x_n))\sigma$ reduces to
		$(\overline{C\sigma'}(x_1, \ldots, x_n))$ if $\sigma$ does not change any of $x_1, \ldots, x_n$ or any of $t_1, \ldots, t_n$.\qedhere

		\todo[inline]{it would work to fix substitutions of $x_i$ by substituting $t_i$ for that instead, as long as the result isn't another $t_i$, but this isn't actually relevant here.}
		
	\end{lemma}

	\begin{prop}
		$\Gamma = \overline{\Gamma}(x_1, \ldots, x_n)$.
	\end{prop}
	\begin{proof}
		By definition, $\Delta$-terms only appear in $\Delta$ and not in $\Gamma$. 
	\end{proof}

	\begin{lemma} $ \Gamma \entails \overline{(\PI(C) \lor C)}(x_1, \ldots, x_n) $.
		\label{lemma:gamma_entails_interpolant}
	\end{lemma}

	\begin{proof}
	By induction on the resultion refutation.

	Base case:
	Either $C \in \Gamma$, then it does not contain $\Delta$-terms.
	Otherwise $C \in \Delta$ and $\PI(C) = \top$.

	Induction step:
	\begin{description}
		\item{Resolution.}
			\begin{prooftree}
				\AxiomCm{C_1: D \lor l}
				\AxiomCm{C_2: E \lor \lnot l'}
				\RightLabelm{\quad l\sigma = l'\sigma}
				\BinaryInfCm{C: (D\lor E)\sigma}
			\end{prooftree}

			By the induction hypothesis, we can assume that:

			$\Gamma \entails \overline{\PI(C_1) \lor (D\lor l)}(x_1, \ldots, x_n)$

			$\Gamma \entails \overline{\PI(C_2) \lor (E\lor \lnot l')}(x_{1}, \ldots, x_n)$

			\begin{enumerate}
				\item $\Pred(l) \in L(\Gamma) \setminus L(\Delta)$:
					Then $\PI(C) = [\PI(C_1) \lor \PI(C_2)]\sigma$. 

					We show that $\Gamma \entails \overline{(\PI(C_1) \lor \PI(C_2) \lor D \lor E)\sigma}(x_1, \ldots, x_n) $.
					This is by lemma \ref{lemma:overline} with $\sigma'$ as in the lemma equivalent to
					$\Gamma \entails \overline{(\PI(C_1) \lor \PI(C_2) \lor D \lor E)}(x_1, \ldots, x_n)\sigma' $.

					By Lemma 11 (Huang) and the induction hypothesis,

					$\Gamma \entails \overline{\PI(C_1)} \lor \overline{D} \lor \overline l$

					$\Gamma \entails \overline{\PI(C_2)} \lor \overline{E} \lor \overline{\lnot l'}$

					As $l\sigma = l'\sigma$, $\overline{l\sigma} = \overline{l'\sigma}$.

					Hence $\Gamma \entails \overline{\PI(C_1)} \lor \overline{D} \lor \overline{\PI(C_2)} \lor \overline{E}$
					and again by Lemma 11 (Huang), 
					$\Gamma \entails \overline{\PI(C_1) \lor D \lor \PI(C_2) \lor E}$.

					Also
					$\Gamma \entails \overline{\PI(C_1) \lor D \lor \PI(C_2) \lor E}\sigma$.
					As $ t_1, \ldots, t_n $ do not appear in $\overline{\PI(C_1) \lor D \lor \PI(C_2) \lor E}$ and these are the only variables where $\sigma$ and $\sigma'$ differs, we get that 
					$\Gamma \entails \overline{\PI(C_1) \lor D \lor \PI(C_2) \lor E}\sigma'$.


				\item $\Pred(l) \in L(\Delta) \setminus L(\Gamma)$:
					Then $\PI(C) = [\PI(C_1) \land \PI(C_2)]\sigma$. 

					We show that $\Gamma \entails \overline{((\PI(C_1) \land \PI(C_2)) \lor D \lor E)\sigma}(x_1, \ldots, x_n) $.
					By lemma \ref{lemma:overline} with $\sigma'$ as in the lemma,
					$\Gamma \entails \overline{((\PI(C_1) \land \PI(C_2)) \lor D \lor E)}(x_1, \ldots, x_n) \sigma'$.

					TODO


			\end{enumerate}

		\item{Paramodulation.}

			\begin{prooftree}
				\AxiomCm{C_1: D \lor s=t}
				\AxiomCm{C_2: E[r]}
				\RightLabel{$\quad \sigma = \mgu(s, r)$}
				\BinaryInfCm{C: (D \lor E[t])\sigma}
			\end{prooftree}

			By the induction hypothesis, we have:

			$\Gamma \entails \overline{\PI(C_1) \lor (D\lor s=t)}$

			$\Gamma \entails \overline{\PI(C_2) \lor (E[r])}$



	easy case:
			$\PI(C) = [ ( s=t \land \PI(C_2) ) \lor (s\neq t \land \PI(C_1)) ]\sigma$

			to show:
			$\Gamma \entails \overline{ [ (( s=t \land \PI(C_2) ) \lor (s\neq t \land \PI(C_1))) \lor (D \lor E[t]) ]\sigma} $

			proof idea: either $s=t$, then also $\PI(C_2)$, or else $s\neq t$, but then also $\PI(C_1)$

			by lemma \ref{lemma:overline} for $\sigma'$ as in lemma, 
			$\Gamma \entails \overline{ (( s=t \land \PI(C_2) ) \lor (s\neq t \land \PI(C_1))) \lor (D \lor E[t]) }\sigma' $

			by lemma 11 (huang)
			$\Gamma \entails [((\overline{s}=\overline{t} \land \overline{\PI(C_2)} ) \lor (\overline{s\neq t} \land \overline{\PI(C_1)})) \lor (\overline{D} \lor \overline{E[t]}) ]\sigma' $

			reformulate:
			$\Gamma \entails ((\overline{s}\sigma'=\overline{t}\sigma' \land \overline{\PI(C_2)}\sigma' ) \lor (\overline{s}\sigma'\neq \overline{t}\sigma' \land \overline{\PI(C_1)}\sigma')) \lor (\overline{D}\sigma' \lor \overline{E[t]}\sigma') $

			By the rule: $s\sigma = r\sigma$, hence also $\overline{s\sigma} = \overline{r\sigma}$ and $\overline{s}\sigma' = \overline{r}\sigma'$ REALLY TRUE? -- think so\dots

			Suppose $M \entails \Gamma$ and $M \not \entails (\overline{D}\sigma' \lor \overline{E[t]}\sigma') $.

			Suppose $M \entails \overline{s}\sigma' = \overline{t}\sigma'$.

			By induction hypothesis (and lemma 11 (huang) and adding the substitution $\sigma'$), 
			$\Gamma \entails \overline{\PI(C_2)}\sigma' \lor \overline{(E[r])}\sigma'$.

		However by assumption $\Gamma \not \entails \overline{E[t]}\sigma'$.

		Hence $\Gamma \not \entails \overline{E[s]}\sigma'$, and
		$\Gamma \not \entails \overline{E[r]}\sigma'$. Therefore $\Gamma \entails \overline{\PI(C_2)}\sigma'$.


		Suppose on the other hand $M \entails \overline{s}\sigma' \neq \overline{t}\sigma'$.

		By the induction hypothesis, 
		$M \entails \overline{\PI(C_1)}\sigma' \lor (\overline{D}\sigma'\lor (\overline{s}=\overline{t})\sigma')$,
		hence then $M \entails \overline{\PI(C_1)}\sigma'$.

		Consequently, 
		$M \entails (\overline{s}\sigma' \neq \overline{t}\sigma' \land \overline{\PI(C_1)}\sigma') \lor (\overline{s}\sigma' = \overline{t}\sigma' \land \overline{\PI(C_2)}\sigma')$.

		By lemma 11 (huang), 
		$M \entails \overline{(s \neq {t} \land {\PI(C_1)} \lor ({s} = {t} \land \PI(C_2))}\sigma'$.

		Hence 
		$\Gamma \entails \overline{(s \neq {t} \land {\PI(C_1)} \lor ({s} = {t} \land \PI(C_2))}\sigma' \lor (\overline{D} \lor \overline{E[t]})\sigma') $.

		IS THIS REALLY WHAT I NEED TO SHOW?


\end{description}
\end{proof}



\subsection{final step of huang's proof}

\begin{thm}
	$Q_1 z_1 \ldots Q_n z_n \PI(\square)^*(z_1, \ldots, z_n)$ is a craig interpolant (order as in huang).
\end{thm}
\begin{proof}
	By lemma \ref{lemma:gamma_entails_interpolant}, $\Gamma \entails \forall x_1 \ldots \forall x_n \overline{\PI(\square)}(x_1, \ldots, x_n)$.

	The terms in $\overline{PI(\square)}$ are either among the $x_i$, $1 \leq i \leq n$ or grey terms or $\Gamma$-terms.
	Let $t$ be a maximal $\Gamma$-term in $\overline{\PI(\square)}$.
	Then it is of the form $f(x_{i_1}, \ldots, x_{i_{n_x}}, u_1, \ldots, u_{n_u}, v_1, \ldots, v_{n_v})$, where $f$ is $\Gamma$-colored, the $x_j$ are as before, the $u_j$ are grey terms and the $v_j$ are $\Gamma$-terms.\todo{basically only need the $x_j$}{}
	Note that the $\Delta$-terms, which are replaced by the $x_{i_1}, \ldots, x_{i_{n_x}}$ are of strictly smaller size than $t$ as they are ``strict'' subterms of $t$.

	In $\PI(\square)^*$, $t$ will be replaced by some $z_j$, which is existentially quantified.
	For this $z_j$, $t$ is a witness as due to the quantifier ordering, all the $x_{i_1}, \ldots, x_{i_{n_x}}$ will be quantified before the existential quantification of $z_j$.
	Therefore $\Gamma \entails Q_1 z_1 \ldots Q_n z_n \PI(\square)^*(z_1, \ldots, z_n)$

\end{proof}




\begin{conj}
	Suppose every variable occurs only once in $\Gamma \cup \Delta$.
	Then the order of the quantifiers for $\PI(\square)^*$ does not matter.
\end{conj}

%\begin{proof}
%	By lemma \ref{lemma:gamma_entails_interpolant}, $\Gamma \entails \forall x_1 \ldots \forall x_n \overline{\PI(\square)}(x_1, \ldots, x_n)$.
%	As these are just universal quantifiers, the order does not matter.
%
%	%Let $t$ be a maximal $\Gamma$-term in $\overline{\PI(\square)}$.
%	%Then it is of the form $f(x_{i_1}, \ldots, x_{i_{n_x}}, u_1, \ldots, u_{n_u}, v_1, \ldots, v_{n_v})$, where the $x_j$ are as before, the $u_j$ are grey terms and the $v_j$ are $\Gamma$-terms.
%\end{proof}

The subterm-relation is reflexive.

\begin{defi}
	%Let $C$ be a clause in $\Gamma \cup \Delta$. 

	%A maximal term $s$ of $C$ is said to be \defiemph{smaller} than a maximal term $t$ of $C$ if a subterm of $s$ occurs in $t$ and $s$ is of smaller length than $t$.
	\label{def:order}


	Let $s$ be a term that is in $\PI(C)$ but not in any predecessor $\PI(C_i)$, $i \in \{1,2\}$. $s$ is smaller than a term $t$ in $\PI(C)$ if $s$ is of strictly smaller length than $t$ and there is a subterm in $s$ which also occurs in $t$.
\end{defi}


\subsection{Half-baked approaches}

\begin{defi}
	Direct interpolation extraction.

	This version of overline and star does NOT overbind variables! If they happen to be in the final interpolant, just overbind them somehow, but not early. these are the only terms that can ``change their color.''

	Convention w.r.t. a clause $C$ which has been derived from $C_1$ and $C_2$:
	$\bar Q_n = Q_1 z_1 \ldots Q_n z_n$, such that the $z_i$ correspond to the maximal terms $t_i$ in $\PI(C)$. Same terms must be overbound by same variable, see 101a for counterexample to per-occurrence-overbinding.
	The $z_i$ are ordered such that
	\begin{compactenum}
	\item the orderings in the $Q_{n_1}$ and $Q_{n_2}$ are respected (no circlular relations can occur in combination with merging as a term is only smaller than another term if it is smaller in length as well, which excludes cycles) 
		\item as well as ordering constraints of terms newly introduced in $\PI(C)$ (i.e.~those that were not present in $\PI(C_1)$ and $\PI(C_2)$). 
	\end{compactenum}
	Basically, use merge sort.


	\begin{itemize}
		\item[Resolution.]~
			\begin{prooftree}
				\AxiomCm{C_1: D \lor l}
				\AxiomCm{C_2: E \lor \lnot l'}
				\RightLabelm{\quad \sigma = \mgu(l, l')}
				\BinaryInfCm{C: (D\lor E)\sigma}
			\end{prooftree}

			$\bar Q_{n_1} \PI(C_1)^*$

			$\bar Q_{n_2} \PI(C_2)^* $

			\begin{enumerate}
				\item $l$ and $l'$ $\Gamma$-colored:

					$\PI(C) \equiv (\PI(C_1) \lor \PI(C_2))\sigma $

					$\PI(C)^* \equiv (\PI(C_1)^* \lor \PI(C_2)^*)\sigma $ (just replace maximal terms)

					intended meaning of $\sigma$: to change the free variables still in the $\PI(C_i)$

					Let $t_1, \ldots, t_{n_1}$ be terms overbound in $\PI(C_1)$ and
					$s_1, \ldots, s_{n_2}$ terms overbound in $\PI(C_2)$.

					$\{ z_1, \ldots, z_n \} = \{t_1, \ldots, t_{n_1}\} \sigma \cup \{s_1, \ldots, s_{n_2}\} \sigma$ $\quad$ // common terms are merged

					order relations as in $C_1, C_2$

					$\bar Q_n \PI(C)^* \equiv \bar Q_n ( \PI(C_1)^* \lor \PI(C_2)^*) $

				\item $l$ and $l'$ $\Delta$-colored:

					similar to first case

				\item $l$ and $l'$ grey:

					$\PI(C) \equiv [(\lnot l' \land \PI(C_1)) \lor (l \land \PI(C_2)) ] \sigma $

					$\PI(C)^* \equiv [ (\lnot l'^* \land \PI(C_1)^*) \lor (l^* \land \PI(C_2)^*) ] \sigma $

					Let $t_1, \ldots, t_{n_1}$ be terms overbound in $\PI(C_1)$,
					$s_1, \ldots, s_{n_2}$ terms overbound in $\PI(C_2)$ and $r_1, \ldots, r_{n_3}$ be the maximal colored terms of $l\sigma$ and $l'\sigma$ (need to apply $\sigma$ here because we there might be grey variables replaced by colored terms) 

					$\{ z_1, \ldots, z_n \} = \{t_1, \ldots, t_{n_1}\} \sigma \cup \{s_1, \ldots, s_{n_2}\} \sigma \cup \{r_1, \ldots, r_{n_3}\}$

					order relations as in $C_1, C_2$ plus:

					\begin{itemize}[+]
						\item If $r_i$ is smaller in length than $t_j$ ($s_j$) and a subterm of $r_i$ occurs in $t_j$ ($s_j$), then $r_i$ is smaller than $t_j$ ($s_j$).

						\item If $r_i$ is larger in length than $t_j$ ($s_j$) and a subterm of $t_j$ ($s_j$) occurs in $r_i$, then $r_i$ is larger than $t_j$ ($s_j$).

						\item If $z_i\sigma \neq z_i$, we have to potentially add new dependencies (cf. 102b). TODO: check only $t_j$ if change in $s_j$ and similar?
					\end{itemize}

					$\bar Q_n \PI(C)^* \equiv \bar Q_n  [ (\lnot l'^* \land \PI(C_1)^*) \lor (l^* \land \PI(C_2)^*) ] \sigma$


			\end{enumerate}
	\end{itemize}

\end{defi}

\begin{conj}
	$Q_1 z_1 \ldots Q_n z_n \PI(\square)^*(z_1, \ldots, z_n)$,
	with the $z_i$ ordered by the terms they replace with ordering defined as in \ref{def:order}, is a craig interpolant.
\end{conj}
\begin{proof}

	By lemma \ref{lemma:gamma_entails_interpolant}, $\Gamma \entails \forall x_1 \ldots \forall x_n \overline{\PI(\square)}(x_1, \ldots, x_n)$.

	The terms in $\overline{PI(\square)}$ are either among the $x_i$, $1 \leq i \leq n$ or grey terms or $\Gamma$-terms.

	Let $t$ be a maximal $\Gamma$-term in $\overline{\PI(\square)}$.
	Then it is of the form $f(x_{i_1}, \ldots, x_{i_{n_x}}, u_1, \ldots, u_{n_u}, v_1, \ldots, v_{n_v})$, where $f$ is $\Gamma$-colored, the $x_j$ are as before, the $u_j$ are grey terms and the $v_j$ are $\Gamma$-terms.

\end{proof}


\begin{prop}
	$\Gamma \entails Q_1 z_1 \ldots Q_n z_n \PI(C)^*(z_1, \ldots, z_n)  \lor C$, quantifiers ordered as in \ref{def:order}, is a craig interpolant.
\end{prop}

\begin{proof}



	Induction.

	Base case: simple.

	Suppose Resolution.
	\begin{prooftree}
		\AxiomCm{C_1: D \lor l}
		\AxiomCm{C_2: E \lor \lnot l'}
		\RightLabelm{\quad \sigma = \mgu(l, l')}
		\BinaryInfCm{C: (D\lor E)\sigma}
	\end{prooftree}

	$\Gamma \entails \bar Q_{n_1} \PI(C_1)^*  \lor D \lor l$

	$\Gamma \entails \bar Q_{n_2} \PI(C_2)^*  \lor E \lor \lnot l'$

	to show:
	$\Gamma \entails \bar Q_n \PI(C)^* \sigma \lor (D \lor E)\sigma$


	Note that a term newly introduced in $\PI(C)$ occurs in either $l$ or $l'$, but not in both.

	Let $t$ be a colored term in $\PI(C)$, which has just been added
	W.l.o.g.\ let it occur in $l$, i.e.\ in $C_1$.



	Case distinction:

			~

			\begin{enumerate}
					\item Suppose $l, l'$ are from $\Gamma$ alone:

			By induction hypothesis:

			$\Gamma \entails ( \bar Q_{n_1} \PI(C_1)^*  \lor D \lor l )\sigma$

			$\Gamma \entails (\bar Q_{n_2} \PI(C_2)^*  \lor E \lor \lnot l')\sigma$

			By resolution:

			$\Gamma \entails (\bar Q_{n_1} \PI(C_1)^*\lor \bar Q_{n_2} \PI(C_2)^*)\sigma  \lor (D \lor E )\sigma$

			
	\begin{description}
		\item [Suppose $t$ is $\Gamma$-colored.] ~

			Then it will be replaced by $x_i$ and existentially quantified.
			It appears in either $\PI(C_1)$ or $\PI(C_2)$.

			$t$ is a witness for $x_i$ because it contains subterms $t_1, \ldots, t_n$. If they are overbound as well, they are so before $t$ and are available here.

			TODO: derive properties using examples 103 or so


		\end{description}

	\end{enumerate}
			\clearpage

			{ \color{gray}
			Then $\sigma$ replaces variables $y_1, \ldots, y_k$ in $E \lor \lnot l'$ with terms that contain~$t$.

			By the induction hypothesis, $\Gamma \entails Q_1 z_1 \ldots Q_{n_2} z_{n_2} \PI(C_2)^*(z_1, \ldots, z_{n_2})  \lor E \lor \lnot l'$.

			Hence $\Gamma \entails (Q_1 z_1 \ldots Q_{n_2} z_{n_2} \PI(C_2)^*(z_1, \ldots, z_{n_2})  \lor E \lor \lnot l' ) \sigma$.

			Also $\Gamma \entails Q_1 z_1 \ldots Q_{n_2} z_{n_2} (\PI(C_2)^*(z_1, \ldots, z_{n_2})\sigma)  \lor E\sigma \lor \lnot l'\sigma$.

			Similarily,
			$\Gamma \entails Q_1 z_1 \ldots Q_{n_1} z_{n_1} (\PI(C_1)^*(z_1, \ldots, z_{n_1})\sigma)  \lor D\sigma \lor l\sigma$

			$\Gamma \entails Q_1 z_1 \ldots Q_{n} z_{n} ( (\lnot l \land \PI(C_2)) \lor (l \land \PI(C_1))) ^*(z_1, \ldots, z_{n})\sigma)  \lor D\sigma \lor l\sigma$

			$l$ basically is the only new thing ($l\sigma = l'\sigma$).

			Either $l$ does not contain any subterms of other terms, then it does not depend on anything and $l$ serves as witness for itself.

			Otherwise it does depend on other terms and we have to make sure that that term is available.
			Depending on another term means that it uses information that is only available from another term,
			i.e.~it contains a subterm of another term. but then that subterm is quantified over before the variable that replaces $t$ is, so it works out.


		\item [$t$ is $\Delta$-colored.]
			Then it is replaced by a universally quantified variable.
			But it ``was already universally quantified'' in the induction hypothesis.
			There, it was some free variable, because that's the only thing that can be substituted, but even with this free var, it worked out.


			%When proving $\Gamma \entails Q_1 z_1 \ldots Q_{n} z_{n} \PI(C)^*(z_1, \ldots, z_{n}$, it will be replaced by an existentially quantified variable, which is ok since $t$ is the witness.
			


	}
\end{proof}

\begin{prop}

	Let $A(x_1, \ldots, x_n)$ be an atom in a relative interpolant.
	A variable occurs in one of the $x_i$ if and only if there are atoms $A(y_1, \ldots, y_n)$ and $A(z_1, \ldots, z_n)$ in $\Gamma$ and $\Delta$ respectively, where $x_i$ can be unified with $z_i$ and $y_i$ such that there is still a variable at that location.

	This means that either the term structure above the variable is the same in the original clauses or there are some variables. Intended meaning: the original clauses prove at least the $x_i$, i.e.~are at least as or more general.
	\medskip

	Special case for outermost variables:

	Let $A(x_1, \ldots, x_n)$ be an atom in a relative interpolant.
	An $x_i$ is a variable if and only if there are atoms $A(y_1, \ldots, y_n)$ and $A(z_1, \ldots, z_n)$ in $\Gamma$ and $\Delta$ respectively, where $y_i$ and $z_i$ are variables.
\end{prop}

need more narrow version: clauses do appear in parent clauses in derivation.



\begin{prop}
	Suppose in a partial interpolant, there are two maximal terms $t_1$ and $t_2$ such that w.l.o.g.~$t_1$ is smaller (as defined in \ref{def:order}) than $t_2$. Then it the final interpolant, an overbinding can be defined where the variable corresponding to $t_1$ is quantified over before the variable corresponding to $t_2$ is.
\end{prop}
