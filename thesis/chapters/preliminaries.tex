\section{Preliminaries}
\label{sec:preliminaries}

Here, we give all required notations and basic concepts which will be used throughout this thesis.

\subsubsection*{Formulas and language}

We work in classical first-order logic with equality.
Formulas are usually denoted by $A$ or $B$, constant symbols by $a$, $b$, $c$ or $d$, function symbols by $f$, $g$ or $h$ and variables by $x$, $y$, $z$, $u$, $v$ or $w$.

The language of a first-order formula $A$ is designated by $\Lang(A)$ and contains all predicate, constant and function symbols that occur in $A$.
For formulas $A_1, \ldots, A_n$, $\Lang(A_1, \ldots, A_n) = \bigcup_{1\varleq i \varleq n} \Lang(A_i)$.
These are also referred to as the \emph{\mbox{non-logical} symbols} of $A$.
The \emph{logical symbols} on the other hand include all logical connectives, quantifiers, the equality symbol ($=$) as well as symbols denoting truth ($\top$) and falsity ($\bot$).
Among the usual symbols for the logical connectives $\land$ (conjunction), $\lor$ (disjunction), $\limpl$ (implication),
we use $A \liff B$ as an abbreviation for $(A\limpl B) \land (B\limpl A)$.
Furthermore, $\semiff$ indicates logical equivalence and syntactic equality is denoted by $\syneq$.
For a set of formulas $\Phi$, $\lnot \Phi$ denotes $\{\lnot A \mid A \in \Phi\}$.



With respect to a formula\nolinebreak{} $A$, an occurrence of a subformula $B$ of $A$ is said to occur \defiemph{positively} if it occurs under an even number of negations and \defiemph{negatively} otherwise.


%\subsubsection*{Terms}

%A term $s$ is a subterm of a term $t$ if $s$ occurs in $t$.
%Moreover, $s$ is a strict subterm of $t$ if $s$ is a subterm of $t$ and $s \neq t$. The superterm relation is the inverse of the subterm relation.


\subsubsection*{Substitutions}


A substitution is a mapping of finitely many variables to terms.
We define named substitutions $\sigma$ of a variable $x$ by a term $t$ in a set-style notation $\sigma = \{ x \mapsto t\}$ such that 
$\varphi\sigma$ denotes a formula or term $\varphi$ where each occurrence of the variable $x$ is replaced by the term $t$.
This is done in a capture avoiding manner, i.e.\ if a variable $y$ occurs free in $t$ and $y$ is also bound in $\varphi$ such that a free occurrence of $x$ is in the scope of this quantifier, the bound variable is renamed by a fresh variable.

Unnamed substitution applications are written as $\varphi\subst{x/t}$.
A substitution $\sigma$ is called trivial on $x$ if $x\sigma = x$. Otherwise it is called non-trivial on $x$.

In some situations, mappings of infinitely many variables to terms are required. We denote such as infinite substitutions.

The domain of a substitution $\sigma$, designated by $\dom(\sigma)$, is the set $\{x \in V \mid x\sigma \neq x\}$, where $V$ denotes the set of all variables.
We refer to the set $\{x\sigma \mid x\in \dom(\sigma)\}$ as the range of sigma, denoted by $\ran(\sigma)$.



A term $s$ is an \defiemph{instance} of a term $t$ if there exists a substitution $\sigma$ such that $t\sigma = s$.
If $s$ is an instance of $t$, then $t$ is an \defiemph{abstraction} of $s$. Note that the abstraction- and instance-relation are reflexive. 

\subsubsection*{Formulas and terms}
The length of a term or formula $\varphi$ is the number of logical and non-logical symbols in $\varphi$.

For formulas or terms $\varphi$, $\varphi\occurat{s}{p}$ denotes $\varphi$ with an occurrence of $s$ at position $p$.
$\varphi\occur{s}$ denotes $\varphi$ where the term $s$ occurs on some set of positions $\Phi$. $\varphi\occur{t}$ denotes $\varphi\occur{s}$ where on each position in $\Phi$, $s$ has been replaced by $t$. Due to its vagueness, this notation is mostly used in order to emphasize that the term $s$ does occur in $\varphi$ in some way.

The function $\FV(\cdot)$ returns the set of free variables for terms and formulas.
Moreover,
$\FS(\cdot)$ returns the set of function symbols for terms, formulas and languages and $\PS(\cdot)$ the set of predicate symbols for formulas and languages.


\subsubsection*{Models}
A model $M$ for a first-order language $\LangSym$ is a pair $(\domainofmodel{M}, \interpretation{M})$, where $\domainofmodel{M}$ is the domain and $\interpretation{M}$ the interpretation, which assigns a domain element to every constant symbol, a function $f : \domainofmodel{M}^n \mapsto \domainofmodel{M}$ to every function symbol of arity $n$ and an $n$-ary relation of domain elements to every predicate symbol of arity $n$ in the language $\LangSym$.

For formulas or sets of formulas $\varphi$, we write $M \entails \varphi$ to denote that $\varphi$ holds in $M$.
For an additional formula or sets of formulas $\psi$, $\varphi \entails \psi$ holds if for every model $M$ of $\varphi$, it holds that $M \entails \psi$.
 $\varphi$ is said to be \defiemph{satisfiable} if there is a model $M$ such that $M \entails \varphi$.


For formulas $A$ with $\FV(A) = \{x_1, \dots, x_n\}$ and a model $M$, $M\entails A$ denotes $M \entails \forall x_1 \quantifierdots \forall x_n A$.
In instances where an explicit assignment $\alpha$ to the free variables is desired,
we write $M_\alpha \entails A$ 
to signify that $M$ entails the formula $A$ where the free variable assignment concurs with $\alpha$ and the free variables not assigned by $\alpha$ are universally quantified.


