

\begin{lemma}[Maehara]
	\label{lemma:maehara}
	Let $\Gamma$ and $\Delta$ be sets of first-order clauses without equality and function symbols such that $\Gamma \vdash \Delta$ is provable in cut-free sequent calculus.
	Then for any partition \parti{\Gamma_1}{\Delta_1}{\Gamma_2}{\Delta_2}
	there is an interpolant $I$ such that
	\begin{compactenum}
	\item $\Gamma_1 \proves \Delta_1, I$ is provable 
		\label{maehcond1}
	\item $\Gamma_2, I \proves \Delta_2$ is provable 
		\label{maehcond2}
	\item $\Lang(I) \subseteq \Lang(\Gamma_1, \Delta_1) \cap \Lang(\Gamma_2, \Delta_2)$
		\label{maehcond3}
	\end{compactenum}
\end{lemma}
\begin{proof}
	We prove this lemma by induction on the number of inferences in a cut-free proof of $\Gamma \proves \Delta$.
	By Lemma \ref{lemma:no_equality_in_proof}, we can assume that no equality symbol occurs in the proof, so equality rules need not be considered. 
	As many cases are similar, we prove some examples only.
	\begin{description}
		\item[\normalfont Base case.]
			Suppose no rules were applied.
			Then $C \vdash D$ is of one of the form
				$A \vdash A$. We give interpolants for any of the four possible partitions:
					\begin{enumerate}
						\item \parti{A}{A}{}{}: $I=\bot$
						\item \parti{}{}{A}{A}: $I=\top$
						\item \parti{}{A}{A}{}: $I=\lnot A$
						\item \parti{A}{}{}{A}: $I=A$
					\end{enumerate}

		\item[\normalfont Structural rules.]
			Suppose the property holds for $n$ rule applications and the $(n+1)$th rule is a structural one.
			We consider some examples:

			\begin{itemize}
				\item The last rule application is an instance of \lkrule{c}{l}. Then it is of the form:
					\begin{prooftree}
						\Axiomm{\Gamma, A, A \fCenter \Delta}
						\RightLabelm{\lkrule{c}{l}}
						\UnaryInfm{\Gamma, A \fCenter \Delta}
					\end{prooftree}

					There are two possible partition schemes: of $\Gamma, A \proves \Delta$:
					\begin{enumerate}
						\item $\partisym = \parti{\Gamma_1, A}{\Delta_1}{\Gamma_2}{\Delta_2}$.
							By the induction hypothesis, we know that there is an interpolant $I$ for the partition \parti{\Gamma_1, A, A}{\Delta_1}{\Gamma_2}{\Delta_2} of the upper sequent.
							$I$ serves as interpolant for $\partisym$ as well.

						\item $\partisym = \parti{\Gamma_1}{\Delta_1}{\Gamma_2, A}{\Delta_2}$.
							By a similar argument, we get that there is an interpolant $I$ for 
							\parti{\Gamma_1}{\Delta_1}{\Gamma_2, A, A}{\Delta_2}, which again is also an interpolant for $\partisym$.

					\end{enumerate}

				\item The last rule application is an instance of \lkrule{w}{r}. Then it is of the form:
					\begin{prooftree}
						\Axiomm{\Gamma \fCenter \Delta}
						\RightLabelm{\lkrule{w}{r}}
						\UnaryInfm{\Gamma \fCenter \Delta, A}
					\end{prooftree}

					By the induction hypothesis, there exists an interpolant $I$ for any partition \parti{\Gamma_1}{\Delta_1}{\Gamma_2}{\Delta_2} of $\Gamma \vdash \Delta$.
					Clearly $I$ remains an interpolant when adding $A$ to either $\Delta_1$ or $\Delta_2$.

			\end{itemize}

		\item[\normalfont Propositional rules.]
			Suppose the property holds for $n$ rule applications and the $(n+1)$th rule is a propositional one.
			We consider some examples:

			\begin{itemize}
				\item The last rule application is an instance of \lkrule{\lnot}{l}. Then it is of the form:
					\begin{prooftree}
						\Axiomm{\Gamma \fCenter \Delta,  A}
						\RightLabelm{\lkrule{\lnot}{l}}
						\UnaryInfm{\lnot A, \Gamma \fCenter \Delta }
					\end{prooftree}

					There are two possible partition schemes of $\Gamma, \lnot A \vdash \Delta$:
					\begin{enumerate}
						\item $\partisym = \parti{\Gamma_1, \lnot A}{\Delta_1}{\Gamma_2}{\Delta_2}$.
							By the induction hypothesis, there exists an interpolant $I$ for the partition \parti{\Gamma_1}{\Delta_1, A}{\Gamma_2}{\Delta_2} of the upper sequent.
							Clearly $I$ is an interpolant for $\partisym$ as well.

						\item $\partisym = \parti{\Gamma_1}{\Delta_1}{\Gamma_2, \lnot A}{\Delta_2}$. A similar argument goes through. 
					\end{enumerate}

				\item The last rule application is an instance of \lkrule{\limpl}{l}. Then it is of the form:
					\begin{prooftree}
						\Axiomm{\Gamma \fCenter \Delta,  A}
						\Axiomm{\Sigma, B \fCenter \Pi}
						\RightLabelm{\lkrule{\limpl}{l}}
						\BinaryInfm{\Gamma, \Sigma, A \limpl B \fCenter \Delta, \Pi }
					\end{prooftree}

					There are two possible partition schemes of $\Gamma, A\limpl B \vdash \Delta$:
					\begin{enumerate}
						\item $\partisym = \parti{\Gamma_1, \Sigma_1, A\limpl B}{\Delta_1, \Pi_1}{\Gamma_2, \Sigma_2}{\Delta_2, \Pi_2}$.
							By the induction hypothesis, there is an interpolant $I_1$ for the partition $\parti{\Gamma_1}{\Delta_1, A}{\Gamma_2}{\Delta_2}$ of the left upper sequent.
							Hence for $I_1$, we have that $\Gamma_1 \fCenter \Delta_1, A, I_1$ and 
							$I_1, \Gamma_2 \fCenter \Delta_2$ are provable.

							Moreover, we also get by the induction hypothesis that there is an interpolant $I_2$ for the partition $\parti{\Sigma_1, B}{\Pi_1}{\Sigma_2}{\Pi_2}$ of the right upper sequent.
							Therefore $\Sigma_1, B \fCenter \Pi_1, I_2$ and $I_2, \Sigma_2 \fCenter \Pi_2$ are provable.

							Using these prerequisites, we first establish that $I_1 \lor I_2$ fulfills conditions \ref{maehcond1} and \ref{maehcond2} of an interpolant for $\partisym$:
							\medskip

							\begin{prooftree}
								\Axiomm{\Gamma_1 \fCenter \Delta_1, A, I_1}
								\Axiomm{\Sigma_1, B \fCenter \Pi_1, I_2}
								\RightLabelm{\lkrule{\limpl}{l}}
								\BinaryInfm{\Gamma_1, \Sigma_1, A\limpl B \fCenter \Delta_1, \Pi_1, I_1, I_2}
								\RightLabelm{\lkrule{\lor}{r}}
								\UnaryInfm{\Gamma_1, \Sigma_1, A\limpl B \fCenter \Delta_1, \Pi_1, I_1 \lor I_2}
							\end{prooftree}
							\medskip

							\begin{prooftree}
								\Axiomm{I_1, \Gamma_2 \fCenter \Delta_2}
								\Axiomm{I_2, \Sigma_2 \fCenter \Pi_2}
								\RightLabelm{\lkrule{\lor}{l}}
								\BinaryInfm{I_1 \lor I_2, \Gamma_2, \Sigma_2 \fCenter \Delta_2, \Pi_2}
							\end{prooftree}
							\medskip

							{
								%\setlength{\abovedisplayskip}{0pt}
								%\setlength{\belowdisplayskip}{0pt}
								%\setlength{\abovedisplayshortskip}{0pt}
								%\setlength{\belowdisplayshortskip}{0pt}


								To show that also condition \ref{maehcond3} is satisfied, consider that by the induction hypothesis, it holds that:
								\begin{align*}
									\Lang(I_1) &\subseteq \Lang(\Gamma_1, \Delta_1, A) \cap \Lang(\Gamma_2, \Delta_2) \\
									\Lang(I_2) &\subseteq \Lang(\Sigma_1, B, \Pi_1) \cap \Lang(\Sigma_2, \Pi_2)
								\end{align*}\nopagebreak
								Therefore
								\begin{align*}
									\Lang(I_1) \cup \Lang(I_2) &\subseteq
									(\Lang(\Gamma_1, \Delta_1, A) \cap \Lang(\Gamma_2, \Delta_2)) \cup ( \Lang(\Sigma_1, B, \Pi_1) \cap \Lang(\Sigma_2, \Pi_2))  \\
									&\Downarrow \\
									\Lang(I_1) \cup \Lang(I_2) &\subseteq
									(\Lang(\Gamma_1, \Delta_1, A) \cup \Lang(\Sigma_1, B, \Pi_1)) \cap (\Lang(\Gamma_2, \Delta_2) \cup \Lang(\Sigma_2, \Pi_2)) \\
									&\Updownarrow \\
									\Lang(I_1 \lor I_2) &\subseteq \Lang(\Gamma_1, \Sigma_1, A\limpl B, \Delta_1, \Pi_1) \cap \Lang(\Gamma_2, \Sigma_2, \Delta_2, \Pi_2)
								\end{align*}

							}

						\item $\partisym = \parti{\Gamma_1, \Sigma_1}{\Delta_1, \Pi_1}{\Gamma_2, \Sigma_2, A\limpl B}{\Delta_2, \Pi_2}$.
							The argument for this case is similar using $I_1 \land I_2$ as interpolant.
					\end{enumerate}

			\end{itemize}

		\item[\normalfont Quantifier rules.]
			Suppose the property holds for $n$ rule applications and the $(n+1)$th rule is a quantifier rule.
			We consider some examples:

			\begin{itemize}
				\item The last rule application is an instance of $\lkrule{\forall}{l}$. Then it is of the form:
					\begin{prooftree}
						\Axiomm{\Gamma, A\subst{x/t} \fCenter \Delta}
						\RightLabelm{\lkrule{\forall}{l}}
						\UnaryInfm{\Gamma, \forall x A \fCenter \Delta}
					\end{prooftree}
					where no free variable of $t$ becomes bound in $A\subst{x/t}$.

					There are two possible partition schemes of $\Gamma, \forall x A \vdash \Delta$:
					\begin{enumerate}
						\item \parti{\Gamma_1, \forall x A}{\Delta_1}{\Gamma_2}{\Delta_2}.
							By the induction hypothesis, there is an interpolant $I$ of the partition $\parti{\Gamma_1, A\subst{x/t}}{\Delta_1}{\Gamma_2}{\Delta_2}$.
							Hence for $I$, 
							$\Gamma_1, A\subst{x/t} \fCenter \Delta_1, I$ and  
							$I, \Gamma_2 \fCenter \Delta_2$ are provable.
							By an application of $\lkrule{\forall}{l}$ to the first sequent we get $\Gamma_1, \forall x A\fCenter \Delta_1, I$, so $I$ satisfies conditions \ref{maehcond1} and \ref{maehcond2} of being an interpolant for $\partisym$.

							In order to show that also $\Lang(I) \subseteq \Lang(\Gamma_1, \forall x A, \Delta_1) \cap \Lang(\Gamma_2, \Delta_2)$, consider that by the induction hypothesis, 
							$\Lang(I) \subseteq \Lang(\Gamma_1, A\subst{x/t}, \Delta_1) \cap \Lang(\Gamma_2, \Delta_2)$.
							As there are no function symbols and since constant symbols are treated as function symbols, $L(\forall x A) = L(A\subst{x/t})$.


						\item \parti{\Gamma_1}{\Delta_1}{\Gamma_2, \forall x A}{\Delta_2}.
							This case can be argued analogously.
					\end{enumerate}

				\item The last rule application is an instance of $\lkrule{\forall}{r}$. Then it is of the form:
					\begin{prooftree}
						\Axiomm{\Gamma\fCenter \Delta, A\subst{x/y} }
						\RightLabelm{\lkrule{\forall}{r}}
						\UnaryInfm{\Gamma\fCenter \Delta, \forall x A }
					\end{prooftree}
					where $y$ does not appear in $\Gamma$, $\Delta$ or $A$.

					There are two possible partition schemes of $\Gamma\vdash \Delta, \forall x A $:
					\begin{enumerate}
						\item $\partisym = \parti{\Gamma_1}{\Delta_1, \forall x A}{\Gamma_2}{\Delta_2}$.
							By the induction hypothesis, there exists an interpolant I of the partition 
							\parti{\Gamma_1}{\Delta_1, A\subst{x/y}}{\Gamma_2}{\Delta_2} of the upper sequent.
							Hence for $I$, 
							$\Gamma_1 \fCenter \Delta_1, A\subst{x/y}, I$ and
							$I, \Gamma_2 \fCenter \Delta_2$ are provable.

						As $y$ does not occur in $\Gamma$ or $\Delta$ and consequently by condition \ref{maehcond3} does not occur in $I$, we may apply the $\lkrule{\forall}{r}$ rule to the former sequent to obtain $\Gamma_1 \fCenter \Delta_1, \forall x A, I$.
							Hence $I$ is an interpolant for $\partisym$ as well.

						\item \parti{\Gamma_1}{\Delta_1}{\Gamma_2}{\Delta_2, \forall x A}.
							This case can be argued analogously.
							\qedhere
					\end{enumerate}


			\end{itemize}

			\begin{comment} % i do not explain why this need not be here
			\item[\normalfont Equality rules.]
				Suppose the property holds for $n$ rule applications and the $(n+1)$th rule is an equality rule.
				We consider an example:

				\begin{itemize}
					\item The last rule application is an instance of $\lkrule{=}{r_1}$. Then it is of the form:
						\begin{prooftree}
							\Axiomm{\Gamma\fCenter \Delta, A\subst{T/t} }
							\Axiomm{\Sigma \fCenter \Pi, s=t}
							\RightLabelm{\lkrule{=}{r_1}}
							\BinaryInfm{\Gamma, \Sigma\fCenter \Delta, \Pi, A\subst{T/s}  }
						\end{prooftree}

						There are two possible partition schemes of $\Gamma, \Sigma \vdash \Delta, \Pi A\subst{T/s} $:
						\begin{enumerate}
							\item $\partisym = \parti{\Gamma_1, \Sigma_1}{\Delta_1, \Pi_1, A\subst{T/s}}{\Gamma_2, \Sigma_2}{\Delta_2, \Pi_2}$.  

								By the induction hypothesis, there is an interpolant $I_1$ for the partition $\parti{\Gamma_1}{\Delta_1, A\subst{T/t}}{\Gamma_2}{\Delta_2}$ of the left upper sequent.
								Hence $\Gamma_1 \fCenter \Delta_1, A\subst{T/t}, I_1$ and $I_1, \Gamma_2 \fCenter \Delta_2$.

								We also get by the induction hypothesis that there is an interpolant $I_2$ for the partition $\parti{\Sigma_1}{\Pi_1, s=t}{\Sigma_2}{\Pi_2}$ of the right upper sequent. Here, we have that 
								$\Sigma_1 \fCenter \Pi_1, s=t, I_2$ and $I_2, \Sigma_2 \fCenter \Pi_2$.

								Now we establish that $I_1 \lor I_2$ is an interpolant for $\partisym$.

								\begin{prooftree}
									\Axiomm{\Gamma_1 \fCenter \Delta_1, A\subst{T/t}, I_1}
									\Axiomm{\Sigma_1 \fCenter \Pi_1, s=t, I_2}
									\RightLabelm{\lkrule{=}{r_2}}
									\BinaryInfm{\Gamma_1, \Sigma_1 \fCenter \Delta_1, \Pi_1, A\subst{T/s}, I_1, I_2}
									\RightLabelm{\lkrule{\lor}{r}}
									\UnaryInfm{\Gamma_1, \Sigma_1 \fCenter \Delta_1, \Pi_1, A\subst{T/s}, I_1 \lor I_2}
								\end{prooftree}

								\begin{prooftree}
									\Axiomm{I_1, \Gamma_2 \fCenter \Delta_2}
									\Axiomm{I_2, \Sigma_2 \fCenter \Pi_2}
									\RightLabelm{\lkrule{\lor}{l}}
									\BinaryInfm{I_1\lor I_2, \Gamma_2, \Sigma_2 \fCenter \Delta_2, \Pi_2}
								\end{prooftree}


								We furthermore get by the induction hypothesis that

								$\Lang(I_1) \subseteq \Lang(\Gamma_1, \Delta_1, A\subst{T/t}) \cap (\Gamma_2, \Delta_2)$

								$\Lang(I_2) \subseteq \Lang(\Sigma_1, \Pi_1, s=t) \cap (\Sigma_2, \Pi_2)$

								$\Lang(I_1 \lor I_2) \subseteq \Lang(\Gamma_1, \Sigma_1, \Delta_1, \Pi_1, A\subst{T/s}) \cap (\Gamma_2 \Sigma_2, \Delta_2, \Pi_2)$

						\end{enumerate}

				\end{itemize}
			\end{comment}
	\end{description}
\end{proof}



