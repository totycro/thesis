\section{Preliminaries}

The language of a first-order formula $A$ is denoted by $L(A)$ and contains all predicate, constant, function and free variable symbols that occur in $A$.
These are also referred to as the \emph{non-logical symbols} of $A$.

An occurrence of term is called \emph{maximal} if it does not occur as subterm of another term.

\section{Craig Interpolation}


\begin{samepage}
	\begin{thm}[Interpolation]
		\label{thm:interpolation}
		Let $\Gamma$ and $\Delta$ be sets of first-order formulas such that $ \Gamma \cup \Delta $ is unsatisfiable.
		Then there exists a first-order formula $I$, called interpolant, such that \nopagebreak[4]
		\begin{compactenum}
		\item $ \Gamma \entails I$ \label{int_1}
		\item $ \Delta \entails \lnot I$  \label{int_2}
		\item $ L(I) \subseteq L(\Gamma) \intersect L(\Delta)$.  \label{int_3}
			\thmqed
		\end{compactenum}
	\end{thm}
\end{samepage}

In the context of interpolation, every non-logical symbol is assigned a color which indicates the its origin(s). 
A non-logical symbol is said to be \emph{$\Gamma$ ($\Delta$)-colored} if it only occurs in $\Gamma$ ($\Delta$) and \emph{grey} in case it occurs in both $\Gamma$ and $\Delta$.
