


\section{Craig Interpolation}

\todo[inline]{TODO: write some text about what interpolation means and that we prove more or less only reverse interpolation, but that's fine by the proposition }

\begin{defi}
	\label{def:interpolant}
	Let $\Gamma$ and $\Delta$ be sets of first-order formulas.
	An \defiemph{interpolant} of $\Gamma$ and $\Delta$ is a first-order formula $I$ such that 
	\begin{enumerate}
		\item $ \Gamma \entails I$ \label{int_1}
		\item $ I \entails \Delta $  \label{int_2}
		\item $ \Lang(I) \subseteq \Lang(\Gamma) \intersect \Lang(\Delta)$.  \label{int_3}
	\end{enumerate}

	\begin{samepage}
		A \defiemph{reverse interpolant} of $\Gamma$ and $\Delta$ is a first-order formula $I$ such that $I$ meets conditions \ref{int_1} and \ref{int_3} of an interpolant as well as:
		\begin{enumerate}[\quad\:1'.]
				\setcounter{enumi}{1}
			\item $ \Delta \entails \lnot I $  \label{int_2prime}
				\qedhere
		\end{enumerate}
	\end{samepage}
\end{defi}

\begin{thm}[Interpolation]
	\label{thm:interpolation_original}
	Let $\Gamma$ and $\Delta$ be sets of first-order formulas such that $ \Gamma \entails \Delta $.
	Then there exists an interpolant for $\Gamma$ and $\Delta$.
\end{thm}
%
%\begin{thm}[Reverse Interpolation]
\begin{restatable}[Reverse Interpolation]{thm}{interpolationRevThm}
		\label{thm:interpolation}
		Let $\Gamma$ and $\Delta$ be sets of first-order formulas such that $ \Gamma \cup \Delta $ is unsatisfiable.
		Then there exists a reverse interpolant for $\Gamma$ and\nolinebreak{} $\Delta$.
\end{restatable}



\begin{prop}
	Theorem \ref{thm:interpolation_original} and \ref{thm:interpolation} are equivalent.
	\label{prop:interpolations_equivalent}
\end{prop}
\begin{proof}
	Let $\Gamma$ and $\Delta$ be sets of first-order formulas such that $ \Gamma \entails \Delta$.
	Then $\Gamma \cup \lnot \Delta$ is unsatisfiable.
	By Theorem \ref{thm:interpolation}, there exists a reverse interpolant $I$ for $\Gamma$ and $\lnot \Delta$.
	As $\lnot \Delta \entails \lnot I$, we get by contraposition that $I \entails \Delta$, hence $I$ is an interpolant for $\Gamma$ and $\Delta$

	For the other direction,
	let $\Gamma$ and $\Delta$ be sets of first-order formulas such that $ \Gamma \cup \Delta$ is unsatisfiable.
	Then $\Gamma \entails \lnot \Delta$, hence by Theorem \ref{thm:interpolation_original}, there exists an interpolant $I$ of $\Gamma$ and $\lnot \Delta$.
	But as thus $ I\entails \lnot \Delta$, we get by contraposition that $\Delta \entails \lnot I$, so $I$ is a reverse interpolant for $\Gamma$ and $\Delta$.
\end{proof}

As the notions of interpolation and reverse interpolation in this sense coincide, we will in the following only speak of interpolation where  will be clear from the context which definition applies.

\begin{remark}
	In this thesis, the equality symbol is treated as a logical symbol. 
	This is justified by the following example:

	Let $\Gamma = \{ a=b \} $ and $\Delta = \{P(a), \lnot P(b)\}$.
	Note that then, $\Lang(\Gamma) \cap \Lang(\Delta)$ does not contain a predicate symbol.
	By regarding $=$ as logical symbol and therefore permitting it to occur in an interpolant despite the fact that it does not occur in $\Delta$ allows for the interpolant $a=b$.
	If we would not permit $=$ in the interpolant, there would be no interpolant for $\Gamma$ and $\Delta$, even though $\Gamma \cup \Delta$ clearly is unsatisfiable.
\end{remark}

\begin{lemma}
	\label{lemma:logically_equivalent_sets}
	Let $\Gamma, \Gamma', \Delta, \Delta'$ be sets of first order formulas such that $\Gamma \semiff \Gamma'$ and $\Delta \semiff \Delta'$ and $\Lang(\Gamma) \cap \Lang(\Delta) = \Lang(\Gamma') \cap \Lang(\Delta')$.
	Then $I$ is an interpolant for $\Gamma$ and $\Delta$ if and only if $I$ is an interpolant for $\Gamma'$ and $\Delta'$.
\end{lemma}
\begin{proof}
	Clearly $\Gamma \entails I$ holds if and only if $\Gamma' \entails I$ and similarly
	$\Delta \entails \lnot I$ holds if and only if $\Delta' \entails \lnot I$.
	As the intersections of the respective languages coincide, the language condition on $I$ is satisfied in both directions.
\end{proof}

\begin{remark}
	In Lemma \ref{lemma:logically_equivalent_sets}, it is not sufficient to require that $\Gamma \semiff \Gamma'$ and $\Delta \semiff \Delta'$. 
	Consider the example where
	$\Gamma = \Delta = \{ \forall x (x=c)\}$
	and 
	$\Gamma' = \Delta' = \{ \forall x (x=d)\}$.
	Then even though $\Gamma$ and $\Gamma'$ as well as $\Delta$ and $\Delta'$ have the same models,
	$\Lang(\Gamma) \cap \Lang(\Delta) = \{c\}$
	whereas
	$\Lang(\Gamma') \cap \Lang(\Delta') = \{d\}$.
	Therefore $\forall x (x=c)$ is an interpolant for $\Gamma$ and $\Delta$ but not for $\Gamma'$ and $\Delta'$.
\end{remark}

In the context of interpolation, every non-logical symbol is assigned a color which indicates its origin(s). 
\begin{defi}[Coloring]
A non-logical symbol is said to be \defiemph{$\Gamma$ ($\Delta$)-colored} if it only occurs in $\Gamma$ ($\Delta$) and \defiemph{grey} in case it occurs in both $\Gamma$ and $\Delta$. A symbol is \defiemph{colored} if it is $\Gamma$- or $\Delta$-colored.
A term is a \defiemph{$\Phi$-term} if its outermost symbol is $\Phi$-colored.

A term $t$ is \defiemph{single-colored} if $t$ is $\Phi$-colored for some $\Phi$ and all colored symbols in $t$ are $\Phi$-colored.
A term $t$ is \defiemph{mixed-colored} if $t$ is $\Phi$-colored for some $\Phi$ and $t$ contains a term which is colored but not $\Phi$-colored.
\mytodo{fix either mixed-colored or multi-colored, currently both is used synonymously}
Note that a multicolored $\Phi$-term consequently is a term whose outermost symbol is $\Phi$-colored and contains a colored but not $\Phi$-colored subterm.

  An occurrence of a term $t$ is called \defiemph{$\Phi$-colored} if $t$ is contained in a maximal $\Phi$-colored term.
	$t$ is called \defiemph{colored} if $t$ is of any color and \defiemph{grey} otherwise.

	A variable is a \defiemph{color-changing} variable if it occurs both in a single-colored $\Phi$-term and a single-colored $\Psi$-term in a given context.
\end{defi}

We sometimes use $\Phi$ and $\Psi$ as colors to emphasise that the reasoning at hand is valid irrespective of the actual color assignment and solely assuming that $\Phi \neq \Psi$. 
%both if $\Phi = \Gamma$ and $\Psi = \Delta  $ as well as if $\Phi = \Delta$ and $\Psi = \Delta$.


