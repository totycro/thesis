\section{Preliminaries}
\todo[inline]{this section contains all the required notation but will just be written up nicely at the end}

The language of a first-order formula $A$ is denoted by $\Lang(A)$ and contains all predicate, constant and function symbols that occur in $A$.
These are also referred to as the \emph{\mbox{non-logical} symbols} of $A$.
The \emph{logical symbols} on the other hand include all logical connectives, quantifiers, the equality symbol ($=$) as well as symbols denoting truth ($\top$) and falsity ($\bot$).

For formulas $A_1, \ldots, A_n$, $\Lang(A_1, \ldots, A_n) = \bigcup_{1\leq i \leq n} \Lang(A_i)$.

An occurrence of $\Phi$-term is called \emph{maximal} if it does not occur as subterm of another $\Phi$-term.

We denote $x_1, \ldots, x_n$ by $\bar x$.

A substitution is a mapping of variables to terms. It is denoted by $\phi\subst{x/t}$, where $\phi$ is a formula or term where each occurrence of the variable $x$ is replaced by the term $t$.
A substitution $\sigma$ is called trivial on $x$ if $x\sigma = x$. Otherwise it is called non-trivial.

A term replacement on the other hand is a mapping of terms to terms. It is denoted by $\phi\termsubst{s/t}$, where $\phi$ is a formula or term where each occurrence of the term $t$ is replaced by the term $s$.
\todo[inline]{is term replacement apt?}

\section{Craig Interpolation}

\todo[inline]{TODO: write some text about what interpolation means and that we prove more or less only reverse interpolation, but that's fine by the proposition }

\begin{defi}
	\label{def:interpolant}
	Let $\Gamma$ and $\Delta$ be sets of first-order formulas.
	An \defiemph{interpolant} of $\Gamma$ and $\Delta$ is a first-order formula $I$ such that 
	\begin{enumerate}
		\item $ \Gamma \entails I$ \label{int_1}
		\item $ I \entails \Delta $  \label{int_2}
		\item $ \Lang(I) \subseteq \Lang(\Gamma) \intersect \Lang(\Delta)$.  \label{int_3}
	\end{enumerate}

	\begin{samepage}
		A \defiemph{reverse interpolant} of $\Gamma$ and $\Delta$ is a first-order formula $I$ such that $I$ meets conditions \ref{int_1} and \ref{int_3} of an interpolant as well as:
		\begin{enumerate}[\quad\:1'.]
				\setcounter{enumi}{1}
			\item $ \Delta \entails \lnot I $  \label{int_2prime}
				\qedhere
		\end{enumerate}
	\end{samepage}
\end{defi}

\begin{thm}[Interpolation]
	\label{thm:interpolation_original}
	Let $\Gamma$ and $\Delta$ be sets of first-order formulas such that $ \Gamma \entails \Delta $.
	Then there exists an interpolant of $\Gamma$ and $\Delta$.
\end{thm}

\begin{thm}[Reverse Interpolation]
	\label{thm:interpolation}
	Let $\Gamma$ and $\Delta$ be sets of first-order formulas such that $ \Gamma \cup \Delta $ is unsatisfiable.
	Then there exists an reverse interpolant of $\Gamma$ and $\Delta$.
\end{thm}


\begin{prop}
	Theorem \ref{thm:interpolation_original} and \ref{thm:interpolation} are equivalent.
	\label{prop:interpolations_equivalent}
\end{prop}
\begin{proof}
	Let $\Gamma$ and $\Delta$ be sets of first-order formulas such that $ \Gamma \entails \Delta$.
	Then $\Gamma \cup \{\lnot B \mid B~\in~\Delta\}$ is unsatisfiable.
	By Theorem \ref{thm:interpolation}, there exists a reverse interpolant $I$ of $\Gamma$ and $\{\lnot B \mid B \in \Delta\}$.
	As $\{\lnot B \mid B \in \Delta\} \entails \lnot I$, we get by contraposition that $I \entails \Delta$, hence $I$ is an interpolant of $\Gamma$ and $\Delta$

	For the other direction,
	let $\Gamma$ and $\Delta$ be sets of first-order formulas such that $ \Gamma \cup \Delta$ is unsatisfiable.
	Then $\Gamma \entails \{\lnot B \mid B \in \Delta\}$, hence by Theorem \ref{thm:interpolation_original}, there exists an interpolant $I$ of $\Gamma$ and $\{\lnot B \mid B \in \Delta\}$.
	But as thus $ I\entails \{\lnot B \mid B \in \Delta\}$, we get by contraposition that $\Delta \entails \lnot I$, so $I$ is a reverse interpolant of $\Gamma$ and $\Delta$.
\end{proof}

As the notion of interpolation and reverse interpolation coincide, we will in the following only speak of interpolation where  will be clear from the context which definition applies.

In the context of interpolation, every non-logical symbol is assigned a color which indicates the its origin(s). 
A non-logical symbol is said to be \emph{$\Gamma$ ($\Delta$)-colored} if it only occurs in $\Gamma$ ($\Delta$) and \emph{grey} in case it occurs in both $\Gamma$ and $\Delta$.
