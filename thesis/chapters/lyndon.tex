\section{Strengthenings of the interpolation theorem}

After Craig's initial result, several stronger versions of the theorem have been published.
\cite{Craig57three} can already be counted among those,
as it defines interpolants equivalently to our Definition~\ref{def:interpolant}, 
but the first publication in \cite{Craig57linear} restricts interpolants only with regard to their predicate symbols, but allows non-common function and constant symbols to occur in it.
This is relevant as later results on the interpolation theorem are only based on \cite{Craig57linear}, which usually is not to be understood as proper restriction of the result as it goes through with the extension of \cite{Craig57three} as well.

%\hl{in lit, first version is used sometimes (\cite{lyndon59}, \cite{Henkin63})} 

Arguably one of the most important strenghtenings is due Lyndon. In \cite{lyndon59}, he showed that the interpolation theorem holds for the following definition of interpolant:

\begin{defi}[Lyndon interpolant]
Let $\Gamma$ and $\Delta$ be sets of first-order formulas. A \defiemph{Lyndon interpolant} of $\Gamma$ and $\Delta$ is a first-order formula I such that the conditions \ref{int_1} and \ref{int_2} of Definition~\ref{def:interpolant} as well as the following:
\begin{enumerate}[\quad\:1'.]
	\setcounter{enumi}{2}
	\item Each predicate symbol occurring positively (negatively) in $I$ occurs positively (negatively) in both $\Gamma$ and $\Delta$.
\end{enumerate}
\end{defi}

proof of this:

in \cite{lyndon59}, based on cut eliminiation (just like craig) and also herbrand's theorem.

in \cite{Henkin63} with a proof that extends \cite{sec:joint_consistency}

~



erroneous treatment of equality in lyndon (cf.\ \cite{motohashi84})

but 

In \cite{oberschelp68}, a related restriction is proved for the equality predicate.

provides information about  degenerate cases (see \cite{})

also corollary about interpolants with no equality


TODO: martin otto ; slagle proof of lyndon




\section*{notes}

henkin gives nice proof basically just like our semantic proof, but with polarity, then extends it to equality.
he shows that interpolation holds with equality, but adds axioms (doesn't say where), so equality has to be allowed in any interpolant.
also further strenghtenings based on nicely writing the formula in some nnf-like style and then defining a form of an interpolant.

