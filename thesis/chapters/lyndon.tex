\section{Strengthenings of the interpolation theorem}
\label{sec:strengthenings}

After Craig's initial result, several stronger versions of the theorem have been published.
\cite{Craig57three} can already be counted among those,
as it defines interpolants equivalently to our Definition~\ref{def:interpolant}, 
whereas the first publication in \cite{Craig57linear} restricts interpolants only with regard to their predicate symbols, but allows non-common function and constant symbols to occur in it.
%This is relevant as some later results on the interpolation theorem are only based on \cite{Craig57linear}, which in many cases is not to be understood as proper restriction of the result.

%\hl{in lit, first version is used sometimes (\cite{lyndon59}, \cite{Henkin63})} 

Arguably one of the most important strengthenings is due Lyndon. In \cite{lyndon59}, he shows the following:

\begin{thm}[Lyndon]
	\label{thm:lyndon}
Let $\Gamma$ and $\Delta$ be sets of first-order formulas such that $\Gamma \entails\nolinebreak \Delta$. 
Then there is a first-order formula $I$ such that the conditions \ref{int_1} and \ref{int_2} of Definition~\ref{def:interpolant} hold for $I$ as well as the following:
\begin{enumerate}
		\renewcommand{\theenumi}{\arabic{enumi}'}
		\setcounter{enumi}{2}
	\item Each predicate symbol occurring positively (negatively) in $I$ occurs positively (negatively) in both $\Gamma$ and $\Delta$.
		\label{int_lyndon_3}
\end{enumerate}
\end{thm}

We do not give a proof here but only proof ideas.
In \cite{lyndon59} and \cite{slagle70}, proofs based on Herbrand's theorem are given:
Starting from two unsatisfiable sets of formulas $\Gamma$ and $\Delta$, unsatisfiable finite subsets are extracted by means of the compactness theorem and a set of unsatisfiable instances of these formulas are produced by Herbrand's theorem.
From these, atoms with predicate symbols which are not contained in $\Lang(\Gamma) \cap \Lang(\Delta)$ are dropped to obtain the desired interpolant.

Theorem~\ref{thm:lyndon} can however also be proven by model-theoretic means similar to the proof of the interpolation theorem given in \ref{sec:joint_consistency}
and is worked out in full detail in \cite{Henkin63} and \cite[Theorem\ 2.2.24]{chang1990model}.

The restriction of the admissible function and constant symbols to the ones in the common language of $\Gamma$ and $\Delta$ is absent in the original formulation of in Theorem~\ref{thm:lyndon}, but can easily be added\footnote{Cf.\ \cite{motohashi84}}.
Therefore it is justified to refer to Lyndon interpolation as a strengthening of Craig interpolation.

It is however not possible to give an restriction on the polarity of the occurrence of constants or function symbol in the interpolant analogous to Theorem~\ref{thm:lyndon}, as the following example shows: 

\begin{exa}[Cf.\ {\cite[p.\ 92]{chang1990model}}]
	Let $\Gamma = \{ \exists x ( x = c \land \lnot P(x)) \}$ and $\Delta = \{ \lnot P(c) \}$.
	Here, the constant $c$ occurs only positively in $\Gamma$ and only negatively in $\Delta$, but must occur in any interpolant.
\end{exa}

Since we regard the equality symbol as a logical symbol, condition \ref{int_lyndon_3} of Theorem~\ref{thm:lyndon} does not apply to it.
Nonetheless
Oberschelp proves in \cite{oberschelp68} that a slightly modified restriction on the polarity of the occurrences of the equality symbol in interpolants is feasible:

\begin{thm}[Oberschelp]
	\label{thm:oberschelp}
	Let $\Gamma$ and $\Delta$ be sets of first-order formulas such that $\Gamma \entails\nolinebreak \Delta$. 
	Then there is a first-order formula $I$ such that the conditions \ref{int_1} and \ref{int_2} of Definition~\ref{def:interpolant} and condition \ref{int_lyndon_3} of Theorem~\ref{thm:lyndon} hold for $I$ as well as the following:

	\begin{enumerate}%[\quad\:1'.]
		\setcounter{enumi}{3}
		\item 
			The equality symbol occurs positively in $I$ only if it occurs positively in $\Gamma$.
		\item
			The equality symbol occurs negatively in $I$ only if it occurs negatively in\nolinebreak{} $\Delta$.
	\end{enumerate}
\end{thm}

The proof can again be given by model-theoretic means in the style of the aforementioned ones.
Example~\ref{exa:degenerate_equality} illustrates these two cases and shows that given these occurrences of the equality symbol, there are sets of formulas which necessitate the equality symbol in their interpolant.
Similar as for Theorem~\ref{thm:lyndon}, a restriction on the function and constant symbols is not given in the original formulation, but can be added as shown in \cite{fujiwara78}.

Note that Theorem~\ref{thm:oberschelp} implies the following corollary on equality-free interpolation:
\begin{corr}
	Let $\Gamma$ and $\Delta$ be sets of first-order formulas such that $\Gamma \entails\nolinebreak \Delta$ and the equality symbol only occurs negatively in $\Gamma$ and only positively in $\Delta$. 
	Then there exists an interpolant $I$ which does not contain the equality symbol.
\end{corr}





%erroneous treatment of equality in lyndon (cf.\ \cite{motohashi84})



%henkin gives nice proof basically just like our semantic proof, but with polarity, then extends it to equality.
%he shows that interpolation holds with equality, but adds axioms (doesn't say where), so equality has to be allowed in any interpolant.
%also further strengthenings based on nicely writing the formula in some nnf-like style and then defining a form of an interpolant.

