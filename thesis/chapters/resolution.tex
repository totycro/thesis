

\section{Resolution}

Resolution calculus, in the formulation as given here, is a sound and complete calculus for first-order logic with equality.
Due to the simplicity of its rules, it is widely used in the area of automated deduction.

\begin{defi}
	A \defiemph{clause} is a finite set of literals. The empty clause will be denoted by $\square$.
	A \defiemph{resolution refutation} of a set of clauses~$\Gamma$ is a derivation of $\square$ consisting of applications of resolution rules (cf.~Figure~\ref{fig:resolution}) starting from clauses in $\Gamma$.

	A clause $C'$ is a successor of a clause $C$ if $C$ occurs in the derivation of $C'$.
	For literals $l$, we say that $l'$ is a successor of $l$ if $l'$ occurs in a successor $C'$ of $C$ and $l'$ is derived from $l$.
	For terms $t$ at position $p$ in clause $C$ we say that $t'$ succeeds $t$ if $t'$ occurs at $p$ in $C'$ where $C'$ succeeds $C$. 
	For clauses, literals and terms, the predecessor relation is the inverse of the successor relation.
\end{defi}

Clauses will usually be denoted by $C$, $D$ or $E$, literals by $l$, $l'$ or $\lambda$ and positions by $p$.

\begin{figure}[htbp]
	\begin{prooftree}
		\LeftLabel{\textit{Resolution:}\quad}
		\AxiomCm{ C \lor l }
		\AxiomCm{ D \lor \lnot l' }
		\RightLabelm{\resruleres \quad \sigma = \mgu(l, l')}
		\BinaryInfCm{ (C \lor D)\sigma }
	\end{prooftree}

	\begin{prooftree}
		\LeftLabel{\textit{Factorisation:}\quad}
		\AxiomCm{ C \lor l \lor l' }
		\RightLabelm{\resrulefac \quad \sigma = \mgu(l, l')}
		\UnaryInfCm{ (C \lor l)\sigma }
	\end{prooftree}

	\begin{prooftree}
		\LeftLabel{\textit{Paramodulation:}\quad}
		\AxiomCm{ D \lor s=t }
		\AxiomCm{ E\occurat{r}{p} }
		\RightLabelm{\resrulepar \quad \sigma = \mgu(s, r)}
		\BinaryInfCm{ (D \lor E\occurat{t}{p})\sigma }
	\end{prooftree}

	\caption{The rules of resolution calculus}
	\label{fig:resolution}
\end{figure}

\begin{samepage}
\begin{thm}
	A clause set $\Gamma$ is unsatisfiable if and only if there is resolution refutation of $\Gamma$.
\end{thm}
\begin{proof}
	See \cite{Rob65}.
\end{proof}
\end{samepage}



\begin{defi}[Tree refutations]
	A resolution refutation is a \defiemph{tree refutation} if every clause is used at most once.
\end{defi}

The following lemma shows that the restriction to tree refutations does not restrict the calculus given that we allow multisets as initial clause sets.
\begin{lemma}
  \label{lemma:bin_tree_deduction}
  Every resolution refutation can be transformed into a tree refutation.
\end{lemma}
\begin{proof}
  Let $\pi$ be a resolution refutation of a set of clauses $\Phi$.
  We show that $\pi$ can be transformed into a tree refutation by induction on the number of clauses that are used multiple times.

  Suppose that no clause is used more than once in $\pi$. Then $\pi$ is a tree refutation.

  Otherwise let $\Psi$ be the set of clauses which is used multiple times.
  Let $C \in \Psi$ be such that no clause $D \in \Psi$ is used in the derivation leading to $C$.
  Let $\chi$ be the derivation leading to $C$.

  Suppose $C$ is used $m$ times.
  We create another resolution refutation $\pi'$ from $\pi$ which contains $m$ copies of $\chi$ and replaces the $i$th use of the clause $C$ by the final clause of the $i$th copy of $\chi$, $1 \varleq i \varleq m$.
  In order to ensure that the sets of variables of the input clauses are disjoint, we rename the variables in each copy of $\chi$ and adapt $\pi'$ accordingly.
  Hence $\pi'$ is a resolution refutation of $\Phi$ where $m-1$ clauses are used more than once.
\end{proof}


\section{Resolution and Interpolation}


In order to apply resolution to arbitrary first-order formulas, they have to be converted to clauses first.
This usually makes use of intermediate normal forms which are defined as follows:

\begin{defi}
	A formula is in \defiemph{Negation Normal Form (NNF)} if negations only occur directly before of atoms.
	A formula is in \defiemph{Conjunctive Normal Form (CNF)} if it is a conjunction of disjunctions of literals.
\end{defi}

In this context, the conjuncts of a CNF-formula are interpreted as clauses.
A well-established procedure for the translation to CNF is comprised of the following steps:

\begin{enumerate}
		\item NNF-Transformation \label{step_nnf_trans}
		\item Skolemisation \label{step_skolem_trans}
		\item CNF-Transformation \label{step_cnf_trans}
\end{enumerate}

Step \ref{step_nnf_trans} can be achieved by solely pushing the negation inwards.
As this transformation yields logically equivalent formulas without affecting the language, by Lemma \ref{lemma:logically_equivalent_sets}, the set of interpolants remains unchanged.
Step \ref{step_skolem_trans} and \ref{step_cnf_trans} on the other hand do not produce logically equivalent formulas since they introduce new symbols.
In this section, we will show that they nonetheless do preserve the set of interpolants.
This fact is vital for the use of resolution-based methods for interpolant computation of arbitrary formulas.


\subsection{Interpolation and Skolemisation}

Skolemisation is a procedure for replacing existential quantifiers by Skolem terms:

\begin{defi}
	Let $V_{\exists x}$ be the set of universally bound variables whose scope includes
	the occurrence of $\exists x$ in a formula.
	The skolemisation of a formula $A$ in NNF, denoted by $\sk(A)$, is the result of replacing every occurrence of an existential quantifier $\exists x$ in $A$ by a term $f(y_1, \ldots, y_n)$ where $f$ is a new Skolem function symbol and $V_{\exists x} = \{y_1, \ldots, y_n\}$.
	In case $V_{\exists x}$ is empty, the occurrence of $\exists x$ is replaced by a new Skolem constant symbol $c$.

	The skolemisation of a set of formulas $\Phi$ is defined to be $\sk(\Phi) = \{ \sk(A) \mid A \in \Phi \}$.
\end{defi}

Note that due to the introduction of Skolem symbols, it is not the case that $\Phi \semiff \sk(\Phi)$.

\begin{prop}
	\label{prop:sk_interpolant}
	Let $\Gamma \cup \Delta$ be unsatisfiable.
	Then $I$ is an interpolant for $\Gamma \cup \Delta$ if and only if it is an interpolant for $\sk(\Gamma) \cup \sk(\Delta)$. 
\end{prop}

\begin{proof}
	Since $\sk(\cdot)$ adds fresh symbols to both $\Gamma$ and $\Delta$ individually,
	none of them are contained in $\Lang(\sk(\Gamma)) \intersect \Lang(\sk(\Delta))$.
	Therefore the language condition on the interpolant is satisfied in both directions.

	%As for any set of formulas $\Phi$, each model of $\Phi$ can be extended to a model of $\sk(\Phi)$ and every model of $\sk(\Phi)$ is a witness for the satisfiability of $\Phi$, $\Phi \entails A$ iff $\sk(\Phi) \entails A$ for a formula $A$.
	%In particular, $\Gamma \entails I$ implies that $\sk(\Gamma) \entails I$
	%and $\Delta \entails \lnot I$ implies that $\sk(\Delta) \entails \lnot I$.

	%As every model of $\sk(\Phi)$ is a witness for the satisfiability of $\Phi$

	We conclude the proof by showing that $\Phi \entails A$ iff $\sk(\Phi) \entails A$ for $\Phi \in \{\Gamma, \Delta\}$ and $A \in \{I, \lnot I\}$.

	Suppose that for a model that $M \entails \sk(\Phi)$ and $\Phi \entails A$.
	Note that the interpretation of the skolem symbols of $\sk(\Phi)$ in $M$ presents witnesses for the corresponding existential quantifiers in $\Phi$.
	Hence $M \entails \Phi$ and consequently $M\entails A$.

	On the other hand, suppose that $M \entails \Phi$ and $\sk(\Phi) \entails A$.
	We assume that $\sk(\Phi)$ only uses Skolem terms which are fresh with respect to $M$.
	Then we can extend $M$ to a model $M'$ of $\sk(\phi)$ by encoding the witness terms for the existential quantifiers in $\Phi$ in the Skolem terms of $\sk(\Phi)$ in $M'$.
	Then $M' \entails \sk(\Phi)$ and thus $M' \entails A$.
	But as $\Lang(A) \subseteq \Lang(M) \subseteq \Lang(M')$, $M$ and $M'$ agree on the interpretation of $A$, hence $M \entails A$.
\end{proof}


\subsection{Interpolation and structure-preserving Normal Form Transformation}

In the following, we describe a common method for transforming a formula $A$ without existential quantifiers into CNF while preserving its structure.
Note that the restriction to formulas without existential quantifiers can easily be established for arbitrary formulas by means of skolemisation and therefore does not limit the applicability of this procedure.

In the following, we use the notational convention that $\{\bar y\} \cup \{\bar z\} = \{\bar x\}$ expressing the intuition that the free variables $\bar x$ of a formula $B$ are comprised of the not necessarily disjoint free variables  $\bar y$ and $\bar z$ of $B$'s direct subformulas.

\begin{defi}
	For every occurrence of a subformula $B$ of a formula $A$ without existential quantifiers, introduce a new atom $L_B(\bar x)$, where $\bar x$ are the free variables occurring in $B$.
	This atom acts as a label for the subformula. 
For each of them, create a defining clause $D_B$:

\begin{itemize}
	\item[If $B$ is atomic:]~

		$D_B\equiv \forall \bar x \big(\lnot B \lor L_B(\bar x)\big) \land \forall \bar x \big(B \lor \lnot L_B(\bar x)\big)  $
	\item[If $B$ is of the form $\lnot G$:]~

		$D_B\equiv \forall \bar x \big(L_B(\bar x) \lor L_G(\bar x)\big) \land \forall \bar x \big(\lnot L_B(\bar x) \lor \lnot L_G(\bar x)\big)$
	\item[If $B$ is of the form $G \land H$:]~

		$D_B\equiv \forall \bar x \big(\lnot L_B(\bar x) \lor L_G(\bar y)\big) \land \forall \bar x \big(\lnot L_B(\bar x) \lor L_H(\bar z)\big) \land \forall \bar x \big(L_B(\bar x) \lor \lnot L_G(\bar y) \lor \lnot L_H(\bar z)\big)  $
	\item[If $B$ is of the form $G \lor H$:]~

		$D_B\equiv \forall \bar x \big(L_B(\bar x) \lor \lnot L_G(\bar y)\big) \land \forall \bar x \big(L_B(\bar x) \lor \lnot L_H(\bar z)\big) \land \forall \bar x \big(\lnot L_B(\bar x) \lor L_G(\bar y) \lor L_H(\bar z)\big)$
	\item[If $B$ is of the form $G \limpl H$:]~

		$D_B\equiv \forall \bar x \big(L_B(\bar x) \lor L_G(\bar y)\big) \land
		\forall \bar x \big(L_B(\bar x) \lor \lnot L_H(\bar z)\big) \land \forall \bar x \big(\lnot L_B(\bar x) \lor \lnot L_G(\bar y) \lor L_H(\bar z)\big)  $
	\item[If $B$ is of the form $\forall x G$:]~

		$D_B\equiv \forall \bar x \forall x \big(\lnot L_B(\bar x) \lor L_G(\bar x, x)\big) \land \forall \bar x \forall x \big(L_B(\bar x) \lor \lnot L_G(\bar x, x)\big)$
\end{itemize}

Let $\subformdefinitional{A}$ be defined as $\bigwedge_{B \in \Sigma(A)} D_B$ and
\defiemph{$\delta(A)$} as $\subformdefinitional{A} \land \forall \bar x L_A(\bar x)$, where $\Sigma(A)$ denotes the set of occurrences of subformulas of $A$.
For a set of formulas without existential quantifiers $\Phi$, let \defiemph{$\delta(\Phi) = \{ \delta(B) \mid B \in \Phi\}$}.
\end{defi}

Note that each of the $D_B$ is in CNF, hence also $\delta(A)$ for any formula $A$ without existential quantifiers.
We continue by working out the logical relations of formulas and their image under $A$:

\begin{lemma}
	Let $M$ be a model of $\delta(A)$ for a formula $A$ without existential quantifiers.
	Then $M \entails A$.
	\label{lemma:delta_model}
\end{lemma}
\begin{proof}

	We show that $M \entails B \liff L_B(\bar x)$ for $B \in \Sigma(A)$.
	As $M\entails \delta(A)$ directly implies that $M\entails L_A$, this proves the lemma.
	Note that also $M \entails \subformdefinitional{A}$.

	The proof is by induction on the structure of $B$.
	For the base case, let $B$ be an atom. Then
	$D_B\equiv \forall \bar x \big(\lnot B \lor L_B(\bar x)\big) \land \forall \bar x \big(B \lor \lnot L_B(\bar x)\big)$, which due to $M \entails D_B$ immediately yields $M\entails B \liff L_B(\bar x)$.

	For the induction step, we illustrate a few cases as the remaining ones are similar.
	\begin{itemize}
		\item Suppose $B$ is of the form $\lnot G$.
			Then:
			\[D_B \equiv \forall \bar x \big(L_B(\bar x) \lor L_G(\bar x)\big) \land\linebreak[0] \forall \bar x \big(\lnot L_B(\bar x) \lor \lnot L_G(\bar x)\big)\]
			By the induction hypothesis, $M \entails G \liff L_G(\bar x)$.
			As $M \entails D_B$, it follows that $M\entails \lnot L_G(\bar x) \liff L_B(\bar x)$,
			so $M\entails \lnot G \liff L_B(\bar x)$
			and $M\entails B \liff L_B(\bar x)$.

		\item Suppose $B$ is of the form $G \lor H$.
			Then:
			\[D_B\equiv \forall \bar x \big(L_B(\bar x) \lor \lnot L_G(\bar y)\big) \land\linebreak[0] \forall \bar x \big(L_B(\bar x) \lor \lnot L_H(\bar z)\big) \land\linebreak[0] \forall \bar x \big(\lnot L_B(\bar x) \lor L_G(\bar y) \lor L_H(\bar z)\big)\]
			We can assume by the induction hypothesis that 
			$M \entails G \liff L_G(\bar x)$ as well as 
			$M \entails H \liff L_H(\bar x)$.
			As $M \entails D_B$, 
			we get that
			$M\entails L_G(\bar y) \limpl L_B(\bar x)$,
			$M\entails L_H(\bar z) \limpl L_B(\bar x)$ and
			$M\entails L_B(\bar x) \limpl (L_G(\bar y) \lor L_H(\bar z))$. 
			Therefore $M \entails L_B(\bar x) \liff\allowbreak (G\lor H)$
			and consequently $M \entails L_B(\bar x) \liff B$.

		\item Suppose $B$ is of the form $\forall x G$.
			Then:
			\[
			D_B\equiv \forall \bar x \forall x \big(\lnot L_B(\bar x) \lor L_G(\bar x, x)\big) \land \forall \bar x \forall x \big(L_B(\bar x) \lor \lnot L_G(\bar x, x)\big)\]
			By the induction hypothesis, $M \entails G \liff L_G(\bar x, x)$.
			Since $M \entails D_B$ and as $x$ does not occur in $L_B(\bar x)$, $M \entails L_B(\bar x) \liff \forall x G$,
			which is nothing else than $M \entails L_B(\bar x) \liff B$.
			\qedhere
	\end{itemize}


\end{proof}

%\begin{defi}
	%For a model $M$ in the language $\Lang(A)$ for a formula $A$, we define $\delta(M)$ to be the extension of $M$ to the language $\Lang(\delta(A))$ by setting $\delta(M) \entails L_B(\bar x)$ if and only if $M \entails B$ for $B\in\Sigma(A)$.
%\end{defi}

\begin{lemma}
	\label{lemma:m_prime}
	Let $A$ be a formula without existential quantifiers and $M_A$ a model in the language $\Lang(A)$.
	Extend $M_A$ to a model $M_A'$ in the language $\Lang(\delta(A))$ such that for $B\in\Sigma(A)$, $M_A\entails L_B(\bar x)$ if and only if $M_A \entails B$.
	Then $M_A' \entails \subformdefinitional{A}$.
\end{lemma}
\begin{proof}
	We proceed by induction on the structure of $A$.
	For the base case, suppose that $A$ is an atom. 
	Then $\subformdefinitional{A} = D_A  =\allowbreak \forall \bar x \big(\lnot A \lor L_A(\bar x)\big) \land\allowbreak \forall \bar x \big(A \lor \lnot L_A(\bar x)\big)$. 
	Consider the case that $M_A' \entails A$. Then by construction of $M_A'$, $M_A'\entails L_A(\bar x)$, hence $D_A$ holds.
	In the case where $M_A' \notentails A$, we know that $M_A' \notentails L_A$, so $D_A$ holds as well.

	For the induction step, consider the following cases. The remaining cases can be argued analoguously.
	\begin{itemize}
		\item $A$ is of the form $G \limpl H$. 
			Then $\subformdefinitional{A} =
			\subformdefinitional{G} \land
			\subformdefinitional{H} \land D_A$.
			By the induction hypothesis, we get that $M_A' \entails \subformdefinitional{G}$ as well as $M_A' \entails \subformdefinitional{H}$.
			It remains to show that $M_A' \entails D_A$, i.e.~
			$M_A' \entails \forall \bar x \big(L_A(\bar x) \lor L_G(\bar y)\big) \land
		\forall \bar x \big(L_A(\bar x) \lor \lnot L_H(\bar z)\big) \land \forall \bar x \big(\lnot L_A(\bar x) \lor \lnot L_G(\bar y) \lor L_H(\bar z)\big)  $.

		Suppose that $M_A' \entails A$.
		Then $M_A' \notentails G$ or $M_A' \entails H$. By construction of $M_A'$, we furthermore have that $M_A' \entails L_B(\bar x)$ and $M_A' \entails \lnot L_G(\bar y) \lor L_H(\bar z)$.

		Otherwise we have that $M_A' \notentails A$, so $M_A' \entails G$ and $M_A' \notentails H$.
		Hence $M_A' \entails \lnot  L_A(\bar x)$, $M_A' \entails L_G(\bar y)$ and $M_A' \notentails L_H(\bar z)$.

		\item $A$ is of the form $\forall x B$.
			Then $\subformdefinitional{A} =
			\subformdefinitional{B} \land D_A$.
			By the induction hypothesis,  $M_A' \entails \subformdefinitional{B}$, and we conclude by showing that $M_A' \entails \forall \bar x \forall x \big(\lnot L_A(\bar x) \lor L_B(\bar x, x)\big) \land \forall \bar x \forall x \big(L_A(\bar x) \lor \lnot L_B(\bar x, x)\big)$:


			Suppose $M_A' \entails A$. Then consequently, $M_A' \entails \forall x B$, so $M_A' \entails L_A(\bar x)$ and $M_A' \entails L_B(\bar x, x)$.
			Otherwise $M_A' \notentails A$. In this case $M_A' \notentails \forall x B$, so $M_A' \notentails L_A(\bar x)$ and $M_A' \notentails L_B(\bar x, x)$.
			\qedhere
	\end{itemize}

\end{proof}




\begin{lemma}
	\label{lemma:definitional}
	Let $A$ be a formula and $\Phi$ a set of formulas without existential quantifiers with such that $\Lang(A) \subseteq \Lang(\Phi)$.
	Then $\Phi \entails A$ if and only if $\delta(\Phi) \entails A$.
\end{lemma}
\begin{proof}
	If $\Phi \entails A$, then $\Phi \cup \{A\}$ is unsatisfiable and thus by the Compactness Theorem,
	there exists a finite $\Phi' \subseteq \Phi$ such that $\Phi' \cup \{A\}$ is unsatisfiable, or in other words $\Phi' \entails A$.
	Extend $\Phi'$ such that $\Lang(A) \subseteq \Lang(\Phi')$.
	Let $B = \bigwedge_{C\in \Phi'} C$.
	We show that $B \entails A$ if and only if $\delta(B) \entails A$ by induction on the structure of $B$.

	For the if-direction, assume that $\delta(B) \entails A$ and let $M$ be a model 
	such that the $\Lang(B)$-reduct of $M$, $M|_{\Lang(B)}$, is a model of $B$.
	%in the language $\Lang(B)$ such that $M \entails B$. 
	Let $M'$ extend $M|_{\Lang(B)}$ as in Lemma \ref{lemma:m_prime} and hence by that lemma, $M' \entails \subformdefinitional{B}$.
	By the construction of $M'$, $M' \entails L_B(\bar x)$, therefore $M' \entails \delta(B)$, so by the induction hypothesis $M' \entails A$. 
	As $\Lang(A) \subseteq \Lang(B)$ and $M'|_{\Lang(B)} = M|_{\Lang(B)}$, $M \entails A$.

	For the only if-direction, assume that $B \entails A$ and let $M$ be a model such that $M\entails \delta(B)$.  
	By Lemma \ref{lemma:delta_model}, $M\entails B$ and hence $M\entails A$.
\end{proof}


\begin{prop}
	\label{prop:delta_interpolant}
	Let $\Gamma \cup \Delta$ be unsatisfiable and contain no existential quantifiers.
	Then $I$ is an interpolant for \mbox{$\Gamma \cup \Delta$} if and only if 
	$I$ is an interpolant for $\delta(\Gamma) \cup \delta(\Delta)$.
\end{prop}
\begin{proof}
	As $\delta$ introduces fresh symbols for each $\Gamma$ and $\Delta$, they do not occur in any interpolant for $\Gamma$ and $\Delta$. 
	This establishes the language condition in both directions.

	Furthermore, Lemma \ref{lemma:definitional} is applicable to interpolants $I$ for $\Gamma \cup \Delta$ due to the language condition and demonstrates that
	$\Gamma \entails I$ if and only if $\delta(\Gamma) \entails I$
	as well as
	$\Delta \entails \lnot I$ if and only if $\delta(\Gamma) \entails \lnot I$, which gives the result.
\end{proof}

At this point, we can summarize the results which enable the use of resolution based methods for calculating interpolants: 

\begin{thm}
	Let $\Gamma \cup \Delta$ be unsatisfiable.
	Then $I$ is an interpolant for \mbox{$\Gamma \cup \Delta$} if and only if 
	$I$ is an interpolant for $\delta(\sk(\Gamma)) \cup \delta(\sk(\Delta))$.
\end{thm}
\begin{proof}
	Immediate by Proposition \ref{prop:delta_interpolant} and Proposition \ref{prop:sk_interpolant}.
\end{proof}

