

\section{Resolution}

Resolution calculus, in the formulation as given here, is a sound and complete calculus for first-order logic with equality.
Due to the simplicity of its rules, it is widely used in the area of automated deduction.

\begin{defi}
	A \defiemph{clause} is a finite set of literals. The empty clause will be denoted by $\square$.
	A \defiemph{resolution refutation} of a set of clauses~$\Gamma$ is a derivation of $\square$ consisting of applications of resolution rules (cf.~figure~\ref{fig:resolution}) starting from clauses in $\Gamma$.
\end{defi}


\begin{thm}
	A clause set $\Gamma$ is unsatisfiable if and only if there is resolution refutation of $\Gamma$.
\end{thm}
\begin{proof}
	See \cite{Rob65}.
\end{proof}

Clauses will usually be denoted by $C$ or $D$, literals by $l$.

\begin{figure}[htbp]
	\begin{prooftree}
		\LeftLabel{\textit{Resolution:}\quad}
		\AxiomCm{ C \lor l }
		\AxiomCm{ D \lor \lnot l' }
		\RightLabelm{\quad \sigma = \mgu(l, l')}
		\BinaryInfCm{ (C \lor D)\sigma }
	\end{prooftree}

	\begin{prooftree}
		\LeftLabel{\textit{Factorisation:}\quad}
		\AxiomCm{ C \lor l \lor l' }
		\RightLabelm{\quad \sigma = \mgu(l, l')}
		\UnaryInfCm{ (C \lor l)\sigma }
	\end{prooftree}

	\begin{prooftree}
		\LeftLabel{\textit{Paramodulation:}\quad}
		\AxiomCm{ C \lor s=t }
		\AxiomCm{ D[r] }
		\RightLabel{$\quad \sigma = \mgu(s, r)$}
		\BinaryInfCm{ (C \lor D[t])\sigma }
	\end{prooftree}

	\caption{The rules of resolution calculus}
	\label{fig:resolution}
\end{figure}


\section{Resolution and Interpolation}


In order to apply resolution to arbitrary first-order formulas, they have to be converted to clauses first.
This usually makes use of intermediate normal forms which are defined as follows:

\begin{defi}
	A formula is in \defiemph{Negation Normal Form (NNF)} if negations only occur directly before of atoms.
	A formula is in \defiemph{Conjunctive Normal Form (CNF)} if it is a conjunction of disjunctions of literals.
\end{defi}

In this context, the conjuncts of a CNF-formula are interpreted as clauses.
A well-established procedure for the translation to CNF is comprised of the following steps:

\begin{enumerate}
		\item NNF-Transformation \label{step_nnf_trans}
		\item Skolemisation \label{step_skolem_trans}
		\item CNF-Transformation \label{step_cnf_trans}
\end{enumerate}

Step \ref{step_nnf_trans} can be achieved by solely pushing the negation inwards.
As this transformation yields an equivalent formula, it clearly has no effect on the interpolants.
Step \ref{step_skolem_trans} and \ref{step_cnf_trans} on the other hand do not produce equivalent formulas since they introduce new symbols.
In this section, we will show that they nonetheless do preserve the set of interpolants.
This fact is vital for the use of resolution-based methods for interpolant computation of arbitrary formulas.


\subsection{Interpolation and Skolemisation}

Skolemisation is a procedure for replacing existential quantifiers with Skolem terms:

\begin{defi}
	Let $V_{\exists x}$ be the set of universally bound variables in the scope of the occurrence of $\exists x$ in a formula.
	The skolemisation of a formula $A$ in NNF, denoted by $\sk(A)$, is the result of replacing every occurrence of an existential quantifier $\exists x$ in $A$ by a term $f(y_1, \ldots, y_n)$ where $f$ is a new Skolem function symbol and $V_{\exists x} = \{y_1, \ldots, y_n\}$.
	In case $V_{\exists x}$ is empty, the occurrence of $\exists x$ is replaced by a new Skolem constant symbol $c$.

	The skolemisation of a set of formulas $\Phi$ is defined to be $\sk(\Phi) = \{ \sk(A) \mid A \in \Phi \}$.
\end{defi}


\begin{prop}
	Let $\Gamma \cup \Delta$ be unsatisfiable.
	Then $I$ is an interpolant for $\Gamma \cup \Delta$ if and only if it is an interpolant for $\sk(\Gamma) \cup \sk(\Delta)$. 
\end{prop}

\begin{proof}
	Since $\sk(\cdot)$ adds fresh symbols to both $\Gamma$ and $\Delta$ individually,
	none of them are contained in $\Lang(\sk(\Gamma)) \intersect \Lang(\sk(\Delta))$.
	Therefore condition \ref{int_3} of theorem \ref{thm:interpolation} is satisfied in both directions.

	As for any set of formulas $\Phi$, each model of $\Phi$ can be extended to a model of $\sk(\Phi)$ and every model of $\sk(\Phi)$ is a witness for the satisfiability of $\Phi$, $\Phi \entails I$ iff $\sk(\Phi) \entails I$.
	Hence conditions \ref{int_1} and \ref{int_2} of theorem \ref{thm:interpolation} remain satisfied for $I$ as well.
\end{proof}


\subsection{Interpolation and structure-preserving Normal Form Transformation}

A common method for transforming a skolemised formula $A$ into CNF while preserving their structure is defined as follows:

\begin{defi}
For every occurrence of a subformula $B$ of $A$, introduce a new atom $L_B$ which acts as a label for the subformula. 
For each of them, create a defining clause $D_B$:
\todo[inline]{does it suffice to not treat universal quantifiers specifically here? (subterms have free variables; possibly need to mention to just pull universal quantifiers outwards to get prenex form and drop quantifiers)}

\begin{itemize}
	\item[If $B$ is atomic:]~

	$D_B\equiv (\lnot B \lor L_B) \land (B \lor \lnot L_B)  $
	\item[If $B$ is $\lnot G$:]~

	$D_B\equiv (L_B \lor L_G) \land (\lnot L_B \lor \lnot L_G)  $
	\item[If $B$ is $G \land H$:]~

		$D_B\equiv (\lnot L_B \lor L_G) \land (\lnot L_B \lor L_H) \land (L_B \lor \lnot L_G \lor \lnot L_H)  $
	\item[If $B$ is $G \lor H$:]~

		$D_B\equiv (L_B \lor \lnot L_G) \land (L_B \lor \lnot L_H) \land (\lnot L_B \lor L_G \lor L_H)  $
	\item[If $B$ is $G \limpl H$:]~

		$D_B\equiv (L_B \lor L_G) \land (L_B \lor \lnot L_H) \land (\lnot L_B \lor \lnot L_G \lor L_H)  $
	\item[If $B$ is $\forall x G$:]~

		$D_B\equiv \forall x (\lnot L_B \lor L_G) \land \forall x(L_B \lor \lnot L_G)$
\end{itemize}

Let \defiemph{$\delta(A)$} be defined as $\bigwedge_{B \in \Sigma(A)} D_B \land L_A$, where $\Sigma(A)$ denotes the set of occurrences of subformulas of $A$.
\end{defi}

\begin{prop}
	\label{prop:definitional_form}
	Let $A$ be a formula. Then $\sk(A)$ is unsatisfiable if and only if $\delta(\sk(A))$ is unsatisfiable.
\end{prop}

\begin{prop}
	Let $\sk(\Gamma) \cup \sk(\Delta)$ be unsatisfiable.
	Then $I$ is an interpolant for \mbox{$\sk(\Gamma) \cup \sk(\Delta)$} if and only if 
	$I$ is an interpolant for $\delta(\sk(\Gamma)) \cup \delta(\sk(\Delta))$.
\end{prop}
\begin{proof}
	As $\delta$ introduces fresh symbols for each $\sk(\Gamma)$ and $\sk(\Delta)$, they must not occur in any interpolant of $\sk(\Gamma)$ and $\sk(\Delta)$. 
	This establishes condition \ref{int_3} of theorem \ref{thm:interpolation} in both directions.

Using proposition \ref{prop:definitional_form}, condition \ref{int_1} and \ref{int_2} of theorem \ref{thm:interpolation} are immediate.
\end{proof}

