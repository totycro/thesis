\documentclass[,%fontsize=11pt,%
	%landscape,
	%DIV8, % mehr text pro seite als defaultyyp
	%DIV10,
	%DIV=calc,%
	draft=false,% final|draft % draft ist platzsparender (kein code, bilder..)
	%titlepage,
	numbers=noendperiod
	11pt,
	a4paper,
	oneside,% apparently, this should stay below some other parameter to have an effect
	openany,
	%]{scrartcl}
]{memoir}



\usepackage[utf8]{inputenc}
\usepackage[T1]{fontenc}
\usepackage[english]{babel}


%\usepackage[urw-garamond]{mathdesign}

\usepackage{lscape}
\usepackage{stackengine}
\usepackage{enumerate}
\usepackage{paralist}
\usepackage{tikz}
\usetikzlibrary{shapes,arrows,backgrounds,graphs,%
	matrix,patterns,arrows,decorations.pathmorphing,decorations.pathreplacing,%
	positioning,fit,calc,decorations.text,shadows%
}


\usepackage{comment} 

\usepackage{etoolbox} % fixes fatal error caused by combining bm, stackengine, hyperref (seriously?)
% http://tex.stackexchange.com/questions/22995/package-incompatibilites-etoolbox-hyperref-and-bm-standalone

\usepackage{etex} % else error on too many packages

% includes
\usepackage{algorithm}
%\usepackage{algorithmic} % conflicts with algpseudocode
\usepackage{algpseudocode}
%\newcommand*\Let[2]{\State #1 $\gets$ #2}
\algrenewcommand\alglinenumber[1]{
{\scriptsize #1}}
\algrenewcommand{\algorithmicrequire}{\textbf{Input:}}
\algrenewcommand{\algorithmicensure}{\textbf{Output:}}


%\usepackage[multiple]{footmisc} % footnotes at the same character separated by ','

\usepackage{multicol}

\usepackage{afterpage}

\usepackage{changepage} % for adjustwidth
\usepackage{caption} % for \ContinuedFloat

\usepackage{tikz}
\usetikzlibrary{shapes,arrows,backgrounds,graphs,%
matrix,patterns,arrows,decorations.pathmorphing,decorations.pathreplacing,%
positioning,fit,calc,decorations.text,shadows%
}

\usepackage{bussproofs}
\EnableBpAbbreviations


\usepackage{amsmath}
\usepackage{amsthm}
\usepackage{amssymb} % the reals
\usepackage{mathtools} % smashoperator

\usepackage{bm} % bm, bold math symbols

\usepackage{thm-restate} % restatable env

% needs extra work and fails on some label here
%\usepackage{cleveref} % cref, apparently better than autoref of hyperref 

\usepackage{nicefrac} % nicefrac

\usepackage{mathrsfs} % mathscr

\usepackage{pst-node} % http://tex.stackexchange.com/questions/35717/how-to-draw-arrows-between-parts-of-an-equation-to-show-the-math-distributive-pr

\usepackage{stackengine}

\usepackage{thmtools} % advanced thm commands (declaretheorem)


\usepackage{nameref} % reference name of thm instead of counter

\usepackage{todonotes}

% conflict with beamer
%\usepackage{paralist} % compactenum

\usepackage{hyperref}
%\hypersetup{hidelinks}  % don't give options to usepackage, it doesn't work with beamer
%\hypersetup{colorlinks=false}  % don't give options to usepackage, it doesn't work with beamer


% \usepackage{enumitem} % labels for enumerate % breaks beamer and memoir itemize


\usepackage{url} 


\usepackage[format=hang,justification=raggedright]{caption}% or e.g. [format=hang]

\usepackage{cancel} % \cancel

\usepackage{lineno}


% commands

% logic etcs
%\newcommand{\ex}[2]{\bigskip\section*{Exercise #1: \begin{minipage}[t]{.80\linewidth} \small \textnormal{\it #2} \end{minipage} } }

\newcommand{\ex}[2]{\bigskip \noindent\textbf{Exercise #1.} \textit{#2} \smallskip}

\newcommand{\comm}[1]{{\color{gray} // #1 }}


\newcommand{\true}[0]{\textbf{1}}
\newcommand{\false}[0]{\textbf{0}}
\newcommand{\tr}{\true}
\newcommand{\fa}{\false}

\newcommand{\ra}{\rightarrow}
\newcommand{\Ra}{\Rightarrow}
\newcommand{\la}{\leftarrow}
\newcommand{\La}{\Leftarrow}

\newcommand{\lra}{\leftrightarrow}
\newcommand{\Lra}{\Leftrightarrow}

\newcommand{\NKZ}{\textbf{NK2}}

%\DeclareMathOperator{\syneq}{\equiv} %spacing seems wrong, therefore defined as newcommand below
\DeclareMathOperator{\limpl}{\supset}
\DeclareMathOperator{\liff}{\lra}
\DeclareMathOperator{\semiff}{\Lra}
\newcommand{\syneq}{\equiv}
\newcommand{\union}{\cup}
\newcommand{\bigunion}{\bigcup}
\newcommand{\intersection}{\cap}
\newcommand{\bigintersection}{\bigcap}
\newcommand{\intersect}{\intersection}
\newcommand{\bigintersect}{\bigintersection}

\newcommand{\powerset}{\mathcal{P}}

\newcommand{\entails}{\vDash}
\newcommand{\notentails}{\nvDash}
\newcommand{\proves}{\vdash}

\newcommand{\vm}{\ensuremath{\vv_\mathcal{M}}}
\newcommand{\Dia}{\ensuremath{\lozenge}}

\newcommand{\spaced}[1]{\ \ #1 \ \ }
\newcommand{\spa}[1]{\spaced{#1}}
\newcommand{\spas}[1]{\;{#1}\;}
\newcommand{\spam}[1]{\;\,{#1}\;\,}

% functions
\DeclareMathOperator{\sk}{sk}
\DeclareMathOperator{\mgu}{mgu}
\DeclareMathOperator{\dom}{dom}
\DeclareMathOperator{\ran}{ran}

\DeclareMathOperator{\id}{id}
\DeclareMathOperator{\Fun}{FS}
\DeclareMathOperator{\Pred}{PS}
\DeclareMathOperator{\Lang}{L}
\DeclareMathOperator{\ar}{ar}
\DeclareMathOperator{\PI}{PI}
\DeclareMathOperator{\LI}{LI}
\DeclareMathOperator{\Congr}{Congr}
\DeclareMathOperator{\Refl}{Refl}
\DeclareMathOperator{\aiu}{au}
\DeclareMathOperator{\expa}{unfold-lift}

\newcommand{\PIinc}{\LI}
\newcommand{\PIincde}{\LIde}

\newcommand{\LIde}{\ensuremath{\LI^\Delta}}

\newcommand{\LIcl}{\ensuremath{\LI_{\operatorname{cl}}}}
\newcommand{\LIclde}{\ensuremath{\LI_{\operatorname{cl}}^\Delta}}

\newcommand{\cll}{\ensuremath{_{\operatorname{LIcl}}}}
\newcommand{\cllde}{\ensuremath{_{\operatorname{LIcl}^\Delta}}}

%\newcommand{\lifi}{\mathop{\ell\text{}i}}
\newcommand{\lifiboth}[1]{\ensuremath{\LIcl(#1)}}
\newcommand{\lifidelta}[1]{\ensuremath{\LIclde(#1)}}


%\DeclareMathOperator{\abstraction}{abstraction}

%\newcommand{\sk}{\ensuremath{\mathrm{sk}}}
%\newcommand{\mgu}{\ensuremath{\mathrm{mgu}}}
%\newcommand{\Fun}{\ensuremath{\mathrm{FS}}}
%\newcommand{\Pred}{\ensuremath{\mathrm{PS}}}
%\newcommand{\PI}{\ensuremath{\mathrm{PI}}}
%\newcommand{\Lang}{\ensuremath{\mathrm{L}}}
%\newcommand{\ar}{\ensuremath{\mathrm{ar}}}

\DeclareMathOperator{\AI}{AI}
\newcommand{\AIde}{\ensuremath{\AI^\Delta}}
\newcommand{\AImatrix}{\ensuremath{\AI_\mathrm{mat}}}
\newcommand{\AImatrixde}{\ensuremath{\AI_\mathrm{mat}^\Delta}}
\newcommand{\AImat}{\AImatrix}
\newcommand{\AImatde}{\AImatrixde}
\newcommand{\AIclause}{\ensuremath{\AI_\mathrm{cl}}}
\newcommand{\AIcl}{\AIclause}
\newcommand{\AIclde}{\AIclausede}
\newcommand{\AIclausede}{\ensuremath{\AIclause^\Delta}}
\newcommand{\fromclause}{\ensuremath{_{\operatorname{AIcl}}}}
\newcommand{\fromclausede}{\ensuremath{_{\operatorname{AIcl}^\Delta}}}
\newcommand{\cl}{\fromclause}
\newcommand{\clde}{\fromclausede}

\newcommand{\Q}{\ensuremath{Q}}

\newcommand{\AIcol}{\ensuremath{\AI_\mathrm{col}}}
\newcommand{\AIcolde}{\AIcol^\Delta}

\newcommand{\AIany}{\ensuremath{\AI_\mathrm{*}}}
\newcommand{\AIanyde}{\AIany^\Delta}

\newcommand{\AIclpre}{\AIclause^\bullet}
\newcommand{\AImatpre}{\AImatrix^\bullet}

\newcommand{\PS}{\Pred}
\newcommand{\FS}{\Fun}

\DeclareMathOperator{\LangSym}{\mathcal{L}}

%\newcommand{\mguarr}{\sim_\ra}
\newcommand{\mguarr}{\mapsto_{\mgu}}


%\newcommand{\Trans}{\ensuremath{\mathrm{T}}}
%\newcommand{\Trans}{\ensuremath{\mathrm{T}}}
\DeclareMathOperator{\Trans}{T}
\DeclareMathOperator{\TransInv}{T^{-1}}

\DeclareMathOperator{\FAX}{F_{Ax}}
\DeclareMathOperator{\EAX}{E_{Ax}}
%\newcommand{\FAX}{\ensuremath{\mathrm{F_{Ax}}}}
%\newcommand{\EAX}{\ensuremath{\mathrm{E_{Ax}}}}

%\newcommand{\TransAll}{\ensuremath{\Trans_{\mathrm{Ax}}}}
\DeclareMathOperator{\TransAll}{\Trans_{Ax}}
%\newcommand{\FAX}{\ensuremath{\mathrm{F_{Ax}}}}

\DeclareMathOperator{\defeq}{\stackrel{\mathrm{def}}{=}}

\newcommand{\subst}[1]{[#1]}
\newcommand{\abstractionOp}[1]{\{#1\}}

\newcommand{\subformdefinitional}[1]{\ensuremath{D_{\Sigma(#1)}}}


%\newcommand{\lift}[3]{\operatorname{Lift}_{#1}(#2; #3)}
%\newcommand{\lift}[3]{\operatorname{Lift}_{#1,#3}(#2)}
%\newcommand{\lift}[3]{\operatorname{Lift}_{#1,#3}[#2]}
%\newcommand{\lift}[3]{\overline{#2}_{#1,#3}}
\newcommand{\lifsym}{\ell}
%\newcommand{\lift}[3]{\lifsym_{#1,#3}[#2]}
\newcommand{\lift}[3]{\lifsym_{#1}^{#3}[#2]}
\newcommand{\liftnovar}[2]{\lifsym_{#1}[#2]}

%\newcommand{\lft}[3]{\lifsym_{#1,#2}[#3]}
\newcommand{\lft}[3]{\lift{#1}{#3}{#2}}
\newcommand{\lifboth}[1]{\lifsym[#1]}

%\newcommand{\lifi}{\mathop{\ell\text{}i}}
%\newcommand{\lifiboth}[1]{\lifi[#1]}
%\newcommand{\lifidelta}[1]{\lifi_\Delta^x[#1]}
%\newcommand{\lifideltanovar}[1]{\lifi_\Delta[#1]}

\newcommand{\lifdelta}[1]{\lift{\Delta}{#1}{x}}
\newcommand{\lifdeltanovar}[1]{\liftnovar{\Delta}{#1}}
\newcommand{\lifgamma}[1]{\lift{\Gamma}{#1}{y}}
\newcommand{\lifgammanovar}[1]{\liftnovar{\Gamma}{#1}}
\newcommand{\lifphinovar}[1]{\liftnovar{\Phi}{#1}}
\newcommand{\lifphi}[1]{\lift{\Phi}{#1}{z}}

\DeclareMathOperator{\arr}{\mathcal{A}}
%\DeclareMathOperator{\arrFinal}{{\mathcal{A}^{\bm*}}}
\DeclareMathOperator{\arrFinal}{{\mathcal{\bm{\hat}A}}}
\DeclareMathOperator{\warr}{\marr}
\DeclareMathOperator{\marr}{\mathcal{M}}

\DeclareMathOperator{\apath}{\leadsto}
\DeclareMathOperator{\mpath}{\leadsto_=}
\DeclareMathOperator{\notapath}{\not\leadsto}
\DeclareMathOperator{\notmpath}{\not\leadsto_=}

\newcommand{\ltArrC}{<_{\arrFinal(C)}}
\newcommand{\ltAC}{<_{\arr(C)}}
\newcommand{\ltArrCOne}{<_{\arrFinal(C_1)}}
\newcommand{\ltArrCTwo}{<_{\arrFinal(C_2)}}
%\newcommand{\ltArrC}{<_{\scalebox{0.6}{$\arrFinal(C)$}}}
\newcommand{\ltArr}{<_{\scalebox{0.6}{$\arrFinal$}}}

\newcommand{\bhat}{\bm\hat}
\newcommand{\bbar}{\bm\bar}
\newcommand{\bdot}{\bm\dot}

%\usepackage{yfonts}
\usepackage{upgreek}
\DeclareMathAlphabet{\mathpzc}{OT1}{pzc}{m}{it}
%\DeclareMathOperator{\pos}{\mathscr{P}}
%\DeclareMathOperator{\pos}{\mathpzc{p}}
%\DeclareMathOperator{\pos}{{\rho}}
\DeclareMathOperator{\pos}{{\operatorname P}}
%\DeclareMathOperator{\pos}{P}
\DeclareMathOperator{\poslit}{\pos_\mathrm{lit}}
\DeclareMathOperator{\posterm}{\pos_\mathrm{term}}
%\newcommand{\poslit}[1]{\ensuremath{p_\text{lit}(#1)}}
%\newcommand{\posterm}[1]{\ensuremath{p_\text{term}(#1)}}
\newcommand{\at}[1]{|_{#1}}

\newcommand{\UICm}[1]{\UnaryInfCm{#1}}
\newcommand{\UnaryInfCm}[1]{\UnaryInfC{$#1$}}
\newcommand{\BICm}[1]{\BinaryInfCm{#1}}
\newcommand{\BinaryInfCm}[1]{\BinaryInfC{$#1$}}
\newcommand{\RightLabelm}[1]{\RightLabel{$#1$}}
\newcommand{\LeftLabelm}[1]{\LeftLabel{$#1$}}
\newcommand{\AXCm}[1]{\AxiomCm{#1}}
\newcommand{\AxiomCm}[1]{\AxiomC{$#1$}}
\newcommand{\mt}[1]{\textnormal{#1}}

\newcommand{\UnaryInfm}[1]{\UnaryInf$#1$}
\newcommand{\BinaryInfm}[1]{\BinaryInf$#1$}
\newcommand{\Axiomm}[1]{\Axiom$#1$}



% math
\newcommand{\calI}{\ensuremath{\mathcal{I}}}

\newcommand{\tupleShort}[2]{\ensuremath{(#1_1,\dotsc,#1_{#2})}}
\newcommand{\tuple}[2]{\ensuremath{(#1_1,\:#1_2\:,\dotsc,\:#1_{#2})}}
\newcommand{\setelements}[2]{\ensuremath{\{#1_1,\:#1_2\:,\dotsc,\:#1_{#2}\}}}
\newcommand{\pathelements}[2]{\ensuremath{ (#1_1,\:#1_2\:,\dotsc,\:#1_{#2}) }}

\newcommand{\elems}[1]{\ensuremath{#1_1,\dotsc, #1_{n}) }}

\newcommand{\defiemph}[1]{\emph{#1}}

\newcommand{\setofbases}{\ensuremath{\mathcal{B}}}
\newcommand{\setofcircuits}{\ensuremath{\mathcal{C}}}

\newcommand{\reals}{\ensuremath{\mathbb{R}}}
\newcommand{\integers}{\ensuremath{\mathbb{Z}}} 
\newcommand{\naturalnumbers}{\ensuremath{\mathbb{N}}}

% general
\newcommand{\zit}[3]{#1\ \cite{#2}, #3}
\newcommand{\zitx}[2]{#1\ \cite{#2}}
\newcommand{\footzit}[3]{\footnote{\zit{#1}{#2}{#3}}}
\newcommand{\footzitx}[2]{\footnote{\zitx{#1}{#2}}}

\newcommand{\ite}{\begin{itemize}}
\newcommand{\ete}{\end{itemize}}

\newcommand{\bfr}{\begin{frame}}
\newcommand{\efr}{\end{frame}}

\newcommand{\ilc}[1]{\texttt{#1}}


% misc

% multiframe
\usepackage{xifthen}% provides \isempty test
% new counter to now which frame it is within the sequence
\newcounter{multiframecounter}
% initialize buffer for previously used frame title
\gdef\lastframetitle{\textit{undefined}}
% new environment for a multi-frame
\newenvironment{multiframe}[1][]{%
\ifthenelse{\isempty{#1}}{%
% if no frame title was set via optional parameter,
% only increase sequence counter by 1
\addtocounter{multiframecounter}{1}%
}{%
% new frame title has been provided, thus
% reset sequence counter to 1 and buffer frame title for later use
\setcounter{multiframecounter}{1}%
\gdef\lastframetitle{#1}%
}%
% start conventional frame environment and
% automatically set frame title followed by sequence counter
\begin{frame}%
\frametitle{\lastframetitle~{\normalfont \Roman{multiframecounter}}}%
}{%
\end{frame}%
}




% http://texfragen.de/hurenkinder_und_schusterjungen
\usepackage[all]{nowidow}



% force no overlong lines:
%\tolerance=1 % tolerance for how badly spaced lines are allowed, less means "better" lines
\tolerance=500 %  need more tolerance for equations
%\emergencystretch=\maxdimen
%\emergencystretch=200pt
%\setlength{\emergencystretch}{3em}
%\hyphenpenalty=10000 % forces no hyphenation
%\hbadness=10000


% http://tex.stackexchange.com/questions/35717/how-to-draw-arrows-between-parts-of-an-equation-to-show-the-math-distributive-pr
\tikzset{square arrow/.style={to path={ -- ++(.0,-.15)  -| (\tikztotarget)}}}
\tikzset{square arrow2/.style={to path={ -- ++(.0,-.25)  -| (\tikztotarget)}}}
%\tikzset{square arrow/.style={to path={ -- ++(00,-.01) -- ++(0.5,-0.1) -- ++(0.5,-0.1) -| (\tikztotarget)},color=red}}


% have arrows from a to b and from c to d here
% just use: texttext\arrowA texttest \arrowB texttext
\newcommand{\arrowA}{\tikz[overlay,remember picture] \node (a) {};}
\newcommand{\arrowB}{\tikz[overlay,remember picture] \node (b) {};}
\newcommand{\drawAB}{
	\tikz[overlay,remember picture]
	{\draw[->,bend left=5,color=red] (a.south) to (b.south);}
	%{\draw[->,square arrow,color=red] (a.south) to (b.south);}
}
\newcommand{\arrowAP}{\tikz[overlay,remember picture] \node (ap) {};}
\newcommand{\arrowBP}{\tikz[overlay,remember picture] \node (bp) {};}
\newcommand{\drawABP}{
	\tikz[overlay,remember picture]
	{\draw[->,bend right=5,color=red] (ap.south) to (bp.south);}
	%{\draw[->,square arrow,color=red] (a.south) to (b.south);}
}

\newcommand{\arrowAB}{\tikz[overlay,remember picture] \node (ab) {};}
\newcommand{\arrowBA}{\tikz[overlay,remember picture] \node (ba) {};}
\newcommand{\drawAABB}{
	\tikz[overlay,remember picture]
	%{\draw[->,bend left=80] (a.north) to (b.north);}
	{\draw[->,square arrow,color=brown] (ab.south) to (ba.south);
	\draw[->,square arrow,color=brown] (ba.south) to (ab.south);}
}


\newcommand{\arrowCD}{\tikz[overlay,remember picture] \node (cd) {};}
\newcommand{\arrowDC}{\tikz[overlay,remember picture] \node (dc) {};}
\newcommand{\drawCCDD}{
	\tikz[overlay,remember picture]
	%{\draw[->,bend left=80] (a.north) to (b.north);}
	{\draw[<->,dashed,square arrow,color=green] (cd.south) to (dc.south); }
}



\newcommand{\arrowC}{\tikz[overlay,remember picture] \node (c) {};}
\newcommand{\arrowD}{\tikz[overlay,remember picture] \node (d) {};}
\newcommand{\drawCD}{
	\tikz[overlay,remember picture]
	{\draw[->,square arrow,color=blue] (c.south) to (d.south);}
}

\newcommand{\arrowE}{\tikz[overlay,remember picture] \node (e) {};}
\newcommand{\arrowF}{\tikz[overlay,remember picture] \node (f) {};}
\newcommand{\drawEF}{
	\tikz[overlay,remember picture]
	{\draw[->,square arrow2,color=orange] (e.south) to (f.south);}
}


\newcommand{\arrAP}{\arrowAP}
\newcommand{\arrBP}{\arrowBP}
\newcommand{\arrA}{\arrowA}
\newcommand{\arrB}{\arrowB}
\newcommand{\arrC}{\arrowC}
\newcommand{\arrD}{\arrowD}
\newcommand{\arrE}{\arrowE}
\newcommand{\arrF}{\arrowF}


\DeclareMathOperator{\simgeq}{\scalebox{0.92}{$\gtrsim$}}

\newcommand{\refsub}[2]{\hyperref[#2]{\ref*{#1}.\ref*{#2}}}

%\newcommand{\sigmarange}[2]{\sigma_{#1}^{#2} }
\newcommand{\sigmarange}[2]{\sigma_{(#1,#2)} }
\newcommand{\sigmaz}[1]{\sigmarange{0}{#1} }
\newcommand{\sigmazi}[0]{\sigmaz{i} }

\DeclareMathOperator{\lit}{lit}

%\def\fCenter{\ \proves\ }
\def\fCenter{\proves}

\newcommand{\prflbl}[2]{\RightLabel{\footnotesize $#1, #2$} }
%\newcommand{\prflblid}[1]{\RightLabel{$#1, \id$} }
\newcommand{\prflblid}[1]{\RightLabel{\footnotesize $#1$} }

\DeclareMathOperator{\resruleres}{res}
\DeclareMathOperator{\resrulefac}{fac}
\DeclareMathOperator{\resrulepar}{par}
\newcommand{\lkrule}[2]{\ensuremath{\operatorname{#1}:#2}} % operatorname fixes spacing issues for =

\newcommand{\parti}[4]{\ensuremath{ \langle (#1; #2), (#3; #4)\rangle  }}

\newcommand{\partisym}{\ensuremath{\chi}}

\newcommand{\occur}[1]{\ensuremath{[#1]}}
\newcommand{\occ}[1]{\occur{#1}}

\newcommand{\occurat}[2]{\ensuremath{{\occur{#1}_{#2}}}}
\newcommand{\occat}[2]{\occurat{#1}{#2}}
\newcommand{\occatp}[1]{\occurat{#1}{p}}
\newcommand{\occatq}[1]{\occurat{#1}{q}}

\newcommand{\colterm}[1]{\zeta_{#1}}



% fix restateable spacing 
%http://tex.stackexchange.com/questions/111639/extra-spacing-around-restatable-theorems

\makeatletter

\def\thmt@rst@storecounters#1{%
%THIS IS THE LINE I ADDED:
\vspace{-1.9ex}%
  \bgroup
        % ugly hack: save chapter,..subsection numbers
        % for equation numbers.
  %\refstepcounter{thmt@dummyctr}% why is this here?
  %% temporarily disabled, broke autorefname.
  \def\@currentlabel{}%
  \@for\thmt@ctr:=\thmt@innercounters\do{%
    \thmt@sanitizethe{\thmt@ctr}%
    \protected@edef\@currentlabel{%
      \@currentlabel
      \protect\def\@xa\protect\csname the\thmt@ctr\endcsname{%
        \csname the\thmt@ctr\endcsname}%
      \ifcsname theH\thmt@ctr\endcsname
        \protect\def\@xa\protect\csname theH\thmt@ctr\endcsname{%
          (restate \protect\theHthmt@dummyctr)\csname theH\thmt@ctr\endcsname}%
      \fi
      \protect\setcounter{\thmt@ctr}{\number\csname c@\thmt@ctr\endcsname}%
    }%
  }%
  \label{thmt@@#1@data}%
  \egroup
}%

\makeatother




\newcommand{\mymark}[1]{\ensuremath{(#1)}}
\newcommand{\markA}{\mymark \circ}
\newcommand{\markB}{\mymark *}
\newcommand{\markC}{\mymark \divideontimes}

\newcommand{\wrong}[1]{{\color{red}WRONG: #1}}
\newcommand{\NB}[1]{{\color{blue}NB: #1}}
\newcommand{\hl}[1]{{\color{orange} #1}}
\newcommand{\mytodo}[1]{{\color{red}TODO: #1}}
\newcommand{\largered}[1]{{

	  \LARGE\bfseries\color{red}
		#1

}}
\newcommand{\largeblue}[1]{{

	  \large\bfseries\color{blue}
		#1

}}




\usepackage{ulem} %  \dotuline{dotty} \dashuline{dashing} \sout{strikethrough}
\normalem

\usepackage{tabu} % tabular also in math mode (and much more)

\usepackage[color]{changebar} %  \cbstart, \cbend
\cbcolor{red}



% http://tex.stackexchange.com/questions/7032/good-way-to-make-textcircled-numbers
\newcommand*\circled[1]{\tikz[baseline=(char.base)]{
\node[shape=circle,draw,inner sep=2pt] (char) {#1};}}



% http://tex.stackexchange.com/questions/43346/how-do-i-get-sub-numbering-for-theorems-theorem-1-a-theorem-1-b-theorem-2

\makeatletter
\newenvironment{subtheorem}[1]{%
  \def\subtheoremcounter{#1}%
  \refstepcounter{#1}%
  \protected@edef\theparentnumber{\csname the#1\endcsname}%
  \setcounter{parentnumber}{\value{#1}}%
  \setcounter{#1}{0}%
  \expandafter\def\csname the#1\endcsname{\theparentnumber.\Alph{#1}}%
  \ignorespaces
}{%
  \setcounter{\subtheoremcounter}{\value{parentnumber}}%
  \ignorespacesafterend
}
\makeatother
\newcounter{parentnumber}


\usepackage{tabularx}% http://ctan.org/pkg/tabularx
\newcolumntype{Y}{>{\centering\arraybackslash}X}

\newcommand{\mycols}[2][3]{
	\noindent\begin{tabularx}{\textwidth}{*{#1}{Y}}
		#2
	\end{tabularx}%
}


\newcommand{\definethms}{

	%\declaretheorem[title=Theorem,qed=$\triangle$,parent=chapter]{thm}
	\newcommand{\thmqed}{$\square$} % for thms without proof
	\newcommand{\propqed}{$\square$} % for props without proof
	\declaretheorem[title=Theorem]{thm}
	\declaretheorem[title=Proposition,sibling=thm]{prop}
	\declaretheorem[title=Conjectured Proposition,sibling=thm]{cprop}

	%\declaretheorem[title=Lemma,parent=chapter]{lemma}
	\declaretheorem[sibling=thm]{lemma}
	\declaretheorem[sibling=thm,title=Conjectured Lemma]{clemma}
	\declaretheorem[title=Corollary,sibling=thm]{corr}
	\declaretheorem[sibling=thm,title=Definition,style=definition,qed=$\triangle$]{defi}
	%\declaretheorem[title=Definition,qed=$\triangle$,parent=chapter]{defi}
	\declaretheorem[title=Example,style=definition,qed=$\triangle$,sibling=thm]{exa}

	\declaretheorem[sibling=thm,title=Conjecture]{conj}

	\declaretheorem[title=Remark,style=remark,numbered=no,qed=$\triangle$]{remark}


}

\usepackage[matha]{mathabx} % the locial operators here have more space around them and [ and ] are thicker, also langle and rangle are a bit nicer; subseteq looks a bit weird

%\usepackage{MnSymbol} % again other symbols


\newcommand{\inference}{\ensuremath{\iota}}

\usepackage{cases} % numcases



% subsections also in toc
\setcounter{tocdepth}{2}
\setsecnumdepth{subsection}


\definethms

\def\proofSkipAmount{ \vskip -0.1em }


%\usepackage{bussproof}

%\usepackage{vaucanson-g}
%\usepackage{amssymb}
\usepackage{latexsym}

% for color-highlighted code
%\usepackage{color} % for grey comments
%\usepackage{alltt}

%\usepackage[doublespacing]{setspace}
%\usepackage[onehalfspacing]{setspace}
%\usepackage[singlespacing]{setspace}


\usepackage{amsthm}


\chapterstyle{madsen}

% define page numbering styles
\makepagestyle{numberCorner}
\makeevenfoot{numberCorner}{\thepage}{}{}
\makeoddfoot{numberCorner}{}{}{\thepage}

\makepagestyle{numberCenter}
%\makeevenfoot{numberCenter}{}{\thepage}{}
%\makeoddfoot{numberCenter}{}{\thepage}{}
%
%\makeevenhead{numberCenter}{\thechapter}{}{\thesection}
%\makeoddhead{numberCenter}{\thesection }{}{\thechapter}
\makeheadrule{numberCenter}{\textwidth}{1pt}

\makeevenhead{numberCenter}{\thepage}{}{\leftmark}
\makeoddhead{numberCenter}{\rightmark}{}{\thepage}


\makeatletter
\makepsmarks{numberCenter}{
	\def\chaptermark##1{\markboth{%
			\ifnum \value{secnumdepth} > -1
			\if@mainmatter
			\chaptername\ \thechapter\ --- %
			\fi
			\fi
	##1}{}}
	\def\sectionmark##1{\markright{%
			\ifnum \value{secnumdepth} > 0
			\thesection. \ %
			\fi
	##1}}
}
\makeatother
\newcommand{\mysetpagestyle}{
	%\pagestyle{numberCorner}
	\pagestyle{numberCenter}
}
\mysetpagestyle





\usepackage{refcheck}

%\settypeblocksize{0.65\stockheight}{0.65\stockwidth}{*}
%\setlrmargins{*}{*}{1.2}
%\setulmargins{*}{*}{1.4}
%\checkandfixthelayout[nearest]


\begin{document}

\tableofcontents

\section{Lemmas other pdf}
\begin{lemma}
	\label{lemma:lifting_order_not_relevant}
	$\lifgammanovar{\lifdeltanovar{\varphi}} = \lifdeltanovar{\lifgammanovar{\varphi}}$.
\end{lemma}

\clearpage

\section{Proof}

\begin{defi}[$\tau(\inference)$]
	For an inference $\inference$ with $\sigma = \mgu(\inference)$, we define the infinite substitution\todo{define infinite substitutions properly and apply definition here}{} $\tau(\inference)$ with $\dom(\tau(\inference)) = \dom(\sigma) \cup \{z_s \mid s\sigma \neq s\}$ as follows for a variable $x$:

	\[
		x\tau(\inference) =
		\begin{cases}
			x\sigma & \text{$x$ is a non-lifting variable} \\
			z_{t\sigma} & \text{$x$ is a lifting variable $z_t$}
		\end{cases} 
	\]
		%\qedhere

	If the inference $\inference$ is clear from the context, we abbreviate $\tau(\inference)$ by $\tau$. 
\end{defi}

\begin{clemma}
	\label{lemma:lifting_tau_commute}
	For a formula or term $\varphi$,
	$\lifboth{\lifboth{ \varphi} \tau} =\nolinebreak \lifboth{ \varphi \tau } $.
\end{clemma}
\begin{proof}
	We proceed by induction.

	\begin{itemize}
		\item Suppose that $t$ is a grey constant or function symbol of the form $f(t_1, \dots, t_n)$.
			Then:
			\begin{align*}
				\lifboth{\lifboth{t}\tau} &= \lifboth{\lifboth{ f(t_1, \dots, t_n)}\tau} \\
																  &= \lifboth{ f(\lifboth{t_1}\tau, \dots, \lifboth{t_n}\tau) } \\
																	&= f(\lifboth{\lifboth{t_1}\tau}, \dots, \lifboth{\lifboth{t_n}\tau}) \\
																 &= f(\lifboth{t_1\tau}, \dots, \lifboth{t_n\tau}) \\
																	&= \lifboth{f(t_1, \dots, t_n)\tau} \\
																	&= \lifboth{t\tau}
			\end{align*}
		\item Suppose that $t$ is a colored constant or function symbol. Then:
			\[
				\lifboth{\lifboth{t}\tau}= \lifboth{z_t\tau} 
				= \lifboth{z_{t\sigma}} 
				= z_{t\sigma} 
				= \lifboth{t\sigma} 
				= \lifboth{t\tau}
			\]
		\item Suppose that $t$ is a variable $x$. Then:
			\[
				\lifboth{\lifboth{t}\tau} = \lifboth{\lifboth{ x}\tau} = \lifboth{x\tau}
			\]
		\item Suppose that $t$ is a lifting variable $z_t$. Then:
			\[
				\lifboth{\lifboth{z_t}\tau} = \lifboth{z_t\tau} 
				\qedhere
			\]
	\end{itemize}

\end{proof}



\begin{defi}[Incremental lifting]
	Let $\pi$ be a resolution refutation of $\Gamma \cup \Delta$.
	We define $\LI(\pi)$ ($\LIcl(\pi)$) to be $\LI(\square)$ ($\LIcl(\square)$), where $\square$ is the empty clause derived in $\pi$.

	Let $C$ be a clause in $\pi$. 
	For a literal $\lambda$ in $C$, we denote the corresponding literal in $\LIcl(C)$ by $\lambda\cll$, which is exists by lemma~\ref{lemma:li_vs_clause_plus_literals_equal}.

	We define $\LI(C)$ and $\lifiboth{C}$ as follows:

	\begin{itemize}
		\item[Base case.]
			If $C \in \Gamma$, $\LI(C) \defeq \bot$.
			If otherwise $C \in \Delta$, $\LI(C) \defeq \top$.

			In any case, $\lifiboth{C} \defeq \lifboth{C}$.
		\item[Resolution.]

			If the clause $C$ is the result of a resolution step $\inference$ of $C_1: D \lor l$ and $C_2: E \lor \lnot l'$ using a unifier $\sigma$ such that $l\sigma =  l'\sigma$, then define $\LI(C)$ and $\lifiboth{C}$ as follows:

			$\lifiboth{C} \defeq \lifboth{(\lifiboth{C_1} \setminus \{l\cll\})\tau} \spam\lor \lifboth{(\lifiboth{C_2}\setminus \{l\cll'\})\tau} $

			\begin{enumerate}

				\item If $l$ is $\Gamma$-colored:
					$\LI(C) \defeq \lifboth{\LI(C_1)\tau}\spas\lor \lifboth{\LI(C_2)\tau} $

				\item If $l$ is $\Delta$-colored:
					$\LI(C) \defeq \lifboth{\LI(C_1)\tau}\spas\land \lifboth{\LI(C_2)\tau} $

				\item If $l$ is grey:
					$\LI(C) \defeq
					(\lifboth{l\cll\tau} \land \lifboth{\LI(C_2)\tau}) \spas\lor
					(\lnot \lifboth{l'\cll\tau}\land \lifboth{\LI(C_1)\tau })
					$

			\end{enumerate}

		\item[Factorisation.]
			If the clause $C$ is the result of a factorisation step $\inference$ of $C_1: l \lor l' \lor D$ using a unifier $\sigma$ such that $l\sigma = l'\sigma$, then $\LI(C) \defeq \lifboth{\LI(C_1)\tau}$ and $\lifiboth{C} \defeq \lifboth{ (\lifiboth{C_1} \setminus\nolinebreak \{l'\cll\})\tau}$.
			\qedhere

	\end{itemize}

\end{defi}

\begin{lemma}
	\label{lemma:li_vs_clause_plus_literals_equal}
	Let $C$ be a clause in a resolution refutation of $\Gamma\cup\Delta$.

	Then for every literal $\lambda$ in $C$, there exists a literal $\lambda\cll$ in\nolinebreak{} $\LIcl(C)$ such that $\lambda\cll = \lifboth{\lambda}$ and for resolved or factorised literals $l$ and $l'$ of a resolution or factorisation inference $\inference$, we have that $\lifboth{l\cll\tau} = \lifboth{l'\cll\tau}$.
\end{lemma}
\begin{proof}
	We proceed by induction.
	\begin{description}
		\item{} Base case.
			For $C\in\Gamma\cup\Delta$, $\LIcl(C)$ is defined to be $\lifboth{C}$.

		\item{} Resolution/Factorisation.
			Suppose the clause $C$ is the result of a resolution or factorisation inference \inference{} of the clauses $\bbar C$ with $\sigma = \mgu(\inference)$.

			%Suppose the clause $C$ is the result of a resolution step \inference{} of $C_1: D \lor l$ and $C_2: E \lor \lnot l'$ with $\sigma = \mgu(\inference)$.

			For every literal in $C$, there exists a predecessor in a clause in $\bbar C$.
			Let $\lambda$ be a literal $C_i$ with $C_i\in\bbar C$, such that $\lambda$ is not the predecessor of the literal being resolved or factorised in $\inference$.
			Then $\lambda\sigma$ is occurs in $C$.

			By the induction hypothesis, $\lifboth{\lambda}$ occurs in $\LIcl(C_i)$.
			The successor of $\lifboth{\lambda}$ in $\LIcl(C)$ is of the form $\lifboth{\lifboth{\lambda}\tau}$.
			But by Lemma~\ref{lemma:lifting_tau_commute}, this is nothing else than $\lifboth{\lambda\tau}$.
			As no lifting variables occur in $\lambda$, we get by the definition of $\tau$ that 
			$\lifboth{\lambda\tau} = \lifboth{\lambda\sigma}$.

			Let $l$ and $l'$ be the resolved or factorised literals.
			In order to show that $\lifboth{l\cll\tau} = \lifboth{l'\cll\tau}$,
			consider that by the induction hypothesis, this is nothing else than
			$\lifboth{\lifboth{l}\tau} = \lifboth{\lifboth{l'}\tau}$.
			But by applying the same argument as above, this is equivalent to
			$\lifboth{l\sigma} = \lifboth{l'\sigma}$, which is implied by $l\sigma =\nolinebreak l'\sigma$.
			\qedhere

			\begin{comment}
				Let $l$ and $l'$ be the resolved or factorised literals.
				By the induction hypothesis,
				$l\cll = \lifboth{l}$
				and
				$l'\cll = \lifboth{l'}$.

				By a similar reasoning as above, we get that $\lifboth{\lifboth{\lambda}\tau} = \lifboth{\lambda\sigma}$ for any literal $\lambda$ in $\bbar C$.

				$\lifboth{\lambda\cll\tau} = \lifboth{\lifboth{\lambda}\tau}$
				But as $l\sigma = l'\sigma$ and

				$\lifboth{\lifboth{l}\tau } =
				\lifboth{\lifboth{l'}\tau }
				$

				As no lifting variables occur in $l$ or $l'$, we get that $l\tau = l'\tau$, which we can lift to $\lifboth{l\tau} = \lifboth{l'\tau}$.

				Note that $l\sigma = l'\sigma$.

				$l\cll = \lifboth{l}$

				$\lifboth{ \lifboth{l} \tau} = \lifboth{l\tau}$

			\end{comment}
	\end{description}
\end{proof}


\begin{defi}
	$\LIde(C)$ ($\lifidelta{C}$) for a clause $C$ is defined as $\LI(C)$ ($\lifiboth{C}$) with the difference that in its inductive definition, every lifting $\lifboth{\varphi}$ for a formula or term $\varphi$ is replaced by a lifting of only the $\Delta$-terms $\lifdeltanovar{\varphi}$.
\end{defi}

\begin{remark}
	Many results involving $\LI(C)$ ($\LIcl(C)$) are valid for $\LIde(C)$ ($\LIclde(C)$) 
	in a formulation which is adapted accordingly.
	This can easily be seen by the following proof idea:

	Let $f_1, \dots, f_n$ be all $\Gamma$-colored function or constant symbols,
	$c$ a fresh constant symbol and $g$ be a fresh $n$-ary function symbol.
	Construct a formula $t = t$ such that $t = g(t_1, \dots, t_n)$,
	such that $t_i = f_i(c_1, \dots, c_m)$ for $1\varleq i \varleq n$ where $m$ is the arity of $f_i$ and $c_j = c$ for $1\varleq j \varleq m$. Let $\Delta' = \Delta$ and apply the desired result to the initial clause sets $\Gamma$ and $\Delta'$.

	Under this construction, every originally $\Gamma$-colored symbol is now grey, which implies that $\LI(C) = \LIde(C)$ as well as $\LIcl(C) = \LIclde(C)$.
	But $\Delta\entails \varphi \semiff \Delta'\entails \varphi$ for any formula $\varphi$.
\end{remark}

\begin{lemma}
	\label{lemma:no_colored_terms}
	Let $C$ be a clause of a resolution refutation $\pi$ of $\Gamma\cup\Delta$.
	$\LI(C)$ and $\LIcl(C)$ do not contain colored symbols.
	$\LIde(C)$ and $\LIclde(C)$ do not contain \mbox{$\Delta$-c}olored symbols.
\end{lemma}
\begin{proof}
	For $\LI(C)$ and $\LIcl(C)$, consider the following:
	In the base case of the inductive definitions of $\LI(C)$ and $\LIcl(C)$, no colored symbols occur.
	In the inductive steps, any colored symbol which is added by $\tau$ to intermediary formulas is lifted.

	For $\LIde(C)$ and $\LIclde(C)$, a similar argument goes through by reading colored as $\Delta$-colored.
\end{proof}

\begin{lemma}
	\label{lemma:substitute_and_lift}
	Let $\sigma$ be a substitution and $F$ a formula without $\Phi$-colored terms such that for a set of formulas $\Psi$, $\Psi \entails F$.
	Then $\Psi \entails \lifphinovar{F\sigma}$.
\end{lemma}
\begin{proof}
	$\lifphinovar{F\sigma}$ is an instance of $F$:
	$\sigma$ substitutes variables either for terms not containing $\Phi$-colored symbols or by terms containing $\Phi$-colored symbols.
	For the first kind, the lifting has no effect.
	For the latter, the lifting only replaces subterms of the terms introduced by the substitution by a lifting variable such that the original structure of $F$ remains invariant as it by assumption does not contain colored terms.
\end{proof}



\begin{lemma}
	\label{lemma:gamma_proves_pide}
	For a clause $C$ in a resolution refutation of $\Gamma\cup\Delta$, 
	$\Gamma\entails \LIde(C) \lor \lifidelta{C}$.
\end{lemma}
\begin{proof}
	We proceed by induction of the strengthening $\Gamma\entails \LIde(C) \lor \lifidelta{C_\Gamma}$.

	\begin{description}
		\item{} Base case.
			For $C\in\Gamma$, $\LIclde(C_\Gamma) = \lifdeltanovar{C} = C$. Hence $\Gamma \entails \LIclde(C_\Gamma)$.

			For $C\in\Delta$, $\LIde(C) = \top$, so $\Gamma \entails \LIde(C)$.

		\item{} Resolution.
			Suppose the clause $C$ is the result of a resolution step \inference{} of $C_1: D \lor l$ and $C_2: E \lor \lnot l'$ with $\sigma = \mgu(\inference)$.

			\newcommand{\clauseOnePrime}{\LIclde( (C_1)_\Gamma )^*}
			\newcommand{\clauseTwoPrime}{\LIclde( (C_2)_\Gamma )^*}

			We define the following abbreviations:

			$\clauseOnePrime = \LIclde( (C_1)_\Gamma \setminus \{l\cllde\} )$

			$\clauseTwoPrime = \LIclde( (C_2)_\Gamma \setminus \{\lnot l'\cllde\} )$

			Hence the induction hypothesis can be stated as follows:

			$\Gamma \entails \LIde(C_1) \lor \clauseOnePrime \lor (l\cllde)_\Gamma$

			$\Gamma \entails \LIde(C_2) \lor \clauseTwoPrime \lor \lnot (l'\cllde)_\Gamma$

			By Lemma~\ref{lemma:no_colored_terms}, $\LIde(C_i)$ and $\LIclde(C_i)$ for $i\in\{1,2\}$ do not contain $\Delta$-colored terms. 
			Hence we are able to apply Lemma~\ref{lemma:substitute_and_lift} in order to obtain

			$\Gamma \stackrel\markA\entails \lifdeltanovar{\LIde(C_1)\tau} \lor \lifdeltanovar{\clauseOnePrime\tau} \lor \lifdeltanovar{(l\cllde)_\Gamma\tau}$

			$\Gamma \stackrel\markB\entails \lifdeltanovar{\LIde(C_2)\tau} \lor \lifdeltanovar{\clauseTwoPrime\tau} \lor \lnot \lifdeltanovar{(l'\cllde)_\Gamma\tau}$

			By Lemma~\ref{lemma:li_vs_clause_plus_literals_equal}, we obtain that
			$\lifdeltanovar{l\cllde\tau} = 
			\lifdeltanovar{l'\cllde\tau}$.

			Now we distinguish cases based on the color of the resolved literal:

			\begin{itemize}
				\item Suppose that $l$ is $\Gamma$-colored.
					Then as
					$\lifdeltanovar{l\cllde\tau} = 
					\lifdeltanovar{l'\cllde\tau}$, 
					we can perform a resolution step on \markA{} and \markB{}, which gives that
					$\Gamma \entails
					\lifdeltanovar{\LIde(C_1)\tau} \spas\lor \lifdeltanovar{\clauseOnePrime\tau} \spam\lor 
					\lifdeltanovar{\LIde(C_2)\tau} \spas\lor \lifdeltanovar{\clauseTwoPrime\tau}$.
					This however is nothing else than $\Gamma\entails \LIde(C) \lor \LIclde(C)$.

				\item Suppose that $l$ is $\Delta$-colored. Then \markA{} and \markB{} simply to the following:

					$\Gamma \entails \lifdeltanovar{\LIde(C_1)\tau} \lor \lifdeltanovar{\clauseOnePrime\tau}$

					$\Gamma \entails \lifdeltanovar{\LIde(C_2)\tau} \lor \lifdeltanovar{\clauseTwoPrime\tau}$

					These however imply that 
					$\Gamma \entails 
					\clauseOnePrime \spas\lor \clauseTwoPrime \spas\lor\allowbreak
					(\lifdeltanovar{\LIde(C_1)\tau} \land
					\lifdeltanovar{\LIde(C_2)\tau} )$, which is nothing else than
					$\Gamma \entails \LIde(C) \lor \LIclde(C)$.

				\item Suppose that $l$ is grey.
					Suppose that $M$ is a model of $\Gamma$ such that
					%$M\notentails \lifdeltanovar{\LIclde(C_1)\tau} \lor \lifdeltanovar{\LIclde(C_2)\tau}$.
					$M\notentails \LIclde(C)$, i.e.\ 
					$M\notentails \lifdeltanovar{\clauseOnePrime\tau} \lor \lifdeltanovar{\clauseTwoPrime\tau}$.
					Then $M \entails \lifdeltanovar{\LIde(C_1)\tau} \lor \lifdeltanovar{l\cllde\tau}$
					as well as 
					$M \entails \lifdeltanovar{\LIde(C_2)\tau} \lor \lnot \lifdeltanovar{l'\cllde\tau}$.

					Due to $\lifdeltanovar{l\cllde\tau} = 
					\lifdeltanovar{l'\cllde\tau}$,
					we obtain that

					$M\entails (\lifdeltanovar{l\cllde\tau} \land \lifdeltanovar{\LIde(C_2)\tau}) \spam\lor
					(\lnot \lifdeltanovar{l'\cllde\tau} \land \lifdeltanovar{\LIde(C_1)\tau})$,
					which is nothing else than $M\entails \LIde(C)$.

			\end{itemize}



		\item{} Factorisation. 
			Suppose the clause $C$ is the result of a factorisation inference $\inference$ of $C_1: l \lor l' \lor D$ with $\sigma=\mgu(\inference)$.

			We introduce the abbreviation
			$\LIclde( (C_1)_\Gamma)^* = \LIclde( (C_1)\Gamma \setminus \{l\cllde, \lnot l'\cllde\} )$
			and express the induction hypothesis as follows:

			$\Gamma \entails \LIde(C_1) \lor \LIclde( (C_1)_\Gamma)^* \lor (l\cllde)_\Gamma \lor \lnot (l'\cllde)_\Gamma$

			By Lemma~\ref{lemma:no_colored_terms}, $\LIde(C_i)$ and $\LIclde(C_i)$ for $i\in\{1,2\}$ do not contain $\Delta$-colored terms. 
			Hence we are able to apply Lemma~\ref{lemma:substitute_and_lift} in order to obtain

			$\Gamma \stackrel\markC\entails \lifdeltanovar{\LIde(C_1)\tau} \lor \lifdeltanovar{\LIclde( (C_1)_\Gamma)^*\tau} \lor \lifdeltanovar{(l\cllde)_\Gamma\tau} \lor \lnot \lifdeltanovar{(l'\cllde)_\Gamma\tau}$

			As by Lemma~\ref{lemma:li_vs_clause_plus_literals_equal} we get that 
			$\lifdeltanovar{l\cllde\tau} = 
			\lifdeltanovar{l'\cllde\tau}$,
			we can perform a factorisation step on \markC{} to obtain that 
			$\Gamma \entails \lifdeltanovar{\LIde(C_1)\tau} \lor \lifdeltanovar{\LIclde( (C_1)_\Gamma)^*\tau} \lor \lifdeltanovar{(l\cllde)_\Gamma\tau}$.
			But this is nothing else than $\Gamma \entails \LIde(C) \lor \LIclde(C_\Gamma)$.
			\qedhere

	\end{description}
\end{proof}

\begin{defi}[Quantifier block]
	\label{def:arrow_quantifier_block}
	Let $C$ be a clause in a resolution refutation $\pi$ of $\Gamma\cup\Delta$
	and $\bar x$ be the $\Delta$-lifting variables and $\bar y$ the $\Gamma$-lifting variables occurring in $\LI(C)$ and $\lifiboth{C}$.
	$\Q(C)$ denotes an arrangement of the elements of  $\{ \forall x_t \mid\nolinebreak x_t \in \bar x\} \cup \{ \exists y_t \mid y_t \in \bar y\}$ such that for two lifting variable $z_s$ and $z_r$, if $s$ is a subterm of $r$, then $z_s$ is listed before $z_r$.
	We denote $\Q(\square)$ by $\Q(\pi)$.
\end{defi}


\begin{lemma}
	\label{lemma:gamma_lifted_lide}
	For a clause $C$ of a resolution refutation of $\Gamma\cup\Delta$, 
	$\lifgammanovar{\LIde(C)} = \LI(C)$ and $\lifgammanovar{\LIclde(C)} = \LIcl(C)$.
\end{lemma}
\begin{proof}
	We proceed by induction.
	\begin{description}
		\item{} Base case.
			For $C \in \Gamma\cup\Delta$, $\LIclde(C) = \lifdeltanovar{C}$.
			By Lemma~\ref{lemma:lifting_order_not_relevant}, $\lifgammanovar{\lifdeltanovar{C}} = \lifboth{C}$,
			so $\lifgammanovar{\LIclde{C}} = \lifboth{C} = \LIclde(C)$.

			$\LIde(C)$ does not contain colored symbols.

		\item{} Inductions step.
			Suppose the clause $C$ is the result of a resolution or factorisation inference \inference{} of the clauses $\bbar C$.

			%$\LIde(C)$ and $\LI(C)$ as well as $\LIclde(C)$ and $\LIcl(C)$ differ only in the liftings, or more precisely:

			Every literal in $\LI(C)$ or $\LIcl(C)$ is of the form $\lifboth{\lambda\tau}$ for some $\lambda$ in $\LI(C_i)$ or $\LIcl(C_i)$ for some $C_i \in \bbar C$.

			Every literal in $\LIde(C)$ or $\LIclde(C)$ is of the form $\lifdeltanovar{\lambda\tau}$ for some $\lambda$ in $\LIde(C_i)$ or $\LIclde(C_i)$ for some $C_i \in \bbar C$.

			%Every literal in $\LI(C)$ ($\LIde(C)$) is of the form $\lifboth{\lambda\tau}$ ($\lifdelta{\lambda\tau}$) for $\lambda$ in $\LI(C_i)$ ($\LIde(C_i)$) or $\LIcl(C_i)$ ($\LIclde(C_i)$), $i\in\{1,2\}$.

			%Every literal in $\LIcl(C)$ ($\LIclde(C)$) is of the form $\lifboth{\lambda\tau}$ ($\lifdelta{\lambda\tau}$) for $\lambda$ in $\LIcl(C_i)$ ($\LIclde(C_i)$), $i\in\{1,2\}$.

			Hence it suffices to show that for a literal $\lambda$ in $\LIde(C_i)$ or $\LIclde(C_i)$ and its corresponding literal $\kappa$ in $\LI(C_i)$ or $\LIcl(C_i)$ for some $C_i \in \bbar C$
			that $\lifgammanovar{ \lifdeltanovar{\lambda\tau} } = \lifboth{\kappa\tau}$.
			%As $\lambda$ corresponds to a literal in $\LIde$ or $\LIclde$, we may by Lemma~\ref{lemma:no_colored_terms} additionally assume that no $\Delta$-colored term occurs in $\lambda$.

			By the induction hypothesis, $\lifgammanovar{\lambda} = \kappa$.
			By Lemma~\ref{lemma:no_colored_terms}, no $\Delta$-terms occur in $\lambda$.
			Hence 
			$\lifboth{\lambda} = \kappa$ and also
			$\lifboth{\lifboth{\lambda}\tau} = \lifboth{\kappa\tau}$.
			By Lemma~\ref{lemma:lifting_tau_commute},
			$\lifboth{\lambda\tau} = \lifboth{\kappa\tau}$, 
			which by Lemma~\ref{lemma:lifting_order_not_relevant} is nothing else than
			$\lifgammanovar{\lifdeltanovar{\lambda\tau}} = \lifboth{\kappa\tau}$.
			\begin{comment}

				For the induction step, it suffices to show that for a literal $\lambda$ with $\lifgammanovar{\lambda} = \kappa$
				that $\lifgammanovar{\lifdeltanovar{\lambda} \tau} = \lifboth{\kappa\tau}$.

				We abbreviate\
				$C_1\setminus \{l\}$ by $C_1^*$ and
				$C_2\setminus \{\lnot l'\}$ by $C_2^*$.

				Note that each literal in $\LIclde(C)$ is of the form $\lifdeltanovar{\lambda\tau}$ for some literal $\lambda$ in $\LIclde(C_1)$ or $\LIclde(C_2)$.
				Hence by the induction hypothesis, $\lifgammanovar{\lambda}$ is contained in $\LIcl(C_1)$ or $\LIcl(C_1)$.
				Note that every literal in $\LIcl(C)$ is of this form $\lifboth{\lifgammanovar{\lambda}\tau}$ for some $\lambda$ in $\LIclde(C_1)$ or $\LIclde(C_2)$.
				We show that $\lifboth{\lifgammanovar{\lambda}\tau} = \lifgammanovar{\lifdeltanovar{\lambda\tau}}$.

				As $\lambda$ is contained in $\LIclde(C_1)$ or $\LIclde(C_2)$, by Lemma~\ref{lemma:no_colored_terms}, $\lambda$ does not contain $\Delta$-colored terms.
				Hence $\lifboth{\lifgammanovar{\lambda}\tau} =
				\lifboth{\lifboth{\lambda}\tau}$.
				But by Lemma~\ref{lemma:lifting_tau_commute}, 
				$\lifboth{\lifboth{\lambda}\tau} = \lifboth{\lambda\tau}$, which by Lemma~\ref{lemma:lifting_order_not_relevant} is nothing else than $\lifgammanovar{\lifdeltanovar{\lambda\tau}}$.


				We now distinguish cases based on the color of the resolved literal:
				\begin{itemize}
					\item Suppose $l$ is $\Gamma$-colored.
				\end{itemize}

				~

				$\LIclde$:

				$\lifgammanovar{\LIclde(C_1)} = \LIcl(C_1)$

				$\lifdeltanovar{\LIclde(C_1)\tau} \subseteq \LIclde(C)$ 

				%$C_1\sigma \subseteq C$

				$\lifboth{\LIcl(C_1)\tau} \subseteq \LIcl(C)$

				to show: $\lifgammanovar{ \LIclde( C)} = \LIcl(C)$

				$ \lifboth{ \lifgammanovar{\LIclde(C_1)} \tau} = \lifboth{ \LIcl(C_1)\tau}$ $\quad$ IH + same op on both sides

				new lemma above

				$ \lifboth{ \lifgammanovar{\LIclde(C_1)} \tau} = 
				\lifboth{ \LIclde(C_1) \tau} $
				~

				$\LIde$:

				\begin{itemize}
					\item Supp $\Gamma$:

						IH: $\lifgammanovar{ \LIde(C_1) } = \LI(C_1)$

						hence also: $\lifboth{ \LIde(C_1) } = \LI(C_1)$ (by lemma: no $\Delta$-terms in \dots)

						+ $\tau$:
						$\lifboth{ \LIde(C_1) }\tau = \LI(C_1)\tau$ 

						+ $\ell$:
						$\lifboth{\lifboth{ \LIde(C_1) }\tau} = \lifboth{\LI(C_1)\tau}$ 

						by new lemma
						$\lifboth{ \LIde(C_1) \tau} = \lifboth{\LI(C_1)\tau}$ 

						hence by Lemma~\ref{lemma:lifting_order_not_relevant}, $\lifgammanovar{\lifdeltanovar{\LIde(C_1)\tau}} \subseteq \LIde(C)$

						hence $\lifgammanovar{\LIde(C)} \subseteq \LIde(C)$



				\end{itemize}

			\end{comment}
			\qedhere
	\end{description}

\end{proof}


\begin{lemma}
	\label{lemma:gamma_proves_quantified_lide}
	For a clause $C$ of a resolution refutation of $\Gamma\cup\Delta$, 
	$\Gamma\entails \Q(C)(\LI(C) \lor \LIcl(C))$.
\end{lemma}
\begin{proof}
	By Lemma~\ref{lemma:gamma_lifted_lide}
	$\lifgammanovar{ \LIde(C) \lor \LIclde(C)} = \LI(C) \lor \LIcl(C)$.

	By Lemma~\ref{lemma:gamma_proves_pide},
	$\Gamma\entails \LIde(C) \lor \LIclde(C)$. 
	Hence the terms in $\LIde(C) \lor \LIclde(C)$ provide witness terms for the $\Gamma$-lifting variables in $\LI(C) \lor \LIcl(C)$, which are existentially quantified in $\Q(C) (\LI(C) \lor \LIcl(C))$.

	Furthermore, the ordering imposed on the quantifiers in $\Q(C)$ implies that if a $\Delta$-lifting variable $x_s$ occurs in a witness term for a $\Gamma$-lifting variable $y_r$, $y_r$ is quantified in the scope of the quantifier of $x_s$ as $s$ is a subterm of $r$.
	This however ensures that the witness terms are valid.
\end{proof}

\begin{lemma}
	\label{lemma:li_symmetry}
	Let $\pi$ be a refutation of $\Gamma\cup\Delta$ and $\bhat \pi$ be $\pi$ with $\bhat \Gamma = \Delta$ and $\bhat \Delta = \Gamma$.
	Then for a clause $C$ in $\pi$ and its corresponding clause $\bhat C$ in $\bhat \pi$, $\Q(C)(\LI(C)) \spas\semiff \Q(\bhat C)(\LI(\bhat C))$.
\end{lemma}
\begin{proof}
	Note that $\LIcl$ is defined irrespective of the coloring, so $\LIcl(C) = \LIcl(\bhat C)$.

	Consider furthermore that liftings variables of $C$ and $\bhat C$ only differ in the variable symbol, but not in the index, and that the quantifier type of any given lifting variable in $C$ is exactly contrary to the corresponding one in $\bhat C$.
	Hence for any formula $\phi$, $\Q(C) \lnot \phi \spas\semiff \lnot \Q(\bhat C) \phi$.

	It remains to show that $\LI(C) \semiff \lnot \LI(\bhat C)$, which we do be induction:

	\begin{itemize}
		\item[Base case.]
			If $C \in \Gamma$, then $\LI(C) = \bot \semiff \lnot \top \semiff \lnot \LI(\bhat C)$ as $\bhat C\in\Delta$. 
			The case for $C\in\Delta$ can be argued analogously.

		\item[Resolution.]
			Suppose the clause $C$ is the result of a resolution step \inference{} of $C_1: D \lor l$ and $C_2: E \lor \lnot l'$ with $\sigma = \mgu(\inference)$.

			As $\tau$ depends only on $\sigma$,
			$\tau$ is the same for both $\pi$ and $\bhat \pi$.

			We now distinguish the following cases:

			\begin{enumerate}

				\item $l$ is $\Gamma$-colored:
					\begin{align*}
						\hspace*{\dimexpr-\leftmargini-\leftmarginii}
						\LI(C)	&= \lifboth{\LI(C_1)\tau}\spas\lor \lifboth{\LI(C_2)\tau} \\
															 &\semiff \lnot (\lnot\lifboth{\LI(C_1)\tau}\spas\land \lnot \lifboth{\LI(C_2)\tau}) \\
															 %&\semiff \lnot (\lifboth{(\lnot \LI(C_1))\tau}\spas\land \lifboth{(\lnot \LI(C_2))\tau}) \\
															 &\semiff \lnot (\lifboth{\LI(\bhat C_1)\tau}\spas\land \lifboth{\LI(\bhat C_2)\tau}) \\
															 &= \lnot \LI(\bhat C)
					\end{align*}

				\item $l$ is $\Delta$-colored:
					This case can be argued analogously

				\item $l$ is grey:
					Note that by Lemma~\ref{lemma:li_vs_clause_plus_literals_equal}, $\lifboth{l\cll\tau} = \lifboth{l'\cll\tau}$ \markB.
					\begin{align*}
						\hspace*{\dimexpr-\leftmargini-\leftmarginii}
						\LI(C) &\stackrel{{\phantom{\markB}}}=
						(\lnot {\lifboth{l'\cll\tau}} \land \lifboth{\LI(C_1)\tau}) \spam\lor 
						(\lifboth{l\cll\tau} \land \lifboth{\LI(C_2)\tau})\\
						&\stackrel{{\markB}}\semiff\,
						({\lifboth{l'\cll\tau}} \lor \lifboth{\LI(C_1)\tau}) \spam\land 
						(\lnot \lifboth{l\cll\tau} \lor \lifboth{\LI(C_2)\tau})\\
						&\stackrel{{\phantom{\markB}}}\semiff \lnot \Big( (\lnot {\lifboth{l'\cll\tau}} \land \lnot \lifboth{\LI(C_1)\tau}) \spam\lor 
						(\lifboth{l\cll\tau} \land \lnot\lifboth{\LI(C_2)\tau}) \Big) \\
						&\stackrel{{\phantom{\markB}}}=\lnot \Big( (\lnot {\lifboth{\bhat l'\cll\tau}} \land \lifboth{\LI(\bhat C_1)\tau}) \spam\lor 
						(\lifboth{\bhat l\cll\tau} \land \lifboth{\LI(\bhat C_2)\tau}) \Big)\\
						& \stackrel{{\phantom{\markB}}}= \LI(\bhat C) 
					\end{align*}


			\end{enumerate}



		\item[Factorisation.]
			Suppose the clause $C$ is the result of a factorisation $\inference$ of $C_1: l \lor l' \lor D$ 
			with $\sigma = \mgu(\inference)$.

			Then $\LI(C) = \lifboth{\LI(C_1)\tau}$, so the construction is not influenced by the coloring and the induction hypothesis gives the result.
			\qedhere
	\end{itemize}
\end{proof}

\begin{thm}
	Let $\pi$ be a resolution refutation of $\Gamma\cup\Delta$.
	Then $\LI( \pi )$ is an interpolant.
\end{thm}
\begin{proof}
	By Lemma~\ref{lemma:gamma_proves_quantified_lide}
	$\Gamma \entails \Q(\pi)(\LI(\pi) \lor \LIcl(\pi))$.
	But as $\LIcl(\pi) = \square$, this simplifies to
	$\Gamma \entails \Q(\pi)\LI(\pi)$.

	By constructing a proof $\bhat \pi$ from $\pi$ with $\bhat \Gamma = \Delta$ and $\bhat \Delta = \Gamma$, we obtain by Lemma~\ref{lemma:gamma_proves_quantified_lide} that $\bhat\Gamma \entails \Q(\bhat \pi)\LI(\bhat \pi)$.
	By Lemma \ref{lemma:li_symmetry}, this however is nothing else than
	$\Delta \entails \lnot \Q(\pi) \LI(\pi)$. 

	As furthermore by construction no colored symbols occur in $\Q(\pi) \LI(\pi)$, this formula is an interpolant for $\Gamma\cup\Delta$.
\end{proof}




\end{document}
