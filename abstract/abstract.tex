\documentclass[,%fontsize=11pt,%
			paper=a4,% 
			%DIV12, % mehr text pro seite als defaultyyp
			DIV12,
			%DIV=calc,%
			%twoside=false,%
			liststotoc,
			bibtotoc,
			draft=false,% final|draft % draft ist platzsparender (kein code, bilder..)
			%titlepage,
			numbers=noendperiod
			]{scrartcl}


\usepackage[utf8]{inputenc}
\usepackage[T1]{fontenc}
\usepackage[english]{babel}

\usepackage{enumerate}
\usepackage{paralist}
\usepackage{tikz}
\usetikzlibrary{shapes,arrows,backgrounds,graphs,%
matrix,patterns,arrows,decorations.pathmorphing,decorations.pathreplacing,%
positioning,fit,calc,decorations.text,shadows%
}


\usepackage{comment} 

\usepackage{etex} % else error on too many packages

% includes
\usepackage{algorithm}
%\usepackage{algorithmic} % conflicts with algpseudocode
\usepackage{algpseudocode}
%\newcommand*\Let[2]{\State #1 $\gets$ #2}
\algrenewcommand\alglinenumber[1]{
{\scriptsize #1}}
\algrenewcommand{\algorithmicrequire}{\textbf{Input:}}
\algrenewcommand{\algorithmicensure}{\textbf{Output:}}


%\usepackage[multiple]{footmisc} % footnotes at the same character separated by ','

\usepackage{multicol}

\usepackage{tikz}
\usetikzlibrary{shapes,arrows,backgrounds,graphs,%
matrix,patterns,arrows,decorations.pathmorphing,decorations.pathreplacing,%
positioning,fit,calc,decorations.text,shadows%
}

\usepackage{bussproofs}
\EnableBpAbbreviations


\usepackage{amsmath}
\usepackage{amsthm}
\usepackage{amssymb} % the reals
\usepackage{mathtools} % smashoperator

\usepackage{pst-node} % http://tex.stackexchange.com/questions/35717/how-to-draw-arrows-between-parts-of-an-equation-to-show-the-math-distributive-pr

\usepackage{stackengine}

\usepackage{thmtools} % advanced thm commands (declaretheorem)


\usepackage{nameref} % reference name of thm instead of counter

\usepackage{todonotes}

% conflict with beamer
%\usepackage{paralist} % compactenum

\usepackage{hyperref}
%\hypersetup{hidelinks}  % don't give options to usepackage, it doesn't work with beamer
%\hypersetup{colorlinks=false}  % don't give options to usepackage, it doesn't work with beamer



% note: this breaks beamer itemize
% \usepackage{enumitem} % labels for enumerate


\usepackage{url} 


\usepackage[format=hang,justification=raggedright]{caption}% or e.g. [format=hang]

\usepackage{cancel} % \cancel

\usepackage{lineno}


% commands

% logic etcs
%\newcommand{\ex}[2]{\bigskip\section*{Exercise #1: \begin{minipage}[t]{.80\linewidth} \small \textnormal{\it #2} \end{minipage} } }

\newcommand{\ex}[2]{\bigskip \noindent\textbf{Exercise #1.} #2 \smallskip}


\newcommand{\true}[0]{\textbf{1}}
\newcommand{\false}[0]{\textbf{0}}
\newcommand{\tr}{\true}
\newcommand{\fa}{\false}

\newcommand{\ra}{\rightarrow}
\newcommand{\Ra}{\Rightarrow}
\newcommand{\la}{\leftarrow}
\newcommand{\La}{\Leftarrow}

\newcommand{\lra}{\leftrightarrow}
\newcommand{\Lra}{\Leftrightarrow}

\newcommand{\NKZ}{\textbf{NK2}}

\DeclareMathOperator{\limpl}{\supset}
\DeclareMathOperator{\liff}{\Lra}
\newcommand{\union}{\cup}
\newcommand{\bigunion}{\bigcup}
\newcommand{\intersection}{\cap}
\newcommand{\bigintersection}{\bigcap}
\newcommand{\intersect}{\intersection}
\newcommand{\bigintersect}{\bigintersection}

\newcommand{\powerset}{\mathcal{P}}

\newcommand{\entails}{\models}
\newcommand{\proves}{\vdash}

\newcommand{\vm}{\ensuremath{\vv_\mathcal{M}}}
\newcommand{\Dia}{\ensuremath{\lozenge}}

\newcommand{\spaced}[1]{\ \ #1 \ \ }
\newcommand{\spa}[1]{\spaced{#1}}
\newcommand{\spas}[1]{\;{#1}\;}

% functions
\DeclareMathOperator{\sk}{sk}
\DeclareMathOperator{\mgu}{mgu}
\DeclareMathOperator{\Fun}{FS}
\DeclareMathOperator{\Pred}{PS}
\DeclareMathOperator{\Lang}{L}
\DeclareMathOperator{\ar}{ar}
\DeclareMathOperator{\PI}{PI}
\DeclareMathOperator{\Congr}{Congr}
\DeclareMathOperator{\Refl}{Refl}
%\newcommand{\sk}{\ensuremath{\mathrm{sk}}}
%\newcommand{\mgu}{\ensuremath{\mathrm{mgu}}}
%\newcommand{\Fun}{\ensuremath{\mathrm{FS}}}
%\newcommand{\Pred}{\ensuremath{\mathrm{PS}}}
%\newcommand{\PI}{\ensuremath{\mathrm{PI}}}
%\newcommand{\Lang}{\ensuremath{\mathrm{L}}}
%\newcommand{\ar}{\ensuremath{\mathrm{ar}}}

\newcommand{\PS}{\Pred}
\newcommand{\FS}{\Fun}

\DeclareMathOperator{\LangSym}{\mathcal{L}}

%\newcommand{\Trans}{\ensuremath{\mathrm{T}}}
%\newcommand{\Trans}{\ensuremath{\mathrm{T}}}
\DeclareMathOperator{\Trans}{T}
\DeclareMathOperator{\TransInv}{T^{-1}}

\DeclareMathOperator{\FAX}{F_{Ax}}
\DeclareMathOperator{\EAX}{E_{Ax}}
%\newcommand{\FAX}{\ensuremath{\mathrm{F_{Ax}}}}
%\newcommand{\EAX}{\ensuremath{\mathrm{E_{Ax}}}}

%\newcommand{\TransAll}{\ensuremath{\Trans_{\mathrm{Ax}}}}
\DeclareMathOperator{\TransAll}{\Trans_{Ax}}
%\newcommand{\FAX}{\ensuremath{\mathrm{F_{Ax}}}}

\DeclareMathOperator{\defeq}{\stackrel{\mathrm{def}}{=}}

\newcommand{\subst}[1]{[#1]}
\newcommand{\termsubst}[1]{\{#1\}}

%\newcommand{\lift}[3]{\operatorname{Lift}_{#1}(#2; #3)}
\newcommand{\lift}[3]{\operatorname{Lift}_{#1,#3}(#2)}

\newcommand{\UICm}[1]{\UnaryInfCm{#1}}
\newcommand{\UnaryInfCm}[1]{\UnaryInfC{$#1$}}
\newcommand{\BICm}[1]{\BinaryInfCm{#1}}
\newcommand{\BinaryInfCm}[1]{\BinaryInfC{$#1$}}
\newcommand{\RightLabelm}[1]{\RightLabel{$#1$}}
\newcommand{\LeftLabelm}[1]{\LeftLabel{$#1$}}
\newcommand{\AXCm}[1]{\AxiomCm{#1}}
\newcommand{\AxiomCm}[1]{\AxiomC{$#1$}}
\newcommand{\mt}[1]{\textnormal{#1}}


% math
\newcommand{\calI}{\ensuremath{\mathcal{I}}}

\newcommand{\tupleShort}[2]{\ensuremath{(#1_1,\dotsc,#1_{#2})}}
\newcommand{\tuple}[2]{\ensuremath{(#1_1,\:#1_2\:,\dotsc,\:#1_{#2})}}
\newcommand{\setelements}[2]{\ensuremath{\{#1_1,\:#1_2\:,\dotsc,\:#1_{#2}\}}}
\newcommand{\pathelements}[2]{\ensuremath{ (#1_1,\:#1_2\:,\dotsc,\:#1_{#2}) }}

\newcommand{\elems}[1]{\ensuremath{#1_1,\dotsc, #1_{n}) }}

\newcommand{\defiemph}[1]{\emph{#1}}

\newcommand{\setofbases}{\ensuremath{\mathcal{B}}}
\newcommand{\setofcircuits}{\ensuremath{\mathcal{C}}}

\newcommand{\reals}{\ensuremath{\mathbb{R}}}
\newcommand{\integers}{\ensuremath{\mathbb{N}}}

% general
\newcommand{\zit}[3]{#1\ \cite{#2}, #3}
\newcommand{\zitx}[2]{#1\ \cite{#2}}
\newcommand{\footzit}[3]{\footnote{\zit{#1}{#2}{#3}}}
\newcommand{\footzitx}[2]{\footnote{\zitx{#1}{#2}}}

\newcommand{\ite}{\begin{itemize}}
\newcommand{\ete}{\end{itemize}}

\newcommand{\bfr}{\begin{frame}}
\newcommand{\efr}{\end{frame}}

\newcommand{\ilc}[1]{\texttt{#1}}


% misc

% multiframe
\usepackage{xifthen}% provides \isempty test
% new counter to now which frame it is within the sequence
\newcounter{multiframecounter}
% initialize buffer for previously used frame title
\gdef\lastframetitle{\textit{undefined}}
% new environment for a multi-frame
\newenvironment{multiframe}[1][]{%
\ifthenelse{\isempty{#1}}{%
% if no frame title was set via optional parameter,
% only increase sequence counter by 1
\addtocounter{multiframecounter}{1}%
}{%
% new frame title has been provided, thus
% reset sequence counter to 1 and buffer frame title for later use
\setcounter{multiframecounter}{1}%
\gdef\lastframetitle{#1}%
}%
% start conventional frame environment and
% automatically set frame title followed by sequence counter
\begin{frame}%
\frametitle{\lastframetitle~{\normalfont \Roman{multiframecounter}}}%
}{%
\end{frame}%
}




% http://texfragen.de/hurenkinder_und_schusterjungen
\usepackage[all]{nowidow}



% force no overlong lines:
%\tolerance=1
%\emergencystretch=\maxdimen
%\hyphenpenalty=10000
%\hbadness=10000


% http://tex.stackexchange.com/questions/35717/how-to-draw-arrows-between-parts-of-an-equation-to-show-the-math-distributive-pr
\tikzset{square arrow/.style={to path={ -- ++(.0,-.15)  -| (\tikztotarget)}}}
\tikzset{square arrow2/.style={to path={ -- ++(.0,-.25)  -| (\tikztotarget)}}}
%\tikzset{square arrow/.style={to path={ -- ++(00,-.01) -- ++(0.5,-0.1) -- ++(0.5,-0.1) -| (\tikztotarget)},color=red}}


% have arrows from a to b and from c to d here
% just use: texttext\arrowA texttest \arrowB texttext
\newcommand{\arrowA}{\tikz[overlay,remember picture] \node (a) {};}
\newcommand{\arrowB}{\tikz[overlay,remember picture] \node (b) {};}
\newcommand{\drawAB}{
	\tikz[overlay,remember picture]
	%{\draw[->,bend left=80] (a.north) to (b.north);}
	{\draw[->,square arrow,color=red] (a.south) to (b.south);}
}

\newcommand{\arrowC}{\tikz[overlay,remember picture] \node (c) {};}
\newcommand{\arrowD}{\tikz[overlay,remember picture] \node (d) {};}
\newcommand{\drawCD}{
	\tikz[overlay,remember picture]
	{\draw[->,square arrow,color=blue] (c.south) to (d.south);}
}

\newcommand{\arrowE}{\tikz[overlay,remember picture] \node (e) {};}
\newcommand{\arrowF}{\tikz[overlay,remember picture] \node (f) {};}
\newcommand{\drawEF}{
	\tikz[overlay,remember picture]
	{\draw[->,square arrow2,color=orange] (e.south) to (f.south);}
}






%\usepackage{bussproof}

%\usepackage{vaucanson-g}
\usepackage{amssymb}
\usepackage{latexsym}

% for color-highlighted code
%\usepackage{color} % for grey comments
%\usepackage{alltt}

%\usepackage[doublespacing]{setspace}
\usepackage[onehalfspacing]{setspace}
%\usepackage[singlespacing]{setspace}
\usepackage{tabularx}
\usepackage{hyperref}
\usepackage{comment}
\usepackage{color}
\usepackage[final]{listings} % sourcecode in document
\usepackage{url}      % for urls
\usepackage{multicol}
\usepackage{float}
\usepackage{caption}
\usepackage{subfigure}
\usepackage{amsmath}
\usepackage{amssymb}

\usepackage{graphicx}

\usepackage[authoryear]{natbib} % \cite ; square|round etc.
%\usepackage[numbers,square]{natbib}
%\usepackage[square, authoryear]{natbib}
%\usepackage[language=english]{biblatex}

%\bibliographystyle{plain}
\bibliographystyle{alpha}
%\bibliographystyle{alphadin}
%\bibliographystyle{dinat}
%\bibliographystyle{chicago}
%\bibliographystyle{plainnat}

\bibdata{bib.bib}

\renewcommand*{\partformat}{\partname\ \thepart\ -}
\let\partheadmidvskip\

		\newcommand{\comp}{\ensuremath{\text{comp}}}
% smaller url style
\makeatletter
\def\url@leostyle{%
\@ifundefined{selectfont}{\def\UrlFont{\sf}}{\def\UrlFont{\small\ttfamily}}}
\makeatother
\urlstyle{leo}

\newcommand{\myfig}[5] {
 \begin{figure}[tbph]
	 \centering
	 \includegraphics[#3]{#1}
	 \caption[#4]{#5}
	 \label{fig:#2}
 \end{figure}
}

\subject{Master Thesis Proposal}
\title{Interpolation in First Order Logic with Equality}
\author{Bernhard Mallinger \medskip \\
Advisor: Ass.Prof.\ Stefan Hetzl}
%\date{13. November 2007}

%\usepackage{fancyhdr}
%\setlength{\headrulewidth}{0.0pt}
\pagestyle{plain}

\definecolor{grey}{gray}{.35} % for grey commnts
\lstset{language=Python,%
escapeinside={@}{@},
extendedchars=false,%
%inputencoding=utf8x,%
basicstyle=\ttfamily\small,%
commentstyle=\color{grey},%
%keywordstyle=,% no bold tt in standard font
%captionpos=b,
tabsize=2,
showstringspaces=false,
breaklines=true,
breakindent=0pt,
numbers=left
}

% just for screen-display!
%\usepackage{newcent}

%\newcommand{\ex}[2]{\section*{Exercise #1} \textbf{#2} }
%\newcommand{\ex}[2]{\subsection*{Exercise #1: #2} }

\newcommand{\myt}[1]{\ensuremath{\;\text{ #1 }\;}}
\newcommand{\myts}[1]{\ensuremath{\text{ #1 }}}

%\setlength{\parindent}{0em}
%\usepackage{thmtools} % actually already in latex_header.tex ...

\usepackage{amsthm}

\theoremstyle{definition}
\newtheorem{thm}{Theorem}


\begin{document}

\maketitle

\section{Motivation and problem statement}
\label{motivation}

After decades of continued research, the area of software verification still lacks effective methods for reasoning about real world programs, which is necessary to prove vital safety or liveness properties.
The emergence of symbolic model checking and bounded model checking constitute considerable advances.
Here, the reachable states of a program are described by means of abstraction, i.e.\ automatically derived predicates overapproximate them. 
Furthermore, in dealing with loops with an a priori unknown number of iterations, it is necessary to infer loop invariants in order to enable giving a meaningful guarantee of the state of the program afterwards.

In recent years, the approach of applying Craig interpolation to solve both of these problems enjoyed increasing popularity, especially after successful applications for instance in \cite{McMillan03} for use in abstraction or \cite{weissenbacher2010} for use in loop invariant generation.

The Interpolation theorem is a long known basic result of mathematical logic.
Interpolants lay bare certain logical relations between formulas or sets of formulas in a concise way. 
This process is fully analytic in the sense that interpolants can efficiently be calculated from proofs.
Leveraging the tremendous progress of automatic deduction systems in the last decades, obtaining the required proofs is feasible.

For practical applicability, often relatively weak formalisms such as propositional logic or equational logic with uninterpreted function symbols are employed.
However for first-order logic with equality, TODO no efficient algorithms for computing interpolants are known, even though a basic procedure is already provided in \cite{craig57linear}.


\begin{comment}
Software verification
Model checking
Derive invariants
interpolation by its nature disregards all but the predicates relevant to a certain property
can be used for predicate refinement in cegar

often restricted to weaker logics, application to more powerful formalisms such as fol with equality is relevant 
\end{comment}

\section{Aim of the work}

TODO
This thesis aims to work towards finding an algorithm to calculate interpolants in first-order logic in the presence of equality.
Currently no procedures of practical applicability are known for this logic,  
This should be accomplished by either improving existing solutions such as the aforementioned procedure by Craig or exploring a novel approach.

Furthermore, a comprehensive account of existing techniques and results will be presented.
This includes different proofs of interpolation results with a focus on constructive proofs which give rise to concrete algorithms.
Non-constructive methods, especially of a model theoretic nature, will be treated in order to form a theoretical baseline and to put the algorithm in perspective.
In this spirit, further corollaries and also applications of the interpolation theorem will be presented.


\section{Methodology and approach}

As the problem at hand is a well-defined mathematical task, standard mathematical methodology applies.

Determined by the results in the development of an algorithm, an implementation is deemed scientifically valuable but most likely beyond the scope of this thesis.


\section{State of the art}

Current research and application is rooted in the fundamental result by Craig \cite{craig57linear}, here given in a formulation suitable for resolution calculus\footnote{Also known as \emph{reverse interpolation}}:

\begin{thm}[Interpolation]
	Let $A$ and $B$ be first-order sentences such that $A \land B$ is refutable. 
	Then there exists an interpolant $I$ such that
	\begin{compactenum}
		\item $ A \limpl I$ is valid 
		\item $I \land B$ is unsatisfiable
		\item the non-logical symbols of $I$ are only those that appear in both $A$ and $B$.
	\end{compactenum}
\end{thm}

This basic result has been proven in different formalisms using different syntactic methods (cf.~e.g.~\cite{craig57linear}; \cite{takeuti1987proof}; \cite{krajivcek1997interpolation}; \cite{Pudlak97}), but also via semantic, model theoretic means (cf.~e.g.~\cite{shoenfield1967mathematical}, section~5.2; \cite{chang1990model}, theorem~2.2.20).
To this end, the interpolation theorem can be seen as a corollary Robinson's joint consistency theorem, but even more, latter can also be proven from the former. 
This suggest a close relation between one the one hand the proof-theoretic and on the other hand the model-theoretic view.

Another major corollary of the interpolation theorem is given by Beth in form of the definability theorem, which shows that the notions of implicit and explicit definition coincide.
In shallow terms, an implicit definitions refers to a definition by usage in a formula, whereas an explicit definition gives a definition in terms of another formula.
The non-trivial direction of this theorem states that implicit definitions can be converted into explicit ones and it can be proved by using the interpolant as explicit definition.


\subsection{Interpolation algorithms}

Constructive proofs of the interpolation theorem directly give rise to algorithms for computing interpolants. For instance in \cite{takeuti1987proof}, the well known Maehara lemma is used in the proof, which forms an efficient procedure for extracting interpolants from first-order LK proofs, but fails in the presence of equality.

TODO: Huang

TODO: HKP

TODO: McMillan




\section{Relevance to the curriculum of Computational Intelligence}

Logic as core machinery of computer science is featured prominently in the curriculum of Computational Intelligence in the mandatory module ``Logic and Computability'' as well as the module ``Logic, Mathematics, and Theoretical Computer Science'', but is also vital in the theoretic foundations of other areas. 

The logic used in this thesis, first-order logic with equality, clearly is among the most common and useful ones; the interpolation theorem is hereby a celebrated result.
As argued in section \ref{motivation}, advancements in this field have direct consequences for the area of formal verification, which is also featured in the curriculum.

The following courses possess a direct relation to the topic of this thesis: 
\begin{itemize}
	\item Formal Methods in Computer Science
	\item Proof Theory 1 
	\item Logic and Computability 
	\item Advanced Mathematical Logic 
\end{itemize}


\nocite{*} % display all entries of bib-file

\bibliography{bib}

\end{document}
