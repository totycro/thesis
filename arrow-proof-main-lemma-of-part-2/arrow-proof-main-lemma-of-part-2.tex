\documentclass[,%fontsize=11pt,%
	paper=a4,% 
	%landscape,
	DIV11, % mehr text pro seite als defaultyyp
	%DIV10, 
	%DIV=calc,%
	twoside=false,%
	liststotoc,
	bibtotoc,
	draft=false,% final|draft % draft ist platzsparender (kein code, bilder..)
	%titlepage,
	numbers=noendperiod
]{scrartcl}

\usepackage{lscape}
\usepackage{stackengine}


\usepackage[utf8]{inputenc}
\usepackage[T1]{fontenc}
\usepackage[english]{babel}

\usepackage{enumerate}
\usepackage{paralist}
\usepackage{tikz}
\usetikzlibrary{shapes,arrows,backgrounds,graphs,%
	matrix,patterns,arrows,decorations.pathmorphing,decorations.pathreplacing,%
	positioning,fit,calc,decorations.text,shadows%
}


\usepackage{comment} 

\usepackage{etoolbox} % fixes fatal error caused by combining bm, stackengine, hyperref (seriously?)
% http://tex.stackexchange.com/questions/22995/package-incompatibilites-etoolbox-hyperref-and-bm-standalone

\usepackage{etex} % else error on too many packages

% includes
\usepackage{algorithm}
%\usepackage{algorithmic} % conflicts with algpseudocode
\usepackage{algpseudocode}
%\newcommand*\Let[2]{\State #1 $\gets$ #2}
\algrenewcommand\alglinenumber[1]{
{\scriptsize #1}}
\algrenewcommand{\algorithmicrequire}{\textbf{Input:}}
\algrenewcommand{\algorithmicensure}{\textbf{Output:}}


%\usepackage[multiple]{footmisc} % footnotes at the same character separated by ','

\usepackage{multicol}

\usepackage{afterpage}

\usepackage{changepage} % for adjustwidth
\usepackage{caption} % for \ContinuedFloat

\usepackage{tikz}
\usetikzlibrary{shapes,arrows,backgrounds,graphs,%
matrix,patterns,arrows,decorations.pathmorphing,decorations.pathreplacing,%
positioning,fit,calc,decorations.text,shadows%
}

\usepackage{bussproofs}
\EnableBpAbbreviations


\usepackage{amsmath}
\usepackage{amsthm}
\usepackage{amssymb} % the reals
\usepackage{mathtools} % smashoperator

\usepackage{bm} % bm, bold math symbols

\usepackage{thm-restate} % restatable env

% needs extra work and fails on some label here
%\usepackage{cleveref} % cref, apparently better than autoref of hyperref 

\usepackage{nicefrac} % nicefrac

\usepackage{mathrsfs} % mathscr

\usepackage{pst-node} % http://tex.stackexchange.com/questions/35717/how-to-draw-arrows-between-parts-of-an-equation-to-show-the-math-distributive-pr

\usepackage{stackengine}

\usepackage{thmtools} % advanced thm commands (declaretheorem)


\usepackage{nameref} % reference name of thm instead of counter

\usepackage{todonotes}

% conflict with beamer
%\usepackage{paralist} % compactenum

\usepackage{hyperref}
%\hypersetup{hidelinks}  % don't give options to usepackage, it doesn't work with beamer
%\hypersetup{colorlinks=false}  % don't give options to usepackage, it doesn't work with beamer


% \usepackage{enumitem} % labels for enumerate % breaks beamer and memoir itemize


\usepackage{url} 


\usepackage[format=hang,justification=raggedright]{caption}% or e.g. [format=hang]

\usepackage{cancel} % \cancel

\usepackage{lineno}


% commands

% logic etcs
%\newcommand{\ex}[2]{\bigskip\section*{Exercise #1: \begin{minipage}[t]{.80\linewidth} \small \textnormal{\it #2} \end{minipage} } }

\newcommand{\ex}[2]{\bigskip \noindent\textbf{Exercise #1.} \textit{#2} \smallskip}

\newcommand{\comm}[1]{{\color{gray} // #1 }}


\newcommand{\true}[0]{\textbf{1}}
\newcommand{\false}[0]{\textbf{0}}
\newcommand{\tr}{\true}
\newcommand{\fa}{\false}

\newcommand{\ra}{\rightarrow}
\newcommand{\Ra}{\Rightarrow}
\newcommand{\la}{\leftarrow}
\newcommand{\La}{\Leftarrow}

\newcommand{\lra}{\leftrightarrow}
\newcommand{\Lra}{\Leftrightarrow}

\newcommand{\NKZ}{\textbf{NK2}}

%\DeclareMathOperator{\syneq}{\equiv} %spacing seems wrong, therefore defined as newcommand below
\DeclareMathOperator{\limpl}{\supset}
\DeclareMathOperator{\liff}{\lra}
\DeclareMathOperator{\semiff}{\Lra}
\newcommand{\syneq}{\equiv}
\newcommand{\union}{\cup}
\newcommand{\bigunion}{\bigcup}
\newcommand{\intersection}{\cap}
\newcommand{\bigintersection}{\bigcap}
\newcommand{\intersect}{\intersection}
\newcommand{\bigintersect}{\bigintersection}

\newcommand{\powerset}{\mathcal{P}}

\newcommand{\entails}{\vDash}
\newcommand{\notentails}{\nvDash}
\newcommand{\proves}{\vdash}

\newcommand{\vm}{\ensuremath{\vv_\mathcal{M}}}
\newcommand{\Dia}{\ensuremath{\lozenge}}

\newcommand{\spaced}[1]{\ \ #1 \ \ }
\newcommand{\spa}[1]{\spaced{#1}}
\newcommand{\spas}[1]{\;{#1}\;}
\newcommand{\spam}[1]{\;\,{#1}\;\,}

% functions
\DeclareMathOperator{\sk}{sk}
\DeclareMathOperator{\mgu}{mgu}
\DeclareMathOperator{\dom}{dom}
\DeclareMathOperator{\ran}{ran}

\DeclareMathOperator{\id}{id}
\DeclareMathOperator{\Fun}{FS}
\DeclareMathOperator{\Pred}{PS}
\DeclareMathOperator{\Lang}{L}
\DeclareMathOperator{\ar}{ar}
\DeclareMathOperator{\PI}{PI}
\DeclareMathOperator{\LI}{LI}
\DeclareMathOperator{\Congr}{Congr}
\DeclareMathOperator{\Refl}{Refl}
\DeclareMathOperator{\aiu}{au}
\DeclareMathOperator{\expa}{unfold-lift}

\newcommand{\PIinc}{\LI}
\newcommand{\PIincde}{\LIde}

\newcommand{\LIde}{\ensuremath{\LI^\Delta}}

\newcommand{\LIcl}{\ensuremath{\LI_{\operatorname{cl}}}}
\newcommand{\LIclde}{\ensuremath{\LI_{\operatorname{cl}}^\Delta}}

\newcommand{\cll}{\ensuremath{_{\operatorname{LIcl}}}}
\newcommand{\cllde}{\ensuremath{_{\operatorname{LIcl}^\Delta}}}

%\newcommand{\lifi}{\mathop{\ell\text{}i}}
\newcommand{\lifiboth}[1]{\ensuremath{\LIcl(#1)}}
\newcommand{\lifidelta}[1]{\ensuremath{\LIclde(#1)}}


%\DeclareMathOperator{\abstraction}{abstraction}

%\newcommand{\sk}{\ensuremath{\mathrm{sk}}}
%\newcommand{\mgu}{\ensuremath{\mathrm{mgu}}}
%\newcommand{\Fun}{\ensuremath{\mathrm{FS}}}
%\newcommand{\Pred}{\ensuremath{\mathrm{PS}}}
%\newcommand{\PI}{\ensuremath{\mathrm{PI}}}
%\newcommand{\Lang}{\ensuremath{\mathrm{L}}}
%\newcommand{\ar}{\ensuremath{\mathrm{ar}}}

\DeclareMathOperator{\AI}{AI}
\newcommand{\AIde}{\ensuremath{\AI^\Delta}}
\newcommand{\AImatrix}{\ensuremath{\AI_\mathrm{mat}}}
\newcommand{\AImatrixde}{\ensuremath{\AI_\mathrm{mat}^\Delta}}
\newcommand{\AImat}{\AImatrix}
\newcommand{\AImatde}{\AImatrixde}
\newcommand{\AIclause}{\ensuremath{\AI_\mathrm{cl}}}
\newcommand{\AIcl}{\AIclause}
\newcommand{\AIclde}{\AIclausede}
\newcommand{\AIclausede}{\ensuremath{\AIclause^\Delta}}
\newcommand{\fromclause}{\ensuremath{_{\operatorname{AIcl}}}}
\newcommand{\fromclausede}{\ensuremath{_{\operatorname{AIcl}^\Delta}}}
\newcommand{\cl}{\fromclause}
\newcommand{\clde}{\fromclausede}

\newcommand{\Q}{\ensuremath{Q}}

\newcommand{\AIcol}{\ensuremath{\AI_\mathrm{col}}}
\newcommand{\AIcolde}{\AIcol^\Delta}

\newcommand{\AIany}{\ensuremath{\AI_\mathrm{*}}}
\newcommand{\AIanyde}{\AIany^\Delta}

\newcommand{\AIclpre}{\AIclause^\bullet}
\newcommand{\AImatpre}{\AImatrix^\bullet}

\newcommand{\PS}{\Pred}
\newcommand{\FS}{\Fun}

\DeclareMathOperator{\LangSym}{\mathcal{L}}

%\newcommand{\mguarr}{\sim_\ra}
\newcommand{\mguarr}{\mapsto_{\mgu}}


%\newcommand{\Trans}{\ensuremath{\mathrm{T}}}
%\newcommand{\Trans}{\ensuremath{\mathrm{T}}}
\DeclareMathOperator{\Trans}{T}
\DeclareMathOperator{\TransInv}{T^{-1}}

\DeclareMathOperator{\FAX}{F_{Ax}}
\DeclareMathOperator{\EAX}{E_{Ax}}
%\newcommand{\FAX}{\ensuremath{\mathrm{F_{Ax}}}}
%\newcommand{\EAX}{\ensuremath{\mathrm{E_{Ax}}}}

%\newcommand{\TransAll}{\ensuremath{\Trans_{\mathrm{Ax}}}}
\DeclareMathOperator{\TransAll}{\Trans_{Ax}}
%\newcommand{\FAX}{\ensuremath{\mathrm{F_{Ax}}}}

\DeclareMathOperator{\defeq}{\stackrel{\mathrm{def}}{=}}

\newcommand{\subst}[1]{[#1]}
\newcommand{\abstractionOp}[1]{\{#1\}}

\newcommand{\subformdefinitional}[1]{\ensuremath{D_{\Sigma(#1)}}}


%\newcommand{\lift}[3]{\operatorname{Lift}_{#1}(#2; #3)}
%\newcommand{\lift}[3]{\operatorname{Lift}_{#1,#3}(#2)}
%\newcommand{\lift}[3]{\operatorname{Lift}_{#1,#3}[#2]}
%\newcommand{\lift}[3]{\overline{#2}_{#1,#3}}
\newcommand{\lifsym}{\ell}
%\newcommand{\lift}[3]{\lifsym_{#1,#3}[#2]}
\newcommand{\lift}[3]{\lifsym_{#1}^{#3}[#2]}
\newcommand{\liftnovar}[2]{\lifsym_{#1}[#2]}

%\newcommand{\lft}[3]{\lifsym_{#1,#2}[#3]}
\newcommand{\lft}[3]{\lift{#1}{#3}{#2}}
\newcommand{\lifboth}[1]{\lifsym[#1]}

%\newcommand{\lifi}{\mathop{\ell\text{}i}}
%\newcommand{\lifiboth}[1]{\lifi[#1]}
%\newcommand{\lifidelta}[1]{\lifi_\Delta^x[#1]}
%\newcommand{\lifideltanovar}[1]{\lifi_\Delta[#1]}

\newcommand{\lifdelta}[1]{\lift{\Delta}{#1}{x}}
\newcommand{\lifdeltanovar}[1]{\liftnovar{\Delta}{#1}}
\newcommand{\lifgamma}[1]{\lift{\Gamma}{#1}{y}}
\newcommand{\lifgammanovar}[1]{\liftnovar{\Gamma}{#1}}
\newcommand{\lifphinovar}[1]{\liftnovar{\Phi}{#1}}
\newcommand{\lifphi}[1]{\lift{\Phi}{#1}{z}}

\DeclareMathOperator{\arr}{\mathcal{A}}
%\DeclareMathOperator{\arrFinal}{{\mathcal{A}^{\bm*}}}
\DeclareMathOperator{\arrFinal}{{\mathcal{\bm{\hat}A}}}
\DeclareMathOperator{\warr}{\marr}
\DeclareMathOperator{\marr}{\mathcal{M}}

\DeclareMathOperator{\apath}{\leadsto}
\DeclareMathOperator{\mpath}{\leadsto_=}
\DeclareMathOperator{\notapath}{\not\leadsto}
\DeclareMathOperator{\notmpath}{\not\leadsto_=}

\newcommand{\ltArrC}{<_{\arrFinal(C)}}
\newcommand{\ltAC}{<_{\arr(C)}}
\newcommand{\ltArrCOne}{<_{\arrFinal(C_1)}}
\newcommand{\ltArrCTwo}{<_{\arrFinal(C_2)}}
%\newcommand{\ltArrC}{<_{\scalebox{0.6}{$\arrFinal(C)$}}}
\newcommand{\ltArr}{<_{\scalebox{0.6}{$\arrFinal$}}}

\newcommand{\bhat}{\bm\hat}
\newcommand{\bbar}{\bm\bar}
\newcommand{\bdot}{\bm\dot}

%\usepackage{yfonts}
\usepackage{upgreek}
\DeclareMathAlphabet{\mathpzc}{OT1}{pzc}{m}{it}
%\DeclareMathOperator{\pos}{\mathscr{P}}
%\DeclareMathOperator{\pos}{\mathpzc{p}}
%\DeclareMathOperator{\pos}{{\rho}}
\DeclareMathOperator{\pos}{{\operatorname P}}
%\DeclareMathOperator{\pos}{P}
\DeclareMathOperator{\poslit}{\pos_\mathrm{lit}}
\DeclareMathOperator{\posterm}{\pos_\mathrm{term}}
%\newcommand{\poslit}[1]{\ensuremath{p_\text{lit}(#1)}}
%\newcommand{\posterm}[1]{\ensuremath{p_\text{term}(#1)}}
\newcommand{\at}[1]{|_{#1}}

\newcommand{\UICm}[1]{\UnaryInfCm{#1}}
\newcommand{\UnaryInfCm}[1]{\UnaryInfC{$#1$}}
\newcommand{\BICm}[1]{\BinaryInfCm{#1}}
\newcommand{\BinaryInfCm}[1]{\BinaryInfC{$#1$}}
\newcommand{\RightLabelm}[1]{\RightLabel{$#1$}}
\newcommand{\LeftLabelm}[1]{\LeftLabel{$#1$}}
\newcommand{\AXCm}[1]{\AxiomCm{#1}}
\newcommand{\AxiomCm}[1]{\AxiomC{$#1$}}
\newcommand{\mt}[1]{\textnormal{#1}}

\newcommand{\UnaryInfm}[1]{\UnaryInf$#1$}
\newcommand{\BinaryInfm}[1]{\BinaryInf$#1$}
\newcommand{\Axiomm}[1]{\Axiom$#1$}



% math
\newcommand{\calI}{\ensuremath{\mathcal{I}}}

\newcommand{\tupleShort}[2]{\ensuremath{(#1_1,\dotsc,#1_{#2})}}
\newcommand{\tuple}[2]{\ensuremath{(#1_1,\:#1_2\:,\dotsc,\:#1_{#2})}}
\newcommand{\setelements}[2]{\ensuremath{\{#1_1,\:#1_2\:,\dotsc,\:#1_{#2}\}}}
\newcommand{\pathelements}[2]{\ensuremath{ (#1_1,\:#1_2\:,\dotsc,\:#1_{#2}) }}

\newcommand{\elems}[1]{\ensuremath{#1_1,\dotsc, #1_{n}) }}

\newcommand{\defiemph}[1]{\emph{#1}}

\newcommand{\setofbases}{\ensuremath{\mathcal{B}}}
\newcommand{\setofcircuits}{\ensuremath{\mathcal{C}}}

\newcommand{\reals}{\ensuremath{\mathbb{R}}}
\newcommand{\integers}{\ensuremath{\mathbb{Z}}} 
\newcommand{\naturalnumbers}{\ensuremath{\mathbb{N}}}

% general
\newcommand{\zit}[3]{#1\ \cite{#2}, #3}
\newcommand{\zitx}[2]{#1\ \cite{#2}}
\newcommand{\footzit}[3]{\footnote{\zit{#1}{#2}{#3}}}
\newcommand{\footzitx}[2]{\footnote{\zitx{#1}{#2}}}

\newcommand{\ite}{\begin{itemize}}
\newcommand{\ete}{\end{itemize}}

\newcommand{\bfr}{\begin{frame}}
\newcommand{\efr}{\end{frame}}

\newcommand{\ilc}[1]{\texttt{#1}}


% misc

% multiframe
\usepackage{xifthen}% provides \isempty test
% new counter to now which frame it is within the sequence
\newcounter{multiframecounter}
% initialize buffer for previously used frame title
\gdef\lastframetitle{\textit{undefined}}
% new environment for a multi-frame
\newenvironment{multiframe}[1][]{%
\ifthenelse{\isempty{#1}}{%
% if no frame title was set via optional parameter,
% only increase sequence counter by 1
\addtocounter{multiframecounter}{1}%
}{%
% new frame title has been provided, thus
% reset sequence counter to 1 and buffer frame title for later use
\setcounter{multiframecounter}{1}%
\gdef\lastframetitle{#1}%
}%
% start conventional frame environment and
% automatically set frame title followed by sequence counter
\begin{frame}%
\frametitle{\lastframetitle~{\normalfont \Roman{multiframecounter}}}%
}{%
\end{frame}%
}




% http://texfragen.de/hurenkinder_und_schusterjungen
\usepackage[all]{nowidow}



% force no overlong lines:
%\tolerance=1 % tolerance for how badly spaced lines are allowed, less means "better" lines
\tolerance=500 %  need more tolerance for equations
%\emergencystretch=\maxdimen
%\emergencystretch=200pt
%\setlength{\emergencystretch}{3em}
%\hyphenpenalty=10000 % forces no hyphenation
%\hbadness=10000


% http://tex.stackexchange.com/questions/35717/how-to-draw-arrows-between-parts-of-an-equation-to-show-the-math-distributive-pr
\tikzset{square arrow/.style={to path={ -- ++(.0,-.15)  -| (\tikztotarget)}}}
\tikzset{square arrow2/.style={to path={ -- ++(.0,-.25)  -| (\tikztotarget)}}}
%\tikzset{square arrow/.style={to path={ -- ++(00,-.01) -- ++(0.5,-0.1) -- ++(0.5,-0.1) -| (\tikztotarget)},color=red}}


% have arrows from a to b and from c to d here
% just use: texttext\arrowA texttest \arrowB texttext
\newcommand{\arrowA}{\tikz[overlay,remember picture] \node (a) {};}
\newcommand{\arrowB}{\tikz[overlay,remember picture] \node (b) {};}
\newcommand{\drawAB}{
	\tikz[overlay,remember picture]
	{\draw[->,bend left=5,color=red] (a.south) to (b.south);}
	%{\draw[->,square arrow,color=red] (a.south) to (b.south);}
}
\newcommand{\arrowAP}{\tikz[overlay,remember picture] \node (ap) {};}
\newcommand{\arrowBP}{\tikz[overlay,remember picture] \node (bp) {};}
\newcommand{\drawABP}{
	\tikz[overlay,remember picture]
	{\draw[->,bend right=5,color=red] (ap.south) to (bp.south);}
	%{\draw[->,square arrow,color=red] (a.south) to (b.south);}
}

\newcommand{\arrowAB}{\tikz[overlay,remember picture] \node (ab) {};}
\newcommand{\arrowBA}{\tikz[overlay,remember picture] \node (ba) {};}
\newcommand{\drawAABB}{
	\tikz[overlay,remember picture]
	%{\draw[->,bend left=80] (a.north) to (b.north);}
	{\draw[->,square arrow,color=brown] (ab.south) to (ba.south);
	\draw[->,square arrow,color=brown] (ba.south) to (ab.south);}
}


\newcommand{\arrowCD}{\tikz[overlay,remember picture] \node (cd) {};}
\newcommand{\arrowDC}{\tikz[overlay,remember picture] \node (dc) {};}
\newcommand{\drawCCDD}{
	\tikz[overlay,remember picture]
	%{\draw[->,bend left=80] (a.north) to (b.north);}
	{\draw[<->,dashed,square arrow,color=green] (cd.south) to (dc.south); }
}



\newcommand{\arrowC}{\tikz[overlay,remember picture] \node (c) {};}
\newcommand{\arrowD}{\tikz[overlay,remember picture] \node (d) {};}
\newcommand{\drawCD}{
	\tikz[overlay,remember picture]
	{\draw[->,square arrow,color=blue] (c.south) to (d.south);}
}

\newcommand{\arrowE}{\tikz[overlay,remember picture] \node (e) {};}
\newcommand{\arrowF}{\tikz[overlay,remember picture] \node (f) {};}
\newcommand{\drawEF}{
	\tikz[overlay,remember picture]
	{\draw[->,square arrow2,color=orange] (e.south) to (f.south);}
}


\newcommand{\arrAP}{\arrowAP}
\newcommand{\arrBP}{\arrowBP}
\newcommand{\arrA}{\arrowA}
\newcommand{\arrB}{\arrowB}
\newcommand{\arrC}{\arrowC}
\newcommand{\arrD}{\arrowD}
\newcommand{\arrE}{\arrowE}
\newcommand{\arrF}{\arrowF}


\DeclareMathOperator{\simgeq}{\scalebox{0.92}{$\gtrsim$}}

\newcommand{\refsub}[2]{\hyperref[#2]{\ref*{#1}.\ref*{#2}}}

%\newcommand{\sigmarange}[2]{\sigma_{#1}^{#2} }
\newcommand{\sigmarange}[2]{\sigma_{(#1,#2)} }
\newcommand{\sigmaz}[1]{\sigmarange{0}{#1} }
\newcommand{\sigmazi}[0]{\sigmaz{i} }

\DeclareMathOperator{\lit}{lit}

%\def\fCenter{\ \proves\ }
\def\fCenter{\proves}

\newcommand{\prflbl}[2]{\RightLabel{\footnotesize $#1, #2$} }
%\newcommand{\prflblid}[1]{\RightLabel{$#1, \id$} }
\newcommand{\prflblid}[1]{\RightLabel{\footnotesize $#1$} }

\DeclareMathOperator{\resruleres}{res}
\DeclareMathOperator{\resrulefac}{fac}
\DeclareMathOperator{\resrulepar}{par}
\newcommand{\lkrule}[2]{\ensuremath{\operatorname{#1}:#2}} % operatorname fixes spacing issues for =

\newcommand{\parti}[4]{\ensuremath{ \langle (#1; #2), (#3; #4)\rangle  }}

\newcommand{\partisym}{\ensuremath{\chi}}

\newcommand{\occur}[1]{\ensuremath{[#1]}}
\newcommand{\occ}[1]{\occur{#1}}

\newcommand{\occurat}[2]{\ensuremath{{\occur{#1}_{#2}}}}
\newcommand{\occat}[2]{\occurat{#1}{#2}}
\newcommand{\occatp}[1]{\occurat{#1}{p}}
\newcommand{\occatq}[1]{\occurat{#1}{q}}

\newcommand{\colterm}[1]{\zeta_{#1}}



% fix restateable spacing 
%http://tex.stackexchange.com/questions/111639/extra-spacing-around-restatable-theorems

\makeatletter

\def\thmt@rst@storecounters#1{%
%THIS IS THE LINE I ADDED:
\vspace{-1.9ex}%
  \bgroup
        % ugly hack: save chapter,..subsection numbers
        % for equation numbers.
  %\refstepcounter{thmt@dummyctr}% why is this here?
  %% temporarily disabled, broke autorefname.
  \def\@currentlabel{}%
  \@for\thmt@ctr:=\thmt@innercounters\do{%
    \thmt@sanitizethe{\thmt@ctr}%
    \protected@edef\@currentlabel{%
      \@currentlabel
      \protect\def\@xa\protect\csname the\thmt@ctr\endcsname{%
        \csname the\thmt@ctr\endcsname}%
      \ifcsname theH\thmt@ctr\endcsname
        \protect\def\@xa\protect\csname theH\thmt@ctr\endcsname{%
          (restate \protect\theHthmt@dummyctr)\csname theH\thmt@ctr\endcsname}%
      \fi
      \protect\setcounter{\thmt@ctr}{\number\csname c@\thmt@ctr\endcsname}%
    }%
  }%
  \label{thmt@@#1@data}%
  \egroup
}%

\makeatother




\newcommand{\mymark}[1]{\ensuremath{(#1)}}
\newcommand{\markA}{\mymark \circ}
\newcommand{\markB}{\mymark *}
\newcommand{\markC}{\mymark \divideontimes}

\newcommand{\wrong}[1]{{\color{red}WRONG: #1}}
\newcommand{\NB}[1]{{\color{blue}NB: #1}}
\newcommand{\hl}[1]{{\color{orange} #1}}
\newcommand{\mytodo}[1]{{\color{red}TODO: #1}}
\newcommand{\largered}[1]{{

	  \LARGE\bfseries\color{red}
		#1

}}
\newcommand{\largeblue}[1]{{

	  \large\bfseries\color{blue}
		#1

}}




\usepackage{ulem} %  \dotuline{dotty} \dashuline{dashing} \sout{strikethrough}
\normalem

\usepackage{tabu} % tabular also in math mode (and much more)

\usepackage[color]{changebar} %  \cbstart, \cbend
\cbcolor{red}



% http://tex.stackexchange.com/questions/7032/good-way-to-make-textcircled-numbers
\newcommand*\circled[1]{\tikz[baseline=(char.base)]{
\node[shape=circle,draw,inner sep=2pt] (char) {#1};}}



% http://tex.stackexchange.com/questions/43346/how-do-i-get-sub-numbering-for-theorems-theorem-1-a-theorem-1-b-theorem-2

\makeatletter
\newenvironment{subtheorem}[1]{%
  \def\subtheoremcounter{#1}%
  \refstepcounter{#1}%
  \protected@edef\theparentnumber{\csname the#1\endcsname}%
  \setcounter{parentnumber}{\value{#1}}%
  \setcounter{#1}{0}%
  \expandafter\def\csname the#1\endcsname{\theparentnumber.\Alph{#1}}%
  \ignorespaces
}{%
  \setcounter{\subtheoremcounter}{\value{parentnumber}}%
  \ignorespacesafterend
}
\makeatother
\newcounter{parentnumber}


\usepackage{tabularx}% http://ctan.org/pkg/tabularx
\newcolumntype{Y}{>{\centering\arraybackslash}X}

\newcommand{\mycols}[2][3]{
	\noindent\begin{tabularx}{\textwidth}{*{#1}{Y}}
		#2
	\end{tabularx}%
}


\newcommand{\definethms}{

	%\declaretheorem[title=Theorem,qed=$\triangle$,parent=chapter]{thm}
	\newcommand{\thmqed}{$\square$} % for thms without proof
	\newcommand{\propqed}{$\square$} % for props without proof
	\declaretheorem[title=Theorem]{thm}
	\declaretheorem[title=Proposition,sibling=thm]{prop}
	\declaretheorem[title=Conjectured Proposition,sibling=thm]{cprop}

	%\declaretheorem[title=Lemma,parent=chapter]{lemma}
	\declaretheorem[sibling=thm]{lemma}
	\declaretheorem[sibling=thm,title=Conjectured Lemma]{clemma}
	\declaretheorem[title=Corollary,sibling=thm]{corr}
	\declaretheorem[sibling=thm,title=Definition,style=definition,qed=$\triangle$]{defi}
	%\declaretheorem[title=Definition,qed=$\triangle$,parent=chapter]{defi}
	\declaretheorem[title=Example,style=definition,qed=$\triangle$,sibling=thm]{exa}

	\declaretheorem[sibling=thm,title=Conjecture]{conj}

	\declaretheorem[title=Remark,style=remark,numbered=no,qed=$\triangle$]{remark}


}

\usepackage[matha]{mathabx} % the locial operators here have more space around them and [ and ] are thicker, also langle and rangle are a bit nicer; subseteq looks a bit weird

%\usepackage{MnSymbol} % again other symbols


\newcommand{\inference}{\ensuremath{\iota}}

\usepackage{cases} % numcases


% subsections also in toc
\setcounter{tocdepth}{2}

\definethms

%\def\proofSkipAmount{ \vskip -0.5em}



%\usepackage{bussproof}

%\usepackage{vaucanson-g}
\usepackage{amssymb}
\usepackage{latexsym}

% for color-highlighted code
%\usepackage{color} % for grey comments
%\usepackage{alltt}

%\usepackage[doublespacing]{setspace}
\usepackage[onehalfspacing]{setspace}
%\usepackage[singlespacing]{setspace}
\usepackage{tabularx}
\usepackage{hyperref}
\usepackage{comment}
\usepackage{color}
\usepackage[final]{listings} % sourcecode in document
\usepackage{url}      % for urls
\usepackage{multicol}
\usepackage{float}
\usepackage{caption}
\usepackage{subfigure}
\usepackage{amsmath}
\usepackage{amssymb}

\usepackage{graphicx}

\usepackage[authoryear]{natbib} % \cite ; square|round etc.
%\usepackage[numbers,square]{natbib}
%\usepackage[square, authoryear]{natbib}
%\usepackage[language=english]{biblatex}

%\bibliographystyle{plain}
\bibliographystyle{alpha}
%\bibliographystyle{alphadin}
%\bibliographystyle{dinat}
%\bibliographystyle{chicago}
%\bibliographystyle{plainnat}

\bibdata{bib.bib}

\renewcommand*{\partformat}{\partname\ \thepart\ -}
\let\partheadmidvskip\

\newcommand{\comp}{\ensuremath{\text{comp}}}
% smaller url style
\makeatletter
\def\url@leostyle{%
\@ifundefined{selectfont}{\def\UrlFont{\sf}}{\def\UrlFont{\small\ttfamily}}}
\makeatother
\urlstyle{leo}

\newcommand{\myfig}[5] {
	\begin{figure}[tbph]
		\centering
		\includegraphics[#3]{#1}
		\caption[#4]{#5}
		\label{fig:#2}
	\end{figure}
}

\setlength{\parindent}{0em}
%\usepackage{thmtools} % actually already in latex_header.tex ...

\usepackage{amsthm}


\usepackage{tikz-qtree}

%\newcommand{\sig}[1]{{#1}_\Sigma}
%\newcommand{\p}[1]{{#1}_\Pi}
\newcommand{\sig}[1]{\stackrel{\Sigma}{#1}}
\newcommand{\p}[1]{\stackrel{\Pi}{#1}}

\newcommand{\e}[1]{\vskip .7em   \subsection*{#1}}

%\def\proofSkipAmount{ \vskip -0.3em}

\usepackage{refcheck}

\begin{document}

\newcommand{\substremarksym}{$\ast$}
\newcommand{\substremarkref}{$(\ast)$}

\newcommand{\lif}[1]{\lift{\Delta}{#1}{x}}
\newcommand{\newterm}{^*}
\newcommand{\de}{^\Delta}

\section{version with merging of reachable components}

\mytodo{need to replace lemma 2 (colored to colored). Plan: adapt lemma for variable in both gamma- and delta-colored terms, then add lemma for multiple variable occurrences, all in single color terms.}

\mytodo{Fix up major proof below with the new lemmas}


\begin{clemma}
	Let $x$ be a variable in $\AIclausede(C)$ which has a grey occurrence and a colored occurrence.
	Then for every grey occurrence $\bhat x$ of $x$, there is an path from a term containing some grey occurrence to a term containing $\bhat x$ using arrows from $\arr(C)$.
	%Also, \circled{2} there is a merge path from every colored occurrence of $x$ to every other colored occurrence of $x$.
\end{clemma}
\begin{proof}
	\newcommand{\one}{\circled{1}}
	\newcommand{\two}{\circled{2}}
	Proof by induction; base case obvious; suppose resolution with usual notation.
	Suppose the paths exist for $C_1$ and $C_2$.

	We consider the different possibilities of introduction of colored occurrences of $x$ and show that in each of them, there is a path from a term containing a grey occurrence to a term containing the colored occurrence.

	\begin{description}
		\item[Suppose $x$ is introduced into a maximal colored term $t$ by means of unification.]
			So $t\sigma$ is a term containing $x$.
			Let $y$ be a variable in $t$ such that $y\sigma$ is a term containing $x$.

			Let $\bhat y$ be the position of $y$ which causes the variable to be changed by the unification algorithm.
			$\bhat y$ is in a resolved literal, say $l$, so we denote it by $l\at{\bhat y}$ and its counterpart in $l'$ by $l'\at{\bhat y}$

			\begin{itemize}
					\item Suppose $l\at{\bhat y}$ is a grey occurrence.

						figure: $P(f(y)) \lor Q(\bhat y)\quad\quad\lnot Q(\cdot)$

						Then by the induction hypothesis, there is an path from a term containing some grey occurrence of $y$ to a term containing $y$ in $t$.
						After applying $\sigma$, the path leads from $y\sigma\occ{x}$ to~$t\sigma\occ{x}$.
						If $y\sigma\occ{x}$ has a grey occurrence of $x$, we are done.
						Otherwise it has a colored occurrence of $x$.
						But as $l\at{\bhat y}\sigma = l'\at{\bhat y}\sigma$, there is a colored occurrence of $x$ in $C_2$. 
						If there also is a grey occurrence, then by the induction hypothesis, there is an arrow from some grey occurrence of $x$ to $l'\at{\bhat y}$ and hence there is a path from that grey occurrence to $t\sigma\occ{x}$.
						If there is no grey occurrence of $x$ in $C_2$, suppose that $x$ originates from $C_1$ and there is a grey occurrence of $x$ in $C_1$, as otherwise we are done.
						As $l'\at{\bhat y}\sigma$ contains $x$ but $l\at{\bhat y}\sigma$, $x$ must occur in $l$, say at $\bdot x$ and its corresponding term in $l'$ is a variable, say $z$, such that $z\sigma = x$.
						$z$ also occurs in $l'\at{\bhat y}\sigma$.
						\begin{itemize}
							\item
								Suppose $l\at{\bdot x}$ is a grey occurrence. Then $l'\at{\bdot x}$ is so as well and by the induction hypothesis, there is a path from a term containing a grey occurrence of $z$ in $C_1$ to $l'\at{\bhat y}\sigma$ and we are done.
							\item
								Otherwise $l\at{\bdot x}$ is a colored occurrence.
								Then so is $l'\at{\bdot x}$ and by Lemma~\ref{lemma:arrow_from_all_colored_to_all_colored}, there is a merge edge between a term containing $l'\at{\bdot x}$ and a term containing $l'\at{\bhat y}\sigma$.
								As there is a grey occurrence of $x$ in $C_1$, by the induction hypothesis, there is a path from a term containing a grey occurrence of $x$ to $l'\at{\bdot x}$ and we are done.
						\end{itemize}



					\item Suppose $l\at{\bhat y}$ is contained in a maximal colored term, say $s\occ{y}$.
						Then in case $t\occ{y} \neq\nolinebreak s\occ{y}$, by Lemma~\ref{lemma:arrow_from_all_colored_to_all_colored}, there is a merge arrow between a term containing $t\occ{y}$ and a term containing $s\occ{y}$.

						As the arrows from terms containing $l\at{\bhat y}$ and the corresponding terms containing $l'\at{\bhat y}$ are merged, we need to show that there is an arrow in the other clause. $l'\at{\bhat y}$ is an abstraction from a term containing $x$.

						\markC
						\begin{itemize}
							\item 
								Suppose $l'\at{\bhat y}$ contains $x$. 

								If there is a grey occurrence of $x$ in $C_2$, we are done as the arrows between $l\at{\bhat y}$ and $l'\at{\bhat y}$ are merged and by the induction hypothesis, $x \apath l'\at{\bhat y}$.

								If there is a grey occurrence of $y$ in $C_1$ and $y\sigma$ contains a grey occurrence of $x$, we are done as by the induction hypothesis, $y \apath t\occ{y}$,
								and after applying $\sigma$, this path leads from a grey occurrence of $x$ to $t\occ{y}\sigma$.

								Otherwise there are no grey occurrences of $x$ and there is nothing to prove.

							\item 
								Suppose $l'\at{\bhat y}$ does not contain $x$.
								Then it contains a variable $v$ such that $v\sigma$ is a term containing $x$.
								

								As $l'\at{\bhat y}$ is by assumption contained in a colored term, where $x$ is introduced, we know that there is an appropriate arrow by Remark~\substremarkref.


						\end{itemize}
				\end{itemize}

			\item [Suppose a term containing $t\occ{x}$ with $t$ colored is in $\ran(\sigma)$.]
				So $y\sigma$ contains $t\occ{x}$ for some $y$, which occurs in w.l.o.g.\ $C_1$.
				Let $\bdot y$ be an arbitrary occurrence of $y$.

				There is an occurrence of $y$ in $l$, say $l\at{\bhat y}$, whose corresponding term $l'\at{\bhat y}$ is an abstraction of $t\occ{x}$. Note that the arrows of $l\at{\bhat y}$ and $l'\at{\bhat y}$ are merged.

				figure:
				$C_1: Q(\dots \bdot y \dots) \lor l\occ{\bhat y}\quad\quad$
				$C_2: \lnot l\occ{\bhat y'}\quad(\bhat y' \text{ is abstraction of } t\occur{x})$

				Now we can argue quite like starting at \markC.
				This establishes that $x \apath l'\at{\bhat y}$ and also $x \apath l\at{\bhat y}$.

				If $y$ and $\bhat y$ are grey occurrences, then by Lemma~\ref{lemma:arrow_from_grey_to_grey}, there is a merge edge and we are done.

				If $\bhat y$ is grey and $\bdot y$ is colored, then by the induction hypothesis, $y \apath \bhat y$ \NB{(which is fine if we order by symbols, not their occurrences)}. 

				If $\bhat y$ is colored and $y$ is grey,
				%and then by the induction hypothesis, there is a path from $y$ to $\bhat y$.
				then by construction of $\arr(C)$, in particular the special treatment of 210g' and as $y\sigma = t\occ{x}$ there is an arrow from any grey occurrence of $x$ to $y$ (if there is one; if there are none, we are done anyway).



				If $\bhat y$ is colored and $y$ is colored, then by Lemma~\ref{lemma:arrow_from_all_colored_to_all_colored}, there is a merge edge between $\bhat y$ and $y$ and we are done.
\qedhere
		\end{description}
\end{proof}

\begin{clemma}
	\label{lemma:arrow_from_all_colored_to_all_colored}
	Let $x$ be a variable in $\AIclausede(C)$.
	Then there is a merge path from every colored occurrence of $x$ to every other colored occurrence of $x$ in $C$.
\end{clemma}
\begin{proof}
	\NB{in this version, we merge also terms of different color. \mytodo{check if this works out }}

	Induction start: by definition.

	Suppose holds for $C_1$ and $C_2$, usual notation.

	We consider introductions of colored occurrences of $x$.

	\begin{description}
		\item [Suppose $x$ is introduced in $t$ by means of unification.]
			$t\sigma$ contains $x$, hence there is a variable $y$ in $t$ such that $y\sigma = s\occ{x}$.

			\cbstart
			Let $\bhat y$ be the position of $y$ which causes the variable to be changed by the unification algorithm.
			$\bhat y$ is in a resolved literal, say $l$, so we denote it by $l\at{\bhat y}$ and its counterpart in $l'$ by $l'\at{\bhat y}$
			\cbend\comm{copied}

			\begin{itemize}
				\item 
					Suppose $l\at{\bhat y}$ is a grey occurrence.
					$l'\at{\bhat y}$ is an abstraction of $s\occ{x}$.
					\mytodo{}

				\item
					Suppose $l\at{\bhat y}$ is a colored occurrence.
					Then by the induction hypothesis, $l\at{\bhat y} \mpath \bdot y$ for every other colored occurrence $\bdot y$ of $y$. As the substitution $s\occ{x}$ for $y$, there are merge edges between all these occurrences.

					For the grey occurrences of $y$ \mytodo{}
					
			\end{itemize}


		\item [Suppose a colored term $t\occ{x}$ containing $x$ is in $\ran(\sigma)$.]
	\end{description}

\end{proof}



\begin{lemma}
	\label{lemma:arrow_from_grey_to_grey}
	Let $x$ be a variable in $\AIclausede(C)$.
	Then there is a merge arrow between every pair of distinct occurrences of $x$,
\end{lemma}
\begin{proof}
	Induction start: by definition.

	Suppose holds for $C_1$ and $C_2$, usual notation.

	Suppose for some grey variable occurrence $x$ that $x\sigma = y$ for some variable $y$ which has a grey occurrence in $C$ (so either it was there in $C_i$ and $y\sigma = y$ or $z\sigma = y$ for some z,  but then some $y$ occurs elsewhere). 

	Then there is a position $\bhat x$ in a resolved literal, say w.l.o.g.\ $l$, such that $l\at{\bhat x} = x$ and $l'\at{\bhat x} = y$.

	\begin{itemize}
		\item
			Suppose that $l\at{\bhat x}$ is a grey occurrence.
			Then so is $l'\at{\bhat x}$.
			By the induction hypothesis, both occurrences have merge edges to all other occurrences of the variable, and these are merged.
			Note that $C_1$ and $C_2$ are variable disjoint, so $x$ does not occur in $C_2$ and $y$ does not occur in $C_1$.

		\item
			Otherwise suppose that $l\at{\bhat x}$ is a colored occurrence.
			Then there are merge edges between all occurrences of $x$ in $C_1$ and $y$ in $C_2$ by construction of the edges, in particular by the special handling of variable renamings.
			\qedhere
	\end{itemize}

\end{proof}



\section{original proof}

Ideas for simplification:

* Lemma for all cases about what is on the other side

\begin{lemma}
	\label{DEPR:lemma:arrow_from_grey_to_colored}
	{

		\LARGE\bfseries\color{red}
		not true in this formulation, we can have $x$, $f(x)$ and $g(x)$ with arrows just from $x$ to the two colored occurrences, even if $f$ and $g$ of same color.


	}
	Let $x$ be a variable in $\AIclause\de(C)$ which has a grey occurrence and a colored occurrence. 
	Then there is an arrow in $\arr(C)$ from a term containing a grey occurrence to a term containing a colored occurrence.
	%Suppose there is a colored and a grey occurrence of $x$.
	%Then for every colored occurrence $p$ of $x$ there is an arrow from some grey occurrence to~$p$ in \mytodo{}.
	\comm{Should also hold for all of $\AI\de$, but is currently not needed in the proof}
\end{lemma}
\begin{proof}
	For clauses $C$ in the initial clause set, $\arr(C)$ is defined to contain an arrow from every grey occurrence to every colored occurrence for every variable occurring in the clause.  

	For the induction step, suppose the lemma holds for $C_1$ and $C_2$. 
	Note that $C_1$ and $C_2$ are variable disjoint.
	\mytodo{how to continue without checking every single case?}

	Note that terms are only changed by means of substitution. 

	If a variable is substituted, it does not occur any further in the derivation. 

	If a variable is substituted by a term containing variables, this is fine because the original arrows still apply for the new terms. 
\end{proof}

\begin{lemma}
{

	\LARGE\bfseries\color{red} (same as above)
		not true in this formulation, we can have $x$, $f(x)$ and $g(x)$ with arrows just from $x$ to the two colored occurrences, even if $f$ and $g$ of same color.


	}
	\label{DEPR:lemma:arrow_from_colored_to_colored}
	Let $x$ be a variable which occurs colored in $\AIclause\de(C)$ and again colored in the same color somewhere else in $\AI\de(C)$.
	Then there is a merge edge between the maximal colored terms containing the two occurrences.
	\comm{This is exactly the case we need, possibly show something more general}
\end{lemma}
\begin{proof}
	\mytodo{}
\end{proof}

\begin{exa}
	\label{exa:lifting_var_only_abstraction_in_arrow_proof}
	$\Gamma = \{ Q(\gamma(x)) \lor P(x), \lnot Q(\gamma(z)), R(\dots)\}$\todo{$R$ only for coloring}

	$\Delta = \{ \lnot P(\delta(y)) \lor R(y), \lnot R(a), Q(\dots) \}$\todo{$Q$ only for coloring}

	$ a \sim x_k, \delta(y) \sim x_i, \delta(a) \sim x_j $

	\begin{prooftree}
		\AxiomCm{ \bot \mid Q(\gamma(x)) \lor P(x) }
		\AxiomCm{ \top \mid \lnot P(x_i) \lor R(y) }

		\BinaryInfCm{ P(x_i) \mid Q(\gamma(x_i)) \lor R(y) }

		\AxiomCm{ \top \mid \lnot R(x_k) }

		\BinaryInfCm{ (\lnot R(x_k) \land P(x_i) ) \lor (R(x_k) \land \top)  \mid Q(\gamma(x_i)) }
		\noLine
		\UnaryInfCm{ P(x_i) \lor R(x_k)  \mid Q(\gamma(x_i)) }

		\AxiomCm{ \bot \mid \lnot Q(\gamma(z)) }

		\BinaryInfCm{ ( \lnot Q(x_j) \land ( P(x_i) \lor R(x_k) )) \lor (Q(x_j) \land \top)   \mid \square }
		\noLine
		\UnaryInfCm{   \lnot Q(x_j) \land (P(x_i) \lor R(x_k)) \mid \square }

	\end{prooftree}

	Gist: When $Q(\gamma(x_i))$ is the only symbol in $\AI\de(\cdot)$, the lifting var means $\delta(x)$, but in the actual derivation, it's $\delta(a)$. however $\tau$ fixes this.
	So before $Q$ is resolved, there is an arrow, but with the wrong lifting var ($x_i$ instead of $x_j$)
\end{exa}

\begin{remark}[\substremarksym]
	Any substitution, in particular $\sigma$, only changes a finite number of variables.
	Furthermore a result of a run of the unification algorithm is acyclic in the sense that if a substitution $u\mapsto t$ is added to the resulting substitution, it is never the case that at a later stage $t\mapsto u$ is added.
	This can easily be seen by considering that at the point when $u\mapsto t$ is added to the resulting substitution, every occurrence of $u$ is replaced by $t$, so $u$ is not encountered by the algorithm at a later stage.

	Therefore in order to show that a statement holds for every $u\mapsto t$ in a unifier $\sigma$, 
	it suffices to show by an induction argument that for every substitution $v\mapsto s$ which is added to the resulting unifier by the unification algorithm that it holds for $v\mapsto s$ under the assumption that it holds for every $w\mapsto r$ such that $w$ occurs in $s$ and $w\mapsto r$ is added to the resulting substitution at a later stage.
\end{remark}


\begin{conj}
	Let $C$ be a clause in a resolution refutation. 
	Suppose that $\AI\de(C)$ contains a maximal $\Gamma$-term $\gamma_j\occ{z_i}$ which contains a lifting variable $z_i$. %(lifting a $\Delta$-term $\delta_i$).
	%Then there is an arrow in $\arrFinal(C)$ from a term containing $z_k$ in $\AI\de(C)$ such that $\delta_k$ is an abstraction of $\delta_i$ to $\gamma_j\occ{z_i}$.
	Then $z_i \ltArrC z_j$.
\end{conj}
\begin{proof}
	We proceed by induction.
	For the base case, note that no multicolored terms occur in initial clauses, so no lifting term can occur inside of a $\Gamma$-term.

	\newcommand{\oldgam}{\bm\tilde\gamma_{j\bm'}} 
	\newcommand{\newgam}{\gamma_j} % this should correspond with the statement, and there, the symbol should be plain (also below)

	Suppose a clause $C$ is the result of a resolution of $C_1: D \lor l$ and $C_2: E \lor \lnot l$ with $l\sigma = l'\sigma$.
	Furthermore suppose that for every lifting term inside a $\Gamma$-term in the clauses $C_1$ and $C_2$ of the refutation, for every term of the form $\gamma_j\occ{z_i}$ we have that $z_i \ltArrCOne z_j$ or $z_i \ltArrCTwo z_j$ respectively. Hence there is an arrow $(p_1, p_2)$ in $\arrFinal(C_1)$ or $\arrFinal(C_2)$ such that $z_i$ is contained in $\pos(p_1)$ and $z_j$ is contained in $\pos(p_2)$.
	In $\AI\de(C)$, $\pos(p_1)$ contains $\lifboth{z_i\sigma}\tau = z_i\tau$ and $\pos(p_2)$ contains $\lifboth{z_j\sigma}\tau = z_j\tau$.
	Hence the indices of the lifting variables might change, but this renaming does not affect the relation of the objects as $\arrFinal(C_1) \cup \arrFinal(C_2) \subseteq \arrFinal(C)$.


	We show that $z_i \ltArrC z_j $ holds true also for every new term of the form $\newgam\occ{z_i}$ for some $j, i$ in $\AI\de(C)$. By ``new'', we mean terms which are not present in $\AI\de(C_1)$ or $\AI\de(C_2)$.
	Note that new terms in $\AI\de(C)$ are of the form $\lifdelta{t\sigma}\tau$ for some $t \in \AI\de(C_1) \cup \AI\de(C_2)$.
	By Lemma~\ref{lemma:no_lifting_vars_in_subst}, $\sigma$ does not introduce lifting variables.
	Hence a new term of the form $\newgam\occ{z_i}$ is created either by introducing a $\Delta$-term into a $\Gamma$-term or by introducing $\newgam\occ{\delta_i}$ via $\sigma$, both followed by the lifting. 
	Note that $\tau$ only substitutes lifting variables by other lifting variables and hence does not introduce lifting variables. Furthermore by Lemma~\ref{lemma:tau_is_specialisation}, $\tau$ only substitutes lifting variables for other lifting variables, whose corresponding term is more specialised. Hence if there exists an arrow from a lifting variable to $\newgam\occ{z_i}$ according to this lemma, it is also an appropriate arrow if $\newgam\occ{z_i}$ is replaced by $\newgam\occ{z_i}\tau$.


	We now distinguish the two cases under which a new term $\newgam\occ{z_i}$ can occur in $\AI\de(C)$:

	\begin{description}
		\item[Suppose for some $\Gamma$-term $\oldgam\occ{u}$ in $\AI\de(C_1)$ or $\AI\de(C_2)$, $u\sigma$ contains a $\Delta$-term.]\hfill\nopagebreak

			Hence we have that $(\oldgam\occ{u})\sigma = \newgam\occ{\delta_i}$ for some $i$.
			Note that the position of $u$ in $\oldgam\occ{u}$ does not necessarily coincide with the position of $\delta_i$ in $\newgam\occ{\delta_i}$ as $u$ might be substituted by $\sigma$ for a grey term containing $\delta_i$.

			We have that $\lifdeltanovar{\oldgam\occ{u}\sigma}\tau = \newgam\occ{z_i}$.

			\newcommand{\hatu}{{\bm\hat u}}
			\newcommand{\hatuPrime}{t_{\bm\hat u}}
			At some well-defined point of application of the unification algorithm, $u$ is substituted by an abstraction of a term which contains $\delta_i$. This occurrence of $u$ is in $l$ and we denote it by $\hatu$.
			We furthermore denote the term at the corresponding position in $l'$ by~$\hatuPrime$.

			We distinguish cases based on the occurrences of $\hatu$ and $\hatuPrime$.
			\begin{itemize}
				\item Suppose $\hatu$ is a grey occurrence.

					\begin{figure}[h]
						\begin{prooftree}
							\AxiomCm{C_1: P(\oldgam\occ{u}) \lor Q(\hatu)}
							\AxiomCm{C_2: \lnot Q(\hatuPrime)}
							\BinaryInfCm{C: P(\newgam\occ{\delta_i})}
						\end{prooftree} 
						\caption{Example for this case}
					\end{figure}

					Then by Lemma~\ref{lemma:arrow_from_grey_to_colored}, there is an arrow from a term containing $u$ to a term containing $\gamma_j\occ{u}$ in $\arrFinal(C)$.
					As $\hatu\sigma$ is a term containing the $\Delta$-term $\delta_i$, the term at the position of $\hatu$ in $\AI\de(C)$ is $\lifboth{\hatu\sigma}\tau$, which by assumption contains $z_i$. 
					But there is an arrow from this term containing $z_i$ to $\newgam\occ{z_i}$, so $z_i \ltArrC z_j$.


					\begin{comment}% version with "less induction hypothesis"
						As $\hatu$ and $\hatuPrime$ are at corresponding positions in the resolved literal, their arrows are merged, so it suffices to show that there is an arrow from a term containing an occurrence of $z_i$ to the position of $\hatuPrime$.

						\begin{itemize} 
							\item Suppose $\hatuPrime$ is a term which does not contain a $\Delta$-term.
								Then as $\hatuPrime\sigma$ is a term containing a $\Delta$-term, we know that $\hatuPrime$ contains a variable, say $v$, for which we have that $v\sigma$ is a term which contains a $\Delta$-term.
								Then by Remark~\substremarkref, we can assume that there is an arrow from a term containing an occurrence of $z_i$ to the position of $\hatuPrime$.

							\item Suppose that $\hatuPrime$ is a term containing a $\Delta$-term.
								Then as $\hatu\sigma=\hatuPrime\sigma$, 

						\end{itemize}
					\end{comment}


				\item Suppose $\hatu$ occurs in a maximal colored term which is a $\Gamma$-term.

					\begin{figure}[h]
						\begin{prooftree}
							\AxiomCm{C_1: P(\oldgam\occ{u}) \lor Q(\gamma_k\occurat{\hatu}{p}) }
							\AxiomCm{C_2: \lnot Q(\gamma_{m}\occurat{\hatuPrime}{p})}
							\BinaryInfCm{C: P(\newgam\occ{\delta_i})}
						\end{prooftree} 
						\begin{prooftree}
							\AxiomCm{C_1: Q(\oldgam\occ{\hatu}) }
							\AxiomCm{C_2: \lnot Q(\gamma_m\occ{\hatuPrime})}
							\BinaryInfCm{C: \square }
							\noLine
							\UnaryInfCm{\comm{\newgam\occ{\delta_i}\text{ occurs in the interpolant}}}
						\end{prooftree} 
						\caption{Examples for this case}
					\end{figure}

					Then either $\hatu$ is the occurrence of $u$ in $\oldgam\occ{\hatu}$ or it occurs in a different $\Gamma$-term~$\gamma_j\occ{\hatu}$.
					In the latter case, by Lemma~\ref{lemma:arrow_from_colored_to_colored}, there is a merge edge between $\oldgam\occ{\hatu}$ and $\gamma_j\occ{\hatu}$. \mytodo{or that other combination, which is fine as well}
					Hence in both cases, it suffices to show that there is an arrow from a term containing an occurrence of $z_i$ to $\hatuPrime$.

					We distinguish on the shape of $\hatuPrime$:
					\begin{itemize}
							\item $\hatuPrime$ is a term which does not contain a $\Delta$-term.
								Then it contains a variable that is be substituted by $\sigma$ by a term which contains a $\Delta$-term as $u\sigma = \hatuPrime\sigma$ is a term containing a $\Delta$-term.
							We denote by $v$ the variable in $\hatuPrime$ which is substituted by a term containing a $\Delta$-term in case $\hatuPrime$ is a grey term.

							In the course of the unification algorithm, there are further unifications of $v$ since we know that $u\sigma = v\sigma$ is a term containing a $\Delta$-term. 
							Therefore by Remark~\substremarkref, we can assume that there is an appropriate arrow to $\hatuPrime$.

						\item $\hatuPrime$ is a term which contains a $\Delta$-term.
							As $\hatuPrime$ occurs in a $\Gamma$-term in $C_1$, say in $\gamma_m\occ{\hatuPrime}$, $C_1$ contains a multicolored $\Gamma$-term.
							Hence the corresponding term in $\AI\de(C_1)$, is of the form $\gamma_m\occ{z_{i\bm '}}$ for some $i\bm'$. 
							Observe that $i\bm'$ in general is not equal to $i$ as demonstrated in Example~\ref{exa:lifting_var_only_abstraction_in_arrow_proof}, even though we have that $\hatuPrime\sigma = u\sigma$.
							This is because the lifting variables in $\AI(\cdot)$ represent abstractions of the terms in the clauses of the resolution derivation (cf.\ Lemma~\ref{lemma:lifting_var_refers_to_abstraction_of_term}). 
							Therefore we only know by the induction hypothesis that $z_{i\bm'} \ltArrCOne \lifboth{\gamma_m\occ{z_{i\bm'}}} = \lifboth{\hatuPrime}$.

							However by Lemma~\ref{lemma:literals_clauses_equal} and due to the fact that $\hatu$ and $\hatuPrime$ respectively occur in the resolved literal, $\lifdeltanovar{\hatu \sigma}\tau = \lifdeltanovar{\hatuPrime \sigma}\tau$.
							As 
							$\lifdeltanovar{\hatu \sigma}\tau = \lifdeltanovar{\delta_i}\tau = z_i\tau $
							as well as
							$ \lifdeltanovar{\hatuPrime \sigma}\tau = \lifdeltanovar{z_{i\bm'}\sigma}\tau = z_{i\bm'}\tau$,
							we must have that $z_i\tau = z_{i\bm'}\tau$.
							As however $u\sigma = \delta_i$, by the definition of $\aiu$, we have that $\{z_i\mapsto z_i\} \in \tau$, so $z_{i\bm'}\tau = z_i$.

							Since $\tau$ is applied to every literal in $\AI\de(C)$ and in $\AI\de(C_1)$ an arrow from a term containing $z_{i\bm'}$ to $\hatuPrime$ exists,
							the same arrow applied to $\AI\de(C)$ points from a term containing $z_{i\bm'}\tau = z_i$ to $\hatuPrime$.
							Therefore $z_i \ltArrC z_j$.


					\end{itemize}


				\item Suppose $\hatu$ occurs in a maximal colored term which is a $\Delta$-term.
						\begin{prooftree}
							\AxiomCm{C_1: P(\oldgam\occ{u}) \lor Q(\delta_k\occurat{\hatu}{p}) }
							\AxiomCm{C_2: \lnot Q(\delta_{m}\occurat{\hatuPrime}{p})}
							\BinaryInfCm{C: P(\newgam\occ{\delta_i})}
						\end{prooftree} 

						By Lemma~\ref{lemma:arrow_for_variables_in_differently_colored_terms}, \mytodo{}


			\end{itemize}

			\textbf{The substitution can also introduce a grey term containing a delta term, make sure to handle that!}

			\textbf{The substitution can also introduce a gamma term containing a delta term, make sure to handle that!}

		\item[Suppose for some variable $v$ in $\AI\de(C_1)$ or $\AI\de(C_2)$, $v\sigma = \newgam\occ{\delta_i}$ for some $i$.]\hfill\\
			As $v$ is affected by the unifier, it occurs in the literal being unified, say w.l.o.g.~in $l$ in $C_1$.
			At some well-defined point in the unification algorithm, $v$ is substituted by an abstraction of $\newgam\occ{\delta_i}$.
			Let $p$ be the position of the occurrence of $v$ in $l$ which causes this substitution.
			Furthermore, let $p'$ be the position corresponding to $p$ in~$l'$.

			Note that any arrow from or to $p'$ also applies to $p$ in $\arrFinal(C)$ and hence to $\newgam\occ{z_i}$ as they are merged due to occurring in the resolved literal.
			So it suffices to show that there is an arrow from an appropriate lifting variable to $p'$. We denote the term at $p'$ by $t$.

			Note that $t\sigma = \newgam\occ{\delta_i}$.
			So $t$ is either a $\Gamma$-term containing a $\Delta$-term, in which case we know that there is an appropriate arrow by the induction hypothesis as $t$ occurs in $l'$ in $C_2$, or $t$ is an abstraction of $\newgam\occ{\delta_i}$, in which case we can assume the existence of an appropriate arrow by Remark~\substremarkref.
			\todo{recheck this paragraphs w.r.t.~$\ltArrC$}
			\qedhere

			\begin{comment}
				We distinguish cases based on the shape of the term at $p'$, which we denote by $t$:
				\begin{itemize}
					\item $t$ is a variable, say $x$. 
						Then $x$ occurs elsewhere in $l'$ and $x\sigma = \newgam\occ{\delta_i}$.
						By Remark~\substremarkref, we can assume that there is an appropriate arrow for $x\sigma$ ad $p'$.

					\item $t$ is a $\Gamma$-term which does not contain a $\Delta$-term.
						$t$ is however an abstraction of $\newgam\occ{\delta_i}$, so it contains at least one variable which is substituted by a term $s$.
						Hence by Remark~\substremarkref, we can assume that an appropriate arrow pointing to $t$ exists.

						We distinguish based on the shape $s$:
						\begin{itemize}
							\item $s$ is a grey or $\Delta$-colored term containing a $\Delta$-term.
								Then a $\Delta$-term is introduced into a $\Gamma$-term by means of unification, so by the respective case of the proof, there is an arrow from an appropriate lifting variable to~$t$.

							\item $s$ is a multicolored $\Gamma$-term.
								Then by Remark~\substremarkref, we can assume that an appropriate arrow pointing to $t$ exists.
						\end{itemize}
					\item $t$ is a $\Gamma$-term containing a $\Delta$-term.
				\end{itemize}
			\end{comment}


	\end{description}

\end{proof}

\section*{something about when i started with connected components}

unification is for resolved literals.

connections between resolved literals and the rest of the clauses is covered by arrows.

if a term enters, merge arrows ensure that everything is propagated.

the special thing about colored occurrences is the fact that they can create multicolored terms in cooperation with grey occurrences..

a variable only occurs in a clause if it was never substituted by anything. Hence in particular all grey occurrences are ``original'' (TODO: renamings of variables)

Let $u$ be a grey occurrence.
let $f(u)$ be a colored occurrence.
either it is original, then we are fine by arrow propagation.
otherwise it has been introduced, but then it has used the network of another variable.

more precisely: %a variable from another clause occurs in a related literal 
a variable $v$ occurs in a related literal in a related position in another clause as $u$ in $f(u)$.
so the variable is substituted by a term containing $u$, say $t\occ{u}$
the arrows at the entry points are merged.

effect: 
$t\occ{u}$ occurs at every grey occurrence of $v$. all arrows mentioning them are merged with the ones mentioning the entry point.
this is justified as the terms there appear ``as they are'', i.e.\ as they are produced at the entry point.

however a colored occurrence cannot be produced from a grey occurrence ($\mgu(x, f(u))$) but only if a grey occ is in the literal and a colored occ is elsewhere in the clause (the network of the other var). but then there are (directed) arrows.


Every variable has a connected network in a clause. 

there is a barrier between colored terms.

~

~



\end{document}
