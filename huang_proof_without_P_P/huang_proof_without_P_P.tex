\documentclass[,%fontsize=11pt,%
	paper=a4,% 
	%landscape,
	%DIV12, % mehr text pro seite als defaultyyp
	DIV14, 
	%DIV=calc,%
	%twoside=false,%
	liststotoc,
	bibtotoc,
	draft=false,% final|draft % draft ist platzsparender (kein code, bilder..)
	%titlepage,
	numbers=noendperiod
]{scrartcl}

\usepackage{lscape}
\usepackage{stackengine}


\usepackage[utf8]{inputenc}
\usepackage[T1]{fontenc}
\usepackage[english]{babel}

\usepackage{enumerate}
\usepackage{paralist}
\usepackage{tikz}
\usetikzlibrary{shapes,arrows,backgrounds,graphs,%
	matrix,patterns,arrows,decorations.pathmorphing,decorations.pathreplacing,%
	positioning,fit,calc,decorations.text,shadows%
}


\usepackage{comment} 

\usepackage{etoolbox} % fixes fatal error caused by combining bm, stackengine, hyperref (seriously?)
% http://tex.stackexchange.com/questions/22995/package-incompatibilites-etoolbox-hyperref-and-bm-standalone

\usepackage{etex} % else error on too many packages

% includes
\usepackage{algorithm}
%\usepackage{algorithmic} % conflicts with algpseudocode
\usepackage{algpseudocode}
%\newcommand*\Let[2]{\State #1 $\gets$ #2}
\algrenewcommand\alglinenumber[1]{
{\scriptsize #1}}
\algrenewcommand{\algorithmicrequire}{\textbf{Input:}}
\algrenewcommand{\algorithmicensure}{\textbf{Output:}}


%\usepackage[multiple]{footmisc} % footnotes at the same character separated by ','

\usepackage{multicol}

\usepackage{afterpage}

\usepackage{changepage} % for adjustwidth
\usepackage{caption} % for \ContinuedFloat

\usepackage{tikz}
\usetikzlibrary{shapes,arrows,backgrounds,graphs,%
matrix,patterns,arrows,decorations.pathmorphing,decorations.pathreplacing,%
positioning,fit,calc,decorations.text,shadows%
}

\usepackage{bussproofs}
\EnableBpAbbreviations


\usepackage{amsmath}
\usepackage{amsthm}
\usepackage{amssymb} % the reals
\usepackage{mathtools} % smashoperator

\usepackage{bm} % bm, bold math symbols

\usepackage{thm-restate} % restatable env

% needs extra work and fails on some label here
%\usepackage{cleveref} % cref, apparently better than autoref of hyperref 

\usepackage{nicefrac} % nicefrac

\usepackage{mathrsfs} % mathscr

\usepackage{pst-node} % http://tex.stackexchange.com/questions/35717/how-to-draw-arrows-between-parts-of-an-equation-to-show-the-math-distributive-pr

\usepackage{stackengine}

\usepackage{thmtools} % advanced thm commands (declaretheorem)


\usepackage{nameref} % reference name of thm instead of counter

\usepackage{todonotes}

% conflict with beamer
%\usepackage{paralist} % compactenum

\usepackage{hyperref}
%\hypersetup{hidelinks}  % don't give options to usepackage, it doesn't work with beamer
%\hypersetup{colorlinks=false}  % don't give options to usepackage, it doesn't work with beamer


% \usepackage{enumitem} % labels for enumerate % breaks beamer and memoir itemize


\usepackage{url} 


\usepackage[format=hang,justification=raggedright]{caption}% or e.g. [format=hang]

\usepackage{cancel} % \cancel

\usepackage{lineno}


% commands

% logic etcs
%\newcommand{\ex}[2]{\bigskip\section*{Exercise #1: \begin{minipage}[t]{.80\linewidth} \small \textnormal{\it #2} \end{minipage} } }

\newcommand{\ex}[2]{\bigskip \noindent\textbf{Exercise #1.} \textit{#2} \smallskip}

\newcommand{\comm}[1]{{\color{gray} // #1 }}


\newcommand{\true}[0]{\textbf{1}}
\newcommand{\false}[0]{\textbf{0}}
\newcommand{\tr}{\true}
\newcommand{\fa}{\false}

\newcommand{\ra}{\rightarrow}
\newcommand{\Ra}{\Rightarrow}
\newcommand{\la}{\leftarrow}
\newcommand{\La}{\Leftarrow}

\newcommand{\lra}{\leftrightarrow}
\newcommand{\Lra}{\Leftrightarrow}

\newcommand{\NKZ}{\textbf{NK2}}

%\DeclareMathOperator{\syneq}{\equiv} %spacing seems wrong, therefore defined as newcommand below
\DeclareMathOperator{\limpl}{\supset}
\DeclareMathOperator{\liff}{\lra}
\DeclareMathOperator{\semiff}{\Lra}
\newcommand{\syneq}{\equiv}
\newcommand{\union}{\cup}
\newcommand{\bigunion}{\bigcup}
\newcommand{\intersection}{\cap}
\newcommand{\bigintersection}{\bigcap}
\newcommand{\intersect}{\intersection}
\newcommand{\bigintersect}{\bigintersection}

\newcommand{\powerset}{\mathcal{P}}

\newcommand{\entails}{\vDash}
\newcommand{\notentails}{\nvDash}
\newcommand{\proves}{\vdash}

\newcommand{\vm}{\ensuremath{\vv_\mathcal{M}}}
\newcommand{\Dia}{\ensuremath{\lozenge}}

\newcommand{\spaced}[1]{\ \ #1 \ \ }
\newcommand{\spa}[1]{\spaced{#1}}
\newcommand{\spas}[1]{\;{#1}\;}
\newcommand{\spam}[1]{\;\,{#1}\;\,}

% functions
\DeclareMathOperator{\sk}{sk}
\DeclareMathOperator{\mgu}{mgu}
\DeclareMathOperator{\dom}{dom}
\DeclareMathOperator{\ran}{ran}

\DeclareMathOperator{\id}{id}
\DeclareMathOperator{\Fun}{FS}
\DeclareMathOperator{\Pred}{PS}
\DeclareMathOperator{\Lang}{L}
\DeclareMathOperator{\ar}{ar}
\DeclareMathOperator{\PI}{PI}
\DeclareMathOperator{\LI}{LI}
\DeclareMathOperator{\Congr}{Congr}
\DeclareMathOperator{\Refl}{Refl}
\DeclareMathOperator{\aiu}{au}
\DeclareMathOperator{\expa}{unfold-lift}

\newcommand{\PIinc}{\LI}
\newcommand{\PIincde}{\LIde}

\newcommand{\LIde}{\ensuremath{\LI^\Delta}}

\newcommand{\LIcl}{\ensuremath{\LI_{\operatorname{cl}}}}
\newcommand{\LIclde}{\ensuremath{\LI_{\operatorname{cl}}^\Delta}}

\newcommand{\cll}{\ensuremath{_{\operatorname{LIcl}}}}
\newcommand{\cllde}{\ensuremath{_{\operatorname{LIcl}^\Delta}}}

%\newcommand{\lifi}{\mathop{\ell\text{}i}}
\newcommand{\lifiboth}[1]{\ensuremath{\LIcl(#1)}}
\newcommand{\lifidelta}[1]{\ensuremath{\LIclde(#1)}}


%\DeclareMathOperator{\abstraction}{abstraction}

%\newcommand{\sk}{\ensuremath{\mathrm{sk}}}
%\newcommand{\mgu}{\ensuremath{\mathrm{mgu}}}
%\newcommand{\Fun}{\ensuremath{\mathrm{FS}}}
%\newcommand{\Pred}{\ensuremath{\mathrm{PS}}}
%\newcommand{\PI}{\ensuremath{\mathrm{PI}}}
%\newcommand{\Lang}{\ensuremath{\mathrm{L}}}
%\newcommand{\ar}{\ensuremath{\mathrm{ar}}}

\DeclareMathOperator{\AI}{AI}
\newcommand{\AIde}{\ensuremath{\AI^\Delta}}
\newcommand{\AImatrix}{\ensuremath{\AI_\mathrm{mat}}}
\newcommand{\AImatrixde}{\ensuremath{\AI_\mathrm{mat}^\Delta}}
\newcommand{\AImat}{\AImatrix}
\newcommand{\AImatde}{\AImatrixde}
\newcommand{\AIclause}{\ensuremath{\AI_\mathrm{cl}}}
\newcommand{\AIcl}{\AIclause}
\newcommand{\AIclde}{\AIclausede}
\newcommand{\AIclausede}{\ensuremath{\AIclause^\Delta}}
\newcommand{\fromclause}{\ensuremath{_{\operatorname{AIcl}}}}
\newcommand{\fromclausede}{\ensuremath{_{\operatorname{AIcl}^\Delta}}}
\newcommand{\cl}{\fromclause}
\newcommand{\clde}{\fromclausede}

\newcommand{\Q}{\ensuremath{Q}}

\newcommand{\AIcol}{\ensuremath{\AI_\mathrm{col}}}
\newcommand{\AIcolde}{\AIcol^\Delta}

\newcommand{\AIany}{\ensuremath{\AI_\mathrm{*}}}
\newcommand{\AIanyde}{\AIany^\Delta}

\newcommand{\AIclpre}{\AIclause^\bullet}
\newcommand{\AImatpre}{\AImatrix^\bullet}

\newcommand{\PS}{\Pred}
\newcommand{\FS}{\Fun}

\DeclareMathOperator{\LangSym}{\mathcal{L}}

%\newcommand{\mguarr}{\sim_\ra}
\newcommand{\mguarr}{\mapsto_{\mgu}}


%\newcommand{\Trans}{\ensuremath{\mathrm{T}}}
%\newcommand{\Trans}{\ensuremath{\mathrm{T}}}
\DeclareMathOperator{\Trans}{T}
\DeclareMathOperator{\TransInv}{T^{-1}}

\DeclareMathOperator{\FAX}{F_{Ax}}
\DeclareMathOperator{\EAX}{E_{Ax}}
%\newcommand{\FAX}{\ensuremath{\mathrm{F_{Ax}}}}
%\newcommand{\EAX}{\ensuremath{\mathrm{E_{Ax}}}}

%\newcommand{\TransAll}{\ensuremath{\Trans_{\mathrm{Ax}}}}
\DeclareMathOperator{\TransAll}{\Trans_{Ax}}
%\newcommand{\FAX}{\ensuremath{\mathrm{F_{Ax}}}}

\DeclareMathOperator{\defeq}{\stackrel{\mathrm{def}}{=}}

\newcommand{\subst}[1]{[#1]}
\newcommand{\abstractionOp}[1]{\{#1\}}

\newcommand{\subformdefinitional}[1]{\ensuremath{D_{\Sigma(#1)}}}


%\newcommand{\lift}[3]{\operatorname{Lift}_{#1}(#2; #3)}
%\newcommand{\lift}[3]{\operatorname{Lift}_{#1,#3}(#2)}
%\newcommand{\lift}[3]{\operatorname{Lift}_{#1,#3}[#2]}
%\newcommand{\lift}[3]{\overline{#2}_{#1,#3}}
\newcommand{\lifsym}{\ell}
%\newcommand{\lift}[3]{\lifsym_{#1,#3}[#2]}
\newcommand{\lift}[3]{\lifsym_{#1}^{#3}[#2]}
\newcommand{\liftnovar}[2]{\lifsym_{#1}[#2]}

%\newcommand{\lft}[3]{\lifsym_{#1,#2}[#3]}
\newcommand{\lft}[3]{\lift{#1}{#3}{#2}}
\newcommand{\lifboth}[1]{\lifsym[#1]}

%\newcommand{\lifi}{\mathop{\ell\text{}i}}
%\newcommand{\lifiboth}[1]{\lifi[#1]}
%\newcommand{\lifidelta}[1]{\lifi_\Delta^x[#1]}
%\newcommand{\lifideltanovar}[1]{\lifi_\Delta[#1]}

\newcommand{\lifdelta}[1]{\lift{\Delta}{#1}{x}}
\newcommand{\lifdeltanovar}[1]{\liftnovar{\Delta}{#1}}
\newcommand{\lifgamma}[1]{\lift{\Gamma}{#1}{y}}
\newcommand{\lifgammanovar}[1]{\liftnovar{\Gamma}{#1}}
\newcommand{\lifphinovar}[1]{\liftnovar{\Phi}{#1}}
\newcommand{\lifphi}[1]{\lift{\Phi}{#1}{z}}

\DeclareMathOperator{\arr}{\mathcal{A}}
%\DeclareMathOperator{\arrFinal}{{\mathcal{A}^{\bm*}}}
\DeclareMathOperator{\arrFinal}{{\mathcal{\bm{\hat}A}}}
\DeclareMathOperator{\warr}{\marr}
\DeclareMathOperator{\marr}{\mathcal{M}}

\DeclareMathOperator{\apath}{\leadsto}
\DeclareMathOperator{\mpath}{\leadsto_=}
\DeclareMathOperator{\notapath}{\not\leadsto}
\DeclareMathOperator{\notmpath}{\not\leadsto_=}

\newcommand{\ltArrC}{<_{\arrFinal(C)}}
\newcommand{\ltAC}{<_{\arr(C)}}
\newcommand{\ltArrCOne}{<_{\arrFinal(C_1)}}
\newcommand{\ltArrCTwo}{<_{\arrFinal(C_2)}}
%\newcommand{\ltArrC}{<_{\scalebox{0.6}{$\arrFinal(C)$}}}
\newcommand{\ltArr}{<_{\scalebox{0.6}{$\arrFinal$}}}

\newcommand{\bhat}{\bm\hat}
\newcommand{\bbar}{\bm\bar}
\newcommand{\bdot}{\bm\dot}

%\usepackage{yfonts}
\usepackage{upgreek}
\DeclareMathAlphabet{\mathpzc}{OT1}{pzc}{m}{it}
%\DeclareMathOperator{\pos}{\mathscr{P}}
%\DeclareMathOperator{\pos}{\mathpzc{p}}
%\DeclareMathOperator{\pos}{{\rho}}
\DeclareMathOperator{\pos}{{\operatorname P}}
%\DeclareMathOperator{\pos}{P}
\DeclareMathOperator{\poslit}{\pos_\mathrm{lit}}
\DeclareMathOperator{\posterm}{\pos_\mathrm{term}}
%\newcommand{\poslit}[1]{\ensuremath{p_\text{lit}(#1)}}
%\newcommand{\posterm}[1]{\ensuremath{p_\text{term}(#1)}}
\newcommand{\at}[1]{|_{#1}}

\newcommand{\UICm}[1]{\UnaryInfCm{#1}}
\newcommand{\UnaryInfCm}[1]{\UnaryInfC{$#1$}}
\newcommand{\BICm}[1]{\BinaryInfCm{#1}}
\newcommand{\BinaryInfCm}[1]{\BinaryInfC{$#1$}}
\newcommand{\RightLabelm}[1]{\RightLabel{$#1$}}
\newcommand{\LeftLabelm}[1]{\LeftLabel{$#1$}}
\newcommand{\AXCm}[1]{\AxiomCm{#1}}
\newcommand{\AxiomCm}[1]{\AxiomC{$#1$}}
\newcommand{\mt}[1]{\textnormal{#1}}

\newcommand{\UnaryInfm}[1]{\UnaryInf$#1$}
\newcommand{\BinaryInfm}[1]{\BinaryInf$#1$}
\newcommand{\Axiomm}[1]{\Axiom$#1$}



% math
\newcommand{\calI}{\ensuremath{\mathcal{I}}}

\newcommand{\tupleShort}[2]{\ensuremath{(#1_1,\dotsc,#1_{#2})}}
\newcommand{\tuple}[2]{\ensuremath{(#1_1,\:#1_2\:,\dotsc,\:#1_{#2})}}
\newcommand{\setelements}[2]{\ensuremath{\{#1_1,\:#1_2\:,\dotsc,\:#1_{#2}\}}}
\newcommand{\pathelements}[2]{\ensuremath{ (#1_1,\:#1_2\:,\dotsc,\:#1_{#2}) }}

\newcommand{\elems}[1]{\ensuremath{#1_1,\dotsc, #1_{n}) }}

\newcommand{\defiemph}[1]{\emph{#1}}

\newcommand{\setofbases}{\ensuremath{\mathcal{B}}}
\newcommand{\setofcircuits}{\ensuremath{\mathcal{C}}}

\newcommand{\reals}{\ensuremath{\mathbb{R}}}
\newcommand{\integers}{\ensuremath{\mathbb{Z}}} 
\newcommand{\naturalnumbers}{\ensuremath{\mathbb{N}}}

% general
\newcommand{\zit}[3]{#1\ \cite{#2}, #3}
\newcommand{\zitx}[2]{#1\ \cite{#2}}
\newcommand{\footzit}[3]{\footnote{\zit{#1}{#2}{#3}}}
\newcommand{\footzitx}[2]{\footnote{\zitx{#1}{#2}}}

\newcommand{\ite}{\begin{itemize}}
\newcommand{\ete}{\end{itemize}}

\newcommand{\bfr}{\begin{frame}}
\newcommand{\efr}{\end{frame}}

\newcommand{\ilc}[1]{\texttt{#1}}


% misc

% multiframe
\usepackage{xifthen}% provides \isempty test
% new counter to now which frame it is within the sequence
\newcounter{multiframecounter}
% initialize buffer for previously used frame title
\gdef\lastframetitle{\textit{undefined}}
% new environment for a multi-frame
\newenvironment{multiframe}[1][]{%
\ifthenelse{\isempty{#1}}{%
% if no frame title was set via optional parameter,
% only increase sequence counter by 1
\addtocounter{multiframecounter}{1}%
}{%
% new frame title has been provided, thus
% reset sequence counter to 1 and buffer frame title for later use
\setcounter{multiframecounter}{1}%
\gdef\lastframetitle{#1}%
}%
% start conventional frame environment and
% automatically set frame title followed by sequence counter
\begin{frame}%
\frametitle{\lastframetitle~{\normalfont \Roman{multiframecounter}}}%
}{%
\end{frame}%
}




% http://texfragen.de/hurenkinder_und_schusterjungen
\usepackage[all]{nowidow}



% force no overlong lines:
%\tolerance=1 % tolerance for how badly spaced lines are allowed, less means "better" lines
\tolerance=500 %  need more tolerance for equations
%\emergencystretch=\maxdimen
%\emergencystretch=200pt
%\setlength{\emergencystretch}{3em}
%\hyphenpenalty=10000 % forces no hyphenation
%\hbadness=10000


% http://tex.stackexchange.com/questions/35717/how-to-draw-arrows-between-parts-of-an-equation-to-show-the-math-distributive-pr
\tikzset{square arrow/.style={to path={ -- ++(.0,-.15)  -| (\tikztotarget)}}}
\tikzset{square arrow2/.style={to path={ -- ++(.0,-.25)  -| (\tikztotarget)}}}
%\tikzset{square arrow/.style={to path={ -- ++(00,-.01) -- ++(0.5,-0.1) -- ++(0.5,-0.1) -| (\tikztotarget)},color=red}}


% have arrows from a to b and from c to d here
% just use: texttext\arrowA texttest \arrowB texttext
\newcommand{\arrowA}{\tikz[overlay,remember picture] \node (a) {};}
\newcommand{\arrowB}{\tikz[overlay,remember picture] \node (b) {};}
\newcommand{\drawAB}{
	\tikz[overlay,remember picture]
	{\draw[->,bend left=5,color=red] (a.south) to (b.south);}
	%{\draw[->,square arrow,color=red] (a.south) to (b.south);}
}
\newcommand{\arrowAP}{\tikz[overlay,remember picture] \node (ap) {};}
\newcommand{\arrowBP}{\tikz[overlay,remember picture] \node (bp) {};}
\newcommand{\drawABP}{
	\tikz[overlay,remember picture]
	{\draw[->,bend right=5,color=red] (ap.south) to (bp.south);}
	%{\draw[->,square arrow,color=red] (a.south) to (b.south);}
}

\newcommand{\arrowAB}{\tikz[overlay,remember picture] \node (ab) {};}
\newcommand{\arrowBA}{\tikz[overlay,remember picture] \node (ba) {};}
\newcommand{\drawAABB}{
	\tikz[overlay,remember picture]
	%{\draw[->,bend left=80] (a.north) to (b.north);}
	{\draw[->,square arrow,color=brown] (ab.south) to (ba.south);
	\draw[->,square arrow,color=brown] (ba.south) to (ab.south);}
}


\newcommand{\arrowCD}{\tikz[overlay,remember picture] \node (cd) {};}
\newcommand{\arrowDC}{\tikz[overlay,remember picture] \node (dc) {};}
\newcommand{\drawCCDD}{
	\tikz[overlay,remember picture]
	%{\draw[->,bend left=80] (a.north) to (b.north);}
	{\draw[<->,dashed,square arrow,color=green] (cd.south) to (dc.south); }
}



\newcommand{\arrowC}{\tikz[overlay,remember picture] \node (c) {};}
\newcommand{\arrowD}{\tikz[overlay,remember picture] \node (d) {};}
\newcommand{\drawCD}{
	\tikz[overlay,remember picture]
	{\draw[->,square arrow,color=blue] (c.south) to (d.south);}
}

\newcommand{\arrowE}{\tikz[overlay,remember picture] \node (e) {};}
\newcommand{\arrowF}{\tikz[overlay,remember picture] \node (f) {};}
\newcommand{\drawEF}{
	\tikz[overlay,remember picture]
	{\draw[->,square arrow2,color=orange] (e.south) to (f.south);}
}


\newcommand{\arrAP}{\arrowAP}
\newcommand{\arrBP}{\arrowBP}
\newcommand{\arrA}{\arrowA}
\newcommand{\arrB}{\arrowB}
\newcommand{\arrC}{\arrowC}
\newcommand{\arrD}{\arrowD}
\newcommand{\arrE}{\arrowE}
\newcommand{\arrF}{\arrowF}


\DeclareMathOperator{\simgeq}{\scalebox{0.92}{$\gtrsim$}}

\newcommand{\refsub}[2]{\hyperref[#2]{\ref*{#1}.\ref*{#2}}}

%\newcommand{\sigmarange}[2]{\sigma_{#1}^{#2} }
\newcommand{\sigmarange}[2]{\sigma_{(#1,#2)} }
\newcommand{\sigmaz}[1]{\sigmarange{0}{#1} }
\newcommand{\sigmazi}[0]{\sigmaz{i} }

\DeclareMathOperator{\lit}{lit}

%\def\fCenter{\ \proves\ }
\def\fCenter{\proves}

\newcommand{\prflbl}[2]{\RightLabel{\footnotesize $#1, #2$} }
%\newcommand{\prflblid}[1]{\RightLabel{$#1, \id$} }
\newcommand{\prflblid}[1]{\RightLabel{\footnotesize $#1$} }

\DeclareMathOperator{\resruleres}{res}
\DeclareMathOperator{\resrulefac}{fac}
\DeclareMathOperator{\resrulepar}{par}
\newcommand{\lkrule}[2]{\ensuremath{\operatorname{#1}:#2}} % operatorname fixes spacing issues for =

\newcommand{\parti}[4]{\ensuremath{ \langle (#1; #2), (#3; #4)\rangle  }}

\newcommand{\partisym}{\ensuremath{\chi}}

\newcommand{\occur}[1]{\ensuremath{[#1]}}
\newcommand{\occ}[1]{\occur{#1}}

\newcommand{\occurat}[2]{\ensuremath{{\occur{#1}_{#2}}}}
\newcommand{\occat}[2]{\occurat{#1}{#2}}
\newcommand{\occatp}[1]{\occurat{#1}{p}}
\newcommand{\occatq}[1]{\occurat{#1}{q}}

\newcommand{\colterm}[1]{\zeta_{#1}}



% fix restateable spacing 
%http://tex.stackexchange.com/questions/111639/extra-spacing-around-restatable-theorems

\makeatletter

\def\thmt@rst@storecounters#1{%
%THIS IS THE LINE I ADDED:
\vspace{-1.9ex}%
  \bgroup
        % ugly hack: save chapter,..subsection numbers
        % for equation numbers.
  %\refstepcounter{thmt@dummyctr}% why is this here?
  %% temporarily disabled, broke autorefname.
  \def\@currentlabel{}%
  \@for\thmt@ctr:=\thmt@innercounters\do{%
    \thmt@sanitizethe{\thmt@ctr}%
    \protected@edef\@currentlabel{%
      \@currentlabel
      \protect\def\@xa\protect\csname the\thmt@ctr\endcsname{%
        \csname the\thmt@ctr\endcsname}%
      \ifcsname theH\thmt@ctr\endcsname
        \protect\def\@xa\protect\csname theH\thmt@ctr\endcsname{%
          (restate \protect\theHthmt@dummyctr)\csname theH\thmt@ctr\endcsname}%
      \fi
      \protect\setcounter{\thmt@ctr}{\number\csname c@\thmt@ctr\endcsname}%
    }%
  }%
  \label{thmt@@#1@data}%
  \egroup
}%

\makeatother




\newcommand{\mymark}[1]{\ensuremath{(#1)}}
\newcommand{\markA}{\mymark \circ}
\newcommand{\markB}{\mymark *}
\newcommand{\markC}{\mymark \divideontimes}

\newcommand{\wrong}[1]{{\color{red}WRONG: #1}}
\newcommand{\NB}[1]{{\color{blue}NB: #1}}
\newcommand{\hl}[1]{{\color{orange} #1}}
\newcommand{\mytodo}[1]{{\color{red}TODO: #1}}
\newcommand{\largered}[1]{{

	  \LARGE\bfseries\color{red}
		#1

}}
\newcommand{\largeblue}[1]{{

	  \large\bfseries\color{blue}
		#1

}}




\usepackage{ulem} %  \dotuline{dotty} \dashuline{dashing} \sout{strikethrough}
\normalem

\usepackage{tabu} % tabular also in math mode (and much more)

\usepackage[color]{changebar} %  \cbstart, \cbend
\cbcolor{red}



% http://tex.stackexchange.com/questions/7032/good-way-to-make-textcircled-numbers
\newcommand*\circled[1]{\tikz[baseline=(char.base)]{
\node[shape=circle,draw,inner sep=2pt] (char) {#1};}}



% http://tex.stackexchange.com/questions/43346/how-do-i-get-sub-numbering-for-theorems-theorem-1-a-theorem-1-b-theorem-2

\makeatletter
\newenvironment{subtheorem}[1]{%
  \def\subtheoremcounter{#1}%
  \refstepcounter{#1}%
  \protected@edef\theparentnumber{\csname the#1\endcsname}%
  \setcounter{parentnumber}{\value{#1}}%
  \setcounter{#1}{0}%
  \expandafter\def\csname the#1\endcsname{\theparentnumber.\Alph{#1}}%
  \ignorespaces
}{%
  \setcounter{\subtheoremcounter}{\value{parentnumber}}%
  \ignorespacesafterend
}
\makeatother
\newcounter{parentnumber}


\usepackage{tabularx}% http://ctan.org/pkg/tabularx
\newcolumntype{Y}{>{\centering\arraybackslash}X}

\newcommand{\mycols}[2][3]{
	\noindent\begin{tabularx}{\textwidth}{*{#1}{Y}}
		#2
	\end{tabularx}%
}


\newcommand{\definethms}{

	%\declaretheorem[title=Theorem,qed=$\triangle$,parent=chapter]{thm}
	\newcommand{\thmqed}{$\square$} % for thms without proof
	\newcommand{\propqed}{$\square$} % for props without proof
	\declaretheorem[title=Theorem]{thm}
	\declaretheorem[title=Proposition,sibling=thm]{prop}
	\declaretheorem[title=Conjectured Proposition,sibling=thm]{cprop}

	%\declaretheorem[title=Lemma,parent=chapter]{lemma}
	\declaretheorem[sibling=thm]{lemma}
	\declaretheorem[sibling=thm,title=Conjectured Lemma]{clemma}
	\declaretheorem[title=Corollary,sibling=thm]{corr}
	\declaretheorem[sibling=thm,title=Definition,style=definition,qed=$\triangle$]{defi}
	%\declaretheorem[title=Definition,qed=$\triangle$,parent=chapter]{defi}
	\declaretheorem[title=Example,style=definition,qed=$\triangle$,sibling=thm]{exa}

	\declaretheorem[sibling=thm,title=Conjecture]{conj}

	\declaretheorem[title=Remark,style=remark,numbered=no,qed=$\triangle$]{remark}


}

\usepackage[matha]{mathabx} % the locial operators here have more space around them and [ and ] are thicker, also langle and rangle are a bit nicer; subseteq looks a bit weird

%\usepackage{MnSymbol} % again other symbols


\newcommand{\inference}{\ensuremath{\iota}}

\usepackage{cases} % numcases


% subsections also in toc
\setcounter{tocdepth}{2}

%\declaretheorem[title=Theorem,qed=$\triangle$,parent=chapter]{thm}
\newcommand{\thmqed}{$\square$} % for thms without proof
\newcommand{\propqed}{$\square$} % for props without proof
\declaretheorem[title=Theorem]{thm}
\declaretheorem[title=Proposition,sibling=thm]{prop}
%\declaretheorem[title=Lemma,parent=chapter]{lemma}
\declaretheorem[sibling=thm]{lemma}
\declaretheorem[title=Corollary,sibling=thm]{corr}
\declaretheorem[sibling=thm,title=Definition,style=definition,qed=$\triangle$]{defi}
%\declaretheorem[title=Definition,qed=$\triangle$,parent=chapter]{defi}
\declaretheorem[title=Example,style=definition,qed=$\triangle$,sibling=thm]{exa}

\declaretheorem[sibling=thm,title=Conjecture]{conj}

%\def\proofSkipAmount{ \vskip -0.5em}



%\usepackage{bussproof}

%\usepackage{vaucanson-g}
\usepackage{amssymb}
\usepackage{latexsym}

% for color-highlighted code
%\usepackage{color} % for grey comments
%\usepackage{alltt}

%\usepackage[doublespacing]{setspace}
\usepackage[onehalfspacing]{setspace}
%\usepackage[singlespacing]{setspace}
\usepackage{tabularx}
\usepackage{hyperref}
\usepackage{comment}
\usepackage{color}
\usepackage[final]{listings} % sourcecode in document
\usepackage{url}      % for urls
\usepackage{multicol}
\usepackage{float}
\usepackage{caption}
\usepackage{subfigure}
\usepackage{amsmath}
\usepackage{amssymb}

\usepackage{graphicx}

\usepackage[authoryear]{natbib} % \cite ; square|round etc.
%\usepackage[numbers,square]{natbib}
%\usepackage[square, authoryear]{natbib}
%\usepackage[language=english]{biblatex}

%\bibliographystyle{plain}
\bibliographystyle{alpha}
%\bibliographystyle{alphadin}
%\bibliographystyle{dinat}
%\bibliographystyle{chicago}
%\bibliographystyle{plainnat}

\bibdata{bib.bib}

\renewcommand*{\partformat}{\partname\ \thepart\ -}
\let\partheadmidvskip\

\newcommand{\comp}{\ensuremath{\text{comp}}}
% smaller url style
\makeatletter
\def\url@leostyle{%
\@ifundefined{selectfont}{\def\UrlFont{\sf}}{\def\UrlFont{\small\ttfamily}}}
\makeatother
\urlstyle{leo}

\newcommand{\myfig}[5] {
	\begin{figure}[tbph]
		\centering
		\includegraphics[#3]{#1}
		\caption[#4]{#5}
		\label{fig:#2}
	\end{figure}
}

\setlength{\parindent}{0em}
%\usepackage{thmtools} % actually already in latex_header.tex ...

\usepackage{amsthm}


\usepackage{tikz-qtree}

%\newcommand{\sig}[1]{{#1}_\Sigma}
%\newcommand{\p}[1]{{#1}_\Pi}
\newcommand{\sig}[1]{\stackrel{\Sigma}{#1}}
\newcommand{\p}[1]{\stackrel{\Pi}{#1}}

\newcommand{\e}[1]{\vskip .7em   \subsection*{#1}}


\def\proofSkipAmount{ \vskip -0.3em}

\newcommand{\lif}[1]{\lift{\Delta}{#1}{x}}
\newcommand{\lifboth}[1]{\lft{\Gamma\cup\Delta}{z}{#1}}

\begin{document}


\section{Proof of the correctness of Huang's algorithm without propositional refutations}


Intuition of $\sigma'$:

If we pull a substitution out of a lifting which replaces $\Delta$-terms, we also have to replace the $\Delta$-terms 
in the ``codomain'' of the substitution. This is the second case in the definition of $\sigma'$ below.

There is just a problem in the following case: $\lif{ f(x)\sigma }$, where $x\sigma = a$ and $f$ is a $\Delta$-symbol.
Then $\lif{ f(x)\sigma } = \lif{ f(a) } = x_i$, but $\lif{f(x)}\sigma = x_j$ with $i\neq j$.
The first case of the definition of $x_j$ then fixes this by replacing $x_j$ with $x_i$. 



\begin{lemma}
	\label{lemma:lif}

	Let $C$ be a clause and $\sigma$ a substitution.
	Let $t_1,\ldots,t_n$ be all maximal $\Delta$-terms in this context, i.e.\ those that occur in $C$ or $C\sigma$,  and 
	$x_1, \ldots, x_n$ the corresponding fresh variables to replace the $t_i$.
	Define $\sigma'$ such that for a variable $z$, 
	\[
		z \sigma' =
		\begin{cases} 
			x_l & \text{ if } z = x_k \text{ and } t_k\sigma = t_l  \\
			\lif{z\sigma} & \text{ otherwise}
		\end{cases} 
	\]

	Then
	$\lif{C\sigma} =
	\lif{C}\sigma'$.
\end{lemma}
Note that the definition of $\sigma'$ only depends on the $x_i$ and $t_i$.
\begin{proof}
	We prove this for an atom $P(s_1, \ldots, s_m)$ in $C$, which works since lifting and substitution commute over binary connectives and into an atom.

	We show that 
	$\lif{s_j \sigma} = \lif{s_j}\sigma'$ for $1 \leq j \leq m$.

	%Let $\phi \sigma^{t}$ denote $\phi[t/y]\sigma[t/x]$ for a fresh variable $y$.

	Note that anything in the term structure above a maximal $\Delta$-term is unaffected by both substitution and the lifting.

	Let $t_i$ be a maximal $\Delta$-term in $s_i \sigma$.

	We show that $ \lif{ t_i \sigma } = \lif{ t_i } \sigma'$, which proves the lemma.

	Let $t_i\sigma = t_j$. Then $\lif{ t_i\sigma} = \lif{t_j} = x_j$.

	We show that $x_j = \lif{t_i}\sigma'$.

	Suppose that $t_i = t_j$, i.e.\ $\sigma$ is trivial on $t_i$.
	Then $i=j$ as the $\Delta$-terms have a unique number.
	Hence $\lif{t_i}\sigma' = x_i \sigma' = x_i = x_j$.


	Otherwise $t_i \neq t_j$. Then $i\neq j$ and  $x_j \neq x_i$.\newline
	$\lif{t_i}\sigma' = x_i \sigma'$.
	By the definition of $\sigma'$, as $t_i\sigma = t_j$, $x_i\sigma' = x_j$.
	%
	%
	\begin{comment}
		Suppose no $\Delta$-colored symbol occurs in $s_j$ or $s_j\sigma$. Then $\lif{s_j\sigma} = s_j\sigma $. this equals $  s_j\sigma'$ as there, only the second case applies, where the lifting doesn't affect the term.

		Suppose a maximal $\Delta$-colored term $t_i$ occurs in $s_j\sigma$ but not in $s_j$ and suppose it's the only one in $s_j\sigma$.
		Then $\lif{ s_j\sigma } = \lif{ s_j\sigma^{t_i} }= s_j\sigma^{t_i} \abstraction{t_i/x_i} = \lif{s_j} \sigma^{t_i} \abstraction{t_i/x_i}$
	\end{comment}
	%Note that if a $\Delta$-term $t_i$ occurs in $s_j$, a $\Delta$-term with the same outermost symbol occurs in $s_j$ at the same position.
\end{proof}


\begin{comment}
	\begin{lemma}
	\label{lemma:lif_literal}
	If $l\sigma$ = $l'\sigma$, then $\lif{l}\sigma' = \lif{l'}\sigma'$ for $\sigma'$ defined as in lemma \ref{lemma:lif}
\end{lemma}
\begin{proof}
	$l\sigma$ = $l'\sigma$

	$\ra \lif{l\sigma} = \lif{l'\sigma}$

	by lemma \ref{lemma:lif}, 
	$\lif{l}\sigma' = \lif{l'}\sigma'$

\end{proof}



\begin{lemma} // currently unused

	$(\lif{C}(x_1, \ldots, x_n))\sigma =
	(\lif{C\sigma'}(x_1, \ldots, x_n))$ if $\sigma$ does not change any of $x_1, \ldots, x_n$ or any of $t_1, \ldots, t_n$.\qedhere

	\todo[inline]{it would work to fix substitutions of $x_i$ by substituting $t_i$ for that instead, as long as the result isn't another $t_i$, but this isn't actually relevant here.}

\end{lemma}

\begin{prop}
	$\Gamma = \lif{\Gamma}$.
\end{prop}
\begin{proof}
	By definition, $\Delta$-terms only appear in $\Delta$ and not in $\Gamma$. 
\end{proof}

\end{comment}

 
\begin{lemma}[corresponds to Lemma 4.8 in thesis and Lemma 11 in Huang]
  \label{lemma:lift_commute}
  Let $A$ and $B$ be first-order formulas and $s$ and $t$ be terms. Then it holds that:
  \begin{enumerate}
    \item $\lift{\Phi}{\lnot A}{x} \semiff{} \lnot \lift{\Phi}{A}{x}$
    \item $\lift{\Phi}{A \circ B}{x} \semiff{} ( \lift{\Phi}{A}{x} \circ \lift{\Phi}{B}{x} )$ for  $\circ \in \{\land, \lor\}$
    \item $\lift{\Phi}{s = t}{x} \semiff{} ( \lift{\Phi}{s}{x} = \lift{\Phi}{t}{x} )$
  \end{enumerate}
\end{lemma}

\begin{lemma}
	Let $s$ and $t$ be terms such that no $x_i$ occurs in them, $\Phi$ a set of formulas and $M$ a model.
	Then $M\entails \lft{\Phi}{x}{s} = \lft{\Phi}{x}{t}$ implies that $M\entails s=t$.
	\label{lemma:lift_equality}
\end{lemma}
\begin{proof}
	Suppose no $\Delta$-term occurs in $s$ or $t$. Then $\lft{\Phi}{x}{s} = s$ 
	and $\lft{\Phi}{x}{t} = t$.

	Otherwise let $t_i$ be a maximal $\Delta$-term in $s$. Suppose it occurs at position $p$. In $\lft{\Phi}{x}{s}$, it is replaced by $x_i$.
	But as $M \entails \lft{\Phi}{x}{s} = \lft{\Phi}{x}{t}$, two situations can arise:
	\begin{compactenum}
	\item $x_i$ occurs at $p$ in $\lft{\Phi}{x}{t}$.
		As $x_i$ does not occur in $t$, it is placed there by the lifting.
		But $x_i$ is only employed in order to replace $t_i$, so at position $p$ in $t$, we have $t_i$.
	\item A term $r$ occurs at $p$ in $\lft{\Phi}{x}{t}$ which does not influence the evaluation of $\lft{\Phi}{x}{t}$ in $M$. This can be the case if $r$ is contained in a subterm of $u$ and in $M$, the function symbol of $u$ is interpreted such that it does not depend on the argument that contains $r$.
		
		But as the maximal $\Delta$-term $t_i$ occurs in $s$ at $p$ and $M \entails \lft{\Phi}{x}{s} = \lft{\Phi}{x}{t}$, there is a function symbol $u'$ in $\lft{\Phi}{x}{s}$ corresponding to $u$ which also does not depend on this argument.

		Hence even though $s$ and $t$ are not syntactically equal, $M\entails s=t$ in this case. \qedhere
	\end{compactenum}

\end{proof}


We use basically the same definition of $\PI$ as Huang with minor adaptions for paramodulation (deviations are marked):
\begin{defi}[Propositional interpolant extraction.]
  Let $\pi$ be a resolution refutation of $\Gamma \cup \Delta$.
  \defiemph{${\PI(\pi)}$} is defined to be $\PI(\square)$, where $\square$ is the empty clause derived in $\pi$.

  For a clause $C$ in $\pi$, \defiemph{$\PI(C)$} is defined as follows:
  \label{def:PI}
  \begin{itemize}
    \item[Base case.]
      If $C \in \Gamma$, $\PI(C) = \bot$.
      If otherwise $C \in \Delta$, $\PI(C) = \top$.
    \item[Resolution.]
      \label{def:PI_resolution}
      %Suppose the clause $C$ is the result of a resolution step. Then it has the following form: 

      % \begin{prooftree}
      %   \AxiomCm{C_1: D \lor l}
      %   \AxiomCm{C_2: E \lor \lnot l'}
      %   \RightLabelm{\quad l\sigma = l'\sigma}
      %   \BinaryInfCm{C: (D\lor E)\sigma}
      % \end{prooftree}
      %\todo{write as prooftree? (not necessary, but nicer)}
      If the clause $C$ is the result of a resolution step of $C_1: D \lor l$ and $C_2: E \lor \lnot l'$ using a unifier $\sigma$ such that $l\sigma = l'\sigma$, then $\PI(C)$ is defined as follows:
      %$\PI(C)$ is defined according to this case distinction:
      \begin{enumerate}
        \item If $l$ is $\Gamma$-colored: $\PI(C) = [\PI(C_1) \lor \PI(C_2)]\sigma$
        \item If $l$ is $\Delta$-colored: $\PI(C) = [\PI(C_1) \land \PI(C_2)]\sigma$
        \item If $l$ is grey: $\PI(C) = [(l \land \PI(C_2)) \lor (\lnot l' \land \PI(C_1)) ]\sigma $
      \end{enumerate}

    \item[Factorisation.]
      If the clause $C$ is the result of a factorisation of $C_1: l \lor l' \lor D$ using a unifier $\sigma$ such that $l\sigma = l'\sigma$, then $\PI(C) = \PI(C_1)\sigma$.

    \item[Paramodulation.]
  \label{def:PI_paramod}
      If the clause $C$ is the result of a paramodulation of $C_1: s=t \lor C$ and $C_2: D\occur{r}$ using a unifier $\sigma$ such that $r\sigma = s\sigma$, then $\PI(C)$ is defined according to the following  case distinction:
      \begin{enumerate}
        \item If $r$ occurs in a maximal $\Delta$-term $h(r)$ in $D\occur{r}$ and $h(r)$ occurs more than once in $D\occur{r} \lor \PI(D\occur{r})$:
          \label{def:PI_paramod_1}
          \newline
          $\PI(C) = [ ( s=t \land \PI(C_2) ) \lor (s\neq t \land \PI(C_1)) ]\sigma \lor (s=t \land h\occur{s} \neq h\occur{t})\sigma$
        \item If $r$ occurs in a maximal $\Gamma$-term $h(r)$ in $D\occur{r}$ and $h(r)$ occurs more than once in $D\occur{r} \lor \PI(D\occur{r})$:
          \label{def:PI_paramod_2}
          \newline
          $\PI(C) = [ ( s=t \land \PI(C_2) ) \lor (s\neq t \land \PI(C_1)) ]\sigma \land (s\neq t \lor h\occur{s} = h\occur{t})\sigma$
        \item Otherwise:
          \label{def:PI_paramod_3}
          \newline
          $\PI(C) = [ ( s=t \land (\PI(C_2) \lor h[s] \neq h[t] ) \lor (s\neq t \land \PI(C_1)) ]\sigma$ \qedhere

      \end{enumerate}
  \end{itemize}
\end{defi}


Now we show the ``main'' lemma of Huang's proof without using a propositional deduction $P_P$.
The remaining part of his proof after this lemma does not use the restriction to propositional deductions and hence goes through.

\begin{lemma}[corresponds to Lemma 12 in Huang and Lemma 4.9 in the thesis]
	Let $\pi$ be a resolution refutation of $\Gamma \cup \Delta$.
	Then for $C \in \pi$,
	$ \Gamma \entails \lif{\PI(C) \lor C} $.
	\label{lemma:gamma_entails_interpolant}
\end{lemma}

\begin{proof}
	By induction on the resolution refutation of the strengthening: $\Gamma \entails \lif{\PI(C) \lor C_\Gamma}$, i.e.\ we only consider literals of $C$ which are contained in $\Lang(\Gamma)$.

	Base case:
	Either $C \in \Gamma$, then it does not contain $\Delta$-terms.
	Otherwise $C \in \Delta$ and $\PI(C) = \top$.

	Induction step:
	\begin{description}
		\item{Resolution.}
			\begin{prooftree}
				\AxiomCm{C_1: D \lor l}
				\AxiomCm{C_2: E \lor \lnot l'}
				\RightLabelm{\quad l\sigma = l'\sigma}
				\BinaryInfCm{C: (D\lor E)\sigma}
			\end{prooftree}

			By the induction hypothesis, we can assume that:

			$\Gamma \entails \lif{\PI(C_1) \lor (D\lor l)_\Gamma}$ and $\Gamma \entails \lif{\PI(C_2) \lor (E\lor \lnot l')_\Gamma}$

			which by Lemma \ref{lemma:lift_commute} implies that

			$\Gamma \stackrel{(*)}\entails \lif{\PI(C_1)} \lor \lif{D_\Gamma} \lor \lif {l_\Gamma}$ and $\Gamma \stackrel{(\circ)}\entails \lif{\PI(C_2)} \lor \lif{E_\Gamma} \lor \lnot \lif{l'_\Gamma}$

			Let $\sigma'$ be defined as in Lemma \ref{lemma:lif} with $t_1, \ldots, t_n$ all $\Delta$-terms in this context (we need that every maximal $\Delta$-term has a distinct index, so take all occurring in $C_1$, $C_2$, $\PI(C_1)$, $\PI(C_2)$, with and without $\sigma$ applied to them).

			Case distinction:

			\begin{enumerate}
				\item $l$ is $\Gamma$-colored.
					Then $\PI(C) = [\PI(C_1) \lor \PI(C_2)]\sigma$. 

					We show that $\Gamma \entails \lif{ (\PI(C_1) \lor \PI(C_2))\sigma \lor (D \lor E)_\Gamma\sigma}$,
					\newline 
					i.e.~$\Gamma \entails \lif{ \Big(\PI(C_1) \lor \PI(C_2) \lor D_\Gamma \lor E_\Gamma\Big)\sigma} $.


					Hence by Lemma \ref{lemma:lif},
					$\Gamma \entails \lif{(\PI(C_1) \lor \PI(C_2) \lor D_\Gamma \lor E_\Gamma)}\sigma' $.

					Since $\sigma = \mgu(l, l')$, $l\sigma$ and $l'\sigma$ are syntactically equal and so $\lif{l\sigma} = \lif{l'\sigma}$.
					
					As by Lemma \ref{lemma:lif} $\lif{l\sigma} = \lif{l}\sigma'$ and $\lif{l'\sigma} = \lif{l'}\sigma'$,
					we get $\lif{l}\sigma' = \lif{l'}\sigma'$.\label{aou5jklah}

					So by applying $\sigma'$ to $(*)$ and $(\circ)$ (note that $l_\Gamma = l$ and $l'_\Gamma = l'$ as they are $\Gamma$-colored), we can perform a resolution step on $\lif{l}\sigma'$ and get

					$\Gamma \entails \lif{\PI(C_1)}\sigma' \lor \lif{D_\Gamma} \sigma' \lor \lif{\PI(C_2)}\sigma' \lor \lif {E_\Gamma} \sigma'$.

					and consequently
				$\Gamma \entails \lif{ \PI(C_1) \lor \PI(C_2) \lor D_\Gamma \lor E_\Gamma}\sigma' $.

				So by Lemma \ref{lemma:lif},

				$\Gamma \entails \lif{ \Big(\PI(C_1) \lor \PI(C_2) \lor D_\Gamma \lor E_\Gamma \Big) \sigma } $.


				\item $l$ is $\Delta$-colored.
					Then $\PI(C) = (\PI(C_1) \land \PI(C_2))\sigma$. 

					We show that $\Gamma \entails \lif{(\PI(C_1) \land \PI(C_2))\sigma \lor (D_\Gamma \lor E_\Gamma)\sigma}$

					which by Lemma \ref{lemma:lift_commute} is equivalent to\newline
					$\Gamma \entails \Big(\lif{\PI(C_1)\sigma} \land \lif{\PI(C_2)\sigma}\Big) \lor \lif{D_\Gamma\sigma} \lor \lif{E_\Gamma\sigma}$

					and by Lemma \ref{lemma:lif} is equivalent to\newline
					$\Gamma \stackrel{\markC}\entails \Big(\lif{\PI(C_1)}\sigma' \land \lif{\PI(C_2)}\sigma'\Big) \lor \lif{D_\Gamma}\sigma' \lor \lif{E_\Gamma}\sigma'$

					As $l$ and $l'$ are $\Delta$-colored, we can simplify $(*)$ and $(\circ)$ as follows and apply $\sigma'$:

					$\Gamma \entails \lif{\PI(C_1)}\sigma' \lor \lif{D_\Gamma}\sigma' $ and $\Gamma \entails \lif{\PI(C_2)}\sigma' \lor \lif{E_\Gamma}\sigma'$

					These clearly imply \markC.

				\item $l$ is grey. Then $\PI(C) = [(l \land \PI(C_2) ) \lor (\lnot l' \land \PI(C_2))]\sigma$.

					We show that $\Gamma \entails \lif{ \Big((l \land \PI(C_2) ) \lor (\lnot l' \land \PI(C_2)) \lor D_\Gamma \lor E_\Gamma\Big)\sigma}$, which by Lemma~\ref{lemma:lift_commute} and Lemma~\ref{lemma:lif} is equivalent to

					$\Gamma \entails \Big(\lif{l}\sigma' \land \lif{\PI(C_2)}\sigma'\Big)\lor\Big(\lnot \lif{l'}\sigma' \land \lif{\PI(C_2)}\sigma'\Big)\lor\lif{D_\Gamma}\sigma' \lor \lif{E_\Gamma}\sigma'$.

					Suppose for a model $M$ of $\Gamma$ that  $M \notentails \lif{D_\Gamma}\sigma'$ and $M\notentails \lif{E_\Gamma}\sigma'$ as otherwise we would be done.
					But then by $(*)$ and $(\circ)$,
					$M \entails \lif{\PI(C_1)}\sigma' \lor \lif{l}\sigma'$ and
					$M \entails\nolinebreak \lif{\PI(C_2)}\sigma' \lor \lnot\lif{l'}\sigma'$.

					As observed in case \ref{aou5jklah}, $\lif{l}\sigma' = \lif{l'}\sigma'$. By a case distinction on the truth value of $\lif{l}\sigma'$, we obtain the result.



			\end{enumerate}

		\item{Factorisation.}
			\begin{prooftree}
				\AxiomCm{C_1: l \lor l' \lor D}
				\RightLabelm{\quad \sigma = \mgu(l, l')}
				\UnaryInfCm{C: (l \lor D)\sigma}
			\end{prooftree}
			Then $\PI(C) = \PI(C_1)\sigma$.

			The induction hypothesis gives that
			$\Gamma \entails \lif{\PI(C_1) \lor l \lor l' \lor D}$.
			Let $\sigma'$ be as in Lemma \ref{lemma:lif}.

			Then $\Gamma \entails \lif{\PI(C_1) \lor l \lor l' \lor D}\sigma'$ and by Lemma \ref{lemma:lif},
			$\Gamma \entails \lif{\PI(C_1)\sigma \lor l\sigma \lor l'\sigma \lor D\sigma}$.

			By Lemma \ref{lemma:lift_commute},
			$\Gamma \entails \lif{\PI(C_1)\sigma} \lor \lif{l\sigma} \lor \lif{l'\sigma} \lor \lif{D\sigma}$.

			As $\sigma = \mgu(l, l')$, $l\sigma$ and $l'\sigma$ are syntactically equal, hence $\lif{l\sigma} = \lif{l'\sigma}$.%\todo[noline,size=\tiny]{syntactically equal? does ``equal'' suffice?. see also $s\sigma=r\sigma$ below}

			But then we can apply a factorisation step and get
			$\Gamma \entails \lif{\PI(C_1)\sigma} \lor \lif{l\sigma} \lor \lif{D\sigma}$ and by Lemma \ref{lemma:lif} and Lemma \ref{lemma:lift_commute}, 
			$\Gamma \entails\nolinebreak \lif{\PI(C_1)\sigma \lor l\sigma \lor D\sigma}$.



		\item{Paramodulation.}
			\begin{prooftree}
				\AxiomCm{C_1: D \lor s=t}
				\AxiomCm{C_2: E\occurat{r}{p}}
				\RightLabel{$\quad \sigma = \mgu(s, r)$}
				\BinaryInfCm{C: (D \lor E\occurat{t}{p})\sigma}
			\end{prooftree}
			By the induction hypothesis, we have:

			$\Gamma \entails \lif{\PI(C_1) \lor (D\lor s=t)_\Gamma}$

			$\Gamma \entails \lif{\PI(C_2) \lor (E\occurat{r}{p})_\Gamma}$

			By Lemma~\ref{lemma:lif} and Lemma~\ref{lemma:lift_commute}, we get that:

			$\Gamma \stackrel{\markA}\entails \lif{\PI(C_1)} \lor \lif{D_\Gamma} \lor \lif{s} = \lif{t}$

			$\Gamma \stackrel{\markB}\entails \lif{\PI(C_2)} \lor \lif{(E\occurat{r}{p})_\Gamma}$

			We distinguish two cases:\nopagebreak
			\begin{enumerate}
				\item Suppose $s$ does not occur in a maximal $\Delta$-term $h\occur{s}$ in $E\occurat{s}{p}$ which occurs more than once in $\PI(E(s)) \lor E\occurat{s}{p}$.

					We show that $\Gamma \entails \lif{ \Big((s=t \land \PI(C_2)) \lor (s\neq t \land \PI(C_1))\Big)\sigma \lor \Big((D \lor E\occurat{t}{p})_\Gamma\Big)\sigma}$, which subsumes the cases 2 and 3 of the definition of $\PI$ for paramodulation.
					By Lemma~\ref{lemma:lift_commute}, we can pull the liftings inwards and by Lemma~\ref{lemma:lif}, we can commute substitution and lifting by employing $\sigma'$ to arrive at

				$\Gamma \entails
				\Big((\lif{s}\sigma')=(\lif{t}\sigma') \land \lif{\PI(C_2)}\sigma'\Big) \lor
				\Big((\lif{s}\sigma')\neq(\lif{t}\sigma') \land \lif{\PI(C_1)}\sigma'\Big) \lor
				\Big(\lif{D_\Gamma}\sigma' \lor \lif{(E\occurat{t}{p})_\Gamma}\sigma'\Big)$

				Let $M$ be a model of $\Gamma$. Let $M \notentails \lif{D_\Gamma}\sigma' \lor \lif{(E\occurat{t}{p})_\Gamma}\sigma'$ as otherwise we would be done. We show that depending on the truth value of  $(\lif{s}) = (\lif{t})$ in $M$, either the first or second conjunct of the above formula holds.

				Suppose that $M \entails (\lif{s}) \neq (\lif{t})$. 
				Then by~\markA, $M \entails \lif{\PI(C_1)}$ and hence $M \entails \lif{\PI(C_1)}\sigma'$.

				On the other hand, suppose that $M \entails (\lif{s}) = (\lif{t})$.
				The following two lemmas show that $M \notentails \lif{E\occurat{r}{p}}\sigma'$, so by~\markB, we get that $M\entails \lif{\PI(C_2)}\sigma'$.
\bigskip

				\begin{lemma}
					\label{aga5tg5ba}
					$M \entails (\lif{s}) = (\lif{t})$ and $M\notentails \lif{E\occurat{t}{p}}$ imply that $M\notentails \lif{E\occurat{s}{p}}$
					or, in case the term at position $p$ in $E$ is contained in a maximal $\Delta$-colored term $g\occur{t}$, $M\entails s=t \;\land\;(\lif{g\occur{s}}) \neq (\lif{g\occur{t}})$.
				\end{lemma}
				\begin{proof} 
					Suppose that the term at $p$ in $E$ is not contained in a $\Delta$-colored term. Then
					$\lif{E\occurat{t}{p}}$ and $\lif{E\occurat{s}{p}}$ only differ at position $p$, where at the first, there is $\lif{t}$, and at the latter, there is $\lif{s}$. But in $M$, they are interpreted the same way, hence $M\entails \lif{E\occurat{t}{p}} \semiff \lif{E\occurat{s}{p}}$, which implies the result.

					Otherwise as $g\occur{t}$ and $g\occur{s}$ in $E\occurat{t}{p}$ and $E\occurat{s}{p}$ respectively  are distinct $\Delta$-terms, they are replaced by distinct variables by the lifting.
					By Lemma \ref{lemma:lift_equality}, $M\entails s=t$, so $M \entails\nolinebreak s=\nolinebreak t \;\land\;(\lif{g\occur{s}})  \neq (\lif{g\occur{t}})$.


				\end{proof} 
				\bigskip

				\begin{lemma}
					$\sigma=\mgu(s, r)$ and $M\notentails \lif{E\occurat{s}{p}}\sigma'$ imply that $M\notentails \lif{E\occurat{r}{p}}\sigma'$.
				\end{lemma}\nopagebreak 
				\begin{proof} 
					By Lemma \ref{lemma:lif}, $M\notentails \lif{(E\occurat{s}{p})\sigma}$. 

					%$\lif{(E\occurat{s}{p})\sigma}$ differs from $\lif{(E\occurat{r}{p})\sigma}$ exactly at $p$. 

					Due to $\sigma=\mgu(s, r)$, both $s\sigma$ and $r\sigma$ are syntactically equal.
%We also have that
% 					$M \entails (E\occurat{s}{p})\sigma \semiff{} (E\occurat{r}{p})\sigma$.
					Suppose they are both not $\Delta$-colored.
Then the lifting does not affect them and 
$\lif{(E\occurat{s}{p})\sigma} = \lif{(E\occurat{r}{p})\sigma}$.
Otherwise the lifting will replace them with the same variable and we as well get that
$\lif{(E\occurat{s}{p})\sigma} = \lif{(E\occurat{r}{p})\sigma}$.

By Lemma $\ref{lemma:lif}$, 
$\lif{(E\occurat{s}{p})}\sigma' = \lif{(E\occurat{r}{p})}\sigma'$, which implies the result.
				\end{proof} 




			\item Otherwise $s$ occurs in a maximal $\Delta$-term $h\occurat{s}{q}$ in $E\occurat{s}{p}$ which occurs more than once in $\PI(E(s)) \lor E\occurat{s}{p}$.


				\newenvironment{lemmaCustomNo}[1]
				{\renewcommand{\thelemma}{\ref{#1}$'$}%
					\addtocounter{lemma}{-1}%
				\begin{lemma}}
				{\end{lemma}}


				Then we have to replace Lemma \ref{aga5tg5ba} by:
				\begin{lemmaCustomNo}{aga5tg5ba}
					$M \entails (\lif{s}) = (\lif{t})$ and $M\notentails \lif{E\occurat{t}{p}}\sigma'$ imply that $M\notentails \lif{E\occurat{s}{p}}\sigma'$ or that $\lif{h\occurat{s}{q}} \neq \lif{h\occurat{t}{q}}$.
				\end{lemmaCustomNo}
				\begin{proof}
					If $\lif{E\occurat{t}{p}}$ and $\lif{E\occurat{s}{p}}$ differ only at position $p$, then the proof of Lemma \ref{aga5tg5ba} applies.
					
					Otherwise position $p$ is in a maximal $\Delta$-term $h\occurat{t}{q}$, such that $h\occurat{t}{q}$ and $h\occurat{s}{q}$ are replaced with distinct variables.
					But then clearly $\lif{h\occurat{s}{q}} \neq \lif{h\occurat{t}{q}}$.
				\end{proof}
 Hence the following holds:

				$\Gamma \entails
				\Big((\lif{s}\sigma')=(\lif{t}\sigma') \land \lif{\PI(C_2)}\sigma'\Big) \lor
				\Big((\lif{s}\sigma')\neq(\lif{t}\sigma') \land \lif{\PI(C_1)}\sigma'\Big) \lor
				\Big((\lif{s}\sigma')=(\lif{t}\sigma') \land (\lif{h\occurat{s}{q}}) \neq (\lif{h\occurat{t}{q}} )\Big) \lor
				\Big(\lif{D_\Gamma}\sigma' \lor \lif{(E\occurat{t}{p})_\Gamma}\sigma'\Big)$
				\qedhere
		\end{enumerate}


		\begin{comment}



			easy case:
			$\PI(C) = [ ( s=t \land \PI(C_2) ) \lor (s\neq t \land \PI(C_1)) ]\sigma$

			to show:
			$\Gamma \entails \lif{ [ (( s=t \land \PI(C_2) ) \lor (s\neq t \land \PI(C_1))) \lor (D \lor E[t]) ]\sigma} $

			proof idea: either $s=t$, then also $\PI(C_2)$, or else $s\neq t$, but then also $\PI(C_1)$

			by lemma \ref{lemma:lif} for $\sigma'$ as in lemma, 
			$\Gamma \entails \lif{ (( s=t \land \PI(C_2) ) \lor (s\neq t \land \PI(C_1))) \lor (D \lor E[t]) }\sigma' $

			by lemma 11 (huang)
			$\Gamma \entails [((\lif{s}=\lif{t} \land \lif{\PI(C_2)} ) \lor (\lif{s\neq t} \land \lif{\PI(C_1)})) \lor (\lif{D} \lor \lif{E[t]}) ]\sigma' $

			reformulate:
			$\Gamma \entails ((\lif{s}\sigma'=\lif{t}\sigma' \land \lif{\PI(C_2)}\sigma' ) \lor (\lif{s}\sigma'\neq \lif{t}\sigma' \land \lif{\PI(C_1)}\sigma')) \lor (\lif{D}\sigma' \lor \lif{E[t]}\sigma') $

			By the rule: $s\sigma = r\sigma$, hence also $\lif{s\sigma} = \lif{r\sigma}$ and $\lif{s}\sigma' = \lif{r}\sigma'$ REALLY TRUE? -- think so\dots

			Suppose $M \entails \Gamma$ and $M \not \entails (\lif{D}\sigma' \lor \lif{E[t]}\sigma') $.

			Suppose $M \entails \lif{s}\sigma' = \lif{t}\sigma'$.

			By induction hypothesis (and lemma 11 (huang) and adding the substitution $\sigma'$), 
			$\Gamma \entails \lif{\PI(C_2)}\sigma' \lor \lif{(E[r])}\sigma'$.

			However by assumption $\Gamma \not \entails \lif{E[t]}\sigma'$.

			Hence $\Gamma \not \entails \lif{E[s]}\sigma'$, and
			$\Gamma \not \entails \lif{E[r]}\sigma'$. Therefore $\Gamma \entails \lif{\PI(C_2)}\sigma'$.


			Suppose on the other hand $M \entails \lif{s}\sigma' \neq \lif{t}\sigma'$.

			By the induction hypothesis, 
			$M \entails \lif{\PI(C_1)}\sigma' \lor (\lif{D}\sigma'\lor (\lif{s}=\lif{t})\sigma')$,
			hence then $M \entails \lif{\PI(C_1)}\sigma'$.

			Consequently, 
			$M \entails (\lif{s}\sigma' \neq \lif{t}\sigma' \land \lif{\PI(C_1)}\sigma') \lor (\lif{s}\sigma' = \lif{t}\sigma' \land \lif{\PI(C_2)}\sigma')$.

			By lemma 11 (huang), 
			$M \entails \lif{s \neq {t} \land {\PI(C_1)} \lor ({s} = {t} \land \PI(C_2))}\sigma'$.

			Hence 
			$\Gamma \entails \lif{(s \neq {t} \land {\PI(C_1)} \lor ({s} = {t} \land \PI(C_2))}\sigma' \lor (\lif{D} \lor \lif{E[t]})\sigma') $.

			is this really what i need to show?
		\end{comment}
\end{description}
\end{proof}



Then the following from the thesis (also same in Huang) seem to go through:

Lemma 4.10: swap $\Gamma$ and $\Delta$ and obtain logical negation as interpolant 

Corollary 4.11: $\Delta \entails \lifgamma{ \lnot \PI(C) \lor C}$ 

Lemma 4.12: not important if lifting delta or gamma terms first 

Thm 4.13: ordering 

\end{document}
