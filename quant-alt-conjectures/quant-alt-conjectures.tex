\documentclass[,%fontsize=11pt,%
	%landscape,
	%DIV8, % mehr text pro seite als defaultyyp
	%DIV10,
	%DIV=calc,%
	draft=false,% final|draft % draft ist platzsparender (kein code, bilder..)
	%titlepage,
	numbers=noendperiod
	12pt,
	a4paper,
	oneside,% apparently, this should stay below some other parameter to have an effect
	openany,
	%]{scrartcl}
]{memoir}



\usepackage[utf8]{inputenc}
\usepackage[T1]{fontenc}
\usepackage[english]{babel}


%\usepackage[bitstream-charter]{mathdesign}


%\renewcommand{\familydefault}{ \sfdefault }
%\renewcommand{\rmdefault}{ppl}


%\usepackage[urw-garamond]{mathdesign}

\usepackage{lscape}
\usepackage{stackengine}
\usepackage{enumerate}
\usepackage{paralist}
\usepackage{tikz}
\usetikzlibrary{shapes,arrows,backgrounds,graphs,%
	matrix,patterns,arrows,decorations.pathmorphing,decorations.pathreplacing,%
	positioning,fit,calc,decorations.text,shadows%
}


\usepackage{comment} 

\usepackage{etoolbox} % fixes fatal error caused by combining bm, stackengine, hyperref (seriously?)
% http://tex.stackexchange.com/questions/22995/package-incompatibilites-etoolbox-hyperref-and-bm-standalone

\usepackage{etex} % else error on too many packages

% includes
\usepackage{algorithm}
%\usepackage{algorithmic} % conflicts with algpseudocode
\usepackage{algpseudocode}
%\newcommand*\Let[2]{\State #1 $\gets$ #2}
\algrenewcommand\alglinenumber[1]{
{\scriptsize #1}}
\algrenewcommand{\algorithmicrequire}{\textbf{Input:}}
\algrenewcommand{\algorithmicensure}{\textbf{Output:}}


%\usepackage[multiple]{footmisc} % footnotes at the same character separated by ','

\usepackage{multicol}

\usepackage{afterpage}

\usepackage{changepage} % for adjustwidth
\usepackage{caption} % for \ContinuedFloat

\usepackage{tikz}
\usetikzlibrary{shapes,arrows,backgrounds,graphs,%
matrix,patterns,arrows,decorations.pathmorphing,decorations.pathreplacing,%
positioning,fit,calc,decorations.text,shadows%
}

\usepackage{bussproofs}
\EnableBpAbbreviations


\usepackage{amsmath}
\usepackage{amsthm}
\usepackage{amssymb} % the reals
\usepackage{mathtools} % smashoperator

\usepackage{bm} % bm, bold math symbols

\usepackage{thm-restate} % restatable env

% needs extra work and fails on some label here
%\usepackage{cleveref} % cref, apparently better than autoref of hyperref 

\usepackage{nicefrac} % nicefrac

\usepackage{mathrsfs} % mathscr

\usepackage{pst-node} % http://tex.stackexchange.com/questions/35717/how-to-draw-arrows-between-parts-of-an-equation-to-show-the-math-distributive-pr

\usepackage{stackengine}

\usepackage{thmtools} % advanced thm commands (declaretheorem)


\usepackage{nameref} % reference name of thm instead of counter

\usepackage{todonotes}

% conflict with beamer
%\usepackage{paralist} % compactenum

\usepackage{hyperref}
%\hypersetup{hidelinks}  % don't give options to usepackage, it doesn't work with beamer
%\hypersetup{colorlinks=false}  % don't give options to usepackage, it doesn't work with beamer


% \usepackage{enumitem} % labels for enumerate % breaks beamer and memoir itemize


\usepackage{url} 


\usepackage[format=hang,justification=raggedright]{caption}% or e.g. [format=hang]

\usepackage{cancel} % \cancel

\usepackage{lineno}


% commands

% logic etcs
%\newcommand{\ex}[2]{\bigskip\section*{Exercise #1: \begin{minipage}[t]{.80\linewidth} \small \textnormal{\it #2} \end{minipage} } }

\newcommand{\ex}[2]{\bigskip \noindent\textbf{Exercise #1.} \textit{#2} \smallskip}

\newcommand{\comm}[1]{{\color{gray} // #1 }}


\newcommand{\true}[0]{\textbf{1}}
\newcommand{\false}[0]{\textbf{0}}
\newcommand{\tr}{\true}
\newcommand{\fa}{\false}

\newcommand{\ra}{\rightarrow}
\newcommand{\Ra}{\Rightarrow}
\newcommand{\la}{\leftarrow}
\newcommand{\La}{\Leftarrow}

\newcommand{\lra}{\leftrightarrow}
\newcommand{\Lra}{\Leftrightarrow}

\newcommand{\NKZ}{\textbf{NK2}}

%\DeclareMathOperator{\syneq}{\equiv} %spacing seems wrong, therefore defined as newcommand below
\DeclareMathOperator{\limpl}{\supset}
\DeclareMathOperator{\liff}{\lra}
\DeclareMathOperator{\semiff}{\Lra}
\newcommand{\syneq}{\equiv}
\newcommand{\union}{\cup}
\newcommand{\bigunion}{\bigcup}
\newcommand{\intersection}{\cap}
\newcommand{\bigintersection}{\bigcap}
\newcommand{\intersect}{\intersection}
\newcommand{\bigintersect}{\bigintersection}

\newcommand{\powerset}{\mathcal{P}}

\newcommand{\entails}{\vDash}
\newcommand{\notentails}{\nvDash}
\newcommand{\proves}{\vdash}

\newcommand{\vm}{\ensuremath{\vv_\mathcal{M}}}
\newcommand{\Dia}{\ensuremath{\lozenge}}

\newcommand{\spaced}[1]{\ \ #1 \ \ }
\newcommand{\spa}[1]{\spaced{#1}}
\newcommand{\spas}[1]{\;{#1}\;}
\newcommand{\spam}[1]{\;\,{#1}\;\,}

% functions
\DeclareMathOperator{\sk}{sk}
\DeclareMathOperator{\mgu}{mgu}
\DeclareMathOperator{\dom}{dom}
\DeclareMathOperator{\ran}{ran}

\DeclareMathOperator{\id}{id}
\DeclareMathOperator{\Fun}{FS}
\DeclareMathOperator{\Pred}{PS}
\DeclareMathOperator{\Lang}{L}
\DeclareMathOperator{\ar}{ar}
\DeclareMathOperator{\PI}{PI}
\DeclareMathOperator{\LI}{LI}
\DeclareMathOperator{\Congr}{Congr}
\DeclareMathOperator{\Refl}{Refl}
\DeclareMathOperator{\aiu}{au}
\DeclareMathOperator{\expa}{unfold-lift}

\newcommand{\PIinc}{\LI}
\newcommand{\PIincde}{\LIde}

\newcommand{\LIde}{\ensuremath{\LI^\Delta}}

\newcommand{\LIcl}{\ensuremath{\LI_{\operatorname{cl}}}}
\newcommand{\LIclde}{\ensuremath{\LI_{\operatorname{cl}}^\Delta}}

\newcommand{\cll}{\ensuremath{_{\operatorname{LIcl}}}}
\newcommand{\cllde}{\ensuremath{_{\operatorname{LIcl}^\Delta}}}

%\newcommand{\lifi}{\mathop{\ell\text{}i}}
\newcommand{\lifiboth}[1]{\ensuremath{\LIcl(#1)}}
\newcommand{\lifidelta}[1]{\ensuremath{\LIclde(#1)}}


%\DeclareMathOperator{\abstraction}{abstraction}

%\newcommand{\sk}{\ensuremath{\mathrm{sk}}}
%\newcommand{\mgu}{\ensuremath{\mathrm{mgu}}}
%\newcommand{\Fun}{\ensuremath{\mathrm{FS}}}
%\newcommand{\Pred}{\ensuremath{\mathrm{PS}}}
%\newcommand{\PI}{\ensuremath{\mathrm{PI}}}
%\newcommand{\Lang}{\ensuremath{\mathrm{L}}}
%\newcommand{\ar}{\ensuremath{\mathrm{ar}}}

\DeclareMathOperator{\AI}{AI}
\newcommand{\AIde}{\ensuremath{\AI^\Delta}}
\newcommand{\AImatrix}{\ensuremath{\AI_\mathrm{mat}}}
\newcommand{\AImatrixde}{\ensuremath{\AI_\mathrm{mat}^\Delta}}
\newcommand{\AImat}{\AImatrix}
\newcommand{\AImatde}{\AImatrixde}
\newcommand{\AIclause}{\ensuremath{\AI_\mathrm{cl}}}
\newcommand{\AIcl}{\AIclause}
\newcommand{\AIclde}{\AIclausede}
\newcommand{\AIclausede}{\ensuremath{\AIclause^\Delta}}
\newcommand{\fromclause}{\ensuremath{_{\operatorname{AIcl}}}}
\newcommand{\fromclausede}{\ensuremath{_{\operatorname{AIcl}^\Delta}}}
\newcommand{\cl}{\fromclause}
\newcommand{\clde}{\fromclausede}

\newcommand{\Q}{\ensuremath{Q}}

\newcommand{\AIcol}{\ensuremath{\AI_\mathrm{col}}}
\newcommand{\AIcolde}{\AIcol^\Delta}

\newcommand{\AIany}{\ensuremath{\AI_\mathrm{*}}}
\newcommand{\AIanyde}{\AIany^\Delta}

\newcommand{\AIclpre}{\AIclause^\bullet}
\newcommand{\AImatpre}{\AImatrix^\bullet}

\newcommand{\PS}{\Pred}
\newcommand{\FS}{\Fun}

\DeclareMathOperator{\LangSym}{\mathcal{L}}

%\newcommand{\mguarr}{\sim_\ra}
\newcommand{\mguarr}{\mapsto_{\mgu}}


%\newcommand{\Trans}{\ensuremath{\mathrm{T}}}
%\newcommand{\Trans}{\ensuremath{\mathrm{T}}}
\DeclareMathOperator{\Trans}{T}
\DeclareMathOperator{\TransInv}{T^{-1}}

\DeclareMathOperator{\FAX}{F_{Ax}}
\DeclareMathOperator{\EAX}{E_{Ax}}
%\newcommand{\FAX}{\ensuremath{\mathrm{F_{Ax}}}}
%\newcommand{\EAX}{\ensuremath{\mathrm{E_{Ax}}}}

%\newcommand{\TransAll}{\ensuremath{\Trans_{\mathrm{Ax}}}}
\DeclareMathOperator{\TransAll}{\Trans_{Ax}}
%\newcommand{\FAX}{\ensuremath{\mathrm{F_{Ax}}}}

\DeclareMathOperator{\defeq}{\stackrel{\mathrm{def}}{=}}

\newcommand{\subst}[1]{[#1]}
\newcommand{\abstractionOp}[1]{\{#1\}}

\newcommand{\subformdefinitional}[1]{\ensuremath{D_{\Sigma(#1)}}}


%\newcommand{\lift}[3]{\operatorname{Lift}_{#1}(#2; #3)}
%\newcommand{\lift}[3]{\operatorname{Lift}_{#1,#3}(#2)}
%\newcommand{\lift}[3]{\operatorname{Lift}_{#1,#3}[#2]}
%\newcommand{\lift}[3]{\overline{#2}_{#1,#3}}
\newcommand{\lifsym}{\ell}
%\newcommand{\lift}[3]{\lifsym_{#1,#3}[#2]}
\newcommand{\lift}[3]{\lifsym_{#1}^{#3}[#2]}
\newcommand{\liftnovar}[2]{\lifsym_{#1}[#2]}

%\newcommand{\lft}[3]{\lifsym_{#1,#2}[#3]}
\newcommand{\lft}[3]{\lift{#1}{#3}{#2}}
\newcommand{\lifboth}[1]{\lifsym[#1]}

%\newcommand{\lifi}{\mathop{\ell\text{}i}}
%\newcommand{\lifiboth}[1]{\lifi[#1]}
%\newcommand{\lifidelta}[1]{\lifi_\Delta^x[#1]}
%\newcommand{\lifideltanovar}[1]{\lifi_\Delta[#1]}

\newcommand{\lifdelta}[1]{\lift{\Delta}{#1}{x}}
\newcommand{\lifdeltanovar}[1]{\liftnovar{\Delta}{#1}}
\newcommand{\lifgamma}[1]{\lift{\Gamma}{#1}{y}}
\newcommand{\lifgammanovar}[1]{\liftnovar{\Gamma}{#1}}
\newcommand{\lifphinovar}[1]{\liftnovar{\Phi}{#1}}
\newcommand{\lifphi}[1]{\lift{\Phi}{#1}{z}}

\DeclareMathOperator{\arr}{\mathcal{A}}
%\DeclareMathOperator{\arrFinal}{{\mathcal{A}^{\bm*}}}
\DeclareMathOperator{\arrFinal}{{\mathcal{\bm{\hat}A}}}
\DeclareMathOperator{\warr}{\marr}
\DeclareMathOperator{\marr}{\mathcal{M}}

\DeclareMathOperator{\apath}{\leadsto}
\DeclareMathOperator{\mpath}{\leadsto_=}
\DeclareMathOperator{\notapath}{\not\leadsto}
\DeclareMathOperator{\notmpath}{\not\leadsto_=}

\newcommand{\ltArrC}{<_{\arrFinal(C)}}
\newcommand{\ltAC}{<_{\arr(C)}}
\newcommand{\ltArrCOne}{<_{\arrFinal(C_1)}}
\newcommand{\ltArrCTwo}{<_{\arrFinal(C_2)}}
%\newcommand{\ltArrC}{<_{\scalebox{0.6}{$\arrFinal(C)$}}}
\newcommand{\ltArr}{<_{\scalebox{0.6}{$\arrFinal$}}}

\newcommand{\bhat}{\bm\hat}
\newcommand{\bbar}{\bm\bar}
\newcommand{\bdot}{\bm\dot}

%\usepackage{yfonts}
\usepackage{upgreek}
\DeclareMathAlphabet{\mathpzc}{OT1}{pzc}{m}{it}
%\DeclareMathOperator{\pos}{\mathscr{P}}
%\DeclareMathOperator{\pos}{\mathpzc{p}}
%\DeclareMathOperator{\pos}{{\rho}}
\DeclareMathOperator{\pos}{{\operatorname P}}
%\DeclareMathOperator{\pos}{P}
\DeclareMathOperator{\poslit}{\pos_\mathrm{lit}}
\DeclareMathOperator{\posterm}{\pos_\mathrm{term}}
%\newcommand{\poslit}[1]{\ensuremath{p_\text{lit}(#1)}}
%\newcommand{\posterm}[1]{\ensuremath{p_\text{term}(#1)}}
\newcommand{\at}[1]{|_{#1}}

\newcommand{\UICm}[1]{\UnaryInfCm{#1}}
\newcommand{\UnaryInfCm}[1]{\UnaryInfC{$#1$}}
\newcommand{\BICm}[1]{\BinaryInfCm{#1}}
\newcommand{\BinaryInfCm}[1]{\BinaryInfC{$#1$}}
\newcommand{\RightLabelm}[1]{\RightLabel{$#1$}}
\newcommand{\LeftLabelm}[1]{\LeftLabel{$#1$}}
\newcommand{\AXCm}[1]{\AxiomCm{#1}}
\newcommand{\AxiomCm}[1]{\AxiomC{$#1$}}
\newcommand{\mt}[1]{\textnormal{#1}}

\newcommand{\UnaryInfm}[1]{\UnaryInf$#1$}
\newcommand{\BinaryInfm}[1]{\BinaryInf$#1$}
\newcommand{\Axiomm}[1]{\Axiom$#1$}



% math
\newcommand{\calI}{\ensuremath{\mathcal{I}}}

\newcommand{\tupleShort}[2]{\ensuremath{(#1_1,\dotsc,#1_{#2})}}
\newcommand{\tuple}[2]{\ensuremath{(#1_1,\:#1_2\:,\dotsc,\:#1_{#2})}}
\newcommand{\setelements}[2]{\ensuremath{\{#1_1,\:#1_2\:,\dotsc,\:#1_{#2}\}}}
\newcommand{\pathelements}[2]{\ensuremath{ (#1_1,\:#1_2\:,\dotsc,\:#1_{#2}) }}

\newcommand{\elems}[1]{\ensuremath{#1_1,\dotsc, #1_{n}) }}

\newcommand{\defiemph}[1]{\emph{#1}}

\newcommand{\setofbases}{\ensuremath{\mathcal{B}}}
\newcommand{\setofcircuits}{\ensuremath{\mathcal{C}}}

\newcommand{\reals}{\ensuremath{\mathbb{R}}}
\newcommand{\integers}{\ensuremath{\mathbb{Z}}} 
\newcommand{\naturalnumbers}{\ensuremath{\mathbb{N}}}

% general
\newcommand{\zit}[3]{#1\ \cite{#2}, #3}
\newcommand{\zitx}[2]{#1\ \cite{#2}}
\newcommand{\footzit}[3]{\footnote{\zit{#1}{#2}{#3}}}
\newcommand{\footzitx}[2]{\footnote{\zitx{#1}{#2}}}

\newcommand{\ite}{\begin{itemize}}
\newcommand{\ete}{\end{itemize}}

\newcommand{\bfr}{\begin{frame}}
\newcommand{\efr}{\end{frame}}

\newcommand{\ilc}[1]{\texttt{#1}}


% misc

% multiframe
\usepackage{xifthen}% provides \isempty test
% new counter to now which frame it is within the sequence
\newcounter{multiframecounter}
% initialize buffer for previously used frame title
\gdef\lastframetitle{\textit{undefined}}
% new environment for a multi-frame
\newenvironment{multiframe}[1][]{%
\ifthenelse{\isempty{#1}}{%
% if no frame title was set via optional parameter,
% only increase sequence counter by 1
\addtocounter{multiframecounter}{1}%
}{%
% new frame title has been provided, thus
% reset sequence counter to 1 and buffer frame title for later use
\setcounter{multiframecounter}{1}%
\gdef\lastframetitle{#1}%
}%
% start conventional frame environment and
% automatically set frame title followed by sequence counter
\begin{frame}%
\frametitle{\lastframetitle~{\normalfont \Roman{multiframecounter}}}%
}{%
\end{frame}%
}




% http://texfragen.de/hurenkinder_und_schusterjungen
\usepackage[all]{nowidow}



% force no overlong lines:
%\tolerance=1 % tolerance for how badly spaced lines are allowed, less means "better" lines
\tolerance=500 %  need more tolerance for equations
%\emergencystretch=\maxdimen
%\emergencystretch=200pt
%\setlength{\emergencystretch}{3em}
%\hyphenpenalty=10000 % forces no hyphenation
%\hbadness=10000


% http://tex.stackexchange.com/questions/35717/how-to-draw-arrows-between-parts-of-an-equation-to-show-the-math-distributive-pr
\tikzset{square arrow/.style={to path={ -- ++(.0,-.15)  -| (\tikztotarget)}}}
\tikzset{square arrow2/.style={to path={ -- ++(.0,-.25)  -| (\tikztotarget)}}}
%\tikzset{square arrow/.style={to path={ -- ++(00,-.01) -- ++(0.5,-0.1) -- ++(0.5,-0.1) -| (\tikztotarget)},color=red}}


% have arrows from a to b and from c to d here
% just use: texttext\arrowA texttest \arrowB texttext
\newcommand{\arrowA}{\tikz[overlay,remember picture] \node (a) {};}
\newcommand{\arrowB}{\tikz[overlay,remember picture] \node (b) {};}
\newcommand{\drawAB}{
	\tikz[overlay,remember picture]
	{\draw[->,bend left=5,color=red] (a.south) to (b.south);}
	%{\draw[->,square arrow,color=red] (a.south) to (b.south);}
}
\newcommand{\arrowAP}{\tikz[overlay,remember picture] \node (ap) {};}
\newcommand{\arrowBP}{\tikz[overlay,remember picture] \node (bp) {};}
\newcommand{\drawABP}{
	\tikz[overlay,remember picture]
	{\draw[->,bend right=5,color=red] (ap.south) to (bp.south);}
	%{\draw[->,square arrow,color=red] (a.south) to (b.south);}
}

\newcommand{\arrowAB}{\tikz[overlay,remember picture] \node (ab) {};}
\newcommand{\arrowBA}{\tikz[overlay,remember picture] \node (ba) {};}
\newcommand{\drawAABB}{
	\tikz[overlay,remember picture]
	%{\draw[->,bend left=80] (a.north) to (b.north);}
	{\draw[->,square arrow,color=brown] (ab.south) to (ba.south);
	\draw[->,square arrow,color=brown] (ba.south) to (ab.south);}
}


\newcommand{\arrowCD}{\tikz[overlay,remember picture] \node (cd) {};}
\newcommand{\arrowDC}{\tikz[overlay,remember picture] \node (dc) {};}
\newcommand{\drawCCDD}{
	\tikz[overlay,remember picture]
	%{\draw[->,bend left=80] (a.north) to (b.north);}
	{\draw[<->,dashed,square arrow,color=green] (cd.south) to (dc.south); }
}



\newcommand{\arrowC}{\tikz[overlay,remember picture] \node (c) {};}
\newcommand{\arrowD}{\tikz[overlay,remember picture] \node (d) {};}
\newcommand{\drawCD}{
	\tikz[overlay,remember picture]
	{\draw[->,square arrow,color=blue] (c.south) to (d.south);}
}

\newcommand{\arrowE}{\tikz[overlay,remember picture] \node (e) {};}
\newcommand{\arrowF}{\tikz[overlay,remember picture] \node (f) {};}
\newcommand{\drawEF}{
	\tikz[overlay,remember picture]
	{\draw[->,square arrow2,color=orange] (e.south) to (f.south);}
}


\newcommand{\arrAP}{\arrowAP}
\newcommand{\arrBP}{\arrowBP}
\newcommand{\arrA}{\arrowA}
\newcommand{\arrB}{\arrowB}
\newcommand{\arrC}{\arrowC}
\newcommand{\arrD}{\arrowD}
\newcommand{\arrE}{\arrowE}
\newcommand{\arrF}{\arrowF}


\DeclareMathOperator{\simgeq}{\scalebox{0.92}{$\gtrsim$}}

\newcommand{\refsub}[2]{\hyperref[#2]{\ref*{#1}.\ref*{#2}}}

%\newcommand{\sigmarange}[2]{\sigma_{#1}^{#2} }
\newcommand{\sigmarange}[2]{\sigma_{(#1,#2)} }
\newcommand{\sigmaz}[1]{\sigmarange{0}{#1} }
\newcommand{\sigmazi}[0]{\sigmaz{i} }

\DeclareMathOperator{\lit}{lit}

%\def\fCenter{\ \proves\ }
\def\fCenter{\proves}

\newcommand{\prflbl}[2]{\RightLabel{\footnotesize $#1, #2$} }
%\newcommand{\prflblid}[1]{\RightLabel{$#1, \id$} }
\newcommand{\prflblid}[1]{\RightLabel{\footnotesize $#1$} }

\DeclareMathOperator{\resruleres}{res}
\DeclareMathOperator{\resrulefac}{fac}
\DeclareMathOperator{\resrulepar}{par}
\newcommand{\lkrule}[2]{\ensuremath{\operatorname{#1}:#2}} % operatorname fixes spacing issues for =

\newcommand{\parti}[4]{\ensuremath{ \langle (#1; #2), (#3; #4)\rangle  }}

\newcommand{\partisym}{\ensuremath{\chi}}

\newcommand{\occur}[1]{\ensuremath{[#1]}}
\newcommand{\occ}[1]{\occur{#1}}

\newcommand{\occurat}[2]{\ensuremath{{\occur{#1}_{#2}}}}
\newcommand{\occat}[2]{\occurat{#1}{#2}}
\newcommand{\occatp}[1]{\occurat{#1}{p}}
\newcommand{\occatq}[1]{\occurat{#1}{q}}

\newcommand{\colterm}[1]{\zeta_{#1}}



% fix restateable spacing 
%http://tex.stackexchange.com/questions/111639/extra-spacing-around-restatable-theorems

\makeatletter

\def\thmt@rst@storecounters#1{%
%THIS IS THE LINE I ADDED:
\vspace{-1.9ex}%
  \bgroup
        % ugly hack: save chapter,..subsection numbers
        % for equation numbers.
  %\refstepcounter{thmt@dummyctr}% why is this here?
  %% temporarily disabled, broke autorefname.
  \def\@currentlabel{}%
  \@for\thmt@ctr:=\thmt@innercounters\do{%
    \thmt@sanitizethe{\thmt@ctr}%
    \protected@edef\@currentlabel{%
      \@currentlabel
      \protect\def\@xa\protect\csname the\thmt@ctr\endcsname{%
        \csname the\thmt@ctr\endcsname}%
      \ifcsname theH\thmt@ctr\endcsname
        \protect\def\@xa\protect\csname theH\thmt@ctr\endcsname{%
          (restate \protect\theHthmt@dummyctr)\csname theH\thmt@ctr\endcsname}%
      \fi
      \protect\setcounter{\thmt@ctr}{\number\csname c@\thmt@ctr\endcsname}%
    }%
  }%
  \label{thmt@@#1@data}%
  \egroup
}%

\makeatother




\newcommand{\mymark}[1]{\ensuremath{(#1)}}
\newcommand{\markA}{\mymark \circ}
\newcommand{\markB}{\mymark *}
\newcommand{\markC}{\mymark \divideontimes}

\newcommand{\wrong}[1]{{\color{red}WRONG: #1}}
\newcommand{\NB}[1]{{\color{blue}NB: #1}}
\newcommand{\hl}[1]{{\color{orange} #1}}
\newcommand{\mytodo}[1]{{\color{red}TODO: #1}}
\newcommand{\largered}[1]{{

	  \LARGE\bfseries\color{red}
		#1

}}
\newcommand{\largeblue}[1]{{

	  \large\bfseries\color{blue}
		#1

}}




\usepackage{ulem} %  \dotuline{dotty} \dashuline{dashing} \sout{strikethrough}
\normalem

\usepackage{tabu} % tabular also in math mode (and much more)

\usepackage[color]{changebar} %  \cbstart, \cbend
\cbcolor{red}



% http://tex.stackexchange.com/questions/7032/good-way-to-make-textcircled-numbers
\newcommand*\circled[1]{\tikz[baseline=(char.base)]{
\node[shape=circle,draw,inner sep=2pt] (char) {#1};}}



% http://tex.stackexchange.com/questions/43346/how-do-i-get-sub-numbering-for-theorems-theorem-1-a-theorem-1-b-theorem-2

\makeatletter
\newenvironment{subtheorem}[1]{%
  \def\subtheoremcounter{#1}%
  \refstepcounter{#1}%
  \protected@edef\theparentnumber{\csname the#1\endcsname}%
  \setcounter{parentnumber}{\value{#1}}%
  \setcounter{#1}{0}%
  \expandafter\def\csname the#1\endcsname{\theparentnumber.\Alph{#1}}%
  \ignorespaces
}{%
  \setcounter{\subtheoremcounter}{\value{parentnumber}}%
  \ignorespacesafterend
}
\makeatother
\newcounter{parentnumber}


\usepackage{tabularx}% http://ctan.org/pkg/tabularx
\newcolumntype{Y}{>{\centering\arraybackslash}X}

\newcommand{\mycols}[2][3]{
	\noindent\begin{tabularx}{\textwidth}{*{#1}{Y}}
		#2
	\end{tabularx}%
}


\newcommand{\definethms}{

	%\declaretheorem[title=Theorem,qed=$\triangle$,parent=chapter]{thm}
	\newcommand{\thmqed}{$\square$} % for thms without proof
	\newcommand{\propqed}{$\square$} % for props without proof
	\declaretheorem[title=Theorem]{thm}
	\declaretheorem[title=Proposition,sibling=thm]{prop}
	\declaretheorem[title=Conjectured Proposition,sibling=thm]{cprop}

	%\declaretheorem[title=Lemma,parent=chapter]{lemma}
	\declaretheorem[sibling=thm]{lemma}
	\declaretheorem[sibling=thm,title=Conjectured Lemma]{clemma}
	\declaretheorem[title=Corollary,sibling=thm]{corr}
	\declaretheorem[sibling=thm,title=Definition,style=definition,qed=$\triangle$]{defi}
	%\declaretheorem[title=Definition,qed=$\triangle$,parent=chapter]{defi}
	\declaretheorem[title=Example,style=definition,qed=$\triangle$,sibling=thm]{exa}

	\declaretheorem[sibling=thm,title=Conjecture]{conj}

	\declaretheorem[title=Remark,style=remark,numbered=no,qed=$\triangle$]{remark}


}

\usepackage[matha]{mathabx} % the locial operators here have more space around them and [ and ] are thicker, also langle and rangle are a bit nicer; subseteq looks a bit weird

%\usepackage{MnSymbol} % again other symbols


\newcommand{\inference}{\ensuremath{\iota}}

\usepackage{cases} % numcases



% subsections also in toc
\setcounter{tocdepth}{2}
\setsecnumdepth{subsection}


\definethms

\def\proofSkipAmount{ \vskip -0.1em }


%\usepackage{bussproof}

%\usepackage{vaucanson-g}
%\usepackage{amssymb}
\usepackage{latexsym}

% for color-highlighted code
%\usepackage{color} % for grey comments
%\usepackage{alltt}

%\usepackage[doublespacing]{setspace}
%\usepackage[onehalfspacing]{setspace}
%\usepackage[singlespacing]{setspace}


\usepackage{amsthm}


\chapterstyle{madsen}


% define page numbering styles
\makepagestyle{numberCorner}
\makeevenfoot{numberCorner}{\thepage}{}{}
\makeoddfoot{numberCorner}{}{}{\thepage}

\makepagestyle{numberCenter}
%\makeevenfoot{numberCenter}{}{\thepage}{}
%\makeoddfoot{numberCenter}{}{\thepage}{}
%
%\makeevenhead{numberCenter}{\thechapter}{}{\thesection}
%\makeoddhead{numberCenter}{\thesection }{}{\thechapter}
\makeheadrule{numberCenter}{\textwidth}{1pt}

\makeevenhead{numberCenter}{\thepage}{}{\leftmark}
\makeoddhead{numberCenter}{\rightmark}{}{\thepage}


\makeatletter
\makepsmarks{numberCenter}{
	\def\chaptermark##1{\markboth{%
			\ifnum \value{secnumdepth} > -1
			\if@mainmatter
			\chaptername\ \thechapter\ --- %
			\fi
			\fi
	##1}{}}
	\def\sectionmark##1{\markright{%
			\ifnum \value{secnumdepth} > 0
			\thesection. \ %
			\fi
	##1}}
}
\makeatother
\newcommand{\mysetpagestyle}{
	%\pagestyle{numberCorner}
	\pagestyle{numberCenter}
}
\mysetpagestyle





\usepackage{refcheck}

%\settypeblocksize{0.67\stockheight}{0.67\stockwidth}{*}
%\setlrmargins{*}{*}{1.2}
%\setulmargins{*}{*}{1.4}
%\checkandfixthelayout[nearest]


\begin{document}

\chapter{ Number of quantifier alternations in Huang and nested }

\section{ Outline }

Goal: try proof Huang and nested equal.

Method: proof for both:

\begin{cprop}
	Let $I$ be an interpolant created by \$algorithm.
	If $I$ contains a term $t$ such that $t$ has $n$ color changes, then $I$ has at least $n$ quantifier alternations.
\end{cprop}

\subsection{generally keep in mind}

\begin{itemize}
	\item Need to define all new terms here: color-changing, single-color, $\Phi$-literal, substitutions from $0$ to $n$
		\begin{itemize}
			\item essentially same position: path from one position to other only contains grey symbol (this def allows for identical position as well)
		\end{itemize}
	\item also note: literal is sometimes used for negated or not negated predicate with terms but in regular formulas with arbitrary connectives
\end{itemize}

\section{ Preliminaries }

Quantifier alternations in $I$ usually assumes the quantifier-alternation-minimising arrangement of quantifiers in $I$ 

\begin{defi}[Color alternation $\ca$]
	Colors $\Gamma$ and $\Delta$, term $t$:

	$\ca(t) \defeq \ca_\bot(t)$

	Let $t = f(t_1, \dots, t_n)$ for constant, function and variable symbols (syntax abuse)


	$\ca_\Phi(t) \defeq 
	\begin{cases}
		\max(\ca_\Phi(t_1), \dots, \ca_\Phi(t_n)) & \text{$f$ is grey} \\
		\max(\ca_\Phi(t_1), \dots, \ca_\Phi(t_n)) & \text{$f$ is of color $\Phi$} \\
		1 + \max(\ca_\Psi(t_1), \dots, \ca_\Psi(t_n)) & \text{$f$ is of color $\Psi$, $\Phi \neq \Psi$} \\
	\end{cases}
	$
\end{defi}

\begin{defi}
	$\PIstepnosigma$ is defined just like $\PIstep$ but without applying any substitution.
\end{defi}
Hence $\PIstepnosigma(\cdot) \sigma = \PIstep(\cdot)$.
$C\nosigma$ is somehow the same, i.e.\ if $C = D\sigma$, then $C\nosigma = D$ where $\sigma$ is derived from the context.

\section{ Random thoughts }

\begin{itemize}
	\item Quantifiers are introduced for lifting variables which actually occur in the interpolant
	\item If term $t$ with $\ca(t) = n$ enters $I$, we need subterm $s$ of $t$ with $\ca(s) = n-1$ to be in $I$ (of course colors of $t$ and $s$ are exactly opposite)
\end{itemize}


\subsection{ Proof}
\begin{itemize}
	\item Induction over $\lifdelta{\PI(C) \lor C}$ and also about $\Gamma$-terms with $\Delta$-lifting vars in that formula. Cf. \texttt{-final}
	\item TODO: describe proof method with $\sigmazi$: which $\PI$? 
		\begin{itemize}
			\item Factorisation:
				easy: just apply $\sigma_i$ for all $i$ to $\PI(C) \lor C$. 
				When done, a literal will be there twice and we can remove it without losing anything
			\item Resolution:
				create propositional structure first.

				Ex.: $C_1 : D \lor l$, $ C_2: \lnot l \lor E$:

				If we talk about properties for which it holds that if they hold for $\PI(C_i) \lor C_i$, $i\in\{1,2\}$, then they also hold for $ A \equiv \Big((l \land \PI(C_2)) \lor (\lnot l \land \PI(C_1))\Big) \lor C\nosigma$, then we can apply $\sigma_i$ for all $i$ to that formula.

				So if we can assume it for $A$ and show it for all $\sigma_i$, we get that it holds for $\PI(C) \lor C$.

		\end{itemize}

		Also: clauses are variable disjoint, so e.g. it's not possible that a color-changing var is created by $\PIstep$

		Also: do it like a few lemmas further down, like $(\PIstepnosigma(\inference, \PI(C_1), \dots, \PI(C_n)) \lor C\nosigma ) \sigmazi$
\end{itemize}


\section{ Proof port attempt from \texttt{-final} }

\largered{ need to show that grey occurrences are in grey literals, all grey literals end up in the interpolant! }

conj: if a $\Delta$-term $t$ occurs in a $\Gamma$-literal in a clause $C$, then $t$ occurs in a grey literal in $\PI(C)$.

evidence:
\begin{itemize}
	\item situation does not occur in $\Gamma$ or $\Delta$
	\item terms are only changed by unificators
	\item $\Delta$- and $\Gamma$-terms are not unifiable, so one of the literals has to have a variable at a grey position when a $\Delta$-term enters a $\Gamma$-literal
	\item that literal has to be grey
	\item QED?
\end{itemize}

\begin{rlemma}
	If a $\Phi$-term $t$ occurs in a $\Psi$-literal in a clause $C$, then $t$ occurs at a grey position in $\PI(C)$.
\end{rlemma}
\begin{proof}
	As all grey literals of clauses involved in a refutation end up in the interpolant, it suffices to show that $t$ occurs at a grey position in a grey literal. 

	Substitutions are applied to all variables, hence we only need to consider terms $t$ which just enter a foreign colored literal.

	\mytodo{propagation 1: $\Phi$-terms vs $\Psi$-terms (in $\Psi$-literals)}

	\mytodo{propagation 2: $\Phi$-terms vs other $\Phi$-terms (in $\Psi$-literals)}

	Induction on refutation and $\sigma$; base case easy.

	Resolution or factorisation inference $\inference$. 
	Let $\lambda$ be a $\Gamma$-literal containing a variable $u$ at position $\bhat u$ such that $u\sigma_i$ contains a $\Delta$-term $t$.

	If the resolved or factorised literals are grey, they become part of $\PI(C)$ and if $t$ occurs grey there, we are done.

	\begin{itemize}
		\item Suppose the resolved literals are $\Gamma$-colored. Then IH.
		\item Suppose the resolved literals are $\Delta$-colored.
			\mytodo{}
		\item Suppose the resolved literals are grey and $t$ does not occur at a grey position in $\lambda\sigma = \lambda'\sigma$.

			\mytodo{}

	\end{itemize}


	~




\end{proof}

\begin{clemma}
	If a $\Phi$-term $t$ occurs in a $\Psi$-literal in a clause $C$, then $t$ occurs at a grey position in a grey literal in $\PI(C) \lor C$.
\end{clemma}
there has to be a variable $u$ in a $\Psi$-literal such that $u\sigma_i$ contains $t$.

\begin{clemma}
	If a variable $u$ occurs in a $\Phi$-literal as well as in a $\Psi$-literal in a clause $C$, then $t$ also occurs at a grey position in a grey literal in $\PI(C)$.
\end{clemma}
\begin{proof}
	Initially not the case.

	Note that we can only resolve/factorise $\Gamma$-/$\Delta$-/grey literals with other $\Gamma$-/$\Delta$-/grey literals as clearly the predicate symbol must be the same for both literals.
	Hence if a variable occurs only in $\Gamma$- or only in $\Delta$-literals, then it can never escape these.
	Hence $u$ certainly is contained in a grey literal.

	Now suppose that $u$ only occurs colored in grey literals.
	Then it occurs in a $\Gamma$-($\Delta$-) term in the original $\Gamma$-($\Delta$-)clauses which contain it.

	As shown before $u$ must occur in some grey literal.
	Suppose it does not occur at a grey position in a grey literal as otherwise we are done.
	Then $u$ only occurs in $\Gamma$-terms in grey literals as 

	\mytodo{it seems that now we have to deal with possible $\Gamma$-terms in $\Delta$-literals and so on $\Ra$ circular reasoning}



	The situation in question arises if some variable $u$ occurs in a $\Gamma$-literal in some clause and some variable $v$ occurs in a $\Delta$-literal in some clause (possibly the same), such that in the unified literals, $u$ and $v$ both occur at the same position in the respective literals.


\end{proof}

\newpage

\section{directly from old proof}

\begin{lemma}
	\label{lemma:col_change}
	Resolution or factorisation step $\inference$ from $\bar C$.

	If $u$ col-change var in $(\PIstepnosigma(\inference, \PI(C_1), \dots, \PI(C_n)) \lor C\nosigma)\sigmazi$, then $u$ also occurs grey in that formula.
\end{lemma}
\begin{proof}
	Abbreviation: $F \equiv (\PIstepnosigma(\inference, \PI(C_1), \dots, \PI(C_n)) \lor C\nosigma)$

	Induction over refutation and $\sigma$; base case easy.

	Step:
	Supp color change var $u$ present in 
	$F\sigmazi$. (could also say introduced, then proof would be somehow different)

	Supp $u$ not grey in 
	$F\sigmazmi$ as otherwise done.
	As a first step, we show that if a (not necessarily color-changing) variable $v$ occurs in a single-colored $\Phi$-term $t\occ{v}$ in $F\sigmazi$,
	then at least one of the following holds:
	\begin{compactenum}
	\item $v$ occurs in some single-colored $\Phi$-term in $F\sigmazmi$ \label{var_occ_1}
	\item there is a color-changing variable $w$ in $F\sigmazmi$ such that $v$ occurs grey in $w\sigma_i$.  \label{var_occ_2}
	\end{compactenum}
	We consider the different cases which can introduce a variable $v$ in a single-colored term $\Phi$: Either it has been there before, it was introduced in a s.c. $\Phi$-colored term, or a s.c.\ $\Phi$-term containing the var is in $\ran(\sigma)$.
	\begin{itemize}
		\item
			Suppose a term $t'\occ{v}$ is present in $F\sigmazmi$ such that $t'\occ{v}\sigma_i = t\occ{v}$.
			Then \ref{var_occ_1} is the case.
		\item
			Suppose a variable $w$ occurs in a single-colored $\Phi$-term in $F\sigmazmi$ such that $v$ occurs grey in $w\sigma_i$.
			Suppose furthermore that \ref{var_occ_1} is not the case, i.e.\ $v$ does not occur in a s.c.\ $\Phi$-term in $F\sigmazmi$, as otherwise we would be done.
			We show that \ref{var_occ_2} is the case.

			As $v$ occurs neither grey nor in a s.c.\ $\Phi$-term in $F\sigmazmi$ but occurs in $\ran(\sigma_i)$, it must occur in $F\sigmazmi$ and this can only be in a single-colored $\Psi$-term.

			As by assumption $v$ occurs grey in $w\sigma_i$, there must be an occurrence $\bhat w$ of $w$ in a resolved or factorised literal, say $\lambda\sigmazmi$ such that for the other resolved literal $\lambda'\sigmazmi$, $\lambda'\sigmazmi\at{\bhat w}$ is a subterm in which $v$ occurs grey.
			But as the occurrence of $v$ in $\lambda'\sigmazmi\at{\bhat w}$ must be contained in a single-colored $\Psi$-term, so is $\lambda\sigmazmi\at{\bhat w}$, hence $z$ occurs in a single-colored $\Psi$-term as well.
			Therefore \ref{var_occ_2} is the case.

		\item
			Suppose there is a variable $z$ in $F\sigmazmi$ such that $v$ occurs in a single-colored $\Phi$-term in $z\sigma_i$.
			Then $z\sigma_i$ occurs in $F\sigmazmi$, but this is a witness for~\ref{var_occ_1}.
	\end{itemize}

	Now recall that we have assumed $u$ to be a color-changing variable in $F\sigmazi$.
	Hence it occurs in a single-colored $\Gamma$-term as well as in a single-colored $\Delta$-term.
	By the reasoning above, this leads to two case:
	\begin{itemize}
		\item In $F\sigmazmi$, $u$ occurs both in some single-colored $\Gamma$-term as well as in some single-colored $\Delta$-term.
			Then we get the result by the induction hypothesis and the fact that $u\not\in\dom(\sigma_i)$ as $u$ does occur in $F\sigmazi$.
		\item Otherwise for some color $\Phi$, $u$ does not occur in a single-colored $\Phi$-term in $F\sigmazmi$.
			Then case \ref{var_occ_2} above must hold and there is some color-changing variable $w$ in $F\sigmazmi$ such that $u$ occurs grey in $w\sigmazi$.
			But then by the induction hypothesis, $w$ occurs grey in $F\sigmazmi$ and hence $u$ occurs grey in $F\sigmazi$.
			\qedhere
	\end{itemize}
\end{proof}

~

\NB{this is the heart of the proof:}\nopagebreak
\begin{lemma}
	If ${\PI(C) \lor C}$ contains a maximal colored occurrence of a $\Gamma$-term $t\occ{s}$ containing $\Delta$-term $s$, then 
	$s$ occurs grey in $\PI(C) \lor C$.
\end{lemma}
\hl{works if todo's are handled in sublemmas}
\begin{proof}
	Note that it suffices to show that at the step where $s$ is introduced as subterm of $t\occ{s}$, $s$ occurs grey in $\PI(C) \lor C$ as any later modification by substitution is applied to both occurrences $s$, so they stay equal throughout the remaining derivation. \mytodo{what if it's in $\PI(C)$ and it disappears due to not being a colored literal?}
	\medskip

	\noindent
	Induction over $\pi$ and $\sigma$.
	\medskip

	\noindent
	Base case: works
	\medskip

	\noindent
	Step:
	Resolution or factorisation inference $\inference$, $\mgu(\inference) = \sigma = \sigma_1\cdots\sigma_n$ 
	The term $t\occ{s}$ is created by one of the following two ways: 

	(we abbreviate 
	$(\PIstepnosigma(\inference, \PI(C_1), \dots, \PI(C_n)) \lor C\nosigma$ by $F$.)
	\begin{itemize}
		\item
			A variable $u$ occurs in $F\sigmazmi$ such that $u\sigma_i = t\occ{s}$.

			Then $u$ occurs in a resolved or factorised literal $\lambda\sigmazmi$ at $\bhat u$ such that at the other resolved or factorised literal $\lambda'\sigmaz{i-1}$, $\lambda'\sigmazmi\at{\bhat u} = t\occ{s}$.
			Then the condition is present at $F\sigmazmi$ and we get the result by the induction hypothesis. \mytodo{add grey here but get for free}

		\item 
			Note that we only consider maximal colored terms.

			\todo{are probably not same $t$ and $s$ as in lemma statement, which isn't technically wrong but confusing}
			Let $t\occ{u}$ be a maximal colored $\Gamma$-term in $F\sigmazmi$ such that in the tree-representation of $t\occ{u}$, the path from the root to $u$ does not contain a node labelled with a $\Delta$-symbol, and $u\sigma_i$ contains a grey occurrence of $s$.


			Suppose that $u$ occurs grey in $F\sigmazmi$.
			Then $s$ occurs grey in $F\sigmazi$ and we are done.

			Now suppose that $u$ does not occur grey in $F\sigmazmi$.

			If $u$ occurs in a single-colored $\Delta$-term in $F\sigmazmi$, then by Lemma~\ref{lemma:col_change}, $u$ also occurs grey in $F\sigmazmi$ and $s$ hence occurs grey in $F\sigmazi$. \mytodo{must be in grey literals}

			Otherwise $u$ only occurs in single-colored $\Gamma$-terms in $F\sigmazmi$. 
			As $u\sigma_i$ contains a grey occurrence of $s$, there is a position $p$ in the two resolved or factorised literals $\lambda$ and $\lambda'$ such that $\lambda\at{p} = u$ and $\lambda'\at{p}$ contains a grey occurrence of $s$.
			Furthermore, the prefix along the path to $p$ is the same in both $\lambda$ and $\lambda'$.
			But as by assumption $u$ only occurs in single-colored $\Gamma$-terms, $\lambda'\at{p}$ does so as well, so $s$ is contained in a single-colored $\Gamma$-term in $F\sigmazmi$.
			Since $s$ is a $\Delta$-term, by the induction hypothesis, $s$ occurs grey in $F\sigmazmi$ and hence also in $F\sigma_i$.
			\mytodo{add grey here but get for free}
			\qedhere
	\end{itemize}

\end{proof}

\clearpage
\section{random}

\begin{clemma}
	Supp $u$ in s.c. $\Phi$-term and that  no variable occurs grey in a grey literal.
	Then $u$ does not occur grey in a $\Psi$-literal.
\end{clemma}
\begin{proof}
	Holds initially.

	Supp holds for first $k$ deductions.

	Supp that then there is some $\Psi$-literal $l$ which contains $v$ at a grey position such that $v\sigma_i$ contains a grey $u$.
	Then $u$ and $v$ face each other in the res/fac literals in at stage $k-1$.
	By IH, there, $u$ does not occur grey in a $\Psi$-literal (and of course not grey in a grey literal). 

	As $v$ occurs grey in a $\Psi$-literal, by contraposition, $v$ does not occur in a s.c. $\Phi$-term.

	\begin{itemize}
		\item
			Supp res/fac lits are $\Psi$-colored.
			Then $u$ does not occur grey here.

			\mytodo{}

		\item
			Supp res/fac lits are grey.
			Then $u$ and $v$ do not occur grey here.
			\mytodo{}

		\item
			Supp res/fac lits are $\Phi$-colored.
			\mytodo{}
	\end{itemize}




\end{proof}


~

\begin{clemma}
	Every variable, which does not occur grey in a grey literal either does not occur in a single-colored $\Phi$-($\Psi$)-term or does not occur grey in a $\Psi$-($\Phi$)-literal.
\end{clemma}
\begin{proof}
	Supp $u$ occurs grey in $\Phi$-literal.
	We show that $u$ does not occur in a s.c.\ $\Psi$-term given that it does not occur grey in a grey literal.

	Supp s.c.\ $\Psi$-term $t\occ{v}$ s.t. $v\sigma_i$ contains grey $u$\todo{basically also grey occ of $v$ with $t\occ{u}$ conceivable, but that should be IH}.
	By IH, $v$ does not occur grey in a a $\Phi$-literal (given it does not occur grey in a grey literal). 

	By IH, there is no s.c.{} $\Psi$-term which contains $u$.
	Hence $t\occ{v}$ is not directly unified with some $t\occ{u}$.

	Supp $v$ occs grey where it is unified. then $u$ occs grey there as well.
	Then it can't be a grey literal, and it can't be a $\Phi$-literal. So it is a $\Psi$-literal.

\end{proof}

~

\begin{clemma}
	If a var occurs in $\Gamma$ but not grey in a grey literal, it does not occur grey in $\Delta$.
\end{clemma}

~

\clearpage

\section{Thursday:}

\begin{clemma}
	\label{last_lemma}
	Suppose a variable $u$ occurs  in $\PI(C) \lor C$ in a s.c.\ $\Phi$-term OR grey in a $\Phi$-literal.
	If $u$ occurs in $\PI(C) \lor C$ in a grey position in a $\Psi$-literal OR in s.c.\ $\Psi$-term in any literal, then it also occurs at a grey position in a grey literal in $\PI(C) \lor C$.
\end{clemma}
\begin{proof}
	ind step; supp $u$ as in statement ubiquitously

	Supp $\Psi$-literal with grey $v$ such that $u$ grey in $v\sigma_i$.
	Other occs of $v$:

	If $v$ also occurs in $\Phi$-term, then it occurs in a $\Phi$-term and in a grey position in a $\Psi$-literal, then by IH, it occurs grey in a grey literal in $F\sigmazmi$.
	By $\sigma_i$, in $F\sigma_i$, $u$ occurs at a grey position in a grey literal in $\PI(C) \lor C$.

	If $v$ also occurs grey in a $\Phi$-literal, then similar argument.

	done?

	~

	Supp s.c.\ $\Psi$-term contains $w$ such that $u$ grey in $w\sigma_i$.
	Supp $u$ not grey in grey literal as otw done.
	other occs of $w$:

	Supp $w$ in s.c. $\Phi$-term: then IH.

	Supp $u$ in $\Psi$-term: then IH

	Supp grey in res/fac lit:
	\begin{quote}
		Supp grey: then both occ grey in grey, done.

		Supp $\Phi$-lit: then IH on $w$

		Supp $\Psi$-: then IH on $u$
	\end{quote}

\end{proof}

\clearpage

{\tiny
\begin{clemma}
	\ref{last_lemma} nochmal gscheid:

	Let $u$ be a variable which occurs in $\PI(C)\lor C$ in some literal in a single-colored $\Phi$-term or grey in a $\Phi$-literal.

	Suppose that $u$ also occurs in $\PI(C) \lor C$ in some literal in a single-colored $\Psi$-term or grey in a $\Psi$-literal.

	Then $u$ occurs grey in a grey literal.

\end{clemma}
Note that $\Phi$ and $\Psi$ are to be read as any pair of different colors, i.e.\ $\Gamma$ and $\Delta$ as well as $\Delta$ and $\Gamma$.
\begin{proof}
	Note that initially, every pair of clauses is variable-disjoint and all symbols of a clause are either all grey or $\Phi$-colored or all grey or $\Psi$-colored, hence the lemma is vacuously true.

	Induction of $\pi$ and $\sigma$.

	Let $u$ occur in $F\sigmazi$ in a s.c.\ $\Phi$-term.
	\hl{Suppose $u$ also occurs in $F\sigmazi$ in a s.c.\ $\Phi$-term.)}
	Suppose that $u$ occurs in $F\sigmazi$ in a s.c.\ $\Psi$-term but not in $F\sigmazmi$ as otherwise we are done by the induction hypothesis \hl{($u$ in s.c.\ $\Psi$-term before)}.
	Then one of the following cases applies (positions of $v$ in res/fac lit):
	\begin{itemize}
		\item There is a s.c.\ $\Psi$-term in $F\sigmazmi$ containing a variable $v$ such that $u$ occurs grey in $v\sigma_i$.

			In the res/fac lits in $\lambda$ and $\lambda'$ respectively, $u$ and $v$ occur at the essentially same position.

			\begin{itemize}
				\item
					If that position is in an s.c. $\Phi$-term, then we are done by the IH on~$v$.
					\hl{$v$ col change}

				\item
					If that position is in an s.c. $\Psi$-term, then we are done by the IH on~$u$.
					\hl{$u$ col change} \NB{also: $u$ is in $\Phi$-term before}

				\item
					Hence suppose that position is grey:

					If it's in a $\Phi$-literal, then we are done by the IH on $v$. \hl{$\Phi$-lit + s.c.\ $\Psi$-term}

					If it's in a $\Psi$-literal, then we are done by the IH on $u$. \hl{$\Psi$-lit + s.c.\ $\Phi$-term} (basically somewhat of propagation of preexisting circumstances)

					Otherwise the occurrence is grey in a grey literal, which we wanted to show. \hl{?}
			\end{itemize}


		\item There is a variable $v$ in $F\sigmazmi$ such that $v\sigma_i$ contains a s.c.\ $\Psi$-term which contains $u$.
			Then $v\sigma_i$ occurs in $F\sigmazmi$ and we are done by the induction hypothesis.
			\hl{$u$ in s.c.\ $\Psi$-term before}
	\end{itemize}

	\mytodo{ formulate as other proof, this part more or less must work out }

	~

	\hl{2nd part:} 

	Let $u$ occur in $F\sigmazi$ in a s.c.\ $\Phi$-term.

	Supp $u$ occs grey in $\Psi$-literal in $F\sigmazi$.
	Supp not in $F\sigmazmi$ as then IH would take effect. \hl{$u$ grey in $\Psi$-lit}

	Then exists $v$ grey in $\Psi$-literal in $F\sigmazmi$ s.t.\ $u$ grey in $v\sigma_i$.
	Distinguish based on occurrence of $u$, $v$ pair in res/fac lit:

	\begin{itemize}
		\item supp in s.c.\ $\Psi$-term. then IH on $u$. \hl{$u$ col change}
		\item supp in s.c.\ $\Phi$-term. then IH on $v$. \hl{$v$ grey in $\Psi$-lit and in s.c.\ $\Phi$-term}
		\item supp grey.
			\begin{itemize}
				\item supp in $\Phi$-lit: then IH on $v$ \hl{$\Psi$-lit + $\Phi$-lit}
				\item supp in $\Psi$-lit: IH on $u$ \hl{$\Psi$-lit + s.c.\ $\Phi$-term}, but basically same as before
				\item supp in grey lit: done

			\end{itemize}

	\end{itemize}

	\mytodo{formulate similarily, in any case the IH or the desired statement is the case}



	~

	\hl{3rd/4th part:} 

	Let $u$ occur in $F\sigmazi$ in grey in a $\Phi$-lit. 
	Either it was there before, or:

	There exists a $\Phi$-lit with a variable $v$ at a grey position such that $v\sigma_i$ contains a grey $u$.
	Occurrences of position of $u$ and $v$ in res/fac lit:

	\begin{itemize}
		\item Supp in s.c.\ $\Psi$-term. Then IH on $v$. \hl{$v$ in s.c. $\Psi$-term and grey in $\Phi$-lit}
		\item Supp in s.c.\ $\Phi$-term. 
	\end{itemize}

\end{proof}

Reformulation sketch:

If some var $u$ occs in $F\sigmazi$ in a s.c.\ $\Phi$-term, then:
\begin{itemize}
	\item was there before
	\item exists col change var which is substituted for $u$ in $F\sigmazi$
	\item exists var grey in $\Phi$-lit and (usually elsewhere) in s.c. $\Psi$-term (or opposite coloring)
\end{itemize}

~

\noindent
Sublemma: If $u$ occurs grey in a $\Phi$-literal in $F\sigmazi$, then at least one of the following statments hold:
\begin{enumerate}
	\item $u$ occurs grey in a $\Phi$-literal in $F\sigmazi$.
		\label{14_1}

	\item 
		\label{14_2}
		A variable $v$ occurs grey in a $\Phi$-lit and grey in a $\Psi$-lit such that $v\sigma_i$ contains a grey occ of $u$.

	\item 
		\label{14_3}
		A variable occurs grey in a $\Phi$-lit and in an s.c. $\Psi$-term, OR

		a variable occurs grey in a $\Psi$-lit and in an s.c. $\Phi$-term.

		Additionally, either this variable is $u$ or it is some variable $v$ such that $v\sigma_i$ contains a grey $u$.

	\item
		\label{14_4}
		$u$ occurs at a grey position in a grey literal.

\end{enumerate}
\noindent
Subproof:
Suppose the lemma precondition. we consider the formation cases for this:
\begin{itemize}
	\item $u$ occurs grey in a $\Phi$-literal in $F\sigmazi$ and this fact is not changed by $\sigma_i$.
		Then we clearly have \ref{14_1}.

	\item Some $\Phi$-literal $\lambda$ in $F\sigmazmi$ contains a variable $v$ at a grey position such that $v\sigma_i$ contains a grey occ of $u$. 
		We distinguish further based on the position $p$ in the literal where $v$ and $u$ face each other.

		\begin{itemize}
			\item Suppose $p$ is in a s.c.\ $\Phi$-term. Then we have \ref{14_3} on $v$.
			\item Suppose $p$ is in a s.c.\ $\Psi$-term. Then we have \ref{14_3} on $u$.
			\item Suppose the res/fac lit is $\Psi$-colored, and the position is grey. Then we have \ref{14_2} on $v$.
			\item Suppose the res/fac lit is $\Phi$-colored, and the position is grey. Then we have \ref{14_1}
			\item Suppose the res/fac lit is grey, and the position is grey. Then we have \ref{14_4}

		\end{itemize}


\end{itemize}
}


\clearpage
\newcommand{\inv}{\ensuremath{\chi}}

In the following, we abbreviate $(\PIstepnosigma(\inference, \PI(C_1), \dots, \PI(C_n)) \lor C\nosigma)$ by $\inv$.

\begin{lemma}
	\label{lemma:var_grey_col_lit}
	Let $\inference$ be an inference in a refutation of $\Gamma\cup\Delta$.
	Suppose that a variable $u$ occurs grey in a $\Phi$-literal in $\inv\sigmazi$.
	Then at least one of the following statements holds:
	\begin{enumerate}
		\item
			\label{14_1}
			The variable $u$ occurs grey in a $\Phi$-literal in $\inv\sigmazmi$.

		\item 
			\label{14_2}
			Some variable $v$ occurs in $\inv\sigmazmi$ grey in a $\Phi$-literal as well as grey in a $\Psi$-literal such that $u$ occurs grey in $v\sigma_i$.

		\item 
			\label{14_3}
			There is a variable $v$ such that $u$ occurs grey in $v\sigma_i$\footnotemark and $v$ occurs in $\inv\sigmazmi$
			either grey in a $\Phi$-literal as well as in a single-colored $\Psi$-term in any literal, 
			or grey in a $\Psi$-literal as well as in a single-colored $\Phi$-term in any literal.
			\footnotetext{Note that this includes the case that $v=u$ and $\sigma_i$ is trivial on $u$.}

		\item
			\label{14_4}
			The variable $u$ occurs at a grey position in a grey literal in $\inv\sigmazmi$.

	\end{enumerate}
\end{lemma}
\begin{proof}
	We consider the different cases which lead to the variable $u$ in a grey position in a $\Phi$-literal in $\inv\sigmazi$:
	\begin{itemize}
		\item
			There already is a $\Phi$-literal in $\inv\sigmazmi$ which contains $u$ at a grey position and $\sigma_i$ does not change this. Then clearly \ref{14_1} is the case.

		\item
			Otherwise there must be a $\Phi$-literal in $\inv\sigmazmi$, which contains a variable $v$ at a grey position such that $u$ occurs grey in $v\sigma_i$.
			Hence in the resolved or factorised literals $\lambda$ and $\lambda'$, there is a position $p$ such that w.l.o.g.\ $\lambda\at{p} = v$ and $\lambda'\at{p}$ contains a grey occurrence of $u$, and $\lambda$ and $\lambda$ coincide along $p$.
			We distinguish based properties of the position $p$:

			\begin{itemize}
				\item
					Suppose that $p$ is contained in a single-colored $\Phi$-term.
					Then $v$ occurs grey in a $\Psi$-literal as well as in a single-colored $\Phi$-term, which suffices for \ref{14_3} as $u$ occurs grey in $v\sigma_i$.

				\item 
					Suppose that $p$ is contained in a single-colored $\Psi$-term.
					Then $u$ occurs grey in a $\Phi$-literal as well in a single-colored $\Psi$-term, which implies \ref{14_3}.

				\item
					Otherwise $p$ is a grey position.
					We distinguish further:

					\begin{itemize}
						\item Suppose that the resolved or factorised literal is $\Phi$-colored.
							Then $u$ occurs grey in a $\Phi$-literal and we have established item \ref{14_3}.

						\item Suppose that the resolved or factorised literal is $\Psi$-colored.
							Then the variable $v$ occurs grey in a $\Phi$-literal as well as grey in a $\Psi$-literal, hence \ref{14_2} is the case.

							Otherwise the resolved or factorised literal is grey and $u$ occurs grey in a grey literal, which is sufficient for \ref{14_4}.
							\qedhere
					\end{itemize}

			\end{itemize}

	\end{itemize}

\end{proof}

\begin{lemma}
	\label{lemma:var_in_sc_term}
	Let $\inference$ be an inference in a refutation of $\Gamma\cup\Delta$.
	Suppose that a variable $u$ occurs in a single-colored $\Phi$-term in $\inv\sigmazi$.
	Then at least one of the following statements holds:
	\begin{enumerate}
		\item
			\label{15_1}
			The variable $u$ occurs in a single-colored $\Phi$-term in $\inv\sigmazmi$.

		\item
			\label{15_2}
			There is a variable $v$ such that $u$ occurs grey in $v\sigma_i$ and $v$ occurs in a single-colored $\Phi$-term as well as in a single-colored $\Psi$-term in $\inv\sigmazmi$.


		\item
			\label{15_4}
			There is a variable $v$ such that $u$ occurs grey in $v\sigma_i$ and $v$ occurs in $\inv\sigmazmi$ in a single-colored $\Phi$-term as well as at a grey position in a $\Psi$-literal.

		\item
			\label{15_3}
			The variable $u$ occurs grey in a $\Phi$-literal in $\inv\sigmazmi$.

		\item
			\label{15_5}
			The variable $u$ occurs grey in a grey literal in $\inv\sigmazmi$.

	\end{enumerate}
\end{lemma}
\begin{proof}
	We consider the different cases which lead to the variable $u$ in a single-colored $\Phi$-term in $\inv\sigmazi$:

	\begin{itemize}
		\item There is a single-colored $\Phi$-term in $\inv\sigmazmi$ which contains $u$ such that $\sigma_i$ does not change this.
			Then \ref{15_1} is the case.

		\item Suppose that there is a single-colored $\Phi$-term in $\inv\sigmazmi$ which contains a variable $v$ such that $u$ occurs grey in $v\sigma_i$.

			Hence in the resolved or factorised literals $\lambda$ and $\lambda'$, there is a position $p$ such that w.l.o.g.\ $\lambda\at{p} = v$ and $\lambda'\at{p}$ contains a grey occurrence of $u$, and $\lambda$ and $\lambda$ coincide along $p$.
			We distinguish based properties of the position $p$:

			\begin{itemize}
				\item Suppose that $p$ is contained in a single-colored $\Phi$-term.
					Then $u$ is contained in a single-colored $\Phi$-term in $\inv\sigmazmi$ and item \ref{15_1} holds.

				\item Suppose that $p$ is contained in a single-colored $\Psi$-term.
					As then $v$ is contained in a single-colored $\Phi$-term as well as in a single-colored $\Psi$-term, \ref{15_2} is the case.

				\item Suppose that $p$ is a grey position.
					We distinguish further:

					\begin{itemize}
						\item Suppose that the resolved or factorised literal is $\Phi$-colored.
							Then $u$ occurs grey in a $\Phi$-literal, which suffices for \ref{15_3}. 

						\item Suppose that the resolved or factorised literal is $\Psi$-colored.
							Then the variable $v$ occurs in a single-colored $\Phi$-term as well as grey in a $\Psi$-literal, which implies \ref{15_4}.

						\item
							Otherwise the resolved or factorised literal is grey. 
							But then $u$ occurs grey in a grey literal and we have established item \ref{15_5}.
					\end{itemize}


				\item Suppose that a variable $w$ occurs in $\inv\sigmazmi$ such that $u$ occurs in a single-colored $\Phi$-term in $w\sigma_i$.
					This can only be the case if $w\sigma$ already occurs in $\inv\sigmazmi$, which implies that \ref{15_1} is the case.
					\qedhere
			\end{itemize}
	\end{itemize}

\end{proof}

\begin{lemma}
	Let $C$ be a clause in the resolution refutation $\pi$ of $\Gamma \cup \Delta$
	and $u$ be a variable which occurs in $\PI(C)\lor C$ in some literal in a single-colored $\Phi$-term or grey in a $\Phi$-literal.

	Suppose that $u$ also occurs in $\PI(C) \lor C$ in some literal in a single-colored $\Psi$-term or grey in a $\Psi$-literal.

	Then $u$ occurs grey in a grey literal.

\end{lemma}
Note that $\Phi$ and $\Psi$ are to be read as any pair of different colors, i.e.\ $\Gamma$ and $\Delta$ as well as $\Delta$ and $\Gamma$.
\begin{proof}
	We proceed by induction over $\pi$. \mytodo{ also this proof strategy as outlined above }

	Note that initially, every pair of clauses is variable-disjoint and all symbols of a clause are either all grey or $\Phi$-colored or all grey or $\Psi$-colored, hence the lemma is vacuously true.

	For the induction step,
	suppose that $u$ occurs in $\chi\sigmazi$ in a single-colored $\Phi$-term or grey in a $\Phi$-literal and 
	that $u$ also occurs in $\chi\sigmazi$ in a single-colored $\Psi$-term or grey in a $\Psi$-literal.

	Then we can deduce by Lemma~\ref{lemma:var_grey_col_lit} and Lemma~\ref{lemma:var_in_sc_term} that at least one of the following statements holds:
	\begin{enumerate}
		\item
			\label{oozoh7Oh1}
			The variable $u$ occurs grey in a $\Phi$-literal in $\inv\sigmazmi$.

		\item 
			\label{oozoh7Oh2}
			Some variable $v$ occurs in $\inv\sigmazmi$ grey in a $\Phi$-literal as well as grey in a $\Psi$-literal such that $u$ occurs grey in $v\sigma_i$.

		\item 
			\label{oozoh7Oh3}
			There is a variable $v$ such that $u$ occurs grey in $v\sigma_i$ and $v$ occurs in $\inv\sigmazmi$
			either grey in a $\Phi$-literal as well as in a single-colored $\Psi$-term in any literal, 
			or grey in a $\Psi$-literal as well as in a single-colored $\Phi$-term in any literal.

		\item
			\label{oozoh7Oh4}
			The variable $u$ occurs at a grey position in a grey literal in $\inv\sigmazmi$.

		\item
			\label{oozoh7Oh5}
			The variable $u$ occurs in a single-colored $\Phi$-term in $\inv\sigmazmi$.

		\item
			\label{oozoh7Oh6}
			There is a variable $v$ such that $u$ occurs grey in $v\sigma_i$ and $v$ occurs in a single-colored $\Phi$-term as well as in a single-colored $\Psi$-term in $\inv\sigmazmi$.


	\end{enumerate}

	\newcommand{\othercase}{$^*$}

	By the same lemmata, we get the same set of statements where $\Phi$ and $\Psi$ are interchanged. We refer to them by the respective number followed by \othercase.

	Suppose that \ref{oozoh7Oh4} is not the case as otherwise we are done since $\sigma_i$ is trivial on $u$ as $u$ occurs in $\inv\sigmazi$.
	Furthermore, there are a number of cases which give the result by the induction hypothesis:
	For the cases \ref{oozoh7Oh2}, \ref{oozoh7Oh3} and \ref{oozoh7Oh6}, we can infer that by the induction hypothesis, there is a grey occurrence of the variable $v$ in a grey literal in $\inv\sigmazmi$, and as $u$ occurs grey in $v\sigma_i$, there is a grey occurrence of $u$ in a grey literal in $\inv\sigmazi$.

	It remains to show that the lemma holds true in case the statements \ref{oozoh7Oh1} or \ref{oozoh7Oh5} as well as \ref{oozoh7Oh1}\othercase or \ref{oozoh7Oh5}\othercase hold.
	But note that in any combination of \ref{oozoh7Oh1} or \ref{oozoh7Oh5} and \ref{oozoh7Oh1}\othercase or \ref{oozoh7Oh5}\othercase in effect yields a situation under which the induction hypothesis again is applicable.
	Hence we may infer that $u$ occurs grey in a grey literal in $\inv\sigmazmi$ and since $\sigma_i$ is trivial $u$ as shown above, $u$ occurs grey in a grey literal in $\inv\sigmazi$.
\end{proof}



\end{document}

